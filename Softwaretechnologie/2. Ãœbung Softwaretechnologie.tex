\documentclass{article}

\usepackage{amsmath,amssymb}
\usepackage{tikz}
\usepackage{pgfplots}
\usepackage{xcolor}
\usepackage[left=2.1cm,right=3.1cm,bottom=3cm,footskip=0.75cm,headsep=0.5cm]{geometry}
\usepackage{enumerate}
\usepackage{enumitem}
\usepackage{marvosym}
\usepackage{tabularx}

\usepackage{listings}
\definecolor{lightlightgray}{rgb}{0.95,0.95,0.95}
\definecolor{lila}{rgb}{0.8,0,0.8}
\definecolor{mygray}{rgb}{0.5,0.5,0.5}
\definecolor{mygreen}{rgb}{0,0.8,0.26}
\lstdefinestyle{java} {language=java}
\lstset{language=java,
	basicstyle=\ttfamily,
	keywordstyle=\color{lila},
	commentstyle=\color{lightgray},
	stringstyle=\color{mygreen}\ttfamily,
	backgroundcolor=\color{white},
	showstringspaces=false,
	numbers=left,
	numbersep=10pt,
	numberstyle=\color{mygray}\ttfamily,
	identifierstyle=\color{blue},
	xleftmargin=.1\textwidth, 
	%xrightmargin=.1\textwidth,
	escapechar=§,
	breaklines=true,
	postbreak=\mbox{\space}
}

\usepackage[utf8]{inputenc}

\renewcommand*{\arraystretch}{1.4}

\newcolumntype{L}[1]{>{\raggedright\arraybackslash}p{#1}}
\newcolumntype{R}[1]{>{\raggedleft\arraybackslash}p{#1}}
\newcolumntype{C}[1]{>{\centering\let\newline\\\arraybackslash\hspace{0pt}}m{#1}}

\newcommand{\E}{\mathbb{E}}
\DeclareMathOperator{\rk}{rk}
\DeclareMathOperator{\Var}{Var}
\DeclareMathOperator{\Cov}{Cov}

\title{\textbf{Softwaretechnologie, Übung 2}}
\author{\textsc{Henry Haustein}}
\date{}

\begin{document}
	\maketitle
	
	\section*{Aufgabe 1}
	Wir erweitern die Klasse \texttt{Book} wie folgt
	\begin{lstlisting}[style=java,tabsize=2]
private boolean isLent;

public boolean getLentStatus() {
	return isLent;
}

public void setLentStatus(boolean status) {
	this.isLent = status;
}
	\end{lstlisting}
	Die Klasse \texttt{Library} wird wie folgt erweitert
	\begin{lstlisting}[style=java,tabsize=2]
public Book search(String title) {
	for (int i = 0; i < counter; i++) {
		if (title.equals(myBooks[i].getTitle())) {
			System.out.println("The book with the title " + myBooks[i].getTitle() + " exists in the library!");
			return myBooks[i];
		}
	}
	System.out.println("The book with the title " + title + " does not exist in the library!");
	return null;
}

public void loan(String title) {
	Book b = search(title);
	if (b.getLentStatus()) {
		System.out.println("Buch ist ausgeliehen");
	}
	else {
		b.setLentStatus(true);
		System.out.println("Buch wurde ausgeliehen");
	}
}
	\end{lstlisting}

	\section*{Aufgabe 2}
	Die Klasse \texttt{Inhabitant} sieht wie folgt aus
	\begin{lstlisting}[style=java,tabsize=2]
public class Inhabitant {
	protected int income;

	public Inhabitant() {
		this.income = 0;
	}

	public int getIncome() {
		return income;
	}

	public void setIncome(int inc) {
		if (inc < 0) {
			throw new IllegalArgumentException("Income negative");
		}
		else {
			this.income = inc;   
		}
	}

	public int taxableIncome() {
		return income;
	}

	public int tax() {
		return (int) Math.floor(Math.max(1,this.taxableIncome() * 0.1));
	}
}
	\end{lstlisting}
	Die Klasse \texttt{Noble}
	\begin{lstlisting}[style=java,tabsize=2]
public class Noble extends Inhabitant {
	public Noble() {
		super();
	}

	public int tax() {
		return (int) Math.floor(Math.max(20,this.taxableIncome() * 0.1));
	}
}
	\end{lstlisting}
	Die Klasse \texttt{King}
	\begin{lstlisting}[style=java,tabsize=2]
public class King extends Inhabitant {
	public King() {
		super();
	}

	public int tax() {
		return 0;
	}
}
	\end{lstlisting}
	Die Klasse \texttt{Peasant}
	\begin{lstlisting}[style=java,tabsize=2]
public class Peasant extends Inhabitant {
	public Peasant() {
		super();
	}
}
	\end{lstlisting}
	Die Klasse \texttt{Serf}
	\begin{lstlisting}[style=java,tabsize=2]
public class Serf extends Peasant {
	public Serf() {
		super();
	}

	public int taxableIncome() {
		return Math.max(0,this.getIncome() - 12);
	}
}
	\end{lstlisting}
	Die Klasse \texttt{Koenigreich}
	\begin{lstlisting}[style=java,tabsize=2]
public class Koenigreich {
	public void steuerbescheid(Inhabitant i) {
		int taxIncome = i.taxableIncome();
		int t = i.tax();
		System.out.println("Einwohner " + i + " hat Einkommen " + i.getIncome() + ", zahlt Steuern auf " + taxIncome + " und zahlt Steuern " + t);
}

	public void main(String[] args) {
		Noble n = new Noble();
		King k = new King();
		Peasant p = new Peasant();
		Serf s = new Serf();
		Serf s2 = new Serf();

		n.setIncome(2000);
		k.setIncome(3000);
		p.setIncome(100);
		s.setIncome(1);
		s2.setIncome(20);

		steuerbescheid(n);
		steuerbescheid(k);
		steuerbescheid(p);
		steuerbescheid(s);
		steuerbescheid(s2);
	}
}
	\end{lstlisting}
	
\end{document}