\documentclass{article}

\usepackage{amsmath,amssymb}
\usepackage{tikz}
\usepackage{pgfplots}
\usepackage{xcolor}
\usepackage[left=2.1cm,right=3.1cm,bottom=3cm,footskip=0.75cm,headsep=0.5cm]{geometry}
\usepackage{enumerate}
\usepackage{enumitem}
\usepackage{marvosym}
\usepackage{tabularx}

\usepackage{listings}
\definecolor{lightlightgray}{rgb}{0.95,0.95,0.95}
\definecolor{lila}{rgb}{0.8,0,0.8}
\definecolor{mygray}{rgb}{0.5,0.5,0.5}
\definecolor{mygreen}{rgb}{0,0.8,0.26}
\lstdefinestyle{java} {language=java}
\lstset{language=java,
	basicstyle=\ttfamily,
	keywordstyle=\color{lila},
	commentstyle=\color{lightgray},
	stringstyle=\color{mygreen}\ttfamily,
	backgroundcolor=\color{white},
	showstringspaces=false,
	numbers=left,
	numbersep=10pt,
	numberstyle=\color{mygray}\ttfamily,
	identifierstyle=\color{blue},
	xleftmargin=.1\textwidth, 
	%xrightmargin=.1\textwidth,
	escapechar=§,
	breaklines=true,
	postbreak=\mbox{\space}
}

\usepackage[utf8]{inputenc}

\renewcommand*{\arraystretch}{1.4}

\newcolumntype{L}[1]{>{\raggedright\arraybackslash}p{#1}}
\newcolumntype{R}[1]{>{\raggedleft\arraybackslash}p{#1}}
\newcolumntype{C}[1]{>{\centering\let\newline\\\arraybackslash\hspace{0pt}}m{#1}}

\newcommand{\E}{\mathbb{E}}
\DeclareMathOperator{\rk}{rk}
\DeclareMathOperator{\Var}{Var}
\DeclareMathOperator{\Cov}{Cov}

\title{\textbf{Softwaretechnologie, Übung 3}}
\author{\textsc{Henry Haustein}}
\date{}

\begin{document}
	\maketitle
	
	\section*{Aufgabe 1}
	\begin{enumerate}[label=(\alph*)]
		\item Testfalltabelle
		\begin{center}
			\begin{tabular}{c|c|c|c|c}
				\textbf{Testfall} & \textbf{erwarteter Status} & \textbf{Klasse} & \textbf{Einkommen} & \textbf{erwartete Ausgabe} \\
				\hline
				E1 & Exception & Einwohner & -1 & IllegalArgumentException \\
				E2 & int & Einwohner & 0 & 1 \\
				E3 & int & Einwohner & 20 & 2 \\
				E4 & int & Einwohner & 25 & 2 \\
				E5 & int & Einwohner & 19 & 1
			\end{tabular}
 		\end{center}
		\item JUnit-Testfallklasse
		\begin{lstlisting}[style=java]
import junit.framework.TestCase;

public class EinwohnerTest extends TestCase { 
	protected Einwohner einwohner;
	
	protected void setUp() {
		this.einwohner = new Einwohner();
	}
	
	public void testE1() {
		try {
			einwohner.setEinkommen(-1);
			fail("No Exception thrown");
		} catch(IllegalArgumentException e){
			System.out.println("Cry :(");
			assertEquals("Du Lappen, kein Einkommen ist negativ!", e.getMessage());
		} catch (Exception e) {
			fail("Wrong Exception thrown");
		}
	}
	// assertEquals(boolean expected, boolean actual)
	public void testE2() {
		einwohner.setEinkommen(0);
		assertEquals(1, einwohner.steuer());
	}
	
	public void testE3() {
		einwohner.setEinkommen(20);
		assertEquals(2, einwohner.steuer());
	}
	
	public void testE4() {
		einwohner.setEinkommen(25);
		assertEquals(2, einwohner.steuer());
	}
	
	public void testE5() {
		einwohner.setEinkommen(19);
		assertEquals(1, einwohner.steuer());
	}
		
}
		\end{lstlisting}
		\item Testfalltabelle \#2 für weitere Klassen
		\begin{center}
			\begin{tabular}{c|c|c|c|c}
				\textbf{Testfall} & \textbf{erwarteter Status} & \textbf{Klasse} & \textbf{Einkommen} & \textbf{erwartete Ausgabe} \\
				\hline
				A1 & Exception & Adel & -1 & IllegalArgumentException \\
				A2 & int & Adel & 0 & 20 \\
				A3 & int & Adel & 221 & 22 \\
				A4 & int & Adel & 225 & 22 \\
				A5 & int & Adel & 19 & 20 \\
				A6 & int & Adel & 220 & 22 \\
				K1 & Exception & Koenig & -1 & IllegalArgumentException \\
				K2 & int & Koenig & 1000 & 0 \\
				L1 & Exception & Koenig & -1 & IllegalArgumentException \\
				L2 & int & Leibeigener & 0 & 1 \\
				L3 & int & Leibeigener & 12 & 1 \\
				L4 & int & Leibeigener & 32 & 2 \\
				L4 & int & Leibeigener & 37 & 2
			\end{tabular}
		\end{center}
		\item Code:
		\begin{lstlisting}[style=java]
import junit.framework.TestSuite;

public class Test {
	public static void main(String[] args) {
		TestSuite suite = new TestSuite();
		suite.addTest(new EinwohnerTest());
		suite.run();
	}
}
		\end{lstlisting}
	\end{enumerate}
	
\end{document}