\documentclass{article}

\usepackage{amsmath,amssymb}
\usepackage{tikz}
\usepackage{pgfplots}
\usepackage{xcolor}
\usepackage[left=2.1cm,right=3.1cm,bottom=3cm,footskip=0.75cm,headsep=0.5cm]{geometry}
\usepackage{enumerate}
\usepackage{enumitem}
\usepackage{marvosym}
\usepackage{tabularx}

\usepackage{listings}
\definecolor{lightlightgray}{rgb}{0.95,0.95,0.95}
\definecolor{lila}{rgb}{0.8,0,0.8}
\definecolor{mygray}{rgb}{0.5,0.5,0.5}
\definecolor{mygreen}{rgb}{0,0.8,0.26}
\lstdefinestyle{java} {language=java}
\lstset{language=java,
	basicstyle=\ttfamily,
	keywordstyle=\color{lila},
	commentstyle=\color{lightgray},
	stringstyle=\color{mygreen}\ttfamily,
	backgroundcolor=\color{white},
	showstringspaces=false,
	numbers=left,
	numbersep=10pt,
	numberstyle=\color{mygray}\ttfamily,
	identifierstyle=\color{blue},
	xleftmargin=.1\textwidth, 
	%xrightmargin=.1\textwidth,
	escapechar=§,
	breaklines=true,
	postbreak=\mbox{\space}
}

\usepackage[utf8]{inputenc}

\renewcommand*{\arraystretch}{1.4}

\newcolumntype{L}[1]{>{\raggedright\arraybackslash}p{#1}}
\newcolumntype{R}[1]{>{\raggedleft\arraybackslash}p{#1}}
\newcolumntype{C}[1]{>{\centering\let\newline\\\arraybackslash\hspace{0pt}}m{#1}}

\newcommand{\E}{\mathbb{E}}
\DeclareMathOperator{\rk}{rk}
\DeclareMathOperator{\Var}{Var}
\DeclareMathOperator{\Cov}{Cov}

\title{\textbf{Softwaretechnologie, Übung 4}}
\author{\textsc{Henry Haustein}}
\date{}

\begin{document}
	\maketitle
	
	\section*{Aufgabe 1}
	\begin{enumerate}[label=(\alph*)]
		\item Problem ist, dass man beim Casten schnell Fehler macht, die beim Kompilieren nicht erkannt werden, aber zur Laufzeit einen Fehler geben:
		\begin{lstlisting}[style=java, tabsize=2]
Bootle beerBottle = new Bottle();
beerBottle.fill (new WhiteWine("Burgunder"));

Beer beer = (Beer) beerBottle.empty();
// ClassCastException
		\end{lstlisting}
		\item Zu jedem Drink gibt es eine Klasse, die die Flasche für diesen Drink bereitstellt.
		\item Datei \texttt{Bottle.java}
		\begin{lstlisting}[style=java,tabsize=2]
public class Bottle<T extends Drink> {
	private T content;
		
	public Bottle() {
		this.content = null;
	}
		
	public boolean isEmpty() {
		if (content == null) {
			return true;
		}
		else {
			return false;
		}
	}
		
	public void fill(T con) {
		if (this.isEmpty()) {
			this.content = con;
		}
		else {
			throw new IllegalStateException("Bottle not empty!");
		}
	}
		
	public T empty() {
		if (this.isEmpty()) {
			throw new IllegalStateException("Bottle must be filled");
		}
		else {
			T oldcontent = content;
			content = null;
			return oldcontent;   
		}
	}
}
		\end{lstlisting}
		Datei \texttt{Drink.java}
		\begin{lstlisting}[style=java, tabsize=2]
public abstract class Drink {
		
}
		\end{lstlisting}
		Datei \texttt{Beer.java}
		\begin{lstlisting}[style=java, tabsize=2]
public class Beer extends Drink {
	private String brewery;
		
	public Beer(String brew) {
		this.brewery = brew;
	}
		
	public String getBrewery() {
		return brewery;
	}
		
	public String toString() {
		return brewery;
	}
}
		\end{lstlisting}
		Datei \texttt{Wine.java}
		\begin{lstlisting}[style=java,tabsize=2]
public abstract class Wine extends Drink {
	private String region;
		
	public Wine(String reg) {
		this.region = reg;
	}
		
	public String getRegion() {
		return region;
	}
		
	public String toString() {
		return region;
	}
}
		\end{lstlisting}
		Datei \texttt{WhiteWine.java}
		\begin{lstlisting}[style=java, tabsize=2]
public class WhiteWine extends Wine {
	public WhiteWine(String reg) {
		super(reg);
	}
}
		\end{lstlisting}
		Datei \texttt{RedWine.java}
		\begin{lstlisting}[style=java,tabsize=2]
public class RedWine extends Wine {
	public RedWine(String reg) {
		super(reg);
	}
}
		\end{lstlisting}
		Datei \texttt{Bar.java}
		\begin{lstlisting}[style=java,tabsize=2]
public class Bar {
	public static void main(String[] args) {
		RedWine rw = new RedWine("Barolo");
		WhiteWine ww = new WhiteWine("Burgunder");
		Bottle<Beer> b = new Bottle<>();

		b.fill(new Beer("Uri"));

		Beer beer = b.empty();
	}
}
		\end{lstlisting}
	\end{enumerate}

	\section*{Aufgabe 2}
	Datei \texttt{Book.java}
	\begin{lstlisting}[style=java]
public class Book implements Comparable<Book>{
	private String isbn;
	private String author;
	private String title;

	public Book(String isbn, String author, String title) {
		if (isbn == null | author == null | title == null) {
			throw new IllegalArgumentException("arguments for book can't be null");
		}
		this.isbn = isbn;
		this.author = author;
		this.title = title;
	}

	//ueberladen des Konstruktors
	public Book(String isbn) {
		if (isbn == null) {
			throw new IllegalArgumentException("arguments for book can't be null");
		}
		this.isbn = isbn;
	}

	public String getIsbn() {
		return this.isbn;
	}

	public void setIsbn(String isbn) {
		this.isbn = isbn;
	}

	public String getAuthor() {
		return this.author;
	}

	public void setAuthor(String author) {
		this.author = author;
	}

	public String getTitle() {
		return this.title;
	}

	public void setTitle(String title) {
		this.title = title;
	}

	public String toString() {
		return this.getIsbn() + " " + this.getAuthor() + " " + this.getTitle();
	}

	// notwendig fuer binaere Suche aus dem Collections-Framework
	public int compareTo(Book book) {
		return this.isbn.compareTo(book.getIsbn());
	}
	\end{lstlisting}
	Datei \texttt{Library.java}
	\begin{lstlisting}[style=java,tabsize=2]
public class Library {
	List<Book> book_list;

	public Library() {
		book_list = new ArrayList<Book>();
	}

	// es sollte nach ISBN sortiert eingefuegt werden, da fuer die binaere Suche so einfacher
	public sortedInsert(Book newBook) {
		book_list.add(newBook);
		Collections.sort(book_list);
	}

	public Book searchForISBN_new(String isbn) {
		int index = Collections.binarySearch(book_list, new Book(isbn,"",""));
		// was passiert, wenn die Suche nix findet?
		// -1 kommt dann zurueck! not fast enough.. :(
		if (index >= 0) {
			return book_list.get(index);
		} else {
			return null; //besser beim Aufruf dann ueberpruefen 
		}
	}

	public Collection<Book> searchForAuthor(String author){
		ArrayList<Book> autorBook_list = new ArrayList<Book>();
		for (Book book : book_list) {
			if(book.getAuthor.equals(author)) {
				autorBook_list.add(book);
			}
		}
		return autorBook_list;
	}
}
	\end{lstlisting}
	
\end{document}