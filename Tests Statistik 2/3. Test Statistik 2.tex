\documentclass{article}

\usepackage{amsmath,amssymb}
\usepackage{tikz}
\usepackage{xcolor}
\usepackage[left=2.1cm,right=3.1cm,bottom=3cm,footskip=0.75cm,headsep=0.5cm]{geometry}
\usepackage{enumerate}
\usepackage{enumitem}
\usepackage{marvosym}
\usepackage{tabularx}
\usepackage{pgfplots}
\pgfplotsset{compat=1.10}
\usepgfplotslibrary{fillbetween}

\usepackage[utf8]{inputenc}

\renewcommand*{\arraystretch}{1.4}

\newcolumntype{L}[1]{>{\raggedright\arraybackslash}p{#1}}
\newcolumntype{R}[1]{>{\raggedleft\arraybackslash}p{#1}}
\newcolumntype{C}[1]{>{\centering\let\newline\\\arraybackslash\hspace{0pt}}m{#1}}

\DeclareMathOperator{\tr}{tr}
\DeclareMathOperator{\Var}{Var}
\DeclareMathOperator{\Cov}{Cov}
\newcommand{\E}{\mathbb{E}}

\title{\textbf{Statistik 2, Test 3}}
\author{\textsc{Henry Haustein}}
\date{}

\begin{document}
	\maketitle
	
	\section*{Aufgabe 1}
	Die erwartungstreuen Schätzer sind $\hat{\mu}_3$ und $\hat{\mu}_4$.
	\begin{align}
		\E(\hat{\mu}_3) &= \E\left(\frac{n^2-1}{(3n-3)(n+1)}(X_1+X_2+X_3)\right) \notag \\
		&= \frac{n^2-1}{3n^2+3n-3n-3}\E(X_1) + \E(X_2) + \E(X_3) \notag \\
		&= \frac{n^2-1}{3(n^2-1)}3\mu \notag \\
		&= \mu \notag \\
		\E(\hat{\mu}_4) &= \E\left( \frac{1}{5}\sum_{i=1}^{5} X_i\right) \notag \\
		&= \frac{1}{5}\cdot 5\mu \notag \\
		&= \mu \notag
	\end{align}

	\section*{Aufgabe 2}
	Die wichtigen Quantile für diese Aufgabe sind $z_{0.975}=1.95996$ für $\alpha=0.05$ und $z_{0.95}=1.64485$ für $\alpha=0.1$. Zusätzlich muss man aufpassen, dass "krumme" $n$ nicht zulässig sind, man muss also entsprechend runden. Die Lösungen der Gleichungen sind
	\begin{enumerate}[label=(\alph*)]
		\item $n \ge 170.73 \Rightarrow n = 171$
		\item $n \ge 120.25 \Rightarrow n = 121$
		\item $\alpha=0.16$ (Die Lösung ist etwas komplizierter zu bestimmen, ich gehe darauf später ein)
		\item $s=7.26048$
	\end{enumerate}
	Nun zur Bestimmung von $\alpha$:
	\begin{itemize}
		\item Lösen der Gleichung $\frac{10}{\sqrt{90}\cdot x \le 1.5}$ ergibt $x \le 1.42$
		\item Suchen in der Formelsammlung auf Seite 23 nach einem $z_\alpha$ was kleiner als 1.42 ist. (Auszug aus dieser Tabelle:)
		\begin{center}
			\begin{tabular}{c|c}
				$\alpha$ & $z_{\alpha}$ \\
				\hline
				0.925 & 1.4395 \\
				0.920 & 1.4051 \\
				0.915 & 1.3722
			\end{tabular}
		\end{center}
		$\Rightarrow 1-\frac{\alpha}{2}=0.92$
		\item Umstellen nach $\alpha$ ergibt $\alpha=0.16$. (Dieses Verfahren ist sehr ungenau, eine Computerlösung ergibt $\alpha=0.15473$.)
	\end{itemize}

	\section*{Aufgabe 3}
	Die Lösung ist
	\begin{itemize}
		\item Der wahre Parameter liegt mit einer Wahrscheinlichkeit von 5\% außerhalb des KI. \textcolor{green!80!black}{wahr}
		\item Die Wahrscheinlichkeit, dass der wahre Parameter innerhalb des KI liegt, ist 97.5\%. \textcolor{red}{falsch, die Wahrscheinlichkeit ist 95\%.}
		\item Der Wert 60 bildet die Mitte des KI. \textcolor{green!80!black}{wahr}
		\item Um das KI zu berechnen, müssen die tabellarischen Werte der Normalverteilung verwendet werden. \textcolor{red}{falsch, es muss die $t$-Verteilung benutzt werden, da $\sigma^2$ unbekannt ist.}
		\item Das KI ist asymmetrisch. \textcolor{red}{falsch, nein das KI ist symmetrisch}
	\end{itemize}

	\section*{Aufgabe 4}
	Die einzige richtige Antwort ist, dass mit steigendem $\alpha$ das KI kleiner wird. Wenn $\alpha$ größer wird, so steigt die Chance, dass man mit dem KI nicht mehr den wahren Wert erreicht. Das klappt natürlich nur, wenn das KI kleiner wird.

	\section*{Aufgabe 5}
	Das KI ist
	\begin{align}
		KI = \left[\bar{X} \mp z_{1-\frac{\alpha}{2}}\frac{\sqrt{\sigma^2}}{\sqrt{n}}\right] \notag
	\end{align}
	Einsetzen der Werte gibt [23.48;25.72].
	
	\section*{Aufgabe 6}
	Mit der Formel für das KI aus Aufgabe muss, mann folgende Gleichung lösen:
	\begin{align}
		1.95996\cdot \frac{\sqrt{\sigma^2}}{\sqrt{n}} &= 5 \notag
	\end{align}
	Umstellen liefert $n\ge 188.23$, also muss $n=189$ sein.
	
\end{document}