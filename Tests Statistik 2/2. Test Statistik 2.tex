\documentclass{article}

\usepackage{amsmath,amssymb}
\usepackage{tikz}
\usepackage{xcolor}
\usepackage[left=2.1cm,right=3.1cm,bottom=3cm,footskip=0.75cm,headsep=0.5cm]{geometry}
\usepackage{enumerate}
\usepackage{enumitem}
\usepackage{marvosym}
\usepackage{tabularx}
\usepackage{pgfplots}
\pgfplotsset{compat=1.10}
\usepgfplotslibrary{fillbetween}

\usepackage[utf8]{inputenc}

\renewcommand*{\arraystretch}{1.4}

\newcolumntype{L}[1]{>{\raggedright\arraybackslash}p{#1}}
\newcolumntype{R}[1]{>{\raggedleft\arraybackslash}p{#1}}
\newcolumntype{C}[1]{>{\centering\let\newline\\\arraybackslash\hspace{0pt}}m{#1}}

\DeclareMathOperator{\tr}{tr}
\DeclareMathOperator{\Var}{Var}
\DeclareMathOperator{\Cov}{Cov}
\newcommand{\E}{\mathbb{E}}

\title{\textbf{Statistik 2, Test 2}}
\author{\textsc{Henry Haustein}}
\date{}

\begin{document}
	\maketitle
	
	\section*{Aufgabe 1}
	\begin{enumerate}[label=(\alph*)]
		\item $\E(X) = -5$
		\item $\E(\E(X)) = \E(-5) = -5$
		\item $\E(\Var(X)) = \E(25) = 25$
		\item $\Var(X) = 25$
		\item $\Var(\E(X)) = \Var(-5) = 0$
		\item $\Var(\Var(X)) = \Var(25) = 0$
		\item $\E(X+7) = \E(X) + 7 = 2$
		\item $\Var(X+7) = \Var(X) = 25$
		\item $\E(-2X) = -2\cdot \E(X) = 10$
		\item $\Var(-2X) = (-2)^2\cdot \Var(X) = 100$
	\end{enumerate}

	\section*{Aufgabe 2}
	Die richtigen Antworten sind:
	\begin{itemize}
		\item Der Erwartungswert des Schätzers $\E(\hat{\vartheta})$ für den Parameter $\vartheta$ ergibt den Parameter selbst.
		\item Der mittlere quadratische Fehler strebt mich wachsendem Stichprobenumfang $n$ gegen 0.
		\item Der mittlere quadratische Fehler ist möglichst klein.
	\end{itemize}

	\section*{Aufgabe 3}
	Die richtige Reihenfolge ist
	\begin{enumerate}[label=\arabic*.]
		\item Aufstellen der Dichtefunktion (stetig) bzw. Wahrscheinlichkeitsfunktion (diskret) für die einzelnen Stichprobenvariablen.
		\item Bilden der Likelihood-Funktion.
		\item Berechnen der Log-Likelihood-Funktion.
		\item Nullsetzen der ersten Ableitung.
		\item Gleichung auflösen nach dem zu schätzenden Parameter $\vartheta$.
		\item Bildung der zweiten Ableitung zur Überprüfung eines Maximums.
	\end{enumerate}

	\section*{Aufgabe 4}
	Die richtige Reihenfolge ist
	\begin{enumerate}[label=\arabic*.]
		\item Annahme eines zugrundeliegenden Modells.
		\item Aufstellen der Zielfunktion.
		\item Bildung und Nullsetzen der ersten Ableitung.
		\item Auflösen der Gleichung nach dem zu schätzenden Parameter $\vartheta$.
		\item Bildung der zweiten Ableitung zur Überprüfung eines Maximums.
	\end{enumerate}

	\section*{Aufgabe 5}
	Das lineare Regressionsmodell wurde wie folgt aufgestellt:
	\begin{align}
		\text{Preis} &= \beta_0 + \beta_1\cdot\text{Quadratmeter} \notag \\
		&= 976\text{ \EUR} + 2191\text{ \EUR}\cdot\text{Quadratmeter} \notag
	\end{align}
	Einsetzen von 60 Quadratmeter ergibt einen Preis von 132436 \EUR.
	
	\section*{Aufgabe 6}
	Die richtigen Antworten sind:
	\begin{itemize}
		\item Abbildung 2 und 4 sind für ein lineares Modell geeignet.
		\item In den Daten sollte keine Heteroskedastizität vorliegen.
		\item Die Varianz in Abbildung 3 steigt mit steigendem $x$.
	\end{itemize}
	
\end{document}