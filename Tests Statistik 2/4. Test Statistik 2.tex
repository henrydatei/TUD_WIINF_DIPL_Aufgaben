\documentclass{article}

\usepackage{amsmath,amssymb}
\usepackage{tikz}
\usepackage{xcolor}
\usepackage[left=2.1cm,right=3.1cm,bottom=3cm,footskip=0.75cm,headsep=0.5cm]{geometry}
\usepackage{enumerate}
\usepackage{enumitem}
\usepackage{marvosym}
\usepackage{tabularx}
\usepackage{pgfplots}
\pgfplotsset{compat=1.10}
\usepgfplotslibrary{fillbetween}

\usepackage[utf8]{inputenc}

\renewcommand*{\arraystretch}{1.4}

\newcolumntype{L}[1]{>{\raggedright\arraybackslash}p{#1}}
\newcolumntype{R}[1]{>{\raggedleft\arraybackslash}p{#1}}
\newcolumntype{C}[1]{>{\centering\let\newline\\\arraybackslash\hspace{0pt}}m{#1}}

\DeclareMathOperator{\tr}{tr}
\DeclareMathOperator{\Var}{Var}
\DeclareMathOperator{\Cov}{Cov}
\newcommand{\E}{\mathbb{E}}

\title{\textbf{Statistik 2, Test 4}}
\author{\textsc{Henry Haustein}}
\date{}

\begin{document}
	\maketitle
	
	\section*{Aufgabe 1}
	Die richtigen Aussagen sind
	\begin{itemize}
		\item Die Ablehnung von $H_0$ führt zu der Annahme von $H_1$.
		\item Wenn die $H_0$-Hypothese beibehalten wird, obwohl sie falsch ist, handelt es sich um einen Fehler 2. Art.
		\item Die Ablehnung von $H_0$ bei Korrektheit der Hypothese führt zu keinem Fehler.\footnote{Meiner Meinung nach ist diese Aussage falsch. $H_0$ wird abgelehnt, aber ist richtig, führt zu einem Fehler 1. Art.}
	\end{itemize}

	\section*{Aufgabe 2}
	Die richtigen Aussagen sind
	\begin{itemize}
		\item Die Prüfgröße ergibt sich mit $T=(n-1)\frac{S^2}{\sigma_0^2}$.
		\item Bei der Durchführung des Tests ergibt sich ein $p$-Wert von 0.01. Zuvor wurde eine Irrtumswahrscheinlichkeit von 5\% festgelegt. In diesem Fall kann man die $H_0$-Hypothese ablehnen.
		\item Die kritischen Werte sind mit $\chi^2_{n-1;\frac{\alpha}{2}}$ und $\chi^2_{n-1;1-\frac{\alpha}{2}}$ gegeben.
	\end{itemize}

	\section*{Aufgabe 3}
	Die richtigen Aussagen sind
	\begin{itemize}
		\item Das $\alpha$ stellt die obere Schranke für die Wahrscheinlichkeit des Fehlers 1. Art dar.
		\item Die Prüfgröße für diesen Fall ist $T=\frac{\bar{X}-\mu_0}{\sigma}\sqrt{n}$.
		\item $H_0$ wird abgelehnt, wenn die Prüfgröße kleiner ist als der kritische Wert.
	\end{itemize}

	\section*{Aufgabe 4}
	Wenn der $p$-Wert kleiner als die Irrtumswahrscheinlichkeit $\alpha$ ist, wird $H_0$ abgelehnt, also gilt hier:
	\begin{itemize}
		\item $\alpha=0.005$: $H_0$ wird nicht abgelehnt
		\item $\alpha=0.05$: $H_0$ wird abgelehnt
		\item $\alpha=0.1$: $H_0$ wird abgelehnt
		\item $\alpha=0.01$: $H_0$ wird nicht abgelehnt
 	\end{itemize}

	\section*{Aufgabe 5}
	\begin{enumerate}[label=(\alph*)]
		\item Da $\alpha=5\%$ und eine obere Schranke für den Fehler 1. Art ist, ist dieser maximal 0.05
		\item Der kritische Wert für einen linksseitigen Test ist $z_\alpha$ ($=-z_{1-\alpha}$), also $z_{0.05}=-1.6449$
	\end{enumerate}
	
	\section*{Aufgabe 6}
	Wenn der $p$-Wert kleiner als die Irrtumswahrscheinlichkeit $\alpha$ ist, wird $H_0$ abgelehnt, also sind hier alle Aussagen richtig.
	
\end{document}