\documentclass{article}

\usepackage{amsmath,amssymb}
\usepackage{tikz}
\usepackage{pgfplots}
\usepackage{xcolor}
\usepackage[left=2.1cm,right=3.1cm,bottom=3cm,footskip=0.75cm,headsep=0.5cm]{geometry}
\usepackage{enumerate}
\usepackage{enumitem}
\usepackage{marvosym}
\usepackage{tabularx}
\usepackage{multirow}
\usepackage[colorlinks = true, linkcolor = blue, urlcolor  = blue, citecolor = blue, anchorcolor = blue]{hyperref}
\usepackage{ulem}

\usepackage{listings}
\definecolor{lightlightgray}{rgb}{0.95,0.95,0.95}
\definecolor{lila}{rgb}{0.8,0,0.8}
\definecolor{mygray}{rgb}{0.5,0.5,0.5}
\definecolor{mygreen}{rgb}{0,0.8,0.26}
\lstdefinestyle{R} {language=R}
\lstset{language=R,
	basicstyle=\ttfamily,
	keywordstyle=\color{lila},
	commentstyle=\color{lightgray},
	stringstyle=\color{mygreen}\ttfamily,
	backgroundcolor=\color{white},
	showstringspaces=false,
	numbers=left,
	numbersep=10pt,
	numberstyle=\color{mygray}\ttfamily,
	identifierstyle=\color{blue},
	xleftmargin=.1\textwidth, 
	%xrightmargin=.1\textwidth,
	escapechar=§,
}

\usepackage[utf8]{inputenc}

\renewcommand*{\arraystretch}{1.4}

\newcolumntype{L}[1]{>{\raggedright\arraybackslash}p{#1}}
\newcolumntype{R}[1]{>{\raggedleft\arraybackslash}p{#1}}
\newcolumntype{C}[1]{>{\centering\let\newline\\\arraybackslash\hspace{0pt}}m{#1}}

\newcommand{\E}{\mathbb{E}}
\DeclareMathOperator{\rk}{rk}
\DeclareMathOperator{\Var}{Var}
\DeclareMathOperator{\Cov}{Cov}
\DeclareMathOperator{\SD}{SD}
\DeclareMathOperator{\Cor}{Cor}

\title{\textbf{Mensch-Computer-Interaktion, Übung 5}}
\author{\textsc{Henry Haustein}, \textsc{Dennis Rössel}}
\date{}

\begin{document}
	\maketitle
	
	\section*{Aufgabe 5.1: Task-Load-Index und System Usability Scale}
	\begin{enumerate}[label=(\alph*)]
		\item TXL, Task: Ein Dokument hochladen \\
		Rating
		\begin{center}
			\begin{tabular}{l|c|c|c}
				& \textbf{User 1} & \textbf{User 2} & \textbf{User 3} \\
				\hline
				Mental Demand & 20 & 15 & 25 \\
				Physical Demand & 5 & 10 & 5 \\
				Temporal Demand & 30 & 25 & 30 \\
				Perfomance & 0 & 0 & 5 \\
				Effort & 5 & 10 & 10 \\
				Frustration & 0 & 0 & 0
			\end{tabular}
		\end{center}
		Weights
		\begin{center}
			\begin{tabular}{l|c|c|c}
				& \textbf{User 1} & \textbf{User 2} & \textbf{User 3} \\
				\hline
				Mental Demand & 1 & 2 & 0 \\
				Physical Demand & 0 & 0 & 0 \\
				Temporal Demand & 2 & 4 & 3 \\
				Perfomance & 5 & 2 & 4 \\
				Effort & 2 & 2 & 3 \\
				Frustration & 5 & 5 & 5
			\end{tabular}
		\end{center}
		Result
		\begin{center}
			\begin{tabular}{l|c|c|c}
				& \textbf{User 1} & \textbf{User 2} & \textbf{User 3} \\
				\hline
				Mental Demand & 20 & 30 & 0 \\
				Physical Demand & 0 & 0 & 0 \\
				Temporal Demand & 60 & 100 & 90 \\
				Perfomance & 0 & 0 & 20 \\
				Effort & 10 & 20 & 30 \\
				Frustration & 0 & 0 & 0 \\
				\hline
				$\Sigma$ & 90 & 150 & 140 \\
				AVG & 6 & 10 & 9.33
			\end{tabular}
		\end{center}
		\item SUS, Task: Ein Dokument hochladen
		\begin{center}
			\begin{tabular}{L{7cm}|c|c|c|c|c|c}
				\textbf{Frage} & $X_{U1}$ & $X_{U2}$ & $X_{U3}$ & $Y_{U1}$ & $Y_{U2}$ & $Y_{U3}$ \\
				\hline
				 I think that I would like to use this system frequently. & 5 & 4 & 5 & 4 & 3 & 4 \\
				 I found the system unnecessarily complex. & 1 & 1 & 1 & 4 & 4 & 4 \\
				 I thought the system was easy to use. & 4 & 4 & 5 & 3 & 3 & 4 \\
				 I think that I would need the support of a technical person to be able to use this system. & 1 & 1 & 1 & 4 & 4 & 4 \\
				 I found the various functions in this system were well integrated. & 5 & 4 & 5 & 4 & 3 & 4 \\
				 I thought there was too much inconsistency in this system. & 1 & 2 & 1 & 4 & 3 & 4 \\
				 I would imagine that most people would learn to use this system very quickly. & 4 & 4 & 4 & 3 & 3 & 3 \\
				 I found the system very cumbersome to use. & 1 & 1 & 1 & 4 & 4 & 4 \\
				 I felt very confident using the system. & 4 & 5 & 5 & 3 & 4 & 4 \\
				 I needed to learn a lot of things before I could get going with this system. & 1 & 2 & 1 & 4 & 3 & 4 \\
				 \hline
				 $\Sigma$ & & & & 37 & 34 & 39 \\
				 \hline
				 Score & & & & 92.5 & 85 & 97.5
			\end{tabular}
		\end{center}
	\end{enumerate}

	\section*{Aufgabe 5.2: }
	Mit R ergibt sich
	\begin{lstlisting}[style=R]
mouse = read.csv2("Testdaten_Mouse.csv")
touch = read.csv2("Testdaten_Touchpad.csv")
cor(mouse$AVG,mouse$ID)
cor(touch$AVG,touch$ID)
summary(lm(AVG ~ ID, data = mouse))
summary(lm(AVG ~ ID, data = touch))
	\end{lstlisting}
	Ergibt eine Korrelation von 0.8633773 bzw. 0.9289729 und
	\begin{center}
\begin{tabular}{l c c}
\hline
 & Maus & Touchpad \\
\hline
(Intercept) & $172.78^{***}$ & $132.20^{***}$ \\
            & $(38.30)$      & $(35.90)$      \\
ID          & $89.05^{***}$  & $122.41^{***}$ \\
            & $(9.83)$       & $(9.22)$       \\
\hline
R$^2$       & $0.75$         & $0.86$         \\
Adj. R$^2$  & $0.74$         & $0.86$         \\
Num. obs.   & $30$           & $30$           \\
\hline
\multicolumn{3}{l}{\scriptsize{$^{***}p<0.001$; $^{**}p<0.01$; $^{*}p<0.05$}}
\end{tabular}
\end{center}
	Fitt's Law passt besser auf die Touchpad-Daten.

	
\end{document}