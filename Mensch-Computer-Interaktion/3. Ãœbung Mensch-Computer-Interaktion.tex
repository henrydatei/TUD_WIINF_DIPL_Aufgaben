\documentclass{article}

\usepackage{amsmath,amssymb}
\usepackage{tikz}
\usepackage{pgfplots}
\usepackage{xcolor}
\usepackage[left=2.1cm,right=3.1cm,bottom=3cm,footskip=0.75cm,headsep=0.5cm]{geometry}
\usepackage{enumerate}
\usepackage{enumitem}
\usepackage{marvosym}
\usepackage{tabularx}
\usepackage{multirow}
\usepackage[colorlinks = true, linkcolor = blue, urlcolor  = blue, citecolor = blue, anchorcolor = blue]{hyperref}
\usepackage{ulem}

\usepackage{listings}
\definecolor{lightlightgray}{rgb}{0.95,0.95,0.95}
\definecolor{lila}{rgb}{0.8,0,0.8}
\definecolor{mygray}{rgb}{0.5,0.5,0.5}
\definecolor{mygreen}{rgb}{0,0.8,0.26}
\lstdefinestyle{java} {language=java}
\lstset{language=java,
	basicstyle=\ttfamily,
	keywordstyle=\color{lila},
	commentstyle=\color{lightgray},
	stringstyle=\color{mygreen}\ttfamily,
	backgroundcolor=\color{white},
	showstringspaces=false,
	numbers=left,
	numbersep=10pt,
	numberstyle=\color{mygray}\ttfamily,
	identifierstyle=\color{blue},
	xleftmargin=.1\textwidth, 
	%xrightmargin=.1\textwidth,
	escapechar=§,
}

\usepackage[utf8]{inputenc}

\renewcommand*{\arraystretch}{1.4}

\newcolumntype{L}[1]{>{\raggedright\arraybackslash}p{#1}}
\newcolumntype{R}[1]{>{\raggedleft\arraybackslash}p{#1}}
\newcolumntype{C}[1]{>{\centering\let\newline\\\arraybackslash\hspace{0pt}}m{#1}}

\newcommand{\E}{\mathbb{E}}
\DeclareMathOperator{\rk}{rk}
\DeclareMathOperator{\Var}{Var}
\DeclareMathOperator{\Cov}{Cov}
\DeclareMathOperator{\SD}{SD}
\DeclareMathOperator{\Cor}{Cor}

\title{\textbf{Mensch-Computer-Interaktion, Übung 3}}
\author{\textsc{Henry Haustein}, \textsc{Dennis Rössel}}
\date{}

\begin{document}
	\maketitle
	
	\section*{Aufgabe 3.1: Zielgruppe definieren}
	\begin{enumerate}[label=(\alph*)]
		\item Unsere Zielgruppe ist nicht wirklich divers, sie besteht nur aus Studenten. Das Portal ist kein Studieninformationsportal, deswegen sind Schüler keine Zielgruppe.
		\item Menschen mit körperlichen Behinderungen (mit geistigen Behinderungen lässt sich nicht wirklich studieren).
		\item $\Rightarrow$ 2 Personas
	\end{enumerate}

	\section*{Aufgabe 3.2: Personas entwickeln}
	1. Persona
	 \begin{enumerate}[label=(\alph*)]
	 	\item Probleme: Steuerrückerstattung vom Finanzamt braucht immer so lange; das Lernen für Klausuren dauert zu lange; bekommt Wutausbrüche wenn seine BWL-Marie (seine Freundin) mit dem Eis auf seine Wildledersitze kleckert
	 	\item Bedürfnisse: sehr schnelles Finden von Informationen, wenn nötig per Premium-Abo
	 	\item Ziele: möglichst schnell das Studium mit minimalem Aufwand abschließen, unbegrenzte Geldmittel können in seinen Erfolg investiert werden
	 	\item demografische Angaben: männlich, 20 Jahre alt, wohnt in München in einer \textit{kleinen} 100 m$^2$-Wohnung, kennt sich nur mit Excel aus
	 	\item Kontext: Zugang vom iPhone, iPad Pro oder MacBook Pro, komplizierter als die Bedienung von Google darf es nicht sein; auf der Suche nach Zusammenfassungen stößt er auf die Webseite
	 	\item Ängste: dass der Bachelor länger als 6 Semester dauert (inklusive 3 Urlaubssemestern für den (Ski-) urlaub in Davos); ein zu langer Klickpfad wirkt abschreckend auf Justus
	 	\item Persönlichkeits-Features: Justus (20), studiert BWL und verbringt viel Zeit mit dem Mercedes seines Vaters und Golf spielen
	 \end{enumerate}
 	2. Persona
 	\begin{enumerate}[label=(\alph*)]
 		\item Probleme: keine Rücksichtnahme auf ihre Einschränkungen
 		\item Bedürfnisse: barrierefreie Dokumente/Räume/Gebäude
 		\item Ziele: so normal wie möglich leben zu können
 		\item demografische Angaben: weiblich, 20 Jahre
 		\item Kontext: Screenreader ließt die Seite vor, Brailletastatur; komplett barrierefreie Webseite; stößt auf unsere Webseite entweder durch Empfehlungen oder Google-Suche
 		\item Ängste: keine barrierefreien Dokumente/Räume/Gebäude, Angst keine Freunde zu finden
 		\item Persönlichkeits-Features: Anna (20) studiert Psychologie und ist geburtsblind.
 	\end{enumerate}
	
\end{document}