\documentclass{article}

\usepackage{amsmath,amssymb}
\usepackage{tikz}
\usepackage{xcolor}
\usepackage[left=2.1cm,right=3.1cm,bottom=3cm,footskip=0.75cm,headsep=0.5cm]{geometry}
\usepackage{enumerate}
\usepackage{enumitem}
\usepackage{marvosym}
\usepackage{tabularx}
\usepackage{pgfplots}
\pgfplotsset{compat=1.10}
\usepgfplotslibrary{fillbetween}
\usepackage[unicode]{hyperref}
\hypersetup{
	colorlinks,
	citecolor=blue,
	filecolor=blue,
	linkcolor=blue,
	urlcolor=blue
}

\usepackage[utf8]{inputenc}

\renewcommand*{\arraystretch}{1.4}

\newcolumntype{L}[1]{>{\raggedright\arraybackslash}p{#1}}
\newcolumntype{R}[1]{>{\raggedleft\arraybackslash}p{#1}}
\newcolumntype{C}[1]{>{\centering\let\newline\\\arraybackslash\hspace{0pt}}m{#1}}

\DeclareMathOperator{\tr}{tr}
\DeclareMathOperator{\Var}{Var}
\DeclareMathOperator{\Cov}{Cov}

\title{\textbf{Öffentliches Recht, Übung 3}}
\author{\textsc{Henry Haustein}}
\date{}

\begin{document}
	\maketitle
	
	\section*{Aufgabe 1: Grundlagen}
	\begin{enumerate}[label=(\alph*)]
		\item Gesetzgebung (Art. 77 I 1 GG), Budgetrecht (Art. 110 GG), Kreationsfunktion (Wahl anderer Staatsorgane) (Art. 63 I, 54 III, 94 I GG), Mitwirkungs- und Zustimmungsfunktion (Art. 23 II GG), Kontrollfunktion: Zitierrecht (Art. 43 GG), Einsetzung von Untersuchungsausschüssen (Art.44 GG), Wehrbeauftragter (Art. 45b GG), Parlamentarisches Kontrollgremium (Art. 45d GG)
		\item Als Zitierungsrecht wird das Recht eines Parlaments oder dessen Ausschüsse bezeichnet, das Erscheinen des Regierungschefs oder eines Ministers im Plenum oder vor den Ausschüssen verlangen zu können. In Deutschland regelt der Art. 43 Abs. 1 Grundgesetz, dass der Bundestag und seine Ausschüsse die Anwesenheit jedes Mitglieds der Bundesregierung verlangen können. Der Beschluss bedarf der Mehrheit des Bundestages.
		\item 4 Jahre (Art. 39 I 1 GG), Beendigung der Legislaturperiode mit Zusammentritt eines neuen Bundestages $\to$ Keine parlamentslose Zeit! (Art. 39 I 2 GG), Vorzeitige Auflösung des Bundestages: Kanzlerneuwahlen (Art. 63 IV 3 GG), Vertrauensantrag ohne absolute Mehrheit (Art. 68 I 1 GG), Kein Selbstauflösungsrecht des Bundestages
		\item Grundsätzlich mit einfacher Mehrheit (Art. 42 II GG) (es sei denn, absolute oder qualifizierte Mehrheit erforderlich)
	\end{enumerate}
	
	Quellen:
	\begin{itemize}
		\item \url{https://de.wikipedia.org/wiki/Zitierungsrecht}
	\end{itemize}

	\section*{Aufgabe 2: Geschäftsordnung}
	Die Abgeordneten unterliegen aber der Geschäftsordnung, die sich der Deutsche Bundestag laut Artikel 40 des Grundgesetzes gibt. Sie regelt unter anderem Redezeiten im Plenum und Verhaltensregeln der Abgeordneten. Die Regeln enthalten keine allgemeine Berufsethik für die Volksvertreter. Ein Katalog listet präzise die Anzeigepflichten und Verbotstatbestände auf und enthält auch eine Regelung für den Fall der Nichtbeachtung.

	Quellen:
	\begin{itemize}
		\item \url{https://www.bundestag.de/parlament/aufgaben/rechtsgrundlagen/go_btg}
	\end{itemize}

	\section*{Aufgabe 3: Wahlen}
	\textit{Die Abgeordneten des Deutschen Bundestages werden in allgemeiner, unmittelbarer, freier, gleicher und geheimer Wahl gewählt. Sie sind Vertreter des ganzen Volkes, an Aufträge und Weisungen nicht gebunden und nur ihrem Gewissen unterworfen.} (Art. 38 Abs. 1 GG). Für die Erklärung der einzelnen Grundsätze siehe Übung 1.
	
	\section*{Aufgabe 4: Bundestagspräsident}
	\begin{itemize}
		\item Der Präsident vertritt den Bundestag und regelt seine Geschäfte. Er wahrt die Würde und die Rechte des Bundestages, fördert seine Arbeiten, leitet die Verhandlungen gerecht und unparteiisch und wahrt die Ordnung im Hause. Er hat beratende Stimme in allen Ausschüssen.
		\item Dem Präsidenten steht das Hausrecht und die Polizeigewalt in allen der Verwaltung des Bundestages unterstehenden Gebäuden, Gebäudeteilen und Grundstücken zu. Der Präsident erläßt im Einvernehmen mit dem Ausschuß für Wahlprüfung, Immunität und Geschäftsordnung eine Hausordnung.
		\item Der Präsident schließt die Verträge, die für die Bundestagsverwaltung von erheblicher Bedeutung sind, im Benehmen mit seinen Stellvertretern ab. Ausgaben im Rahmen des Haushaltsplanes weist der Präsident an.
		\item Der Präsident ist die oberste Dienstbehörde der Bundestagsbeamten. Er ernennt und stellt die Bundestagsbeamten nach den gesetzlichen und allgemeinen Verwaltungsvorschriften ein und versetzt sie in den Ruhestand. Auch die nichtbeamteten Bediensteten des Bundestages werden von dem Präsidenten eingestellt und entlassen. Maßnahmen nach Satz 2 und 3 trifft der Präsident, soweit Beamte des höheren Dienstes oder entsprechend eingestufte Angestellte betroffen sind, im Benehmen mit den stellvertretenden Präsidenten, soweit leitende Beamte (A16 und höher) oder entsprechend eingestufte Angestellte eingestellt, befördert bzw. höhergestuft werden, mit Zustimmung des Präsidiums.
		\item Absatz 4 gilt auch für die dem Wehrbeauftragten beigegebenen Beschäftigten. Maßnahmen nach Absatz 4 Satz 4 erfolgen im Benehmen mit dem Wehrbeauftragten. Für die Bestellung, Ernennung, Umsetzung, Versetzung und Zurruhesetzung des Leitenden Beamten ist das Einvernehmen mit dem Wehrbeauftragten erforderlich. Der Wehrbeauftragte hat das Recht, für alle Entscheidungen nach Absatz 4 Vorschläge zu unterbreiten.
		\item Ist der Präsident verhindert, vertritt ihn einer seiner Stellvertreter aus der zweitstärksten Fraktion.
	\end{itemize}

	Quellen:
	\begin{itemize}
		\item \url{https://www.bundestag.de/parlament/aufgaben/rechtsgrundlagen/go_btg/go03-245160}
	\end{itemize}

	\section*{Aufgabe 5: Abgeordnete}
	\begin{enumerate}[label=(\alph*)]
		\item Das Grundgesetz stattet die Abgeordneten des Deutschen Bundestages mit dem sogenannten freien Mandat aus. Die Abgeordneten sind somit nicht an Aufträge und Weisungen gebunden.
		\item Das Grundgesetz legt in Artikel 38 eindeutig fest, dass die Abgeordneten an keine Weisungen gebunden "und nur ihrem Gewissen unterworfen" sind. Sie dürfen also zu keinem Abstimmungsverhalten gezwungen werden - auch nicht durch ihre Fraktion. Andererseits betont auch das Bundesverfassungsgericht die Bedeutung der Fraktionen als "maßgebliche Faktoren der politischen Willensbildung". Wenn ein einzelner Abgeordneter politischen Einfluss ausüben und gestalten wolle, brauche er die abgestimmte Unterstützung anderer Abgeordneter, wie sie Fraktionen organisieren. "Das freie Mandat und die Gleichheit der Abgeordneten werden deshalb durch die Anforderungen der in Fraktionen organisierten parlamentarischen Arbeit mit geprägt", erklärten die Richter 2004 in einem Urteil. Sie betonten zugleich, dass die Freiheit und Gleichheit der Abgeordneten auch "innerhalb der Fraktion bei Abstimmungen und bei einzelnen Abweichungen von der Fraktionsdisziplin erhalten" bleibe.
		\item In Deutschland genießen nach Art. 46 Abs. 1 Grundgesetz (GG) und § 36 StGB sowohl Bundestags- als auch Landtagsabgeordnete, sowie Mitglieder der Bundesversammlung wegen ihrer Äußerungen Indemnität. Sie dürfen also wegen einer Abstimmung oder einer Äußerung, die sie im Parlament oder dessen Ausschüssen getan haben, zu keiner Zeit – also auch nicht nach Ablauf des Mandats – gerichtlich oder dienstlich verfolgt oder sonst außerhalb des Parlaments zur Verantwortung gezogen werden. Sie gilt für jedes gerichtliche Verfahren, also einschließlich strafrechtlicher und zivilrechtlicher Klagen. Die einzigen Ausnahmen sind Verleumdungen gemäß § 187 StGB sowie Verunglimpfung des Andenkens Verstorbener gemäß § 189 StGB. Die Indemnität ist ein Strafausschließungsgrund und kann im Gegensatz zur Immunität weder vom Parlament noch von einer anderen Stelle aufgehoben werden. Die Indemnität soll sicherstellen, dass die Abgeordneten nur nach ihrem Gewissen handeln können, und die Funktionsfähigkeit des Parlaments gewährleisten. Exekutive und Judikative wird die Möglichkeit genommen, wegen angeblicher oder tatsächlicher Vergehen Einfluss auf Abstimmungsverhalten und Zusammensetzung des Parlaments zu nehmen. Insoweit dient die Indemnität auch der Gewaltenteilung. Den Ehrenschutz und die Arbeitsdisziplin stellt das Parlament selbst sicher. Nach Maßgabe der Geschäftsordnung kommen etwa Ordnungsruf, Ruf zur Sache, Wortentziehung und Saalverweis in Betracht. Ein Abgeordneter des Deutschen Bundestages oder auch ein Mitglied der Bundesversammlung hat parlamentarische Immunität, die ihn vor der Strafverfolgung, jedoch nicht vor zivilrechtlichen Ansprüchen schützt. Die Immunität (Art. 46 Abs. 2 GG) schützt aber nicht den Abgeordneten selbst vor Strafe (im Gegensatz zur Indemnität), sondern soll die Arbeitsfähigkeit des Parlaments sicherstellen. Sie kann daher auch vom jeweiligen Parlament aufgehoben werden.
	\end{enumerate}

	Quellen:
	\begin{itemize}
		\item \url{https://www.bundestag.de/parlament/aufgaben/rechtsgrundlagen/go_btg}
		\item \url{https://www.tagesschau.de/inland/fraktionsdisziplin-101.html}
		\item \url{https://de.wikipedia.org/wiki/Indemnit%C3%A4t}
		\item \url{https://de.wikipedia.org/wiki/Politische_Immunit%C3%A4t}
	\end{itemize}

	\section*{Aufgabe 6: Fraktionen und Gruppen}
	\begin{enumerate}[label=(\alph*)]
		\item Die Voraussetzngen sind:
		\begin{itemize}
			\item Die Fraktionen sind Vereinigungen von mindestens fünf vom Hundert der Mitglieder des Bundestages, die derselben Partei oder solchen Parteien angehören, die auf Grund gleichgerichteter politischer Ziele in keinem Land miteinander im Wettbewerb stehen.
			\item Die Bildung einer Fraktion, ihre Bezeichnung, die Namen der Vorsitzenden, Mitglieder und Gäste sind dem Präsidenten schriftlich mitzuteilen.
		\end{itemize}
		\item Fraktionen: 5\%, Gruppen: $< 5\%$. Für beide gilt:
		\begin{itemize}
			\item Die Bildung einer Fraktion/Gruppe, ihre Bezeichnung, die Namen der Vorsitzenden, Mitglieder und Gäste sind dem Präsidenten schriftlich mitzuteilen.
			\item Fraktionen/Gruppen können Gäste aufnehmen, die bei der Feststellung der Fraktionsstärke nicht mitzählen, jedoch bei der Bemessung der Stellenanteile (§ 12) zu berücksichtigen sind.
		\end{itemize}
	\end{enumerate}

	Quellen:
	\begin{itemize}
		\item \url{https://www.bundestag.de/parlament/aufgaben/rechtsgrundlagen/go_btg/go04-245158}
	\end{itemize}

	\section*{Aufgabe 7: Auflösung}
	\begin{enumerate}[label=(\alph*)]
		\item Das Grundgesetz sieht zwei Möglichkeiten einer Auflösung des Bundestages vor - nach Art. 63 Abs. 4 Satz 3 GG und nach Art. 68 Abs. 1 Satz 1 GG. In beiden Fällen liegt die Entscheidung, den Bundestag aufzulösen, beim Bundespräsidenten. Es gibt also weder eine automatische Auflösung noch ein Selbstauflösungsrecht des Parlaments; dieses hat es jedoch in der Hand, eine Auflösung zu verhindern, indem es mit der Mehrheit seiner Mitglieder einen neuen Kanzler wählt (vgl. Art. 68 Abs. 1 Satz 2 GG). Nach erfolgter Auflösung müssen "innerhalb von sechzig Tagen" (Art. 39 Abs. 1 Satz 4 GG) Neuwahlen stattfinden. Der "aufgelöste" Bundestag bleibt bis zum Zusammentritt des neuen Bundestages bestehen, wie sich aus Art. 39 Abs. 1 Satz 2 GG ergibt; es gibt also keine parlamentslose Zeit.
		\begin{itemize}
			\item Nach Art. 68 Abs. 1 Satz 1 GG "kann der Bundespräsident auf Vorschlag des Bundeskanzlers binnen einundzwanzig Tagen den Bundestag auflösen", wenn "ein Antrag des Bundeskanzlers, ihm das Vertrauen auszusprechen, nicht die Zustimmung der Mehrheit der Mitglieder des Bundestages" findet. Darüber hinaus enthält Art. 68 Abs. 1 Satz 1 GG nach Ansicht des Bundesverfassungsgerichts jedoch noch ein ungeschriebenes sachliches Tatbestandsmerkmal: "Die politischen Kräfteverhältnisse im Bundestag müssen seine [d.h. des Bundeskanzlers] Handlungsfähigkeit so beeinträchtigen oder lähmen, dass er eine vom stetigen Vertrauen der Mehrheit getragene Politik nicht sinnvoll zu verfolgen vermag."
			\item Art. 63 Abs. 4 Satz 3 GG betrifft die Situation, dass im Falle der Wahl eines neuen Kanzlers – der nicht nur beim Zusammentritt eines neuen Bundestages, sondern auch bei einem Rücktritt des Kanzlers eintreten kann -, die Mehrheit der Mitlglieder des Bundestages nicht dem Wahlvorschlag des Bundespräsidenten gefolgt ist, innerhalb von 14 Tagen einen anderen Bundeskanzler nicht gewählt hat und in einem daraufhin stattfindenden Wahlgang der Gewählte nicht die Stimmen der Mehrheit der Mitglieder des Bundestages auf sich vereinigt hat. Dann "hat der Bundespräsident binnen sieben Tagen entweder ihn zu ernennen oder den Bundestag aufzulösen" (Art. 63 Abs. 4 Satz 3 GG).
		\end{itemize}
		\item Die Vertrauensfrage ist in vielen parlamentarischen Demokratien ein Instrument der Regierung zur Disziplinierung des Parlaments. Sie kann von einer Regierung dem Parlament gestellt werden, um festzustellen, ob es mit ihrer Haltung grundsätzlich noch übereinstimmt, und so die Abklärung gravierender Konflikte herbeiführen. Ein negatives Ergebnis führt häufig zum Rücktritt der Regierung oder zu Neuwahlen. In Deutschland spricht man von einer Vertrauensfrage im Sinne von Art. 68 Grundgesetz (GG), wenn der Bundeskanzler oder die Bundeskanzlerin beim Bundestag den Antrag stellt, ihm das Vertrauen auszusprechen. Er kann mit der Vertrauensfrage oder schon mit ihrer bloßen Androhung die ihn tragende Parlamentsmehrheit disziplinieren. Wird sie nicht positiv beantwortet, kann er dem Bundespräsidenten vorschlagen, den Bundestag aufzulösen.
		\item Der Bundestag kann sich nicht selbst auflösen. Unter anderem mit Verweis auf die schlechte Erfahrung häufiger Parlamentsauflösungen und Regierungswechsel in der Weimarer Republik ist bei der Entstehung des Grundgesetzes ein solches Recht verworfen worden. Die Einführung eines Selbstauflösungsrechts des Bundestages durch Grundgesetzänderung wird aus verfassungspolitischer Sicht überwiegend abgelehnt, weil es dem Repräsentationsprinzip zuwiderlaufe und zu Inkonsistenzen im politischen System führe.
	\end{enumerate}

	Quellen:
	\begin{itemize}
		\item \url{https://www.bundestag.de/resource/blob/513604/7524c478f79b7163c2d4f883b3826ea4/aufloesung-des-bundestages-und-vorzeitige-wahlen-data.pdf}
		\item \url{https://de.wikipedia.org/wiki/Vertrauensfrage}
		\item \url{https://de.wikipedia.org/wiki/Deutscher_Bundestag#Repr%C3%A4sentationsprinzip_und_Selbstaufl%C3%B6sung}
	\end{itemize}

	\section*{Aufgabe 9: Ausschüsse}
	\begin{enumerate}[label=(\alph*)]
		\item Die Zusammensetzung des Ältestenrates und der Ausschüsse sowie die Regelung des Vorsitzes in den Ausschüssen ist im Verhältnis der Stärke der einzelnen Fraktionen vorzunehmen. Derselbe Grundsatz wird bei Wahlen, die der Bundestag vorzunehmen hat, angewandt.
		\item Das Grundgesetz schreibt einen Auswärtigen Ausschuss, einen EU-Ausschuss, einen Verteidigungs- und einen Petitionsausschuss vor.
		\item Der Untersuchungsausschuss im Deutschen Bundestag ist ein Bundestagsausschuss, welcher im Wesent\-lichen der parlamentarischen Kontrolle gegenüber der vollziehenden Gewalt dient. Aufgabe des Untersuchungsausschusses ist es, Sachverhalte, deren Aufklärung im öffentlichen Interesse liegt, zu untersuchen und dem Bundestag darüber Bericht zu erstatten.
	\end{enumerate}

	Quellen:
	\begin{itemize}
		\item \url{https://www.bundestag.de/parlament/aufgaben/rechtsgrundlagen/go_btg/go04-245158}
		\item \url{https://de.wikipedia.org/wiki/Bundestagsausschuss}
		\item \url{https://de.wikipedia.org/wiki/Untersuchungsausschuss}
	\end{itemize}

	\section*{Aufgabe 10: Bundesrat}
	\begin{enumerate}[label=(\alph*)]
		\item \textit{Die Länder haben das Recht der Gesetzgebung, soweit dieses Grundgesetz nicht dem Bunde Gesetzgebungsbefugnisse verleiht.} (Artikel 70 Abs. 1 GG)
		\item Der verfassungspolitische Rang und die Bedeutung des Bundesrates ergeben sich hauptsächlich aus seinen Mitentscheidungsrechten bei Zustimmungsgesetzen. Diese Gesetze können nur zustande kommen, wenn Bundesrat und Bundestag sich einig sind. Bei einem endgültigen Nein des Bundesrates sind Zustimmungsgesetze gescheitert. Welche Gesetze zustimmungsbedürftig sind, ist ausdrücklich und abschließend im Grundgesetz geregelt. Im Wesentlichen lassen sich drei Gruppen unterscheiden:
		\begin{itemize}
			\item Gesetze, die die Verfassung ändern.
			\item Gesetze, die in bestimmter Weise Auswirkungen auf die Finanzen der Länder haben.
			\item Gesetze, für deren Umsetzung in die Organisations- und Verwaltungshoheit der Länder eingegriffen wird.
		\end{itemize}
		Beschließt der Bundesrat mit der absoluten Mehrheit (Mehrheit der Mitglieder) seiner Stimmen Einspruch einzulegen, kann der Einspruch nur mit der absoluten Mehrheit im Bundestag (Mehrheit der Mitglieder = Kanzlermehrheit) überstimmt werden. Legt der Bundesrat den Einspruch mit einer Zwei-Drittel-Mehrheit ein, müssen für die Zurückweisung des Einspruchs im Bundestag zwei Drittel der abgegebenen Stimmen zusammen kommen, mindestens jedoch die Stimmen der Hälfte aller Mitglieder.
		\item \textit{Durch den Bundesrat wirken die Länder bei der Gesetzgebung und Verwaltung des Bundes und in Angelegenheiten der Europäischen Union mit.} (Art. 50 GG)
		\item Ein imperatives Mandat ist ein Mandat, bei dem ein Abgeordneter an inhaltliche Vorgaben der von ihm Vertretenen gebunden ist. Damit kann sowohl der Bindungszwang eines Delegierten an die ihn entsendenden Partei-Vereinsgliederungen als auch der eines Abgeordneten an den direkten Willen des wählenden Bürgers gemeint sein. Folgt der Mandatsträger nicht der Linie der ihn entsendenden Organisationsgliederung oder dem Wählerwillen, kann er abgesetzt werden.
		\item Der Bundesrat besteht aus Mitgliedern der Regierungen der Länder, die sie bestellen und abberufen. Sie können durch andere Mitglieder ihrer Regierungen vertreten werden (vgl. Art. 51 GG). Ein Mitglied des Bundesrates darf nicht gleichzeitig Mitglied des Deutschen Bundestages sein. Das Mitglied muss einen Sitz und eine Stimme in einer Landesregierung haben. Die Anzahl der Stimmen für jedes Land ist nach seiner Einwohnerzahl gestaffelt, ohne diese jedoch proportional abzubilden.
		\item Mehrheit der Stimmen erforderlich (Art. 52 III 1 GG), Stimmenabgabe eines Landes muss einheitlich sein (Art. 51 III 2 GG)
	\end{enumerate}

	Quellen: 
	\begin{itemize}
		\item \url{https://www.bundesrat.de/DE/aufgaben/gesetzgebung/gesetzgebung-node.html}
		\item \url{https://www.bundesrat.de/DE/aufgaben/gesetzgebung/zust-einspr/zust-einspr-node.html}
		\item \url{https://www.bpb.de/politik/grundfragen/deutsche-demokratie/39348/bundesrat}
		\item \url{https://de.wikipedia.org/wiki/Imperatives_Mandat}
		\item \url{https://de.wikipedia.org/wiki/Bundesrat_(Deutschland)#Mitglieder_und_Stimmenverteilung_auf_die_L%C3%A4nder}
	\end{itemize}
	
\end{document}