\documentclass{article}

\usepackage{amsmath,amssymb}
\usepackage{tikz}
\usepackage{xcolor}
\usepackage[left=2.1cm,right=3.1cm,bottom=3cm,footskip=0.75cm,headsep=0.5cm]{geometry}
\usepackage{enumerate}
\usepackage{enumitem}
\usepackage{marvosym}
\usepackage{tabularx}
\usepackage{pgfplots}
\pgfplotsset{compat=1.10}
\usepgfplotslibrary{fillbetween}

\usepackage[utf8]{inputenc}

\renewcommand*{\arraystretch}{1.4}

\newcolumntype{L}[1]{>{\raggedright\arraybackslash}p{#1}}
\newcolumntype{R}[1]{>{\raggedleft\arraybackslash}p{#1}}
\newcolumntype{C}[1]{>{\centering\let\newline\\\arraybackslash\hspace{0pt}}m{#1}}

\DeclareMathOperator{\tr}{tr}
\DeclareMathOperator{\Var}{Var}
\DeclareMathOperator{\Cov}{Cov}

\title{\textbf{Öffentliches Recht, Übung 1}}
\author{\textsc{Henry Haustein}}
\date{}

\begin{document}
	\maketitle
	
	\section*{Aufgabe 1: Staatsbegriff}
	Die drei Merkmale sind Staatsgebiet (natürlicher, abgegrenzter Teil der Erdoberfläche, die beherrschbar und zum dauerhaften Aufenthalt von Menschen geeignet ist), Staatsvolk (auf Dauer angelegter Zusammenschluss von Menschen auf einem Gebiet, welche mit diesem rechtlich sowie untereinander verbunden sind) und Staatsgewalt (originäre, unteilbare Herrschaftsmacht über das Gebiet und die dort befindlichen Personen).

	\section{Aufgabe 2: Demokratieprinzip}
	\begin{enumerate}[label=(\alph*)]
		\item in den Artikeln 20, 21, 28 und 38ff
		\item unmittelbar durch Wahlen, mittelbar durch Organe der Gesetzgebung, Polizei, Justiz
		\item Parlament ist vom Volk gewählt und zeitlich legitimiert und bildet das Volk ab. Mehrheitsentscheid, Gleichheit der Staatsbürger
		\item Parteien sind Vereinigungen von Bürgern, die dauernd oder für längere Zeit für den Bereich des Bundes oder eines Landes auf die politische Willensbildung Einfluß nehmen und an der Vertretung des Volkes im Deutschen Bundestag oder einem Landtag mitwirken wollen, wenn sie nach dem Gesamtbild der tatsächlichen Verhältnisse, insbesondere nach Umfang und Festigkeit ihrer Organisation, nach der Zahl ihrer Mitglieder und nach ihrem Hervortreten in der Öffentlichkeit eine ausreichende Gewähr für die Ernsthaftigkeit dieser Zielsetzung bieten. Mitglieder einer Partei können nur natürliche Personen sein.
		\item Das Parteienprivileg des Art. 21 GG stattet die politischen Parteien in Deutschland wegen ihrer besonderen Bedeutung für die parlamentarische Demokratie mit einer erhöhten Schutz- und Bestandsgarantie aus. Insbesondere legt Art. 21 Abs. 4 GG die Entscheidung über die Verfassungswidrigkeit einer politischen Partei ausschließlich in die Hand des Bundesverfassungsgerichts. Bis zur Entscheidung des Bundesverfassungsgerichts ist von der Verfassungsmäßigkeit der Partei auszugehen. Insofern kommt dieser Entscheidung konstitutive Bedeutung zu.
		\item Allgemeinheit (Aktives und passives Wahlrecht für alle deutschen Staatsbürger), Unmittelbarkeit (Keine Zwischeninstanz, wie Wahlmänner), Freiheit (Wahl ohne Druck und Zwang), Gleichheit (Zähl- und Erfolgswertgleichheit), Geheimheit (Stimmabgabe geheim, keine Offenbarungspflicht des Wählers)
	\end{enumerate}

	\section*{Aufgabe 3}
	\begin{enumerate}[label=(\alph*)]
		\item Artikel 20 III GG
		\item Der Rechtsstaat des Grundgesetzes ist nicht nur formeller, sondern vor allem auch materieller Rechts\-staat (Grundrechte!). Formelle Rechtsstaatlichkeit meint die Bindung der staatlichen Machtausübung an Recht und Gesetz (vor allem Zuständigkeiten und Verfahren betreffend), die materielle Rechtsstaatl\-ichkeit hingegen beschreibt bestimmte materielle Ausformungen und Anforderungen an die Gestaltung des Staates und der Ausübung staatlicher Macht wie etwa die Ausübung staatlicher Gewalt unter Beachtung der grundrechtlich gewährleisteten Garantien und unter Berücksichtigung des Übermaßver\-botes. 
		\item ??
		\item Ein formelles Gesetz (auch: Gesetz im formellen Sinn) ist jede Regelung, die im Rahmen eines förmlichen Gesetzgebungsverfahrens zustande gekommen ist. Rechtsverordnungen dienen der Entlastung des Gesetzgebers. Sie erscheinen historisch erstmals durch die Bestätigung der Gewaltenteilung im modernen Staat. Die Exekutive wird gesetzlich dazu ermächtigt, technische Fragen, Einzelheiten zu regeln, die den Parlamentsalltag unnötig behindern würden. Körperschaften, Anstalten und Stiftungen des öffentlichen Rechts sind durch Gesetz ermächtigt, Satzungen zur Regelung ihrer eigenen Angelegenheiten zu erlassen. Ungeschriebenes Recht ist nirgendwo aufgeschrieben, aber man hält sich daran (z.B. Kultur, ...).
		\item Bundesrecht hat einen Geltungsvorrang vor Landesrecht (Art. 31 GG)
		\item Verhältnismäßigkeit = Übermaßverbot; zentrale Rolle beim Grundrechtsschutz
		\item Legitimer Zweck (die staatliche Maßnahme verfolgt einen legitimen Zweck), Geeignetheit (die staatliche Maßnahme erreicht bzw. fördert den Zweck), Erforderlichkeit (es gibt kein milderes Mittel, das den Zweck gleich wirksam erreicht wie die Maßnahme), Angemessenheit (die Wertigkeit des Zwecks steht in einem angemessenen Verhältnis zur Schwere des Eingriffs)
		\item Die Ausübung der Staatsgewalt ist in Deutschland nicht nur horizontal zwischen Legislative, Exekutive und Judikative, sondern auch vertikal zwischen dem Bund und den Ländern geteilt
		\item Verwaltung darf nicht gegen bestehende Gesetze handeln, kein Handeln gegen Gesetz
		\item Verwaltung darf nur handeln, wenn ein entsprechendes Gesetz dies gestattet, kein Handeln ohne Gesetz
		\item Klarheit und Beständigkeit staatlicher Entscheidungen, Norm bzw. Rechtsakt muss so klar sein, dass für den Betroffenen die Rechtslage erkennbar ist und er sein Verhalten daran ausrichten kann
		\item Natürlich, sonst ist Willkür Tür und Tor geöffnet
		\item Art. 103 II GG
		\item Echte Rückwirkung (Rückbewirkung von Rechtsfolgen), Unechte Rückwirkung (Tatbestandliche Rück\-anknüpfung)
		\item Art. 19 IV GG: Justizgewährungsanspruch, Flankierung durch Art. 97 GG (sachliche und persönliche Unabhängigkeit des Richters)
	\end{enumerate}
	
\end{document}