\documentclass{article}

\usepackage{amsmath,amssymb}
\usepackage{tikz}
\usepackage{pgfplots}
\usepackage{xcolor}
\usepackage[left=2.1cm,right=3.1cm,bottom=3cm,footskip=0.75cm,headsep=0.5cm]{geometry}
\usepackage{enumerate}
\usepackage{enumitem}
\usepackage{marvosym}
\usepackage{tabularx}
\usepackage{multirow}
\usepackage[colorlinks = true, linkcolor = blue, urlcolor  = blue, citecolor = blue, anchorcolor = blue]{hyperref}

\usepackage{listings}
\definecolor{lightlightgray}{rgb}{0.95,0.95,0.95}
\definecolor{lila}{rgb}{0.8,0,0.8}
\definecolor{mygray}{rgb}{0.5,0.5,0.5}
\definecolor{mygreen}{rgb}{0,0.8,0.26}
\lstdefinestyle{java} {language=java}
\lstset{language=java,
	basicstyle=\ttfamily,
	keywordstyle=\color{lila},
	commentstyle=\color{lightgray},
	stringstyle=\color{mygreen}\ttfamily,
	backgroundcolor=\color{white},
	showstringspaces=false,
	numbers=left,
	numbersep=10pt,
	numberstyle=\color{mygray}\ttfamily,
	identifierstyle=\color{blue},
	xleftmargin=.1\textwidth, 
	%xrightmargin=.1\textwidth,
	escapechar=§,
}

\usepackage[utf8]{inputenc}

\renewcommand*{\arraystretch}{1.4}

\newcolumntype{L}[1]{>{\raggedright\arraybackslash}p{#1}}
\newcolumntype{R}[1]{>{\raggedleft\arraybackslash}p{#1}}
\newcolumntype{C}[1]{>{\centering\let\newline\\\arraybackslash\hspace{0pt}}m{#1}}

\newcommand{\E}{\mathbb{E}}
\DeclareMathOperator{\rk}{rk}
\DeclareMathOperator{\Var}{Var}
\DeclareMathOperator{\Cov}{Cov}
\DeclareMathOperator{\SD}{SD}
\DeclareMathOperator{\Cor}{Cor}

\title{\textbf{Grundlagen des Finanzmanagements, Übung 5}}
\author{\textsc{Henry Haustein}}
\date{}

\begin{document}
	\maketitle
	
	\section*{Aufgabe 14.5: MM II: Verschuldung, Arbitrage \& Unternehmenswert}
	\begin{enumerate}[label=(\alph*)]
		\item Da beide Unternehmen (nahezu) identisch sind, müssen auch die Unternehmenswerte identisch sein. Der Unternehmenswert für Alpha Industries ist 10 Millionen Aktien $\cdot$ 22 \EUR/Aktie = 220 Mio. \EUR. Damit ist dann der Preis pro Aktie für Omega Industries:
		\begin{align}
			P = \frac{220\text{ Mio. \EUR} - 60 \text{ Mio. \EUR}}{20\text{ Mio. Aktien}} = 8\text{ \EUR} \notag
		\end{align}
		\item Omega Industries ist also überbewertet. Man sollte also Omega Industries verkaufen und auf Kredit Alpha Industries kaufen. Das klappt natürlich nur bei einem vollkommenen Kapitalmarkt.
	\end{enumerate}
	
	\section*{Aufgabe 2K213: Modigliani/Miller}
	\begin{enumerate}[label=(\alph*)]
		\item Der Eigenkapitalkostensatz berechnet sich durch
		\begin{align}
			r_{E}^{Villabajo} = \frac{\text{Nettogewinn}}{E^{Villabajo}} = \frac{1\text{ Mio. \EUR}}{0.1\text{ Mio. Aktien}\cdot 58.82\text{ \EUR/Aktie}} = 0.17 \notag
		\end{align}
		Der Marktwert des Eigenkapitals von Villariba ist
		\begin{align}
			E^{Villariba} = \frac{\text{Nettogewinn}}{r_{E}^{Villariba}} = \frac{0.5\text{ Mio. \EUR} - 0.1\cdot 0.379\text{ Mio. \EUR}}{0.17} = \text{2 718 235.29 \EUR} \notag
		\end{align}
		Die durchschnittlichen Kapitalkosten sind
		\begin{align}
			r_{WACC} &= \frac{E}{E+D}\cdot r_E + \frac{D}{E+D}\cdot r_D \notag \\
			r_{WACC}^{Villabajo} &= \frac{\text{5 882 000 \EUR}}{\text{5 882 000 \EUR} + 0 \text{ \EUR}} \cdot 0.17 + \frac{0\text{ \EUR}}{\text{5 882 000 \EUR} + 0 \text{ \EUR}} \cdot 0.1 = 0.17 \notag \\
			r_{WACC}^{Villariba} &= \frac{\text{2 718 235.29 \EUR}}{\text{2 718 235.29 \EUR} + \text{379 000 \EUR}} \cdot 0.17 + \frac{\text{379 000 \EUR}}{\text{2 718 235.29 \EUR} + \text{379 000 \EUR}} \cdot 0.1 = 0.1614 \notag
		\end{align}
		\item Eine Arbitrage ist möglich weil das WACC von Villabajo größer als das WACC von Villariba ist. Also werden alle Anteile von Villariba verkauft, was einen Erlös von
		\begin{align}
			C = w\cdot E_{Villariba} = 0.1\cdot \text{2 718 235.29 \EUR} = \text{271 823.53 \EUR} \notag
		\end{align}
		ergibt. Zusätzlich wird noch ein Kredit in Höhe von
		\begin{align}
			NOM = C\cdot \frac{L_{alt} - L_{neu}}{L_{neu} + 1} = \text{271 823.53 \EUR}\cdot\frac{0.1394-0}{0+1} = \text{37 892.20 \EUR} \notag
		\end{align}
		aufgenommen ($L_{alt}=\frac{D_{Villariba}}{E_{Villariba}}=\frac{\text{379 000 \EUR}}{\text{2 718 235.29 \EUR}}=0.1394$ und $L_{neu}=\frac{D_{Villabajo}}{E_{Villabajo}}=\frac{0\text{ \EUR}}{\text{5 882 000 \EUR}}=0$). Mit diesem Geld wird jetzt Villabajo gekauft und es ergibt sich ein Arbitragegewinn von
		\begin{align}
			\text{Arbitragegewinn} &= \frac{C+NOM}{E_{Villabajo}}\cdot\text{Nettogewinn}_{Villabajo} - NOM\cdot r_f - w\cdot \text{Nettogewinn}_{Villariba} \notag \\
			&= \frac{\text{271 823.53 \EUR} +\text{37 892.20 \EUR} }{\text{5 882 000 \EUR}}\cdot \text{1 000 000 \EUR} - \text{37 892.20 \EUR}\cdot 0.1 - 0.1\cdot \text{462 100 \EUR} \notag \\
			&= \text{2 655.61 \EUR} \notag
		\end{align}
		Individueller Zinssatz:
		\begin{align}
			r_i &= \frac{\frac{C+NOM}{E_{Villabajo}}\cdot\text{Nettogewinn}_{Villabajo} - NOM\cdot r_f }{C} \notag \\
			&= \frac{\frac{\text{271 823.53 \EUR} +\text{37 892.20 \EUR} }{\text{5 882 000 \EUR}}\cdot \text{1 000 000 \EUR} - \text{37 892.20 \EUR}\cdot 0.1}{\text{271 823.53 \EUR}} \notag \\
			&= 0.1798 \notag
		\end{align}
		Damit ergibt sich ein Barwert von $\frac{\text{Arbitragegewinn}}{r_i} = \frac{\text{2 655.61 \EUR}}{0.1798}=\text{14 769.80 \EUR}$.
		\item Die Aufgabe unterscheidet sich nicht meiner Meinung nach nicht von (a).
		\item Der Wert des Unternehmens ist
		\begin{align}
			V_L = \text{5 882 000 \EUR} + \text{100 000 \EUR}\cdot 0.35 = \text{5 917 000 \EUR} \notag
		\end{align}
		Das Fremdkapital hat einen Wert von 100 000 \EUR, damit hat das Eigenkapital einen Wert von 5 817 000 \EUR.
		\item Die traditionelle These der optimalen Kapitalstruktur geht davon aus, dass es für jedes Unternehmen einen optimalen Verschuldungsgrad (und somit eine optimale Kapitalstruktur) gibt. Dieser Verschuldungsgrad beschreibt eine Finanzierungssituation mit minimalen Finanzierungskosten. Dabei werden bestimmte, in der Realität zu beobachtende Verhaltensweisen der Eigen- und Fremdkapitalgeber angenommen. Bei gegebenem Gesamtkapital ist ein Unternehmen unter Berücksichtigung der Sensibilität der Kapitalgeber für das Verschuldungsrisiko (= Kapitalstrukturrisiko) und niedriger Ausgangsverschuldung in der Lage, durch Substitution von "teurem" Eigenkapital durch "billiges" Fremdkapital die durchschnittlichen Gesamtkapitalkosten zu minimieren und damit den Marktwert des gesamten Unternehmens zu maximieren (Ertragswertkonzept). Solange weder Eigenkapital- noch Fremdkapitalgeber einen Anlass sehen, ihre Rendite- bzw. Zinsforderungen zu verändern, lassen sich c.p. die durchschnittlichen Gesamtkapitalkosten des Unternehmens durch eine zunehmende Verschuldung senken (Bereich I). Dies gilt auch dann noch, wenn sich die nicht substituierten Eigenkapitalanteile verteuern, weil die Eigenkapitalgeber eine Prämie für das wachsende Verschuldungsrisiko fordern (Bereich II). Die Vorteilhaftigkeit des fortgesetzten Austausches von Eigenkapital durch Fremdkapital nimmt mit zunehmendem Verschuldungsgrad bei konstanten Zinsforderungen der Fremdkapitalgeber allerdings ab, weil die Eigenkapitalgeber das erhöhte Kapitalstrukturrisiko erkennen und für die weitere Kapitalüberlassung höhere Risikoprämien verlangen. Ab einem bestimmten Verschuldungsgrad erkennen selbst die Fremdkapitalgeber das erhöhte Verschuldungsrisiko und verlangen für die weitere Kapitalüberlassung höhere Risikoprämien bzw. eine höhere Fremdkapitalverzinsung (Bereich III). Durch eine weitere Kapitalumschichtung können die durchschnittlichen Gesamtkapitalkosten nicht mehr gesenkt werden. Aufgrund der beschriebenen Verhaltensmuster von Eigen- und Fremdkapitalgebern nehmen die durchschnittlichen Gesamtkapitalkosten im Bereich II ein - wenn auch nicht eindeutig bestimmbares - Minimum an. Der Bereich II wird auch als Bereich optimaler Kapitalstrukturen bezeichnet. (Quelle: \url{http://www.wirtschaftslexikon24.com/d/traditionelle-these/traditionelle-these.htm})
	\end{enumerate}
	
	\section*{Aufgabe 3K250: Kapitalstruktur}
	\begin{enumerate}[label=(\alph*)]
		\item Wenn wir in der Modigliani/Miller-Welt sind, dann müssten die Eigenkapitalkosten linear mit dem Verschuldungsgrad steigen:
		\begin{align}
			r_E = r_U + \underline{\frac{D}{E}}_{\text{Verschuldungsgrad}}\left(r_U-r_D\right) \notag
		\end{align}
		Es müssen also die folgenden Gleichungen erfüllt sein:
		\begin{align}
			\label{eq1} 0.29 &= r_U + 1\left(r_U-r_D\right) \\
			\label{eq2} 0.38 &= r_U + 2\left(r_U-r_D\right) \\
			\label{eq3} 0.47 &= r_U + 3\left(r_U-r_D\right)
		\end{align}
		Aus \eqref{eq1} und \eqref{eq2} kann man $r_U=0.2$ und $r_D=0.11$ ermitteln und diese Lösung erfüllt auch \eqref{eq3}. Damit gilt dann
		\begin{itemize}
			\item Verschuldungsgrad 0: $r_E=0.2 + 0(0.2-0.11)=0.2$
			\item Verschuldungsgrad40: $r_E=0.2 + 4(0.2-0.11)=0.56$
		\end{itemize}
		\item Die WACC's der Unternehmen sind
		\begin{align}
			r_{WACC}^{Mars} &= \frac{4\text{ Mio. \EUR}}{20\text{ Mio. \EUR}} = 0.2 \notag \\
			r_{WACC}^{Venus} &= \frac{4\text{ Mio. \EUR}}{18\text{ Mio. \EUR}} = 0.22 \notag
		\end{align}
		Durch den Verkauf von Mars-Aktien und den Kauf von Venus-Aktien liegt eine Arbitrage-Möglichkeit vor. Verkauf von Mars-Aktien liefert
		\begin{align}
			C = 0.2\cdot 10\text{ Mio. \EUR} = 2 \text{ Mio. \EUR} \notag
		\end{align}
		Mittel frei, diese werden durch einen Kredit weiter aufgestockt:
		\begin{align}
			NOM = C\cdot\frac{L_{alt}-L_{neu}}{L_{neu} + 1} = 2\text{ Mio. \EUR}\cdot\frac{\frac{10 \text{ Mio. \EUR}}{10\text{ Mio. \EUR}}-\frac{3\text{ Mio. \EUR}}{15\text{ Mio. \EUR}}}{\frac{3}{15}+1} = \text{1 333 333.33 \EUR} \notag
		\end{align}
		Der Arbitragegewinn ist dann
		\begin{align}
			\text{Arbitragegewinn} &= \frac{C+NOM}{E_{Venus}}\cdot\text{Nettogewinn}_{Venus} - NOM\cdot r_f - w\cdot \text{Nettogewinn}_{Mars} \notag \\
			&= \frac{2\text{ Mio. \EUR} + \text{1 333 333.33 \EUR}}{15\text{ Mio. \EUR}}\cdot (4\text{ Mio.} - 0.1\cdot 3\text{ Mio. \EUR}) \notag \\
			& - \text{1 333 333.33 \EUR} \cdot 0.1 - 0.2\cdot (4\text{ Mio. \EUR} - 0.1\cdot 10\text{ Mio. \EUR}) \notag \\
			&= \text{88 888.89 \EUR} \notag
		\end{align}
		Individueller Zinssatz:
		\begin{align}
			r_i &= \frac{\frac{C+NOM}{E_{Venus}}\cdot\text{Nettogewinn}_{Venus} - NOM\cdot r_f }{C} \notag \\
			&= \frac{\frac{2\text{ Mio. \EUR} + \text{1 333 333.33 \EUR}}{15\text{ Mio. \EUR}}\cdot (4\text{ Mio.} - 0.1\cdot 3\text{ Mio. \EUR}) - \text{1 333 333.33 \EUR} \cdot 0.1}{2 \text{ Mio. \EUR}} \notag \\
			&= 0.3444 \notag
		\end{align}
		Damit ergibt sich ein Barwert von $\frac{\text{Arbitragegewinn}}{r_i} = \frac{\text{88 888.89 \EUR}}{0.3444}=\text{258 097.82 \EUR}$.
		\item Minimierung des WACC:
		\begin{itemize}
			\item Verschuldungsgrad 0: $r_{WACC}=0.2$
			\item Verschuldungsgrad 0.5: $r_{WACC}=\frac{2}{3}\cdot 0.25 + \frac{1}{3}\cdot 0.07 = 0.19$
			\item Verschuldungsgrad 1: $r_{WACC}=\frac{1}{2}\cdot 0.28 + \frac{1}{2}\cdot 0.08 = 0.18$
			\item Verschuldungsgrad 2: $r_{WACC}=\frac{1}{3}\cdot 0.33 + \frac{2}{3}\cdot 0.09 = 0.17$
			\item Verschuldungsgrad 3: $r_{WACC}=\frac{1}{4}\cdot 0.39 + \frac{3}{4}\cdot 0.11 = 0.18$
			\item Verschuldungsgrad 4: $r_{WACC}=\frac{1}{5}\cdot 0.48 + \frac{4}{5}\cdot 0.13 = 0.20$
		\end{itemize}
		Optimal ist also Verschuldungsgrad 2. Das Eigenkapital hat einen Wert von 800 000 Aktien $\cdot$ 40\EUR/Aktie = 32 Mio. \EUR, was $\frac{1}{3}$ des Gesamtkapitals entsprechen soll, damit muss das Fremdkapital einen Wert von 64 Mio. \EUR\, haben, dazu müssen noch 48 Mio. \EUR\, aufgenommen werden.
		\item Der erwartete Erlös ist
		\begin{align}
			\E(\text{Erlös}) = 0.2\cdot 60 \text{ Mio. \EUR} + 0.5 \cdot 85\text{ Mio. \EUR} + 0.3\cdot 105\text{ Mio. \EUR} = 86 \text{ Mio. \EUR} \notag
		\end{align}
		Auf den Kredit müssen Zinsen gezahlt werden, sodass gilt
		\begin{align}
			\E(r) = 0.28 &= \frac{\E(\text{Erlös}) - 50\text{ Mio. \EUR} - \text{Zinsen}}{50\text{ Mio. \EUR}} \notag \\
			\text{Zinsen} &= 22\text{ Mio. \EUR} \notag
		\end{align}
		Das entspricht einer Renditeforderung von $\frac{22\text{ Mio. \EUR}}{50 \text{ Mio. \EUR}} = 0.44$.
	\end{enumerate}
	
\end{document}