\documentclass{article}

\usepackage{amsmath,amssymb}
\usepackage{tikz}
\usepackage{pgfplots}
\usepackage{xcolor}
\usepackage[left=2.1cm,right=3.1cm,bottom=3cm,footskip=0.75cm,headsep=0.5cm]{geometry}
\usepackage{enumerate}
\usepackage{enumitem}
\usepackage{marvosym}
\usepackage{tabularx}
\usepackage{parskip}

\usepackage{listings}
\definecolor{lightlightgray}{rgb}{0.95,0.95,0.95}
\definecolor{lila}{rgb}{0.8,0,0.8}
\definecolor{mygray}{rgb}{0.5,0.5,0.5}
\definecolor{mygreen}{rgb}{0,0.8,0.26}
%\lstdefinestyle{java} {language=java}
\lstset{language=R,
	basicstyle=\ttfamily,
	keywordstyle=\color{lila},
	commentstyle=\color{lightgray},
	stringstyle=\color{mygreen}\ttfamily,
	backgroundcolor=\color{white},
	showstringspaces=false,
	numbers=left,
	numbersep=10pt,
	numberstyle=\color{mygray}\ttfamily,
	identifierstyle=\color{blue},
	xleftmargin=.1\textwidth, 
	%xrightmargin=.1\textwidth,
	escapechar=§,
	%literate={\t}{{\ }}1
	breaklines=true,
	postbreak=\mbox{\space}
}

\usepackage[colorlinks = true, linkcolor = blue, urlcolor  = blue, citecolor = blue, anchorcolor = blue]{hyperref}
\usepackage[utf8]{inputenc}

\renewcommand*{\arraystretch}{1.4}

\newcolumntype{L}[1]{>{\raggedright\arraybackslash}p{#1}}
\newcolumntype{R}[1]{>{\raggedleft\arraybackslash}p{#1}}
\newcolumntype{C}[1]{>{\centering\let\newline\\\arraybackslash\hspace{0pt}}m{#1}}

\newcommand{\E}{\mathbb{E}}
\DeclareMathOperator{\rk}{rk}
\DeclareMathOperator{\Var}{Var}
\DeclareMathOperator{\Cov}{Cov}

\title{\textbf{Lösungsvorschlag Klausur \textit{Mathematik 3 für Wirtschaftsingenieure} WS 2021/22}}
\author{\textsc{Henry Haustein}}
\date{}

\begin{document}
	\maketitle
	
	\textit{Ein Hinweis vorweg: Ich habe das Modul nie besucht, bekomme aber immer wieder Fragen dazu. Ich habe mich deswegen dazu entschieden mal die aktuellste Altklausur durchzuarbeiten und meinen Lösungsvorschlag zu veröffentlichen. Dieser sollte definitiv mit Vorsicht genossen werden, ich habe mir das Wissen an einem Vormittag angeeignet/zusammengegoogelt und konnte natürlich auch nicht alle Aufgaben mit WolframAlpha gegenchecken. Falls jemand einen Fehler entdeckt, schreibt gerne einen Kommentar dazu.}
	
	\section*{Aufgabe 1}
	\begin{enumerate}[label=(\alph*)]
		\item Ein Polynom $p$ in diesem Vektorraum sieht wie folgt aus: $p(x) = ax^5 + bx^4 + cx^3 + dx^2 + ex + f$. Wenn wir die beiden Eigenschaften mal einsetzen erhalten wir die folgenden Gleichungen:
		\begin{align}
			p(0) = f &= 0 \notag \\
			p(1) = a + b + c + d + e + f &= 0 \notag
		\end{align}
		Wir haben also 2 Gleichungen mit 6 Unbekannten, die Dimension des Vektorraums ist also 4. Wir haben 4 freie Variablen, nennen wir sie $\alpha$, $\beta$, $\gamma$, $\delta$ und es ergibt sich:
		\begin{align}
			a &= \alpha \notag \\
			b &= \beta \notag \\
			c &= \gamma \notag \\
			d &= \delta \notag \\
			e &= -\alpha - \beta - \gamma - \delta \notag
		\end{align}
		Setzen wir das ein in $p(x)$ und sortieren um:
		\begin{align}
			p(x) &= \alpha x^5 + \beta x^4 + \gamma x^3 + \delta x^2 + (-\alpha - \beta - \gamma - \delta)x \notag \\
			p(x) &= \alpha(x^5 - x) + \beta (x^4 - x) + \gamma (x^3 - x) + \delta (x^2 - x) \notag
		\end{align}
		Und daraus können wir unsere Basis ablesen: $B = \left\lbrace \left(x^5-x\right), \left(x^4-x\right), \left(x^3-x\right), \left(x^2-x\right) \right\rbrace$.
		\item Eine Norm, die auf jedem Vektorraum funktioniert ist die 1-Norm $\Vert p\Vert_1 = \vert a\vert + \vert b\vert + \vert c\vert + \vert d\vert + \vert e\vert + \vert f\vert$. Die sollte auch auf diesem speziellen Vektorraum gültig sein.
	\end{enumerate}

	\section*{Aufgabe 2}
	\begin{enumerate}[label=(\alph*)]
		\item Wir suchen eine Funktionenreihe, die in der 2-Norm gegen $f$ konvergiert, da schauen wir uns zuerst an, was die Norm von $f$ ist:
		\begin{align}
			\Vert f\Vert_2 &= \sqrt{\int_0^1 \vert f(x)\vert^2 \,dx} \notag \\
			&= \sqrt{\int_0^1 x^2\,dx} \notag \\
			&= \sqrt{F(1) - F(0)} \quad\text{ mit } F(x) = \frac{1}{3}x^3 \notag \\
			&= \sqrt{\frac{1}{3}} \notag
		\end{align}
		\textit{(Im nachfolgenden werde ich jetzt einige Entscheidungen fällen, die man sicherlich auch anders fällen kann und dann auf eine andere Lösung kommt.)} Um jetzt $f_n$ zu finden, müssen wir diese Schritte rückwärts gehen. Das $\frac{1}{3}$ unter der Wurzel kommt durch $F_n(1) - F_n(0)$. Ich finde es immer bequem wenn $F_n(0)=0$ ist, das heißt wir müssen eine Funktion finden, deren Stammfunktion an der Stelle 1 den Wert $\frac{1}{3}$ annimmt und irgendwie muss in diese Funktion auch noch ein $n$ rein. Das ganze soll auch noch konvergieren, also $F_n(1) \xrightarrow{n\to\infty} \frac{1}{3}$ und $F_n(0) \xrightarrow{n\to\infty} 0$.
		
		Um es mir leichter zu machen, würde ich gerne $F_n(1)$ und $F_n(0)$ unabhängig von $n$ machen, dann haben wir eine konstante Folge von Stammfunktionen vorliegen. Wie werden wir das $n$ los? Im wesentlichen haben wir zwei Möglichkeiten:
		\begin{itemize}
			\item Wir machen es uns einfach und setzen z.B. $f_n(x) = \frac{n}{n}\cdot x$. Das $n$ kürzt sich weg und nach dem Integrieren kommt das $n$ auch nicht mehr wieder, die Stammfunktionen $F_n$ sind unabhängig von $n$.
			\item Wir basteln uns unsere $f_n$ so, dass erst nach dem Integrieren das $n$ verschwindet, z.B. in einer solchen Funktion hier: $f_n = (n+1)\cdot x^n$. Gut, nach dem Integrieren ist das $n$ immer noch enthalten ($F_n(x) = x^{n+1}$), aber wir interessieren uns ja hauptsächlich für $F_n(1)$ und da ist das $n$ im Exponenten von $x$ völlig egal.
		\end{itemize}
		Ich entscheide mich für die zweite Möglichkeit. Da wir eigentlich $[f_n(x)]^2$ integrieren, versuchen wir erstmal diese Funktion zu designen. Ein guter Kandidat ist zum Beispiel:
		\begin{align}
			[f_n(x)]^2 &= \frac{1}{3}\cdot (n+1)\cdot x^n \notag
		\end{align}
		Die Stammfunktion ist $F_n(x) = \frac{1}{3} x^{n+1}$ und $F_n(0)=0$ und $F_n(1) = \frac{1}{3}$. Das sind genau die Eigenschaften, die wir haben wollen.
		
		Von diesem Punkt aus müssen wir nur noch $f_n(x)$ bestimmen:
		\begin{align}
			f_n(x) = \sqrt{\frac{1}{3}} \cdot \sqrt{n+1} \cdot x^{\frac{n}{2}} \notag
		\end{align}
		Damit ergibt sich dann:
		\begin{align}
			\Vert f\Vert_2 &= \sqrt{\int_0^1 \vert f_n(x)\vert^2 \,dx} \notag \\
			&= \sqrt{\int_0^1 \left[\sqrt{\frac{1}{3}} \cdot \sqrt{n+1} \cdot x^{\frac{n}{2}}\right]^2\,dx} \notag \\
			&= \sqrt{\int_0^1 \frac{1}{3} \cdot (n+1) \cdot x^n \,dx} \notag \\
			&= \sqrt{F(1) - F(0)} \quad\text{ mit } F(x) = \frac{1}{3}x^{n+1} \notag \\
			&= \sqrt{\frac{1}{3}} \notag
		\end{align}
		\item Da uns bei der Maximumsnorm nur das Maximum interessiert, bin ich immer geneigt, Funktionen zu suchen, bei denen ich genau weiß, wo und wie groß das Maximum ist. Besonders schöne Funktionen sind hier konstante Funktionen. Wenn wir $f_n = n$ setzen, ergibt sich:
		\begin{align}
			\Vert f_n\Vert_\infty = \Vert n\Vert_\infty = n \xrightarrow{n \to\infty} \infty \notag
		\end{align}
	\end{enumerate}

	\section*{Aufgabe 3}
	Wir lösen diese Aufgabe mit dem Eulerschen Ansatz mit Störfunktion.
	\begin{itemize}
		\item Lösung des homogenen Problems: Ansatz $y = e^{\lambda t}$. Einsetzen ergibt:
		\begin{align}
			\frac{d^2}{dt^2} \left(e^{\lambda t}\right) + \frac{d}{dt} \left(e^{\lambda t}\right) - 2e^{\lambda t} &= 0 \notag \\
			\lambda^2 e^{\lambda t} + \lambda e^{\lambda t} - 2e^{\lambda t} &= 0 \notag \\
			(\lambda^2 + \lambda - 2)\underbrace{e^{\lambda t}}_{\neq 0} &= 0 \notag \\
			\Rightarrow \lambda_1 &= 1 \notag \\
			\Rightarrow \lambda_2 &= -2 \notag
		\end{align}
		Daraus ergibt sich die homogene Lösung
		\begin{align}
			y_h = c_1e^t + c_2e^{-2t} \notag
		\end{align}
		\item Jetzt kümmern wir uns um die Partikulärlösung. Die Störfunktion ist $b(t) = 3te^t$, damit ist der Ansatz $y_p = Ae^t + Bte^t$. Wir haben eine Resonanz durch $Be^t$ und $c_1e^t$, damit ändert sich der Ansatz zu $y_p = Ate^t + Bt^2e^t$.
		
		Um den Ansatz gleich in die Differentialgleichung einzusetzen, brauchen wir zuerst die Ableitungen des Ansatzes:
		\begin{align}
			y_p' &= Ae^t + Ate^t + Bt^2e^t + 2Bte^t \notag \\
			y_p'' &= A(2e^t + te^t) + B(2e^t + t^2e^t + 4te^t) \notag
		\end{align}
		Einsetzen und zusammenfassen ergibt:
		\begin{align}
			A(2e^t + te^t) + B(2e^t + t^2e^t + 4te^t) + Ae^t + Ate^t + Bt^2e^t + 2Bte^t - 2(Ate^t + Bt^2e^t) &= 3te^t \notag \\
			(3A + 2B)e^t + 6Bte^t &= 3te^t \notag
		\end{align}
		Koeffizientenvergleich:
		\begin{align}
			3A + 2B &= 0 \notag \\
			6B &= 3 \notag
		\end{align}
		$\Rightarrow A = -\frac{1}{3}$ und $B=\frac{1}{2}$. Damit ist die Partikulärlösung
		\begin{align}
			y_p = \frac{1}{2}t^2e^t - \frac{1}{3}te^t \notag
		\end{align}
		Und zusammengesetzt mit der homogenen Lösung ergibt sich die Lösung der Differentialgleichung:
		\begin{align}
			y = c_1e^t + c_2e^{-2t} + \frac{1}{2}t^2e^t - \frac{1}{3}te^t \notag
		\end{align}
		\item Lösung des Anfangswertproblems: Dazu brauchen wir auch erstmal wieder die Ableitung von $y$:
		\begin{align}
			y' = \frac{1}{6}e^t\left(6c_1 + 3t^2 + 4t - 2\right) - 2c_2e^{-2t} \notag
		\end{align}
		Setzen wir die Anfangswerte ein, so ergibt sich:
		\begin{align}
			y(0) &= c_1 + c_2 \overset{!}{=} 0 \notag \\
			y'(0) &= \frac{1}{6}(6c_1 - 2) - 2c_2 \overset{!}{=} 0 \notag
		\end{align}
		Daraus ergibt sich $c_1 = \frac{1}{9}$ und $c_2=-\frac{1}{9}$. Und damit lautet die Lösung des Anfangswertproblems
		\begin{align}
			y = \frac{1}{9}e^t - \frac{1}{9}e^{-2t} + \frac{1}{2}t^2e^t - \frac{1}{3}te^t \notag
		\end{align}
	\end{itemize}

	\section*{Aufgabe 4}
	Formen wir das erstmal ein bisschen um und nutzen $1 = \frac{d}{dx} x = x'$:
	\begin{align}
		x^2y' + xy - 1 &= 0 \notag \\
		xy' + y &= -\frac{1}{x} \notag \\
		xy' + x'y &= -\frac{1}{x} \notag
	\end{align}
	Links sieht wie das Ergebnis der Produktregel aus:
	\begin{align}
		(xy)' &= -\frac{1}{x} \notag \\
		xy &= \int -\frac{1}{x} \,dx \notag \\
		xy &= -\ln(x) + c_1 \notag \\
		y &= \frac{-\ln(x) + c_1}{x} \notag
	\end{align}
	
	\section*{Aufgabe 5}
	Wir wollen hier den Satz von Fubini anwenden, dazu sollten wir aber die Bedingung $y\le 6-x$ zu $x\le 6-y$ umformen. Dann ergibt sich
	\begin{align}
		\int_G \frac{1}{(x+y+1)^2} \,dxdy &= \int_{\frac{1}{2}x}^{2x}\int_{-\infty}^{6-y} \frac{1}{(x+y+1)^2} \,dxdy \notag
	\end{align}
	Finden wir zuerst mal eine Stammfunktion bezüglich $x$ von $\frac{1}{(x+y+1)^2}$. Dazu substituieren wir $u=x+y+1$ und $du=dx$ und erhalten
	\begin{align}
		\int \frac{1}{(x+y+1)^2} \,dx &= \int \frac{1}{u^2} du \notag \\
		&= -\frac{1}{u} \notag \\
		&= -\frac{1}{x+y+1} \notag
	\end{align}
	Damit löst sich das innere Integral zu
	\begin{align}
		\int_{-\infty}^{6-y} \frac{1}{(x+y+1)^2} \,dx &= -\frac{1}{6-y+y+1} - 0 \notag \\
		&= -\frac{1}{7} \notag
	\end{align}
	Die Stammfunktion bezüglich $y$ von $-\frac{1}{7}$ ist $-\frac{1}{7}y$ und damit löst sich das äußere Integral zu
	\begin{align}
		\int_{\frac{1}{2}x}^{2x} -\frac{1}{7}\, dy &= -\frac{1}{7}(2x) - \left(-\frac{1}{7}\left(\frac{1}{2}x\right)\right) \notag \\
		&= -\frac{2}{7}x + \frac{1}{14}x \notag \\
		&= -\frac{3}{14}x \notag
	\end{align}

	\section*{Aufgabe 6}
	Um zu testen, ob $T$ stetig ist, müssen wir beweisen, dass $\Vert T(v)\Vert_W \le L\cdot \Vert v\Vert_W$ für ein frei wählbares, aber festes $L$ ist.\footnote{Das $L$ heißt \textsc{Lipschitz}-Konstante.} Angewendet hier auf die Aufgabe müssen wir also zeigen oder widerlegen (was deutlich einfacher ist), ob
	\begin{align}
		\Vert x\cdot f'(x)\Vert_\infty \le L\cdot \Vert f(x)\Vert_\infty \notag
	\end{align}
	Da wir uns hier in $C[0,1]$ bewegen, ist $x$ maximal 1 und wir können sagen, dass
	\begin{align}
		\Vert x\cdot f'(x)\Vert_\infty \le \Vert f'(x)\Vert_\infty \notag
	\end{align}
	ist. Was wir also eigentlich zeigen oder widerlegen wollen, ist
	\begin{align}
		\Vert f'(x)\Vert_\infty \le L\cdot \Vert f(x)\Vert_\infty \notag
	\end{align}
	Wenn wir die Aussage widerlegen wollen, dann müssen wir Funktionen finden, die eine möglichst große Ableitung besitzen, sodass das $L$ auch besonders groß wird. Die Hoffnung ist, dass wir $\Vert f'(x)\Vert \to \infty$ treiben können, weil dann lässt sich kein festes $L$ mehr finden ($L=\infty$ ist offensichtlich nicht fest). Als Funktionen mit großen Ableitungen fallen mit Potenzfunktionen ein, am besten finde ich die Exponentialfunktion, die hat auch eine einfache Ableitung. Insbesondere hat $f_n(x) = e^{nx}$ die Ableitung $n\cdot e^{nx}$ und für $n\to\infty$ strebt auch die Ableitung Richtung unendlich. Auch das Maximum lässt sich hier sehr bequem berechnen, es liegt bei 1.
	
	Das sieht schon mal ganz vielversprechend aus, setzen wir das mal in die obige Ungleichung ein und gucken was passiert:
	\begin{align}
		\Vert f'(x)\Vert_\infty &\le L\cdot \Vert f(x)\Vert_\infty \notag \\
		\Vert n\cdot e^{nx}\Vert_\infty &\le L\cdot \Vert e^{nx}\Vert_\infty \notag \\
		n\cdot e^n &\le L\cdot e^n \notag \\
		n &\le L \notag
	\end{align}
	Da wir $n$ bis in alle Unendlichkeit treiben können, wird sich auch nie ein festes $L$ finden lassen, dass die Ungleichung für \textit{alle} Funktionen erfüllt. Damit ist $T$ nicht stetig.
	
\end{document}