\documentclass{article}

\usepackage{amsmath,amssymb}
\usepackage{tikz}
\usepackage{pgfplots}
\usepackage{xcolor}
\usepackage[left=2.1cm,right=3.1cm,bottom=3cm,footskip=0.75cm,headsep=0.5cm]{geometry}
\usepackage{enumerate}
\usepackage{enumitem}
\usepackage{marvosym}
\usepackage{tabularx}
\usepackage{multirow}
\usepackage[colorlinks = true, linkcolor = blue, urlcolor  = blue, citecolor = blue, anchorcolor = blue]{hyperref}

\usepackage{listings}
\definecolor{lightlightgray}{rgb}{0.95,0.95,0.95}
\definecolor{lila}{rgb}{0.8,0,0.8}
\definecolor{mygray}{rgb}{0.5,0.5,0.5}
\definecolor{mygreen}{rgb}{0,0.8,0.26}
\lstdefinestyle{java} {language=java}
\lstset{language=java,
	basicstyle=\ttfamily,
	keywordstyle=\color{lila},
	commentstyle=\color{lightgray},
	stringstyle=\color{mygreen}\ttfamily,
	backgroundcolor=\color{white},
	showstringspaces=false,
	numbers=left,
	numbersep=10pt,
	numberstyle=\color{mygray}\ttfamily,
	identifierstyle=\color{blue},
	xleftmargin=.1\textwidth, 
	%xrightmargin=.1\textwidth,
	escapechar=§,
}

\usepackage[utf8]{inputenc}

\renewcommand*{\arraystretch}{1.4}

\newcolumntype{L}[1]{>{\raggedright\arraybackslash}p{#1}}
\newcolumntype{R}[1]{>{\raggedleft\arraybackslash}p{#1}}
\newcolumntype{C}[1]{>{\centering\let\newline\\\arraybackslash\hspace{0pt}}m{#1}}

\newcommand{\E}{\mathbb{E}}
\DeclareMathOperator{\rk}{rk}
\DeclareMathOperator{\Var}{Var}
\DeclareMathOperator{\Cov}{Cov}
\DeclareMathOperator{\SD}{SD}
\DeclareMathOperator{\Cor}{Cor}

\title{\textbf{Instrumente des Finanzmanagements, Tutorium 4}}
\author{\textsc{Henry Haustein}}
\date{}

\begin{document}
	\maketitle
	
	\section*{Aufgabe 18.3: Investitionsplanung}
	Der Leverage ist $\frac{D}{E}=2.6$, also z.B. $D=26$ und $E=10$. Damit ergibt sich der WACC
	\begin{align}
		r_{WACC} &= \frac{10}{10+26}\cdot 8.5\% + \frac{26}{10+26}\cdot 7\%\cdot (1-0.35) \notag \\
		&= 5.65\% \notag
	\end{align}
	Damit ist der Barwert
	\begin{align}
		BW &= \frac{1.5\text{ Mio. \EUR}}{0.0565-0.025} \notag \\
		&= 47.62\text{ Mio. \EUR}\notag
	\end{align}

	\section*{Aufgabe 18.4: Die Methode des gewichteten Durchschnitts der Kapitalkosten}
	\begin{enumerate}[label=(\alph*)]
		\item Der WACC ist
		\begin{align}
			r_{WACC} &= \frac{10.8}{14.4}\cdot 10\% + \frac{14.4-10.8}{14.4}\cdot 6.1\%\cdot (1-0.35) \notag \\
			&= 8.49\% \notag
		\end{align}
		\item Damit ist der Kapitalwert
		\begin{align}
			V_0 &= \frac{50\text{ Mio. \EUR}}{1.0849} + \frac{100\text{ Mio. \EUR}}{1.0849^2} + \frac{70\text{ Mio. \EUR}}{1.0849^3} \notag \\
			&= 185.8671\text{ Mio. \EUR} \notag
		\end{align}
		\item Der Verschuldungsgrad ist $d=\frac{D}{V}=\frac{14.4-10.8}{14.4}=\frac{1}{4}$, damit ergibt sich folgende Tabelle:
		\begin{center}
			\begin{tabular}{l|c|c|c|c}
				$t$ & 0 & 1 & 2 & 3 \\
				\hline
				$FCF_t$ & -100 & 50 & 100 & 70 \\
				$V_t$ & 185.8671 & $\frac{100}{1.0849} + \frac{70}{1.0849^2}=151.6472$ & $\frac{70}{1.0849}=64.5221$ & 0 \\
				$D_t=V_t\cdot d$ & 46.4668 & 37.9118 & 16.1305 & 0
			\end{tabular}
		\end{center}
	\end{enumerate}

	\section*{Aufgabe 18.11: Der APV bei anderen Verschuldungsstrategien}
	\begin{enumerate}[label=(\alph*)]
		\item Berechnung der FCF
		\begin{center}
			\begin{tabular}{l|r|r|r}
				& \textbf{Jahr 0} & \textbf{Jahr 1-9} & \textbf{Jahr 10} \\
				\hline
				Einnahmen & 0.00 & 145.00 & 145.00 + 300.00 \\
				- Ausgaben & -0.00 & -0.00 & -0.00 \\
				- Abschreibungen & -0.00 & -60.00 & -60.00 \\
				\hline
				= EBIT & 0.00 & 85.00 & 385.00 \\
				- Steuer ($\tau=35\%$) & -0.00 & -29.75 & -134.75 \\
				- Investitionen & -600.00 & -0.00 & -0.00 \\
				 + Abschreibungen & -0.00 & -60.00 & -60.00 \\
				 $\Delta$ Nettoumlauf & -50.00 & 0.00 & 50.00 \\
				 \hline
				 = FCF & -650.00 & 115.25 & 360.25
			\end{tabular}
		\end{center}
		Der Zins ist $r=5\% + 1.57\cdot (11\%-5\%)=14.42\%$, damit ist der Barwert
		\begin{align}
			BW &= -650 + \sum_{i=1}^{9}\frac{115.25}{1.1442^i} + \frac{360.25}{1.1442^{10}} \notag \\
			&= 5.1331 \notag
		\end{align}
		\item Der Steuervorteil in jedem Jahr ist $TS=D\cdot \tau\cdot r_D=400\cdot 0.35\cdot 0.09=12.6$, damit ist der Barwert
		\begin{align}
			BW &= \sum_{i=1}^{10} \frac{12.6}{1.09^i} \notag \\
			&= 115.5963 \notag
		\end{align}
		und damit ist der Wert des Projektes $V_0=5.1331 + 115.5963=120.7294$.
	\end{enumerate}

	\section*{Aufgabe 1K299: APV / FTE / WACC}
	\begin{enumerate}[label=(\alph*)]
		\item EK-Kosten verschuldet (vor Steuern): $r_E=6\% + 1.5\cdot (10\%-6\%)=12\%$ \\
		EK-Kosten unverschuldet (nach Steuern):
		\begin{align}
			r_U &= \frac{E}{E+D(1-\tau)} \cdot r_E + \frac{D(1-\tau)}{E+D(1-\tau)} \cdot r_D \notag \\
			&= \frac{1}{1+1\cdot 0.75}\cdot 12\% + \frac{1\cdot 0.75}{1+1\cdot 0.75}\cdot 6\% \notag \\
			&= 9.43\% \notag
		\end{align}
		\item Der FCF ist
		\begin{center}
			\begin{tabular}{l|r}
				Umsatz & 0.9 \\
				- Kosten & -0.3 \\
				- Abschreibungen & -0.01 \\
				\hline
				= EBIT & 0.59 \\
				- Steuern ($\tau=0.25$) & -0.1475 \\
				+ Abschreibungen & 0.01 \\
				- Investitionen & -0.01 \\
				\hline
				= FCF & 0.4425
			\end{tabular}
		\end{center}
		Weil $D=const.$ und mit 5 Mio. Anfangsinvestition mit $L=1$, ergibt sich für den Barwert der Steuervorteile: $BW(TS) = D\cdot \tau = 2.5\cdot 0.25 = 0.625$ und der Wert des Unternehmens ist
		\begin{align}
			V_0 &= \frac{0.4425}{0.0943} + 0.625 \notag \\
			&= 5.3175 \notag
		\end{align}
		\item mehr Steuern $\to$ weniger FCF $\to$ höheres Insolvenzrisiko \\
		häufige Anpassung von $D$, da $L=const.$ $\to$ höhere Transaktionskosten/Emissionskosten
		\item Wir brauchen den FCFE
			\begin{center}
			\begin{tabular}{l|r}
				Umsatz & 0.9 \\
				- Kosten & -0.3 \\
				- Abschreibungen & -0.01 \\
				- Zinsen & $0.06\cdot 2.5$ \\
				\hline
				= EBIT & 0.44 \\
				- Steuern ($\tau=0.25$) & -0.11 \\
				+ Abschreibungen & 0.01 \\
				- Investitionen & -0.01 \\
				\hline
				= FCF & 0.33
			\end{tabular}
		\end{center}
		Damit gilt
		\begin{align}
			E &= \frac{FCFE}{r_E} = \frac{0.33}{0.12} = 2.75 \notag \\
			V_0 &= \frac{E}{1-d} = \frac{2.75}{0.5} = 5.5 \notag
		\end{align}
		\item Der WACC ist
		\begin{align}
			r_{WACC} &= \frac{1}{2}\cdot 12\% + \frac{1}{2}\cdot 6\%\cdot (1-0.25) \notag \\
			&= 8.25\% \notag
		\end{align}
		Und damit
		\begin{align}
			V_0 &= \frac{FCF}{r_{WACC}} \notag \\
			&= \frac{0.4425}{0.0825} \notag \\
			&= 5.3636 \notag
		\end{align}
		\item WACC bei $L=const.$ \\
		APV bei $D=const.$ \\
		FTE bei $TS\neq const.$
	\end{enumerate}
	
\end{document}