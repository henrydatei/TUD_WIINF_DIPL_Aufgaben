\documentclass{article}

\usepackage{amsmath,amssymb}
\usepackage{tikz}
\usepackage{pgfplots}
\usepackage{xcolor}
\usepackage[left=2.1cm,right=3.1cm,bottom=3cm,footskip=0.75cm,headsep=0.5cm]{geometry}
\usepackage{enumerate}
\usepackage{enumitem}
\usepackage{marvosym}
\usepackage{tabularx}
\usepackage{multirow}
\usepackage[colorlinks = true, linkcolor = blue, urlcolor  = blue, citecolor = blue, anchorcolor = blue]{hyperref}

\usepackage{listings}
\definecolor{lightlightgray}{rgb}{0.95,0.95,0.95}
\definecolor{lila}{rgb}{0.8,0,0.8}
\definecolor{mygray}{rgb}{0.5,0.5,0.5}
\definecolor{mygreen}{rgb}{0,0.8,0.26}
\lstdefinestyle{java} {language=java}
\lstset{language=java,
	basicstyle=\ttfamily,
	keywordstyle=\color{lila},
	commentstyle=\color{lightgray},
	stringstyle=\color{mygreen}\ttfamily,
	backgroundcolor=\color{white},
	showstringspaces=false,
	numbers=left,
	numbersep=10pt,
	numberstyle=\color{mygray}\ttfamily,
	identifierstyle=\color{blue},
	xleftmargin=.1\textwidth, 
	%xrightmargin=.1\textwidth,
	escapechar=§,
}

\usepackage[utf8]{inputenc}

\renewcommand*{\arraystretch}{1.4}

\newcolumntype{L}[1]{>{\raggedright\arraybackslash}p{#1}}
\newcolumntype{R}[1]{>{\raggedleft\arraybackslash}p{#1}}
\newcolumntype{C}[1]{>{\centering\let\newline\\\arraybackslash\hspace{0pt}}m{#1}}

\newcommand{\E}{\mathbb{E}}
\DeclareMathOperator{\rk}{rk}
\DeclareMathOperator{\Var}{Var}
\DeclareMathOperator{\Cov}{Cov}
\DeclareMathOperator{\SD}{SD}
\DeclareMathOperator{\Cor}{Cor}

\title{\textbf{Lösung Probeklausur WS1516}}
\author{\textsc{Henry Haustein}}
\date{}

\begin{document}
	\maketitle

	\section*{Aufgabe 1: Projektbewertung}
	\begin{enumerate}[label=(\alph*)]
		\item Die Fremdkapitalbetas sind:
		\begin{align}
			\beta_D^{Binder} &= \frac{0.08 + 0.15}{2} = 0.115 \notag \\
			\beta_D^{TING} &= \frac{0.45 + 0.33}{2} = 0.39 \notag \\
			\beta_D^{Headbook} &= \frac{0.23 + 0.15}{2} = 0.19 \notag \\
			\beta_D^{baluu} &= \frac{0.33 + 0.23}{2} = 0.28 \notag
		\end{align}
		\item Die Unternehmensbetas sind:
		\begin{align}
			\beta^{Binder} &= \frac{500}{1600}\cdot 5 + \frac{1100}{1600} \cdot 0.115 = 1.6416 \notag \\
			\beta^{TING} &= \frac{230}{300}\cdot 3.4 + \frac{70}{300} \cdot 0.39 = 2.6977 \notag \\
			\beta^{Headbook} &= \frac{1000}{2500}\cdot 4 + \frac{1500}{2500} \cdot 0.19 = 1.714 \notag \\
			\beta^{baluu} &= \frac{750}{1150}\cdot 3 + \frac{400}{1150} \cdot 0.28 = 2.0539 \notag
		\end{align}
		\item Das Branchenbeta ist damit $\beta = \frac{1.6416 + 2.6977 + 1.714 + 2.0539}{4} = 2.0268$.
		\item In den einzelnen Zuständen ist die Rendite:
		\begin{center}
			\begin{tabular}{l|c|c|c|c}
				& \textbf{Z1} & \textbf{Z2} & \textbf{Z3} & \textbf{Z4} \\
				\hline
				Wahrscheinlichkeit & 50\% & 30\% & 19\% & 1\% \\
				\hline
				Rendite & 10\% & 25\% & -10\% & -50\%
			\end{tabular}
		\end{center}
		Damit ist der Erwartungswert:
		\begin{align}
			\E(r_M) &= 0.5\cdot10\% + 0.3\cdot 25\% + 0.19\cdot -10\% + 0.01\cdot -50\% \notag \\
			&= 10.1\% \notag
		\end{align}
		\item Die Marktrisikoprämie $\E(r_M)-r_f$ ist 9.5\%, damit ist $r_f=0.6\%$ und damit gilt nach CAPM:
		\begin{align}
			r_U &= 0.6\% + 2.0268\cdot 9.5\% \notag \\
			&= 19.8546\% \notag
		\end{align}
		\item Es gilt:
		\begin{align}
			V_0 &= \frac{FCF}{r_U-g} = \frac{40\text{ Mio. \EUR}}{0.198546 - 0.1} \notag \\
			&= 405.9018\text{ Mio. \EUR} \notag
		\end{align}
		\item $V_0 = E + D \Rightarrow E = 265.9018\text{ Mio. \EUR}$. Wir brauchen noch den Fremdkapitalzinssatz $r_D = 0.6\%\cdot 0.23\cdot 9.5\% = 2.785\%$ und damit gilt:
		\begin{align}
			r_U &= \frac{E}{V}\cdot r_E + \frac{D}{V}\cdot r_D \notag \\
			19.8546\% &= \frac{265.9018\text{ Mio. \EUR}}{405.9018\text{ Mio. \EUR}}\cdot r_E + \frac{140\text{ Mio. \EUR}}{405.9018\text{ Mio. \EUR}}\cdot 2.785\% \notag \\
			r_E &= 28.8419\% \notag
		\end{align}
		und weiterhin
		\begin{align}
			r_E &= r_f + \beta_E\cdot (\E(r_M) - r_f) \notag \\
			28.8419\% &= 0.6\% + \beta_E\cdot 9.5\% \notag \\
			\beta_E &= 2.9728\notag
		\end{align}
		\item Die Annahme ist ein vollkommener Kapitalmarkt, damit keine Transaktionskosten, homogene Erwartungen und Geld kann zum risikofreien Zins in unbegrenzter Höhe geliehen werden.
	\end{enumerate}
	
	\section*{Aufgabe 3: CAPM und Portfoliotheorie}
	\begin{enumerate}[label=(\alph*)]
		\item Die Rendite der A-Bank $r_A$ und der B-Bau $r_B$ ist:
		\begin{align}
			r_A &= 0.2\cdot -5\% + 0.5\cdot 15\% + 0.3\cdot 5\% = 8\% \notag \\
			r_B &= 0.2\cdot 7\% + 0.5\cdot 18\% + 0.3\cdot -6\% = 8.6\% \notag \\
			\SD(r_A) &= \sqrt{0.2(-5\% - 8\%)^2 + 0.5(15\% - 8\%)^2 + 0.3(5\% - 8\%)^2} = \sqrt{61} = 7.8102\% \notag \\
			\SD(r_B) &= \sqrt{0.2(7\% - 8.6\%)^2 + 0.5(18\% - 8.6\%)^2 + 0.3(-6\% - 8.6\%)^2} = \frac{2\sqrt{679}}{5} = 10.4231\% \notag
		\end{align}
		\item Die Kovarianz und Korrelation ist:
		\begin{align}
			\Cov(r_A,r_B) &= 0.2(-5\% - 8\%)(7\% - 8.6\%) + 0.5(15\% - 8\%)(18\% - 8.6\%) + 0.3(5\% - 8\%)(-6\% - 8.6\%) \notag \\
			&= 50.2\%^2 \notag \\
			\Cor(r_A,r_B) &= \frac{\Cov(r_A,r_B)}{\SD(r_A)\cdot \SD(r_B)} = \frac{50.2\%^2}{7.8102\%\cdot 10.4231\%} \notag \\
			&= 0.6167 \notag
		\end{align}
		\item Bei einer Korrelation von 1 muss man kein Portfolio erstellen, es ist auch keine Diversifikation möglich \\
		Bei einer Korrelation von $<1$ sinkt die Volatilität des Portfolios aufgrund der Diversifikation \\
		Bei einer Korrelation von $-1$ kann man mit dem Portfolio Gewinne ohne Risiko einfahren
		\item Es gilt:
		\begin{align}
			8\% &= 1\% + \beta_A\cdot (10\% - 1\%) \notag \\
			\beta_A &= 0.7778 \notag \\
			8.6\% &= 1\% + \beta_B\cdot (10\% - 1\%) \notag \\
			\beta_B &= 0.8444 \notag
		\end{align}
		und es gilt:
		\begin{align}
			\beta_A &= \Cor(r_A, r_M)\cdot\frac{\SD(r_A)}{\SD(r_M)} \notag \\
			0.7778 &= 0.5\cdot \frac{7.8102\%}{\SD(r_M)} \notag \\
			\SD(r_M) &= 5.0207\% \notag
		\end{align}
		\item Rendite und Standardabweichung\footnote{Man könnte sicherlich eine Formel für $\Var(X+Y+Z)$ herleiten, aber da der risikofreie Zinssatz immer gleich ist und zu nichts korreliert, kann man den einfach in der Berechnung der Varianz/Standardabweichung weglassen.} des Portfolios sind:
		\begin{align}
			r_P &= 0.3\cdot 8\% + 0.6\cdot 8.6\% + 0.1\cdot 1\% \notag \\
			&= 7.66\%\notag \\
			\SD(r_P) &= \sqrt{0.3^2\cdot (7.8102\%)^2 + 0.6^2\cdot (10.4231\%)^2 + 2\cdot 0.3\cdot 0.6\cdot 50.2\%^2} \notag \\
			&= 7.9166 \% \notag
		\end{align}
		Damit gilt:
		\begin{align}
			7.66\% &= 1\% + \beta_P\cdot (10\% - 1\%) \notag \\
			\beta_P &= 0.74 \notag
		\end{align}
		und weiterhin
		\begin{align}
			\beta_P &= \Cor(r_P, r_M)\cdot\frac{\SD(r_P)}{\SD(r_M)} \notag \\
			0.74 &= \Cor(r_P, r_M)\cdot\frac{7.9166\%}{5.0207\%} \notag \\
			\Cor(r_P, r_M) &= 0.4693 \notag
		\end{align}
		\item Da ich mir für 1\% Geld leihe, muss mein Portfolio eine Rendite von 8.5\% abwerfen, also
		\begin{align}
			8.5\% &= \alpha \cdot 8\% + (1-\alpha)\cdot 8.6\% \notag \\
			\alpha &= 0.1667 \notag
		\end{align}
	\end{enumerate}
	
\end{document}