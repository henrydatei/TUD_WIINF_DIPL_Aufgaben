\documentclass{article}

\usepackage{amsmath,amssymb}
\usepackage{tikz}
\usepackage{pgfplots}
\usepackage{xcolor}
\usepackage[left=2.1cm,right=3.1cm,bottom=3cm,footskip=0.75cm,headsep=0.5cm]{geometry}
\usepackage{enumerate}
\usepackage{enumitem}
\usepackage{marvosym}
\usepackage{tabularx}
\usepackage{parskip}

\usepackage{listings}
\definecolor{lightlightgray}{rgb}{0.95,0.95,0.95}
\definecolor{lila}{rgb}{0.8,0,0.8}
\definecolor{mygray}{rgb}{0.5,0.5,0.5}
\definecolor{mygreen}{rgb}{0,0.8,0.26}
%\lstdefinestyle{java} {language=java}
\lstset{language=python,
	basicstyle=\ttfamily,
	keywordstyle=\color{lila},
	commentstyle=\color{lightgray},
	stringstyle=\color{mygreen}\ttfamily,
	backgroundcolor=\color{white},
	morekeywords={split, as, dot, read\_csv, loc, between, groupby, sort\_values},
	showstringspaces=false,
	numbers=left,
	numbersep=10pt,
	tabsize=2,
	numberstyle=\color{mygray}\ttfamily,
	identifierstyle=\color{blue},
	xleftmargin=.1\textwidth, 
	%xrightmargin=.1\textwidth,
	escapechar=§,
	%literate={\t}{{\ }}1
	breaklines=true,
	postbreak=\mbox{\space},
	literate=%
	{Ö}{{\"O}}1
	{Ä}{{\"A}}1
	{Ü}{{\"U}}1
	{ß}{{\ss}}1
	{ü}{{\"u}}1
	{ä}{{\"a}}1
	{ö}{{\"o}}1
}

\usepackage[colorlinks = true, linkcolor = blue, urlcolor  = blue, citecolor = blue, anchorcolor = blue]{hyperref}
\usepackage[utf8]{inputenc}

\renewcommand*{\arraystretch}{1.4}

\newcolumntype{L}[1]{>{\raggedright\arraybackslash}p{#1}}
\newcolumntype{R}[1]{>{\raggedleft\arraybackslash}p{#1}}
\newcolumntype{C}[1]{>{\centering\let\newline\\\arraybackslash\hspace{0pt}}m{#1}}

\newcommand{\E}{\mathbb{E}}
\DeclareMathOperator{\rk}{rk}
\DeclareMathOperator{\Var}{Var}
\DeclareMathOperator{\Cov}{Cov}

\title{\textbf{Prescriptive Analytics, Seminar 3}}
\author{\textsc{Henry Haustein}}
\date{}

\begin{document}
	\maketitle
	\begin{lstlisting}
# files from first session
from InputData import InputData
from OutputData import OutputJob
# Additionally
import numpy

	\end{lstlisting}
	\section*{Aufgabe 1: Codierung und Bewertung einer Lösung}
	\begin{enumerate}[label=(\alph*)]
		\item Klasse \texttt{Solution}
		\begin{lstlisting}
class Solution:
	OutputJobs = {}
	Makespan = None
	TotalTardiness = None
	TotalWeightedTardiness = None
	Permutation = None

	def __init__(self, inputJobs, reihenfolge):
		self.Makespan = -1
		self.TotalTardiness = -1
		self.TotalWeightedTardiness = -1
		self.Permutation = reihenfolge

		for idx, value in enumerate(inputJobs):
			self.OutputJobs[idx] = OutputJob(value)

	def __str__(self):
		return f"Solution({self.OutputJobs = }, {self.Makespan = }, {self.TotalTardiness = }, {self.TotalWeightedTardiness = }, {self.Permutation = })"

#### Führen Sie im Anschluss folgenden Code aus ####
data = InputData("InputFlowshopSIST.json")
Permutation = [x-1 for x in [6,5,7,4,8,3,9,2,10,1,11]]
DevilSolution = Solution(data.InputJobs, Permutation) 
print(DevilSolution)
		\end{lstlisting}
		\item Klasse \texttt{EvaluationLogic}. Diese Klasse ist ziemlich sinnlos, sie enthält keine Attribute, sondern nur Funktionen
		\begin{lstlisting}
class EvaluationLogic:    
	def DefineStartEnd(self, currentSolution):    
		#####
		# schedule first job: starts when finished at previous stage
		firstJob = currentSolution.OutputJobs[currentSolution.Permutation[0]]
		firstJob.EndTimes = numpy.cumsum([firstJob.ProcessingTime(x) for x in range(len(firstJob.EndTimes))])
		firstJob.StartTimes[1:] = firstJob.EndTimes[:-1]
		#####
		# schedule further jobs: starts when finished at previous stage and the predecessor is no longer on the considered machine
		for j in range(1,len(currentSolution.Permutation)):
			currentJob = currentSolution.OutputJobs[currentSolution.Permutation[j]]
			previousJob = currentSolution.OutputJobs[currentSolution.Permutation[j-1]]
			# first machine
			currentJob.StartTimes[0] = previousJob.EndTimes[0]
			currentJob.EndTimes[0] = currentJob.StartTimes[0] + currentJob.ProcessingTime(0)
			# other machines
			for i in range(1,len(currentJob.StartTimes)):
				currentJob.StartTimes[i] = max(previousJob.EndTimes[i], currentJob.EndTimes[i-1])
				currentJob.EndTimes[i] = currentJob.StartTimes[i] + currentJob.ProcessingTime(i)
		#####
		# Save Makespan and return Solution
		currentSolution.Makespan = currentSolution.OutputJobs[currentSolution.Permutation[-1]].EndTimes[-1]

EvaluationLogic().DefineStartEnd(DevilSolution)
print(DevilSolution.Makespan)
		\end{lstlisting}
		\item Die Dokumentation des \texttt{csv}-Moduls gibt es hier: \url{https://docs.python.org/3/library/csv.html}
		\begin{lstlisting}
import csv

def WriteSolToCsv(self):
	with open("output.csv", "w") as file:
		writer = csv.writer(file)
		writer.writerow(["Machine", "Job", "Start_Setup", "End_Setup", "Start", "End"])
		for maschine in range(5):
			for jobID in self.Permutation:
				currentJob = self.OutputJobs[jobID]
				writer.writerow([maschine + 1, jobID, currentJob.StartSetups[maschine], currentJob.EndSetups[maschine], currentJob.StartTimes[maschine], currentJob.EndTimes[maschine]])

setattr(Solution, "WriteSolToCsv", WriteSolToCsv)

DevilSolution.WriteSolToCsv()
		\end{lstlisting}
		\item Super simpel, wenn man weiß, was man machen muss. Ein bisschen Dokumentation hätte dem externen Quellcode schon gut getan
		\begin{lstlisting}
from Gantt import *

GanttChartFromCsv("output.csv", "gantt.html")
		\end{lstlisting}
	\end{enumerate}

	\section*{Aufgabe 2: Dispatching Rules zur Erstellung von Lösungen}
	\begin{enumerate}[label=(\alph*)]
		\item Es gibt viele Funktionen die eine zufällige Permutation erzeugen. Ich habe mich hier für \texttt{numpy.random.shuffle()} entschieden.
		\begin{lstlisting}
def ROS(anzahlFolgen, inputData):
	bestSolution = None
	reihenfolge = numpy.arange(11)
	for i in range(anzahlFolgen):
		numpy.random.shuffle(reihenfolge)
		sol = Solution(inputData.InputJobs, reihenfolge)
		EvaluationLogic().DefineStartEnd(sol)
		if bestSolution == None or bestSolution.Makespan > sol.Makespan:
			bestSolution = sol

	print("Reihenfolge " + str(bestSolution.Permutation) + " ergibt Makespan " + str(bestSolution.Makespan))
	return sol

ROS(50, data)
		\end{lstlisting}
		\item Wir kombinieren \texttt{ROS} mit \texttt{Itertools} um alle Permutationen durchlaufen zu lassen:
		\begin{lstlisting}
from itertools import permutations

def checkAllPermutations(inputData):
	bestSolution = None
	allPermutations = permutations(range(11))
	print("Rechenzeit: " + str(len(allPermutations) * 0.0011259999999992942 / 60) + " Minuten")
	for reihenfolge in allPermutations:
		sol = Solution(inputData.InputJobs, reihenfolge)
		EvaluationLogic().DefineStartEnd(sol)
		if bestSolution == None or bestSolution.Makespan > sol.Makespan:
			bestSolution = sol

	print("Reihenfolge " + str(bestSolution.Permutation) + " ergibt Makespan " + str(bestSolution.Makespan))
	return sol
		\end{lstlisting}
		Um die Rechenzeit halbwegs exakt abgeben zu können, müssen wir ermitteln, wie lange es für eine Permutation dauert. Dafür gibt es \texttt{time}:
		\begin{lstlisting}
import time

start = time.process_time()
reihenfolge = range(11)
sol = Solution(data.InputJobs, reihenfolge)
EvaluationLogic().DefineStartEnd(sol)
print(time.process_time() - start)
		\end{lstlisting}
		\item First-come-first-serve ist ziemlich einfach. Die Reihenfolge ist einfach 1, 2, 3, ...
		\begin{lstlisting}
def FCFS(inputData):
	reihenfolge = range(11)
	sol = Solution(inputData.InputJobs, reihenfolge)
	EvaluationLogic().DefineStartEnd(sol)

	return sol

FCFSSol = FCFS(data)
print(FCFSSol)
		\end{lstlisting}
		Für Shortest-Processing-Time müssen wir erst alle ProcessingTimes für jeden Job aufaddieren und danach sortieren:
		\begin{lstlisting}
def SPT(inputData):
	procTimes = {}
	for job in inputData.InputJobs:
		procTimes[job.JobId] = sum(value[1] for value in job.Operations)
	reihenfolge = {k-1: v for k, v in sorted(procTimes.items(), key=lambda item: item[1])}
	print(reihenfolge)
	sol = Solution(inputData.InputJobs, list(reihenfolge.keys()))
	EvaluationLogic().DefineStartEnd(sol)

	return sol

SPTSol = SPT(data)
print(SPTSol)
		\end{lstlisting}
		Longest-Processing-Time ist fast das selbe, nur, dass wir hier anders herum sortieren:
		\begin{lstlisting}
def LPT(inputData):
	procTimes = {}
	for job in inputData.InputJobs:
		procTimes[job.JobId] = sum(value[1] for value in job.Operations)
	reihenfolge = {k-1: v for k, v in sorted(procTimes.items(), key=lambda item: item[1], reverse = True)}
	print(reihenfolge)
	sol = Solution(inputData.InputJobs, list(reihenfolge.keys()))
	EvaluationLogic().DefineStartEnd(sol)

	return sol

LPTSol = LPT(data)
print(LPTSol)
		\end{lstlisting}
		\item NEH-Heuristik war schon gegeben
		\item Wieder so eine völlig sinnlose Klasse mit nur einer Funktion
		\begin{lstlisting}
class ConstructiveHeuristic:
	inputData = None

	def __init__(self, inputData):
		self.inputData = inputData

	def useRule(self, name):
		if name  == "ros":
			return ROS(10, self.inputData)
		if name == "fcfs":
			return FCFS(self.inputData)
		if name == "spt":
			return SPT(self.inputData)
		if name == "lpt":
			return LPT(self.inputData)
		if name == "neh":
			return NEH(self.inputData)
		\end{lstlisting}
	\end{enumerate}

	\section*{Aufgabe 3: Liefertreue}
	\begin{enumerate}[label=(\alph*)]
		\item schon gegeben
		\item DueDates sammeln und nach diesen sortieren
		\begin{lstlisting}
def EDD(inputData):
	dueTimes = {}
	for job in inputData.InputJobs:
		dueTimes[job.JobId] = job.DueDate
	reihenfolge = {k-1: v for k, v in sorted(dueTimes.items(), key=lambda item: item[1])}
	print(reihenfolge)
	sol = Solution(inputData.InputJobs, list(reihenfolge.keys()))
	EvaluationLogic().DefineStartEnd(sol)
	EvaluationLogic().CalculateTardiness(sol)

	return sol

SPTSol = SPT(data)
print(SPTSol)
		\end{lstlisting}
	\end{enumerate}

\end{document}