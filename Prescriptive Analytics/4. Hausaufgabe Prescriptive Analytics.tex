\documentclass{article}

\usepackage{amsmath,amssymb}
\usepackage{tikz}
\usepackage{pgfplots}
\usepackage{xcolor}
\usepackage[left=2.1cm,right=3.1cm,bottom=3cm,footskip=0.75cm,headsep=0.5cm]{geometry}
\usepackage{enumerate}
\usepackage{enumitem}
\usepackage{marvosym}
\usepackage{tabularx}
\usepackage{parskip}

\usepackage{listings}
\definecolor{lightlightgray}{rgb}{0.95,0.95,0.95}
\definecolor{lila}{rgb}{0.8,0,0.8}
\definecolor{mygray}{rgb}{0.5,0.5,0.5}
\definecolor{mygreen}{rgb}{0,0.8,0.26}
%\lstdefinestyle{java} {language=java}
\lstset{language=python,
	basicstyle=\ttfamily,
	keywordstyle=\color{lila},
	commentstyle=\color{lightgray},
	stringstyle=\color{mygreen}\ttfamily,
	backgroundcolor=\color{white},
	morekeywords={split, as, dot, read\_csv, loc, between, groupby, sort\_values},
	showstringspaces=false,
	numbers=left,
	numbersep=10pt,
	tabsize=2,
	numberstyle=\color{mygray}\ttfamily,
	identifierstyle=\color{blue},
	xleftmargin=.1\textwidth, 
	%xrightmargin=.1\textwidth,
	escapechar=§,
	%literate={\t}{{\ }}1
	breaklines=true,
	postbreak=\mbox{\space},
	literate=%
	{Ö}{{\"O}}1
	{Ä}{{\"A}}1
	{Ü}{{\"U}}1
	{ß}{{\ss}}1
	{ü}{{\"u}}1
	{ä}{{\"a}}1
	{ö}{{\"o}}1
}

\usepackage[colorlinks = true, linkcolor = blue, urlcolor  = blue, citecolor = blue, anchorcolor = blue]{hyperref}
\usepackage[utf8]{inputenc}

\renewcommand*{\arraystretch}{1.4}

\newcolumntype{L}[1]{>{\raggedright\arraybackslash}p{#1}}
\newcolumntype{R}[1]{>{\raggedleft\arraybackslash}p{#1}}
\newcolumntype{C}[1]{>{\centering\let\newline\\\arraybackslash\hspace{0pt}}m{#1}}

\newcommand{\E}{\mathbb{E}}
\DeclareMathOperator{\rk}{rk}
\DeclareMathOperator{\Var}{Var}
\DeclareMathOperator{\Cov}{Cov}

\title{\textbf{Prescriptive Analytics, Hausaufgabe 4}}
\author{\textsc{Henry Haustein}}
\date{}

\begin{document}
	\maketitle
	
	\section*{Aufgabe 1: Zeitmessung}
	\begin{lstlisting}
from Solver import *
import timeit

data = InputData("VFR20_10_1_SIST.json") 
solver = Solver(data, 1008)

localSearch = IterativeImprovement(data, 'BestImprovement', ['Insertion'])
iteratedGreedy = IteratedGreedy(
	inputData = data, 
	numberJobsToRemove = 2, 
	baseTemperature = 0.8, 
	maxIterations = 10,
	localSearchAlgorithm = localSearch)

start = timeit.default_timer()
solver.RunLocalSearch(
	constructiveSolutionMethod='NEH',
	algorithm=iteratedGreedy)
ende = timeit.default_timer()
print(f"Runtime: {ende - start} seconds")
	\end{lstlisting}
	
	\section*{Aufgabe 2: Stoppkriterium}
	Code im Jupyter-Notebook:
	\begin{lstlisting}
from Solver import *

data = InputData("VFR20_10_1_SIST.json") 
solver = Solver(data, 1008)

localSearch = IterativeImprovement(data, 'BestImprovement', ['Insertion'])
iteratedGreedy = IteratedGreedy(
	inputData = data, 
	numberJobsToRemove = 2, 
	baseTemperature = 0.8, 
	maxIterations = 10,
	maxIterationsWithoutImprovement = 2,
	localSearchAlgorithm = localSearch)

solver.RunLocalSearch(
	constructiveSolutionMethod='NEH',
	algorithm=iteratedGreedy)
	\end{lstlisting}
	Aktualisierter Code in \texttt{ImprovementAlgorithm.py} (\texttt{\_\_init\_\_}-Methode)
	\begin{lstlisting}
def __init__(self, inputData, numberJobsToRemove, baseTemperature, maxIterations, maxIterationsWithoutImprovement = 1000000, localSearchAlgorithm = None):
	super().__init__(inputData)

	self.NumberJobsToRemove = numberJobsToRemove
	self.BaseTemperature = baseTemperature
	self.MaxIterations = maxIterations
	self.MaxIterationsWithoutImprovement = maxIterationsWithoutImprovement

	if localSearchAlgorithm is not None:
		self.LocalSearchAlgorithm = localSearchAlgorithm
	else:
		self.LocalSearchAlgorithm = IterativeImprovement(self.InputData, neighborhoodTypes=[]) # IterativeImprovement without a neighborhood does not modify the solution
	\end{lstlisting}
	\texttt{Run}-Methode (veränderte Zeilen: 6, 7, 15 und 24)
	\begin{lstlisting}
def Run(self, currentSolution):
	currentSolution = self.LocalSearchAlgorithm.Run(currentSolution)

	currentBest = self.SolutionPool.GetLowestMakespanSolution().Makespan
	iteration = 0
	withoutImprovement = 0
	while iteration < self.MaxIterations and withoutImprovement < self.MaxIterationsWithoutImprovement:
		removedJobs, partialPermutation = self.Destruction(currentSolution)
		newSolution = self.Construction(removedJobs, partialPermutation)

		newSolution = self.LocalSearchAlgorithm.Run(newSolution)

		if newSolution.Makespan < currentSolution.Makespan:
			currentSolution = newSolution
			withoutImprovement = 0

			if newSolution.Makespan < currentBest:
				print(f'New best solution in iteration {iteration}: {currentSolution}')
				self.SolutionPool.AddSolution(currentSolution)
				currentBest = newSolution.Makespan

		elif self.AcceptWorseSolution(currentSolution.Makespan, newSolution.Makespan):
			currentSolution = newSolution
			withoutImprovement += 1

		iteration += 1

	return self.SolutionPool.GetLowestMakespanSolution()
	\end{lstlisting}

	\section*{Aufgabe 3: Rechenstudie}
	Sammeln der Daten
	\begin{lstlisting}
from Solver import *
import timeit

data = InputData("VFR20_10_1_SIST.json") 

rows = []
for neighboorhood in ["None", "Insertion", "TaillardInsertion"]:
	for numberJobsToRemove in [2, 3, 4]:
		for baseTemperature in [0.5, 1]:
			for maxIterations in [1, 10]:
				for iteration in range(3):
					seed = numberJobsToRemove * maxIterations * iteration
					solver = Solver(data, seed)
					if neighboorhood == "None":
						localSearch = None
					else:
						localSearch = IterativeImprovement(data, 'BestImprovement', [neighboorhood])
					iteratedGreedy = IteratedGreedy(
						inputData = data, 
						numberJobsToRemove = numberJobsToRemove, 
						baseTemperature = baseTemperature, 
						maxIterations = maxIterations,
						maxIterationsWithoutImprovement = 100,
						localSearchAlgorithm = localSearch)

					start = timeit.default_timer()
					solver.RunLocalSearch(
						constructiveSolutionMethod='NEH',
						algorithm=iteratedGreedy)
					ende = timeit.default_timer()

					# build dict
					row = {}
					row["Iteration"] = iteration
					row["LocalSearch"] = neighboorhood
					row["NumberOfJobsToRemove"] = numberJobsToRemove
					row["BaseTemperature"] = baseTemperature
					row["MaxIterations"] = maxIterations
					row["Seed"] = seed
					row["Makespan"] = solver.SolutionPool.GetLowestMakespanSolution().Makespan
					row["Runtime"] = ende - start
					rows.append(row)
	\end{lstlisting}
	Verarbeitung mittels Pandas
	\begin{lstlisting}
import pandas as pd
df = pd.DataFrame(rows)
df
df.groupby(["LocalSearch", "NumberOfJobsToRemove", "BaseTemperature", "MaxIterations"]).mean()
	\end{lstlisting}

\end{document}