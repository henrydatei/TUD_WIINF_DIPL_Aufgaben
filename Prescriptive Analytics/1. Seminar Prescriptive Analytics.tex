\documentclass{article}

\usepackage{amsmath,amssymb}
\usepackage{tikz}
\usepackage{pgfplots}
\usepackage{xcolor}
\usepackage[left=2.1cm,right=3.1cm,bottom=3cm,footskip=0.75cm,headsep=0.5cm]{geometry}
\usepackage{enumerate}
\usepackage{enumitem}
\usepackage{marvosym}
\usepackage{tabularx}
\usepackage{parskip}

\usepackage{listings}
\definecolor{lightlightgray}{rgb}{0.95,0.95,0.95}
\definecolor{lila}{rgb}{0.8,0,0.8}
\definecolor{mygray}{rgb}{0.5,0.5,0.5}
\definecolor{mygreen}{rgb}{0,0.8,0.26}
%\lstdefinestyle{java} {language=java}
\lstset{language=python,
	basicstyle=\ttfamily,
	keywordstyle=\color{lila},
	commentstyle=\color{lightgray},
	stringstyle=\color{mygreen}\ttfamily,
	backgroundcolor=\color{white},
	morekeywords={split, as, dot, read\_csv, loc, between, groupby, sort\_values},
	showstringspaces=false,
	numbers=left,
	numbersep=10pt,
	tabsize=2,
	numberstyle=\color{mygray}\ttfamily,
	identifierstyle=\color{blue},
	xleftmargin=.1\textwidth, 
	%xrightmargin=.1\textwidth,
	escapechar=§,
	%literate={\t}{{\ }}1
	breaklines=true,
	postbreak=\mbox{\space},
	literate=%
	{Ö}{{\"O}}1
	{Ä}{{\"A}}1
	{Ü}{{\"U}}1
	{ß}{{\ss}}1
	{ü}{{\"u}}1
	{ä}{{\"a}}1
	{ö}{{\"o}}1
}

\usepackage[colorlinks = true, linkcolor = blue, urlcolor  = blue, citecolor = blue, anchorcolor = blue]{hyperref}
\usepackage[utf8]{inputenc}

\renewcommand*{\arraystretch}{1.4}

\newcolumntype{L}[1]{>{\raggedright\arraybackslash}p{#1}}
\newcolumntype{R}[1]{>{\raggedleft\arraybackslash}p{#1}}
\newcolumntype{C}[1]{>{\centering\let\newline\\\arraybackslash\hspace{0pt}}m{#1}}

\newcommand{\E}{\mathbb{E}}
\DeclareMathOperator{\rk}{rk}
\DeclareMathOperator{\Var}{Var}
\DeclareMathOperator{\Cov}{Cov}

\title{\textbf{Prescriptive Analytics, Seminar 1}}
\author{\textsc{Henry Haustein}}
\date{}

\begin{document}
	\maketitle
	
	\section*{Aufgabe 1: Podcast-Sendezeit}
	\begin{enumerate}[label=(\alph*)]
		\item Lösung mittels f-String (\url{https://zetcode.com/python/fstring/}):
		\begin{lstlisting}
sendezeit = 70
result = 2 * (70 - 3*3)
print(f'Bei einer Podcastlänge von {sendezeit} Minuten ergeben sich {result/60:.2f} Stunden Sendezeit pro Woche')
		\end{lstlisting}
		\item mittel \texttt{.format()}
		\begin{lstlisting}
print("Bei einer Podcastlänge von {} Minuten ergeben sich {} Stunden Sendezeit pro Woche".format(sendezeit, round(result/60,2)))
		\end{lstlisting}
		
	\end{enumerate}

	\section*{Aufgabe 2: Chaos in der Mailingliste}
	\begin{enumerate}[label=(\alph*)]
		\item mittels List Comprehension (\url{https://www.w3schools.com/python/python_lists_comprehension.asp})
		\begin{lstlisting}
mail = "service@klara_Klarna.com"
[name.split(".")[0] for name in mail.split("@")[1].split("_")]
		\end{lstlisting}
		\item mittels \texttt{zip()}
		\begin{lstlisting}
namen = ["Klara", "Sebastian", "Britta", "Klaus"]
mails = ["service@klara_Klarna.com", "Sebastian_Duesentrieb@gmail.com", "bwiebritta@yahoo.com", "Klaus@peter.de"]
result = {}
for name, mail in zip(namen, mails):
	result[name] = mail
result
		\end{lstlisting}
	\end{enumerate}
	
	\section*{Aufgabe 3: Anpassung der Werbeeinnahmen}
	\begin{enumerate}[label=(\alph*)]
		\item mittels Definition einer Funktion
		\begin{lstlisting}
input_price = 333
def preisanpassung(oldPrice):
	if oldPrice <= 400:
		return oldPrice * 1.1
	if 400 < oldPrice and oldPrice <= 600:
		return oldPrice * 1.2
	if 600 < oldPrice:
		return oldPrice * 1.25

preisanpassung(333)
		\end{lstlisting}
		\item wieder mittels List Comprehension
		\begin{lstlisting}
oldPrices = [350, 300, 600, 800]
[preisanpassung(oldPrice) for oldPrice in oldPrices]
		\end{lstlisting}
		\item mittels \texttt{numpy.dot()} welches unabhängig von der Länge der Inputvektoren ist. Die Musterlösung des Lehrstuhls wird falsch, wenn es nicht mehr 4, sondern 5 verschiedene Werbepakete gibt
		\begin{lstlisting}
import numpy as np

sales = [3, 5, 7, 4]
oldPrice = [350, 300, 600, 800]
newPrice = [385, 330, 720, 1000]

def umsatz(newprices, oldprices, sales):
	diff = [new - old for new, old in zip(newprices, oldprices)]
	return np.dot(diff, sales)

umsatz(newPrice, oldPrice, sales)
		\end{lstlisting}
		\item mittels \texttt{lambda}-Funktion
		\begin{lstlisting}
umsatz = lambda preis, vkz: sum([preis[i]*vkz[i] for i in range(len(sales))])

umsatz(sales, newPrice) - umsatz(sales, oldPrice)
		\end{lstlisting}
	\end{enumerate}
	
	\section*{Aufgabe 4: Expansionswahnsinn}
	\begin{enumerate}[label=(\alph*)]
		\item mittels Pandas
		\begin{lstlisting}
import pandas as pd
births = pd.read_csv("births.csv")
births["age"] = 2022 - births["Year"]
zielguppe = births.loc[births["age"].between(25, 35)]
zielguppe["Count"].sum()
		\end{lstlisting}
		\item mit Pandas geht das besser als mit einem Dictionary
		\begin{lstlisting}
zielguppe.groupby(["age"]).sum().sort_values("Count", ascending = False)
		\end{lstlisting}
	\end{enumerate}

\end{document}