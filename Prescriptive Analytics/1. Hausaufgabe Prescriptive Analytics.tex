\documentclass{article}

\usepackage{amsmath,amssymb}
\usepackage{tikz}
\usepackage{pgfplots}
\usepackage{xcolor}
\usepackage[left=2.1cm,right=3.1cm,bottom=3cm,footskip=0.75cm,headsep=0.5cm]{geometry}
\usepackage{enumerate}
\usepackage{enumitem}
\usepackage{marvosym}
\usepackage{tabularx}
\usepackage{parskip}

\usepackage{listings}
\definecolor{lightlightgray}{rgb}{0.95,0.95,0.95}
\definecolor{lila}{rgb}{0.8,0,0.8}
\definecolor{mygray}{rgb}{0.5,0.5,0.5}
\definecolor{mygreen}{rgb}{0,0.8,0.26}
%\lstdefinestyle{java} {language=java}
\lstset{language=python,
	basicstyle=\ttfamily,
	keywordstyle=\color{lila},
	commentstyle=\color{lightgray},
	stringstyle=\color{mygreen}\ttfamily,
	backgroundcolor=\color{white},
	morekeywords={split, as, dot, read\_csv, loc, between, groupby, sort\_values},
	showstringspaces=false,
	numbers=left,
	numbersep=10pt,
	tabsize=2,
	numberstyle=\color{mygray}\ttfamily,
	identifierstyle=\color{blue},
	xleftmargin=.1\textwidth, 
	%xrightmargin=.1\textwidth,
	escapechar=§,
	%literate={\t}{{\ }}1
	breaklines=true,
	postbreak=\mbox{\space},
	literate=%
	{Ö}{{\"O}}1
	{Ä}{{\"A}}1
	{Ü}{{\"U}}1
	{ß}{{\ss}}1
	{ü}{{\"u}}1
	{ä}{{\"a}}1
	{ö}{{\"o}}1
}

\usepackage[colorlinks = true, linkcolor = blue, urlcolor  = blue, citecolor = blue, anchorcolor = blue]{hyperref}
\usepackage[utf8]{inputenc}

\renewcommand*{\arraystretch}{1.4}

\newcolumntype{L}[1]{>{\raggedright\arraybackslash}p{#1}}
\newcolumntype{R}[1]{>{\raggedleft\arraybackslash}p{#1}}
\newcolumntype{C}[1]{>{\centering\let\newline\\\arraybackslash\hspace{0pt}}m{#1}}

\newcommand{\E}{\mathbb{E}}
\DeclareMathOperator{\rk}{rk}
\DeclareMathOperator{\Var}{Var}
\DeclareMathOperator{\Cov}{Cov}

\title{\textbf{Prescriptive Analytics, Hausaufgabe 1}}
\author{\textsc{Henry Haustein}}
\date{}

\begin{document}
	\maketitle
	
	\section*{Aufgabe 1: Listen \& Sortierungen}
	\begin{enumerate}[label=(\alph*)]
		\item mittels List Comprehension
		\begin{lstlisting}
### Ergänzen Sie hier Ihren Code, um die gewünschte Ausgabe zu erzielen. ###
thisList2 = [element[::-1] for element in thisList]

### Führen Sie anschließend nachfolgenden Programmcode aus, um Ihre Lösung zu überprüfen ###
print(thisList2)
		\end{lstlisting}
		\item noch mehr List Comprehension
		\begin{lstlisting}
### Ergänzen Sie hier Ihren Code, um die gewünschte Ausgabe zu erzielen. ###
thisList3 = [element for element in thisList if type(element) == str]

### Führen Sie anschließend nachfolgenden Programmcode aus, um Ihre Lösung zu überprüfen ###
print(thisList3)
		\end{lstlisting}
		\item mittels \texttt{lambda}-Funktion
		\begin{lstlisting}
import random
evaluations = [('Otto', 3.779, 1.238, 0.49), ('Level', 3.961, 5.725, 0.233), ('Leseesel', 3.935, 1.482, 2.41), ('Kajak', 1.989, 2.66, 0.656), ('Mueller', 2.024, 1.427, 3.013), ('Retter', 3.297, 2.36, 3.179), ('Rotor', 1.733, 4.218, 4.972), ('Effe', 3.311, 3.197, 3.991), ('Meier', 5.956, 2.554, 4.622)]

### Bitte ergänzen Sie hier Ihren Code ###
sortiert = sorted(evaluations, key = lambda x: x[1] + x[2] + x[3], reverse = True)
print(sortiert[0])
		\end{lstlisting}
		\item für sowas gibt auch Pakete... aber die sollen wir nicht nutzen
		\begin{lstlisting}
thisList3Corrupted = ['Otto', ['Level', 'Leseesel', 'Kajak'], ['Mueller'], 'Retter', 'Rotor', ['Effe', 'Meier']]

### Bitte ergänzen Sie hier Ihren Code ###
def flatten(x):
	liste = []
	for element in x:
		if type(element) is list:
			liste.extend(flatten(element))
		else:
			liste.append(element)
	return liste

### Führen Sie anschließend nachfolgenden Programmcode aus, um Ihre Lösung zu überprüfen ###
thisList3Repaired = flatten(thisList3Corrupted)
print(thisList3Repaired)
		\end{lstlisting}
	\end{enumerate}

	\section*{Aufgabe 2: Einfache Funktionen}
	\begin{enumerate}[label=(\alph*)]
		\item Wir benutzen wieder den Index-Operator \texttt{[::-1]}. Eigentlich schreibt man den ersten Buchstaben einer Funktion klein (lowerCamelCase) und den ersten Buchstaben einer Klasse groß (UpperCamelCase).
		\begin{lstlisting}
thisList = ['Otto', 'Level', 'Leseesel', [1, 2, 3], 'Kajak', 'Mueller', 'Retter', 'Rotor', [5, 6, 7], 'Effe', 'Meier']

### Bitte ergänzen Sie hier Ihren Programmcode ###
def CheckForPalindroms(x):
	liste = []
	for idx, value in enumerate(x):
		if type(value) is str:
			if value.lower() == value[::-1].lower():
				liste.append(idx)
	return liste

### Führen Sie anschließend nachfolgenden Programmcode aus, um Ihre Lösung zu überprüfen ###
CheckForPalindroms(thisList)
		\end{lstlisting}
		\item Dass der folgende Quelltext beim Ausführen einen Fehler gibt, ist beabsichtigt und soll auch so sein:
		\begin{lstlisting}
### Bitte ergänzen Sie hier Ihren Programmcode ###
def splitNumber(x):
	try:
		ganzeZahl = int(x)
		rest = x - ganzeZahl
		return ganzeZahl, rest
	except:
		print("Bitte nur Zahlen eingeben")
		raise TypeError("Input has wrong type!")

### Führen Sie anschließend nachfolgenden Programmcode aus, um Ihre Lösung zu überprüfen ###
print(splitNumber(3.56))
print(splitNumber("Hallo"))
		\end{lstlisting}
	\end{enumerate}

	\section*{Aufgabe 3: Klassendefinition \& weitere Funktionen}
	\begin{enumerate}[label=(\alph*)]
		\item Klasse \texttt{Rechteck}
		\begin{lstlisting}
### Bitte ergänzen Sie hier Ihren Programmcode ###
import json

class Rechteck:
	id = None
	width = None
	height = None
	area = None

	def __init__(self, id, w, h):
		self.id = id
		self.width = w
		self.height = h
		self.area = self.DetermineArea()

	def DetermineArea(self):
		return self.height * self.width

	def __str__(self):
		return f"Rechteck {self.id} mit Breite {self.width}, Höhe {self.height}, Fläche {self.area}"

	def Load(filename):
		liste = []
		with open(filename) as file:
			data = json.load(file)
			for id in data.keys():
				r = Rechteck(id, data[id]["width"], data[id]["height"])
				liste.append(r)
		return liste

### Führen Sie anschließend nachfolgenden Programmcode aus, um Ihre Lösung zu überprüfen ###

rectangulars = Load('Rectangulars.json')
for rec in rectangulars:
	print(rec)

		\end{lstlisting}
		\item Cäsar-Chiffre
		\begin{lstlisting}
verschluesselteNachricht = "Ylmny Buomuozauvy aymwbuzzn! Scjjcy :)"
schluessel = 20

### Bitte ergänzen Sie hier Ihren Programmcode ###
def verschieben(x, anzahl):
	if (x.isupper()):
		return chr((ord(x) + anzahl - 65) % 26 + 65)
	else:
		return chr((ord(x) + anzahl - 97) % 26 + 97)

def Decrypt(x, anzahl):
	result = ""
	for letter in x:
		if letter not in [" ", ":", ")", "!"]:
			result += verschieben(letter, -anzahl)
		else:
			result += letter
	return result

### Führen Sie anschließend nachfolgenden Programmcode aus, um Ihre Lösung zu überprüfen ###
Decrypt(verschluesselteNachricht, schluessel)
		\end{lstlisting}
	\end{enumerate}

\end{document}