\documentclass{article}

\usepackage{amsmath,amssymb}
\usepackage{tikz}
\usepackage{pgfplots}
\usepackage{xcolor}
\usepackage[left=2.1cm,right=3.1cm,bottom=3cm,footskip=0.75cm,headsep=0.5cm]{geometry}
\usepackage{enumerate}
\usepackage{enumitem}
\usepackage{marvosym}
\usepackage{tabularx}
\usepackage{parskip}

\usepackage{listings}
\definecolor{lightlightgray}{rgb}{0.95,0.95,0.95}
\definecolor{lila}{rgb}{0.8,0,0.8}
\definecolor{mygray}{rgb}{0.5,0.5,0.5}
\definecolor{mygreen}{rgb}{0,0.8,0.26}
%\lstdefinestyle{java} {language=java}
\lstset{language=python,
	basicstyle=\ttfamily,
	keywordstyle=\color{lila},
	commentstyle=\color{lightgray},
	stringstyle=\color{mygreen}\ttfamily,
	backgroundcolor=\color{white},
	morekeywords={split, as, dot, read\_csv, loc, between, groupby, sort\_values},
	showstringspaces=false,
	numbers=left,
	numbersep=10pt,
	tabsize=2,
	numberstyle=\color{mygray}\ttfamily,
	identifierstyle=\color{blue},
	xleftmargin=.1\textwidth, 
	%xrightmargin=.1\textwidth,
	escapechar=§,
	%literate={\t}{{\ }}1
	breaklines=true,
	postbreak=\mbox{\space},
	literate=%
	{Ö}{{\"O}}1
	{Ä}{{\"A}}1
	{Ü}{{\"U}}1
	{ß}{{\ss}}1
	{ü}{{\"u}}1
	{ä}{{\"a}}1
	{ö}{{\"o}}1
}

\usepackage[colorlinks = true, linkcolor = blue, urlcolor  = blue, citecolor = blue, anchorcolor = blue]{hyperref}
\usepackage[utf8]{inputenc}

\renewcommand*{\arraystretch}{1.4}

\newcolumntype{L}[1]{>{\raggedright\arraybackslash}p{#1}}
\newcolumntype{R}[1]{>{\raggedleft\arraybackslash}p{#1}}
\newcolumntype{C}[1]{>{\centering\let\newline\\\arraybackslash\hspace{0pt}}m{#1}}

\newcommand{\E}{\mathbb{E}}
\DeclareMathOperator{\rk}{rk}
\DeclareMathOperator{\Var}{Var}
\DeclareMathOperator{\Cov}{Cov}

\title{\textbf{Prescriptive Analytics, Seminar 2}}
\author{\textsc{Henry Haustein}}
\date{}

\begin{document}
	\maketitle
	
	\section*{Aufgabe 1: Podcast-Management}
	\begin{enumerate}[label=(\alph*)]
		\item Klasse \texttt{Podcast} gleich mit privaten Attributen:
		\begin{lstlisting}
class Podcast():
	# nicht nötig, aber finde ich besser lesbar, insb. bei größeren Klassen
	__episode = None
	__length = None
	__moderator = None
	__adverts = None
	__title = None

	def __init__(self, e, t, l, m):
		self.__episode =int(e)
		self.__length = int(l)
		self.__moderator = str(m)
		self.__title = str(t)
		self.__adverts = 0

	def display(self):
		print("Podcast Folge Nr.: {} mit ModeratorIn: {}".format(self.__episode, self.__moderator))
		print("Thema der Podcastfolge: {}".format(self.__title))
		print("Länge der Podcastfolge: {}".format(self.__length))
		print("Aktuell verkaufte Werbeblöcke: {}".format(self.__adverts))

	def setAdverts(self, werbebloecke):
		self.__length = self.__length + 3*werbebloecke
		self.__adverts = self.__adverts + werbebloecke

	def getAdverts(self):
		print(self.__adverts)

	def cut(self, minuten):
		if self.__length - minuten <= 30:
			print("Keine Kürzung möglich")
		else:
			self.__length = self.__length - minuten

newpodcast = Podcast(23, "Traveling Salesman Problems", 70, "O. Peratio")
newpodcast.display()
		\end{lstlisting}
		\item siehe (a)
		\item Klasse \texttt{SpecialPodcast}
		\begin{lstlisting}
class SpecialPodcast(Podcast):
	__specialGuest = None

	def __init__(self, e, t, l, m, s):
		# Alternative 1
		self.__episode = int(e)
		self.__length = int(l)
		self.__moderator = str(m)
		self.__title = str(t)
		self.__adverts = 0
		# Alternative 2 (funktioniert nicht, weil Attribute von Podcast privat sind)
		# Podcast(e, t, l, m)
		self.__specialGuest = s

	def changeGuest(self, newGuest):
		self.__specialGuest = newGuest

	def display(self):
		print("Podcast Folge Nr.: {} mit ModeratorIn: {}".format(self.__episode, self.__moderator))
		print("Thema der Podcastfolge: {}".format(self.__title))
		print("Länge der Podcastfolge: {}".format(self.__length))
		print("Aktuell verkaufte Werbeblöcke: {}".format(self.__adverts))
		print("SpecialGuest der Podcastfolge ist: {}".format(self.__specialGuest))

sp = SpecialPodcast(1, "Vehicle Routing Problem", 65, "M. Ipler", "Dr. Best")
sp.changeGuest("Dr. Secondbest")
sp.display()
		\end{lstlisting}
	\end{enumerate}

	\section*{Aufgabe 2: Einführung des Planungsproblems}
	\begin{enumerate}[label=(\alph*)]
		\item mittels \texttt{with} (\url{https://preshing.com/20110920/the-python-with-statement-by-example/})
		\begin{lstlisting}
import json

with open('InputFlowshopSIST.json') as json_file:
	data = json.load(json_file)
	print(data)
		\end{lstlisting}
		Das \texttt{with}-Statement wird gerne bei der Arbeit mit Dateien genommen, da man nach dem Öffnen einer Datei diese auch wieder schließen müsste. Aber das wird häufig vergessen und \texttt{with} macht das automatisch im Hintergrund. Man hätte also auch schreiben können:
		\begin{lstlisting}
import json
			
json_file =  open('InputFlowshopSIST.json')
data = json.load(json_file)
print(data)
json_file.close()
		\end{lstlisting}
		\item Ich habe mich hier für die Klassen \texttt{Job}, \texttt{Maschine} und \texttt{Flowshop} entschieden:
		\begin{lstlisting}
class Job:
	id = None
	setupTimes = None
	processingTimes = None
	dueDate = None
	tardCosts = None

	def __init__(self, id, setup, processing, due, costs):
		self.id = int(id)
		self.setupTimes = setup
		self.processingTimes = processing
		self.dueDate = due
		self.tardCosts = costs

	def __str__(self):
		return f"Job({self.id = }, {self.setupTimes = }, {self.processingTimes = }, {self.dueDate = }, {self.tardCosts = })"

class Maschine:
	id = None

	def __init__(self, id):
		self.id = id

	def __str__(self):
		return f"Maschine({self.id = })"

class Flowshop:
	name = None
	nMaschines = None
	nJobs = None
	jobList = []

	def __init__(self, name, anzahlMaschinen, anzahlJobsProAuftrag, jobs):
		self.name = name
		self.nMaschines = int(anzahlMaschinen)
		self.nJobs = int(anzahlJobsProAuftrag)
		self.jobList = jobs

	def __str__(self):
		return f"Flowshop({self.name = }, {self.nMaschines = }, {self.nJobs = }, {self.jobList = })"
		\end{lstlisting}
		\item Warum ich jetzt hier eine Klasse erstellen soll, weiß ich nicht, eine Funktion hätte es auch getan
		\begin{lstlisting}
class InputData:
	flowshop = None
	filename = None

	def __init__(self, filename):
		self.filename = filename
		with open(filename) as file:
			data = json.load(file)
			liste = []
			for job in data["Jobs"]:
				j = Job(job["Id"], job["SetupTimes"], job["ProcessingTimes"], job["DueDate"], job["TardCosts"])
				liste.append(j)
			self.flowshop = Flowshop(data["Name"], data["nMachines"], data["nJobs"], liste)

input = InputData("InputFlowshopSIST.json")
		\end{lstlisting}
		\item Diese Klasse ist noch unnötiger...
		\begin{lstlisting}
class OutputJob:
	job = None

	def __init__(self, job):
		self.job = job

	def display(self):
		print(self.job)

OutputJob(input.flowshop.jobList[0]).display()
		\end{lstlisting}
	\end{enumerate}

\end{document}