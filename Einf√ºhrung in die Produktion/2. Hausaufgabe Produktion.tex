\documentclass{article}

\usepackage{amsmath,amssymb}
\usepackage{tikz}
\usepackage{pgfplots}
\usepackage{xcolor}
\usepackage[left=2.1cm,right=3.1cm,bottom=3cm,footskip=0.75cm,headsep=0.5cm]{geometry}
\usepackage{enumerate}
\usepackage{enumitem}
\usepackage{marvosym}
\usepackage{tabularx}
\usepackage[amsmath,thmmarks,standard]{ntheorem}
\usepackage{mathtools}

\usepackage[utf8]{inputenc}

\renewcommand*{\arraystretch}{1.4}
\newcommand{\E}{\mathbb{E}}

\newcolumntype{L}[1]{>{\raggedright\arraybackslash}p{#1}}
\newcolumntype{R}[1]{>{\raggedleft\arraybackslash}p{#1}}
\newcolumntype{C}[1]{>{\centering\let\newline\\\arraybackslash\hspace{0pt}}m{#1}}

\DeclareMathOperator{\tr}{tr}
\DeclareMathOperator{\Var}{Var}
\DeclareMathOperator{\Cov}{Cov}
\renewcommand{\E}{\mathbb{E}}

\newtheorem{thm}{Theorem}
\newtheorem{lem}{Lemma}

\title{\textbf{Einführung in die Produktion, Hausaufgabe 2}}
\author{\textsc{Henry Haustein}}
\date{}

\begin{document}
	\maketitle
	
	\section*{Aufgabe 2}
	\begin{enumerate}[label=(\alph*)]
		\item Um bei gegebenen $r_B$ das $r_K$ zu berechnen, müssen wir folgende Gleichung lösen:
		\begin{align}
			1 &= 0.0429r_B - 0.0006r_B^2 + (0.0003r_B^2 + 0.0357)r_K \notag \\
			r_K &= \frac{-0.0429r_B + 0.0006r_B^2-1}{0.0003r_B^2 + 0.0357} \notag
		\end{align}
		Es ergibt sich
		\begin{center}
			\begin{tabular}{c|c}
				$r_B$ & $r_K$ \\
				\hline
				0 & 28.011 \\
				10 & 9.6043 \\
				20 & 2.4534 \\
				30 & 0.8276 \\
				40 & 0.4731 \\
				50 & 0.4518
			\end{tabular}
		\end{center}
		\begin{center}
			\begin{tikzpicture}
				\begin{axis}[
					xmin=0, xmax=50, xlabel=$r_B$,
					ymin=0, ymax=30, ylabel=$r_K$,
					samples=400,
					axis x line=middle,
					axis y line=middle,
					domain=0:50,
					]
					\addplot[mark=x,blue] coordinates {
						(0,28.011)
						(10,9.6043)
						(20,2.4534)
						(30,0.8276)
						(40,0.4731)
						(50,0.4518)
					};
				\end{axis}
			\end{tikzpicture}
		\end{center}
		\item Die Kostenfunktion ist $K=2r_K + 2.7r_B$. Daraus ergibt sich die Lagrangefunktion als $L=2r_K + 2.7r_B + \lambda(0.0429r_B - 0.0006r_B^2 + 0.0003r_B^2r_K + 0.0357 r_K - 1)$. Die Optimalitätsbedingungen sind dann
		\begin{align}
			\frac{\partial L}{\partial r_K} &= 2 + 0.0003r_B^2\lambda + 0.0357\lambda = 0 \\
			\frac{\partial L}{\partial r_B} &= 2.7 + 0.0429\lambda + 0.0012 r_B\lambda + 0.0006r_Br_K\lambda =0  \\
			\frac{\partial L}{\partial \lambda} &= 0.0429r_B - 0.0006r_B^2 + 0.0003r_B^2r_K + 0.0357 r_K - 1 = 0
		\end{align}
		Damit eine optimale Produktion an den zu prüfenden Punkten möglich ist, müssen die Gleichungen (1) bis (3) an der zu prüfenden Stelle genau 0 sein. Die entsprechenden $\lambda$'s in (1) und (2) müssen gleich sein. Für den Punkt $(8.4271,11.9479)$ ergibt sich
		\begin{align}
			\frac{\partial L}{\partial r_K} &=0 \Rightarrow \lambda = -35.0848 \notag \\
			\frac{\partial L}{\partial r_B} &=0 \Rightarrow \lambda = -28.9702 \notag
		\end{align}
		Wir sparen uns die dritte Bedingung zu überprüfen, denn der Punkt $(8.4271,11.9479)$ kann also kein Maximum sein. Die Kosten hier sind $K=46.6490$. Für den Punkt $(9.9854,9.6239)$ ergibt sich
		\begin{align}
			\frac{\partial L}{\partial r_K} &=0 \Rightarrow \lambda = -30.4820 \notag \\
			\frac{\partial L}{\partial r_B} &=0 \Rightarrow \lambda = -30.4821 \notag \\
			\frac{\partial L}{\partial \lambda} = 0 \notag
		\end{align}
		Der Punkt $(9.9854,9.6239)$ ist ein Optimum. Die Kosten liegen hier bei $K=46.2084$.
	\end{enumerate}
	
\end{document}