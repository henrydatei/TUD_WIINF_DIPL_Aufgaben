\documentclass{article}

\usepackage{amsmath,amssymb}
\usepackage{tikz}
\usetikzlibrary{er,positioning}
\usepackage{pgfplots}
\usepackage{xcolor}
\usepackage[left=2.1cm,right=3.1cm,bottom=3cm,footskip=0.75cm,headsep=0.5cm]{geometry}
\usepackage{enumerate}
\usepackage{enumitem}
\usepackage{marvosym}
\usepackage{tabularx}

\usepackage{listings}
\definecolor{lightlightgray}{rgb}{0.95,0.95,0.95}
\definecolor{lila}{rgb}{0.8,0,0.8}
\definecolor{mygray}{rgb}{0.5,0.5,0.5}
\definecolor{mygreen}{rgb}{0,0.8,0.26}
\lstdefinestyle{sql} {language=sql}
\lstset{language=sql,
	basicstyle=\ttfamily,
	keywordstyle=\color{lila},
	commentstyle=\color{lightgray},
	stringstyle=\color{mygreen}\ttfamily,
	backgroundcolor=\color{white},
	showstringspaces=false,
	numbers=left,
	numbersep=10pt,
	numberstyle=\color{mygray}\ttfamily,
	identifierstyle=\color{blue},
	xleftmargin=.1\textwidth, 
	%xrightmargin=.1\textwidth,
	escapechar=§,
}

\usepackage[utf8]{inputenc}

\renewcommand*{\arraystretch}{1.4}

\newcolumntype{L}[1]{>{\raggedright\arraybackslash}p{#1}}
\newcolumntype{R}[1]{>{\raggedleft\arraybackslash}p{#1}}
\newcolumntype{C}[1]{>{\centering\let\newline\\\arraybackslash\hspace{0pt}}m{#1}}

\newcommand{\E}{\mathbb{E}}
\DeclareMathOperator{\rk}{rk}
\DeclareMathOperator{\Var}{Var}
\DeclareMathOperator{\Cov}{Cov}
\DeclareMathOperator{\Hul}{Hul}

\def\ojoin{\setbox0=\hbox{$\bowtie$}%
	\rule[-.02ex]{.25em}{.4pt}\llap{\rule[\ht0]{.25em}{.4pt}}}
\def\leftouterjoin{\mathbin{\ojoin\mkern-5.8mu\bowtie}}
\def\rightouterjoin{\mathbin{\bowtie\mkern-5.8mu\ojoin}}
\def\fullouterjoin{\mathbin{\ojoin\mkern-5.8mu\bowtie\mkern-5.8mu\ojoin}}

\title{\textbf{Datenbanken, SQL-Praktikum}}
\author{\textsc{Henry Haustein}}
\date{}

\begin{document}
	\maketitle
	
	\section*{Aufgabe 1}
	Gesucht sind Name und Einwohnerzahl aller Städte in absteigender Reihenfolge ihrer Einwohnerzahl. [3051]
	\begin{lstlisting}[style=sql,tabsize=2]
SELECT name, population
FROM City
ORDER BY population DESC
	\end{lstlisting}

	\section*{Aufgabe 2}
	Ermitteln Sie alle Städte (Name), die sowohl an einem Fluss als auch an einem See liegen. [2]
	\begin{lstlisting}[style=sql,tabsize=2]
SELECT city
FROM located
WHERE river IS NOT NULL AND lake IS NOT NULL
	\end{lstlisting}

	\section*{Aufgabe 3}
	Gesucht sind Name und Einwohnerzahl alle deutschen Städte in absteigender Reihenfolge ihrer Einwohnerzahl. [86]
	\begin{lstlisting}[style=sql,tabsize=2]
SELECT name, population
FROM city
WHERE country = 'D'
ORDER BY population DESC
	\end{lstlisting}
	
	\section*{Aufgabe 4}
	 Ermitteln Sie alle Sprachen (Name), die in Ländern Europas gesprochen werden. [24]
	\begin{lstlisting}[style=sql,tabsize=2]
SELECT DISTINCT l.name
FROM language l 
INNER JOIN encompasses e ON l.country = e.country 
WHERE e.continent = 'Europe'
	\end{lstlisting}
	
	\section*{Aufgabe 5}
	Ermitteln Sie alle Sprachen (Name), die in Ländern der EU gesprochen werden, und in wie vielen Ländern (der EU) sie gesprochen werden, absteigend geordnet nach Anzahl der Länder. [11]
	\begin{lstlisting}[style=sql,tabsize=2]
SELECT l.name, count(m.country)
FROM language AS l 
INNER JOIN is_member AS m ON l.country = m.country 
WHERE m.organization = 'EU' AND m.type = 'member' 
GROUP BY l.name 
ORDER BY 2 DESC
	\end{lstlisting}
	
	\section*{Aufgabe 6}
	Gesucht sind Name, Einwohnerzahl und Name des Landes aller Hauptstädte, in denen mehr als 30\% der Bevölkerung des Landes leben, in absteigender Reihenfolge ihrer Einwohnerzahl. [17]
	\begin{lstlisting}[style=sql,tabsize=2]
SELECT s.name, s.population, c.name
FROM city AS s 
INNER JOIN country c ON s.name = c.capital AND s.country = c.code 
WHERE s.population > 0.3 * c.population
ORDER BY s.population DESC
	\end{lstlisting}
	
	\section*{Aufgabe 7}
	Bestimmen Sie alle Länder (Name) mit mindestens einem über 4.000 m hohen Berg. [21]
	\begin{lstlisting}[style=sql,tabsize=2]
SELECT DISTINCT c.name
FROM country c 
INNER JOIN geo_mountain gm ON c.code = gm.country 
INNER JOIN mountain m ON gm.mountain = m.name
WHERE m.height > 4000
	\end{lstlisting}
	
	\section*{Aufgabe 8}
	Gesucht sind alle Länder (Name) mit mindestens einer Stadt, welche mehr Einwohner hat als die Hauptstadt des Landes. [18]
	\begin{lstlisting}[style=sql,tabsize=2]
SELECT DISTINCT c.name
FROM country c 
INNER JOIN city s ON c.code = s.country 
INNER JOIN city hs ON c.capital = hs.name AND c.code = hs.country
WHERE s.population > hs.population
	\end{lstlisting}
	
	\section*{Aufgabe 9}
	Ermitteln Sie die Namen aller Länder mit Millionenstädten sowie der jeweiligen Gesamtpopulation aller Millionenstädte eines Landes in Reihenfolge dieser Gesamtpopulation. [68]
	\begin{lstlisting}[style=sql,tabsize=2]
SELECT c.name, SUM(s.population)
FROM city s 
INNER JOIN country c ON s.country = c.code 
WHERE s.population > 1000000
GROUP BY c.name
ORDER BY 2
	\end{lstlisting}
	
	\section*{Aufgabe 10}
	Gesucht sind alle Länder (Name), die an Deutschland grenzen, sowie deren gesamte Grenzlänge. [9]
	\begin{lstlisting}[style=sql,tabsize=2]
SELECT c.name, sum(b2.length) AS laenge
FROM country c 
INNER JOIN borders b1 ON c.code = b1.country1 OR c.code = b1.country2 
INNER JOIN borders b2 ON c.code = b2.country1 OR c.code = b2.country2
WHERE c.code != 'D' AND (b1.country1 = 'D' OR b1.country2 = 'D')
GROUP BY c.name	
	\end{lstlisting}
	
	\section*{Aufgabe 11}
	Ermitteln Sie die Namen aller Länder, deren Anteil an Einwohnern, welche in den Millionenstädten des Landes leben, größer als 30\% ist, aufsteigend geordnet nach diesem Anteil. [6]
	\begin{lstlisting}[style=sql,tabsize=2]
SELECT c.name, sum(s.population)*100/c.population AS anteil 
FROM city s 
INNER JOIN country c ON s.country = c.code 
WHERE s.population > 1000000 
GROUP BY c.name
HAVING sum(s.population) > 0.3 * c.population
ORDER BY anteil
	\end{lstlisting}
	
	\section*{Aufgabe 12}
	Gesucht sind alle Länder (Name), die mehr Seen als Berge haben, in aufsteigender Reihenfolge der Anzahl der Seen. Es sollen nur Länder berücksichtigt werden, in denen es sowohl Seen als auch Berge gibt. [12]
	\begin{lstlisting}[style=sql,tabsize=2]
SELECT c.name, count(DISTINCT gl.lake) AS "anzahl see", 
count(DISTINCT gm.mountain) AS "anzahl berge"
FROM country c 
INNER JOIN geo_lake gl ON c.code = gl.country 
INNER JOIN geo_mountain gm ON c.code = gm.country
GROUP BY c.name
HAVING count(DISTINCT gl.lake) > count(DISTINCT gm.mountain) 
ORDER BY "anzahl see"
	\end{lstlisting}
	
	\section*{Aufgabe 13}
	Bestimmen Sie alle Länder (Name), die mehr Nachbarstaaten als Deutschland haben. [3]
	\begin{lstlisting}[style=sql,tabsize=2]
SELECT c.name, count(*) AS "anzahl nachbarn"
FROM country c 
INNER JOIN borders b ON c.code = b.country1 OR c.code = b.country2 
GROUP BY c.name
HAVING count(*) > (SELECT count(*)
	FROM borders b
	WHERE b.country1 = 'D' OR b.country2 = 'D')
	\end{lstlisting}
	
	\section*{Aufgabe 14}
	Bestimmen Sie die Stadt mit der größten Einwohnerzahl und das Land, in dem sie liegt. [1]
	\begin{lstlisting}[style=sql,tabsize=2]
SELECT c.name, s.name 
FROM country c 
INNER JOIN city s ON c.code = s.country
ORDER BY s.population DESC
LIMIT 1
	\end{lstlisting}
	oder
	\begin{lstlisting}[style=sql, tabsize=2]
SELECT c.name, s.name
FROM country c 
INNER JOIN city s ON c.code = s.country 
WHERE s.population = (SELECT MAX(population) FROM city)
	\end{lstlisting}
	
\end{document}