\documentclass{article}

\usepackage{amsmath,amssymb}
\usepackage{tikz}
\usetikzlibrary{er,positioning}
\usepackage{pgfplots}
\usepackage{xcolor}
\usepackage[left=2.1cm,right=3.1cm,bottom=3cm,footskip=0.75cm,headsep=0.5cm]{geometry}
\usepackage{enumerate}
\usepackage{enumitem}
\usepackage{marvosym}
\usepackage{tabularx}

\usepackage{listings}
\definecolor{lightlightgray}{rgb}{0.95,0.95,0.95}
\definecolor{lila}{rgb}{0.8,0,0.8}
\definecolor{mygray}{rgb}{0.5,0.5,0.5}
\definecolor{mygreen}{rgb}{0,0.8,0.26}
\lstdefinestyle{sql} {language=sql}
\lstset{language=sql,
	basicstyle=\ttfamily,
	keywordstyle=\color{lila},
	commentstyle=\color{lightgray},
	stringstyle=\color{mygreen}\ttfamily,
	backgroundcolor=\color{white},
	showstringspaces=false,
	numbers=left,
	numbersep=10pt,
	numberstyle=\color{mygray}\ttfamily,
	identifierstyle=\color{blue},
	xleftmargin=.1\textwidth, 
	%xrightmargin=.1\textwidth,
	escapechar=§,
}

\usepackage[utf8]{inputenc}

\renewcommand*{\arraystretch}{1.4}

\newcolumntype{L}[1]{>{\raggedright\arraybackslash}p{#1}}
\newcolumntype{R}[1]{>{\raggedleft\arraybackslash}p{#1}}
\newcolumntype{C}[1]{>{\centering\let\newline\\\arraybackslash\hspace{0pt}}m{#1}}

\newcommand{\E}{\mathbb{E}}
\DeclareMathOperator{\rk}{rk}
\DeclareMathOperator{\Var}{Var}
\DeclareMathOperator{\Cov}{Cov}

\def\ojoin{\setbox0=\hbox{$\bowtie$}%
	\rule[-.02ex]{.25em}{.4pt}\llap{\rule[\ht0]{.25em}{.4pt}}}
\def\leftouterjoin{\mathbin{\ojoin\mkern-5.8mu\bowtie}}
\def\rightouterjoin{\mathbin{\bowtie\mkern-5.8mu\ojoin}}
\def\fullouterjoin{\mathbin{\ojoin\mkern-5.8mu\bowtie\mkern-5.8mu\ojoin}}

\title{\textbf{Datenbanken, Übung 5}}
\author{\textsc{Henry Haustein}}
\date{}

\begin{document}
	\maketitle
	
	\section*{Aufgabe 1}
	\begin{enumerate}[label=(\alph*)]
		\item Versuch, ob man einen Primärschlüssel hinzufügen kann. Wenn AB ein Schlüssel ist, dann kommt kein Fehler, ansonsten wirft das DBMS einen Fehler zurück:
		\begin{lstlisting}[style=sql]
ALTER TABLE R ADD PRIMARY KEY (A,B);
		\end{lstlisting}
		\item Alle Tupel finden, die die funktionale Abhängigkeit verletzten. Werden keine Tupel gefunden, so gilt die funktionale Abhängigkeit
		\begin{lstlisting}[style=sql]
SELECT * FROM R t1, R t2 
WHERE ((t1.D = t2.D) AND (t1.E = t2.E)) 
AND (t1.B <> t2.B);
		\end{lstlisting}
	\end{enumerate}

	\section*{Aufgabe 2}
	Berechnung der Hülle von $B$, $B^+$, liefert $\{A,B,C,D,E,F\}$. Die Attribute $G$ und $H$ werden nicht erreicht. Somit wäre $(BG)^+ = R$ ein Schlüssel der Relation und damit ist $ABG$ nicht minimal.

	\section*{Aufgabe 3}
	\begin{enumerate}[label=(\alph*)]
		\item Es gelten $A\to B,C$ und $C\to A,B$.
		\item $R_1\cap R_2 = \{B\}$ und für eine verlustfreie Zerlegung müsste entweder $B\to A$ oder $B\to C$ gelten. Beides ist nicht der Fall.
		\item Ist sie nicht. Um $A\to C$ oder $C\to A$ zu prüfen, müsste zuerst ein JOIN ausgeführt werden.
		\item $FD = \{B\to A\}$, also z.B. solche Tupel:
		\begin{center}
			\begin{tabular}{c|c|c}
				\textbf{A} & \textbf{B} & \textbf{C} \\
				\hline
				2 & 3 & 4 \\
				5 & 4 & 6 \\
				7 & 8 & 10
			\end{tabular}
		\end{center}
	\end{enumerate}
	
	\section*{Aufgabe 4}
	\begin{enumerate}[label=(\alph*)]
		\item $\{AG\}$, $\{CG\}$, $\{FG\}$, $\{EG\}$
		\item ?
	\end{enumerate}
	
\end{document}