\documentclass{article}

\usepackage{amsmath,amssymb}
\usepackage{tikz}
\usepackage{pgfplots}
\usepackage{xcolor}
\usepackage[left=2.1cm,right=3.1cm,bottom=3cm,footskip=0.75cm,headsep=0.5cm]{geometry}
\usepackage{enumerate}
\usepackage{enumitem}
\usepackage{marvosym}
\usepackage{tabularx}

\usepackage{listings}
\definecolor{lightlightgray}{rgb}{0.95,0.95,0.95}
\definecolor{lila}{rgb}{0.8,0,0.8}
\definecolor{mygray}{rgb}{0.5,0.5,0.5}
\definecolor{mygreen}{rgb}{0,0.8,0.26}
\lstdefinestyle{java} {language=java}
\lstset{language=java,
	basicstyle=\ttfamily,
	keywordstyle=\color{lila},
	commentstyle=\color{lightgray},
	stringstyle=\color{mygreen}\ttfamily,
	backgroundcolor=\color{white},
	showstringspaces=false,
	numbers=left,
	numbersep=10pt,
	numberstyle=\color{mygray}\ttfamily,
	identifierstyle=\color{blue},
	xleftmargin=.1\textwidth, 
	%xrightmargin=.1\textwidth,
	escapechar=§,
}

\usepackage[utf8]{inputenc}

\renewcommand*{\arraystretch}{1.4}

\newcolumntype{L}[1]{>{\raggedright\arraybackslash}p{#1}}
\newcolumntype{R}[1]{>{\raggedleft\arraybackslash}p{#1}}
\newcolumntype{C}[1]{>{\centering\let\newline\\\arraybackslash\hspace{0pt}}m{#1}}

\newcommand{\E}{\mathbb{E}}
\DeclareMathOperator{\rk}{rk}
\DeclareMathOperator{\Var}{Var}
\DeclareMathOperator{\Cov}{Cov}

\title{\textbf{Datenbanken, Übung 1}}
\author{\textsc{Henry Haustein}}
\date{}

\begin{document}
	\maketitle
	
	\section*{Aufgabe 1}
	\begin{enumerate}[label=(\alph*)]
		\item Es werden Daten wie Text, Zeit, Ort, Account, App/Desktop, verwendete Hashtags, erwähnte User, Anzahl Likes/Retweets, ... generiert. Man kann diese Daten in Datenbanken speichern. Wenn jemand auf Like derückt, werden auch da Daten erzeugt: Account, welcher Tweet, Zeit, Ort, App/Desktop, ... 
		\item Wenn der Tweet gelöscht wird, was passiert dann mit den Likes/Antworten/Retweets, die sich auf den gelöschten Tweet beziehen? Nutzerkonto kann aber einfach gelöscht werden.
	\end{enumerate}

	\section*{Aufgabe 2}
	\begin{enumerate}[label=(\alph*)]
		\item Performance ist aber katastrophal.
		\item Wird vielleicht jetzt noch nicht benötigt, aber in Zukunft.
		\item Welcher Entwickler kann denn bitte kein SQL?
		\item Es können aber sehr schnell Inkonsistenzen auftreten.
	\end{enumerate}

	\section*{Aufgabe 3}
	A und B haben jeweils 1000 \EUR\, und A überweist B 100 \EUR, während B A 200 \EUR\, überweist. Wenn die Transaktionen gleichzeitig ausgeführt werden, wird bei A das Geld abgezogen, aber weil die Transaktion von B nach A gleichzeitig läuft, ließt B den alten Kontostand von A ein, und erhöht diesen. Selbiges läuft auf der Seite von A ab: Es wird der alte Kontostand von B eingelesen, bei dem noch nicht das Geld für die Transaktion abgezogen war, weil der Schreibzugriff noch nicht erfolgt war.
	
	\section*{Aufgabe 4}
	\begin{enumerate}[label=(\alph*)]
		\item Externe Ebene: stellt die Daten dar, z.B. in Tabellen \\
		Logische Ebene: enthält die Relationen \\
		Physische Ebene: kümmert sich um die Speicherung der Daten
		\item Änderungen an Schichten sollen keine Auswirkungen auf die andere Schicht haben
	\end{enumerate}
	
\end{document}