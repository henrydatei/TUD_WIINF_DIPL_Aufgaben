\documentclass{article}

\usepackage{amsmath,amssymb}
\usepackage{tikz}
\usetikzlibrary{er,positioning}
\usepackage{pgfplots}
\usepackage{xcolor}
\usepackage[left=2.1cm,right=3.1cm,bottom=3cm,footskip=0.75cm,headsep=0.5cm]{geometry}
\usepackage{enumerate}
\usepackage{enumitem}
\usepackage{marvosym}
\usepackage{tabularx}

\usepackage{listings}
\definecolor{lightlightgray}{rgb}{0.95,0.95,0.95}
\definecolor{lila}{rgb}{0.8,0,0.8}
\definecolor{mygray}{rgb}{0.5,0.5,0.5}
\definecolor{mygreen}{rgb}{0,0.8,0.26}
\lstdefinestyle{sql} {language=sql}
\lstset{language=sql,
	basicstyle=\ttfamily,
	keywordstyle=\color{lila},
	commentstyle=\color{lightgray},
	stringstyle=\color{mygreen}\ttfamily,
	backgroundcolor=\color{white},
	showstringspaces=false,
	numbers=left,
	numbersep=10pt,
	numberstyle=\color{mygray}\ttfamily,
	identifierstyle=\color{blue},
	xleftmargin=.1\textwidth, 
	%xrightmargin=.1\textwidth,
	escapechar=§,
}

\usepackage[utf8]{inputenc}

\renewcommand*{\arraystretch}{1.4}

\newcolumntype{L}[1]{>{\raggedright\arraybackslash}p{#1}}
\newcolumntype{R}[1]{>{\raggedleft\arraybackslash}p{#1}}
\newcolumntype{C}[1]{>{\centering\let\newline\\\arraybackslash\hspace{0pt}}m{#1}}

\newcommand{\E}{\mathbb{E}}
\DeclareMathOperator{\rk}{rk}
\DeclareMathOperator{\Var}{Var}
\DeclareMathOperator{\Cov}{Cov}
\DeclareMathOperator{\Hul}{Hul}

\def\ojoin{\setbox0=\hbox{$\bowtie$}%
	\rule[-.02ex]{.25em}{.4pt}\llap{\rule[\ht0]{.25em}{.4pt}}}
\def\leftouterjoin{\mathbin{\ojoin\mkern-5.8mu\bowtie}}
\def\rightouterjoin{\mathbin{\bowtie\mkern-5.8mu\ojoin}}
\def\fullouterjoin{\mathbin{\ojoin\mkern-5.8mu\bowtie\mkern-5.8mu\ojoin}}

\title{\textbf{Datenbanken, Übung 6}}
\author{\textsc{Henry Haustein}}
\date{}

\begin{document}
	\maketitle
	
	\section*{Aufgabe 1}
	\begin{enumerate}[label=(\alph*)]
		\item Linksreduktion:
		\begin{itemize}
			\item $CE\in\Hul(F_1,D)$: nein
			\item $CE\in\Hul(F_1,C)$: nein
		\end{itemize}
		Rechtsreduktion:
		\begin{itemize}
			\item $A\to BC$
			\begin{itemize}
				\item $B\in\Hul(F_1-(A\to BC) \cup (A\to C), A)$: nein
				\item $C\in\Hul(F_1-(A\to BC) \cup (A\to B), A)$: ja, $A\to B\to CD$ $\Rightarrow$ neue Menge $F_1'$
			\end{itemize} 
			\item $B\to CD$
			\begin{itemize}
				\item $C\in\Hul(F_1'-(B\to CD) \cup (B\to D), B)$: nein
				\item $D\in\Hul(F_1'-(B\to CD) \cup (B\to C), B)$: nein
			\end{itemize}
			\item $CD\to CE$
			\begin{itemize}
				\item $C\in\Hul(F_1'-(CD\to CE) \cup (CD\to E), CD)$: ja, $C\to C$ $\Rightarrow$ neue Menge $F_1''$
				\item $E\in\Hul(F_1''-(CD\to E) \cup (CD\to \emptyset), CD)$: nein
			\end{itemize}
			\item $B\to EF$
			\begin{itemize}
				\item $E\in\Hul(F_1''-(B\to EF) \cup (B\to F), B)$: ja, $B\to CD\to E$ $\Rightarrow$ neue Menge $F_1'''$
				\item $F\in\Hul(F_1'''-(B\to F) \cup (B\to \emptyset), B)$: nein
			\end{itemize}
		\end{itemize}
		Streichen von leeren Mengen auf der rechten Seite: nichts zu tun \\
		Zusammenfassen
		\begin{align}
			F_1''' = \{A\to B, B\to CDF, CD\to E\} \notag
		\end{align}
		\item Linksreduktion
		\begin{itemize}
			\item $BDE\in\Hul(F_2,C)$: nein
			\item $BDE\in\Hul(F_2,B)$: ja, $B\to C\to DE$ und $B\to B$ $\Rightarrow$ neue Menge $F_2'$
		\end{itemize}
		Rechtsreduktion
		\begin{itemize}
			\item $A\to BE$
			\begin{itemize}
				\item $B\in\Hul(F_2'-(A\to BC) \cup (A\to C), A)$: nein
				\item $C\in\Hul(F_2'-(A\to BC) \cup (A\to B), A)$: ja, $A\to B\to C$ $\Rightarrow$ neue Menge $F_2''$
			\end{itemize}
			\item $B\to BDE$
			\begin{itemize}
				\item $B\in\Hul(F_2''-(B\to BDE) \cup (B\to DE), B)$: ja, $B\to B$ $\Rightarrow$ neue Menge $F_2'''$
				\item $D\in\Hul(F_2'''-(B\to DE) \cup (B\to E), B)$: ja, $B\to C\to DE$ $\Rightarrow$ neue Menge $F_2^{(4)}$
				\item $E\in\Hul(F_2^{(4)}-(B\to E) \cup (B\to \emptyset), B)$: ja, $B\to C\to DE$ $\Rightarrow$ neue Menge $F_2^{(5)}$
			\end{itemize}
			\item $D\to F$
			\begin{itemize}
				\item $D\in\Hul(F_2^{(5)} - (D\to F) \cup (D\to\emptyset), D)$: nein
			\end{itemize}
			\item $E\to EG$
			\begin{itemize}
				\item $E\in\Hul(F_2^{(5)} - (E\to EG) \cup (E\to G), E)$: ja, $E\to E$ $\Rightarrow$ neue Menge $F_2^{(6)}$
				\item $G\in\Hul(F_2^{(6)} - (E\to G) \cup (E\to \emptyset), E)$: nein
			\end{itemize}
			\item $B\to C$
			\begin{itemize}
				\item $C\in\Hul(F_2^{(6)} - (B\to C) \cup (B\to\emptyset), B)$: nein
			\end{itemize}
			\item $C\to DE$
			\begin{itemize}
				\item $D\in\Hul(F_2^{(6)} - (C\to DE) \cup (C\to E), C)$: nein
				\item $E\in\Hul(F_2^{(6)} - (C\to DE) \cup (C\to D), C)$: nein
			\end{itemize}
		\end{itemize}
		Entfernen der FDs mit leerer Menge rechts: $B\to\emptyset$ wird entfernt \\
		Zusammenfassen:
		\begin{align}
			F_2^{(6)} = \{A\to B, D\to F, E\to G, B\to C, C\to DE\} \notag
		\end{align}
	\end{enumerate}

	\section*{Aufgabe 2}
	\begin{enumerate}[label=(\alph*)]
		\item Ist die Tabelle in erster Normalform? ja \\
		Um zu überprüfen, ob die Tabelle auch in 2. NF ist, müssen zuerst die funktionalen Abhängigkeiten und daraus ein Schlüssel bestimmt werden:
		\begin{itemize}
			\item Signatur $\to$ Titel
			\item Benutzer $\to$ Straße, PLZ, Ort
			\item Vorgang $\to$ Datum, Benutzer
			\item PLZ $\to$ Ort
		\end{itemize}
		$\Rightarrow$ Schlüssel: \{Signatur, Vorgang\}. Wir sehen, dass Benutzer/Datum/Straße/PLZ/Ort nur von \{Vorgang\} funktional abhängig ist, aber nicht von dem kompletten Schlüssel \{Signatur, Vorgang\}. Ähnliches gilt für den Titel. Damit ist die Relation nicht in 2. NF, aber wir können sie in die 2. NF bringen:
		\begin{itemize}
			\item $R_1:$ \underline{Vorgang}, Datum, Benutzer, Straße, Ort, PLZ
			\item $R_2:$ \underline{Signatur}, Titel
			\item $R_3:$ \underline{Vorgang, Signatur}
		\end{itemize}
		\item Betrachten wir die Relation Benutzer $\to$ Straße, PLZ, Ort. Hier ist weder Benutzer ein Superschlüssel, noch ist \{Straße, PLZ, Ort\} Teil eines Kandidatenschlüssels. Selbiges gilt für PLZ $\to$ Ort. Diese funktionalen Abhängigkeiten müssen noch in eigene Relationen gesteckt werden:
		\begin{itemize}
			\item $R_1$: \underline{Benutzer}, Straße, PLZ
			\item $R_2$: \underline{PLZ}, Ort
			\item $R_3$: \underline{Vorgang}, Datum, Benutzer
		\end{itemize}
	\end{enumerate}

	\section*{Aufgabe 3}
	\begin{enumerate}[label=(\alph*)]
		\item Die Tabelle ist nicht mal in 1. NF, also unnormalisiert, da Produkt nicht atomar ist.
		\item Verk\_Nr, Produkt\_Name $\to$ Umsatz
		\item Die Relation
		\begin{itemize}
			\item $R_1$: \underline{Verk\_Nr}, Verk\_Name, Verk\_Ort
			\item $R_2$: \underline{Verk\_Nr, Produkt\_Name}, Umsatz
		\end{itemize}
		ist schon in 3. NF.
		\item Neue Produkte können nicht eingefügt werden ohne sie verkauft zu haben und wenn Verkäufer kündigen werden auch die Produkte gelöscht. $\Rightarrow$ INSERT- und DELETE-Anomalie
	\end{enumerate}
	
	\section*{Aufgabe 4}
	Die Liste der Determinanten ist $\{A, B, C\}$. Die Kandidatenschlüssel sind $\{A\}$ oder $\{B\}$. Damit ist $C$ kein Schlüssel und die funktionale Abhängigkeit $C\to D$ wird in eine eigene Relation ausgelagert. Damit sind die Relationen dann
	\begin{itemize}
		\item $R_1$: \underline{$A$}, $B$, $C$ (man könnte auch $B$ als Schlüssel wählen)
		\item $R_2$: \underline{$C$}, $D$
	\end{itemize}
	
\end{document}