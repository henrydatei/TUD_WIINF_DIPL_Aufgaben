\documentclass{article}

\usepackage{amsmath,amssymb}
\usepackage{tikz}
\usepackage{xcolor}
\usepackage[left=2.1cm,right=3.1cm,bottom=3cm,footskip=0.75cm,headsep=0.5cm]{geometry}
\usepackage{enumerate}
\usepackage{enumitem}

\usepackage[utf8]{inputenc}

\renewcommand*{\arraystretch}{1.4}

\title{\textbf{Einführung in die Informatik, Übung 6}}
\author{\textsc{Henry Haustein}}
\date{}

\begin{document}
	\maketitle
	
	\section*{Aufgabe 6.1}
	
	\begin{enumerate}[label=(\alph*)]
		\item $L=\{ab,abcab,abcab\vert c\vert ab,abcabcab\vert c\vert abcab, abcabcabcabcab\vert c\vert abcabcab\}$
		\item nein, denn $f(0)=f(1)$, aber $0\neq 1$
		\item nein, denn $f(\cdot)\neq c$
	\end{enumerate}
	
	\section*{Aufgabe 6.2}
	
	\begin{enumerate}[label=(\alph*)]
		\item nein, denn $ba\in \{a,b\}^\ast$, aber $ba\notin \{a\}^\ast\cdot\{b\}^\ast$
		\item ja, trivial
		\item nein, denn $ba\in \{a,b\}^\ast$, aber $ba\notin \{a\}^\ast\cup\{b\}^\ast$
	\end{enumerate}
	
	\section*{Aufgabe 6.3}
	
	\begin{enumerate}[label=(\alph*)]
		\item Die Sprache besteht aus den Wörtern $a$, $ba$ und $b$ \\
		$\Rightarrow$ $L=\{a,ba,b\}$
		\item Die Sprache besteht aus dem Wort $a$ und den Wörtern, die mit $b$ enden, wo aber davor mindesten 0 $a$'s stehen \\
		$\Rightarrow$ $L=\{a,a^ib\mid i\ge 0\}$
		\item Die Sprache besteht aus Wörtern, die aus Wiederholungen von $abc$'s gebildet sind, wobei $abc$ mindestens einmal vorkommt \\
		$\Rightarrow$ $L=\{(abc)^n\mid n\ge 1\}$
	\end{enumerate}
	
	\section*{Aufgabe 6.4}
	
	\begin{enumerate}[label=(\alph*)]
		\item $(aaa)^\ast$
		\item $(a^+b^+)^5$
		\item $\Sigma^\ast\cdot (aaa)\cdot [\Sigma^\ast\cdot (bab)\cdot\Sigma^\ast\cdot (bab)\cdot\Sigma^\ast]^+\cdot \Sigma^\ast$
		\item $\overline{(abba)}\cdot\Sigma^\ast$
		\item $b^\ast a^\ast$
	\end{enumerate}

\end{document}