\documentclass{article}

\usepackage{amsmath,amssymb}
\usepackage{tikz}
\usepackage{xcolor}
\usepackage[left=2.1cm,right=3.1cm,bottom=3cm,footskip=0.75cm,headsep=0.5cm]{geometry}
\usepackage{enumerate}
\usepackage{enumitem}
\usepackage{marvosym}
\usepackage{tabularx}

\usepackage[utf8]{inputenc}

\renewcommand*{\arraystretch}{1.4}

\newcolumntype{L}[1]{>{\raggedright\arraybackslash}p{#1}}
\newcolumntype{R}[1]{>{\raggedleft\arraybackslash}p{#1}}
\newcolumntype{C}[1]{>{\centering\let\newline\\\arraybackslash\hspace{0pt}}m{#1}}

\title{\textbf{Einführung in die Informatik, Übung 9}}
\author{\textsc{Henry Haustein}}
\date{}

\begin{document}
	\maketitle
	
	\section*{Aufgabe 9.1}
	
	\begin{enumerate}[label=(\alph*)]
		\item $(N_1\cup \{S\}\cup N_2 \cup \{T\},\Sigma,P_1\cup \{S\to\varepsilon,S\to SS_1\}\cup P_2\cup \{T\to SS_2\},T)$
		\item Veränderungen in den Grammatiken sind mit \textcolor{red}{rot} dargestellt.\\
		$\cdot$: $(N_1\cup N_2\cup \{S\},\Sigma,P_1\cup P_2\cup \{S\to S_1S_2\},S)$ \\
		$(\cdot)^\ast$: $(N_1\cup N_2\cup \{S\}\cup \textcolor{red}{\{T\}},\Sigma,P_1\cup P_2\cup \{S\to S_1S_2\}\cup \textcolor{red}{\{T\to\varepsilon,T\to TS\}},\textcolor{red}{T})$ \\
		$(\cdot)^\ast\cup$: $(N_1\cup N_2\cup \{S\}\cup \{T\}\cup \textcolor{red}{\{A\}},\Sigma,P_1\cup P_2\cup \{S\to S_1S_2\}\cup \{T\to\varepsilon,T\to TS\}\cup \textcolor{red}{\{A\to T,A\to S_1\}},\textcolor{red}{A})$
	\end{enumerate}
	
	\section*{Aufgabe 9.2}
	
	\begin{enumerate}[label=(\alph*)]
		\item $T_1=\{S,R\}$, $T_2=\{S,R,V\}$, $T_3=\{S,R,V,W\}=T_4=...$
		\item $ab\in L(G)\Rightarrow L(G)\neq\emptyset$ (alternative Begründung: $S\in T_3=T_4=...$)
	\end{enumerate}

	\section*{Aufgabe 9.3}
	
	\begin{center}
		\begin{tabular}{l|L{3cm}|L{3cm}|L{3cm}|L{3cm}}
			& $P_1$ & $P_2$ & $P_3$ & $P_4$ \\
			\hline
			kontextfrei & \checkmark & \checkmark & \checkmark & \checkmark \\
			\hline
			rechtslinear & mehr als 1 nichtterminales Symbol & \checkmark & mehr als 1 nichtterminales Symbol & mehr als 1 nichtterminales Symbol \\
			\hline
			Chomsky-Normalform & $B\to BB$ geht nicht & $S\to bB$ geht nicht & $A\to aB$ geht nicht & $A\to SC$ geht nicht, wenn $S\to\epsilon\in P_4$
		\end{tabular}
	\end{center}

	\section*{Aufgabe 9.4}
	
	\begin{enumerate}[label=(\alph*)]
		\item siehe Tabelle\\
		\begin{center}
			\begin{tabular}{l|L{3cm}|L{3cm}|L{3cm}|L{3cm}}
				& \textbf{Typ 0} & \textbf{Typ 1} & \textbf{Typ 2} & \textbf{Typ 3} \\
				\hline
				$G_1$ & \checkmark & \checkmark & $Ca\to aC\notin N\times (N\cup\Sigma)^\ast$ & nicht rechtslinear \\
				\hline
				$G_2$ & \checkmark & \checkmark & \checkmark & nicht rechtslinear \\
				\hline
				$G_3$ & \checkmark & \checkmark & \checkmark & \checkmark
			\end{tabular}
		\end{center}
		\item Vermutung: $G$ vom Typ $i$ $\Leftrightarrow$ $L(G)$ vom Typ $i$. Zumindest für den Typ 3 scheint dies zu stimmen.
	\end{enumerate}

	\section*{Aufgabe 9.5}
	
	\begin{enumerate}[label=(\alph*)]
		\item $aaabba\in L(G)$
		\begin{center}
			\begin{tabular}{c|cccccc}
				& $a$ & $a$ & $a$ & $b$ & $b$ & $a$ \\
				& 1 & 2 & 3 & 4 & 5 & 6 \\
				\hline
				1 & $A,B$ & $S,M$ & $X$ & $S,M$ & $X$ & $S,M$ \\
				2 & & $A,B$ & $S,M$ & $X$ & $S,M$ & $X$ \\
				3 & & & $A,B$ & $S,M$ & $X$ & $\emptyset$ \\
				4 & & & & $B$ & $\emptyset$ & $\emptyset$ \\
				5 & & & & & $B$ & $\emptyset$ \\
				6 & & & & & & $A,B$
			\end{tabular}
		\end{center}
		\item $aabbaa\notin L(G)$
		\begin{center}
			\begin{tabular}{c|cccccc}
				& $a$ & $a$ & $b$ & $b$ & $a$ & $a$ \\
				& 1 & 2 & 3 & 4 & 5 & 6 \\
				\hline
				1 & $A,B$ & $S,M$ & $X$ & $S,M$ & $X$ & $\emptyset$ \\
				2 & & $A,B$ & $S,M$ & $X$ & $\emptyset$ & $\emptyset$ \\
				3 & & & $B$ & $\emptyset$ & $\emptyset$ & $\emptyset$ \\
				4 & & & & $B$ & $\emptyset$ & $\emptyset$ \\
				5 & & & & & $A,B$ & $S,M$ \\
				6 & & & & & & $A,B$
			\end{tabular}
		\end{center}
	\end{enumerate}

\end{document}