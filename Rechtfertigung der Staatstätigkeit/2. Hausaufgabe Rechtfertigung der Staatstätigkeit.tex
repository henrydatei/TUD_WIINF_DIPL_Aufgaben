\documentclass{article}

\usepackage{amsmath,amssymb}
\usepackage{tikz}
\usepackage{pgfplots}
\usepackage{xcolor}
\usepackage[left=2.1cm,right=3.1cm,bottom=3cm,footskip=0.75cm,headsep=0.5cm]{geometry}
\usepackage{enumerate}
\usepackage{enumitem}
\usepackage{marvosym}
\usepackage{tabularx}

\usepackage[utf8]{inputenc}

\renewcommand*{\arraystretch}{1.4}

\newcolumntype{L}[1]{>{\raggedright\arraybackslash}p{#1}}
\newcolumntype{R}[1]{>{\raggedleft\arraybackslash}p{#1}}
\newcolumntype{C}[1]{>{\centering\let\newline\\\arraybackslash\hspace{0pt}}m{#1}}

\title{\textbf{Rechtfertigung der Staatstätigkeit, Hausaufgabe 2}}
\author{\textsc{Henry Haustein}}
\date{}

\begin{document}
	\maketitle
	
	\section*{Aufgabe 2}
	\begin{enumerate}[label=(\alph*)]
		\item Der Monopolist setzt $GE=GK$. Die Grenzkosten sind gegeben, kümmern wir uns um den Grenzerlös:
		\begin{align}
			E &= p\cdot x \notag \\
			&= (a-bx)\cdot x \notag \\
			&= ax - bx^2 \notag \\
			GE &= a - 2bx \notag
		\end{align}
		Setzen wir ein
		\begin{align}
			GE &= GK \notag \\
			a-2bx &= cx+d \notag \\
			a-d &= cx+2bx \notag \\
			x_M &= \frac{a-d}{c+2b} \notag
		\end{align}
		Aus sozialer Sicht wäre $GK=GZB$ wünschenswert, also
		\begin{align}
			cx+d &= a-bx \notag \\
			cx+bx &= a-d \notag \\
			x_{opt} &= \frac{a-d}{c+b} \notag
		\end{align}
		\item Eine Subvention senkt die Grenzkosten des Monopolisten. Man muss also solange subventionieren, bis $GE=GK-s \Rightarrow x_{opt}=\frac{a-d}{b+c}$ ergibt.
		\begin{align}
			a-2bx_S &= cx_S+d-s \notag \\
			a-d+s &= cx_S + 2bx_S \notag \\
			x_S &= \frac{a-d+s}{c+2b} \notag \\
			&\overset{!}{=} \frac{a-d}{b+c} \notag \\
			(c+2b)(a-d) &= (a-d+s)(b+c) \notag \\
			a-d+s &= \frac{(c+2b)(a-d)}{b+c} \notag \\
			s &= \frac{(c+2b)(a-d)}{b+c}-a+d \notag \\
			&= \frac{b(a-d)}{b+c} \notag
		\end{align}
		\item Subventionsbedarf: $S = s\cdot x = \frac{bx(a-d)}{b+c}$
		\begin{center}
			\begin{tikzpicture}
				\draw[->] (0,-3) -- (5,-3) node[above] {$x$};
				\draw[->] (0,-3) -- (0,5) node[right] {$p$};
				
				\draw[blue] (0,4) -- (5,0);
				\draw[red] (0,1) -- (5,1);
				\draw[dashed,red] (0,-2.6) -- (5,-2.6);
				\draw[green!80!black] (0,4) -- (4,-3);
				
				\draw[dotted] (1.72,-3) node[below] {$x_M$} to (1.72,2.62) -- (0,2.62);
				\draw[dotted] (3.77,-3) node[below] {$x_{opt}$} to (3.77,1);
				
				\draw[->] (0.6,1) -- (0.6,-2.6);		
			\end{tikzpicture} \\
			\textcolor{blue}{GZB}, \textcolor{red}{GK/um $s$ reduzierte Grenzkosten}, \textcolor{green!80!black}{GE}
		\end{center}
		\item Diese Subvention ist teuer und "schwierig zu verkaufen": Man belohnt einen Monopolisten dafür, dass er ein Monopol hat, statt ihn zu zerschlagen. Etwas billiger wird es, wenn man nur die Einheiten zwischen $x_M$ und $x_{opt}$ subventioniert.
		\begin{center}
			\begin{tikzpicture}
				\draw[->] (0,-3) -- (5,-3) node[above] {$x$};
				\draw[->] (0,-3) -- (0,5) node[right] {$p$};
				
				\draw[blue] (0,4) -- (5,0);
				\draw[red] (0,1) -- (1.72,1) -- (1.72,-2.6) -- (5,-2.6);
				\draw[green!80!black] (0,4) -- (4,-3);
				
				\draw[dotted] (1.72,-3) node[below] {$x_M$} to (1.72,2.62) -- (0,2.62);
				\draw[dotted] (3.77,-3) node[below] {$x_{opt}$} to (3.77,1);
			\end{tikzpicture} \\
			\textcolor{blue}{GZB}, \textcolor{red}{subventionierte Grenzkosten}, \textcolor{green!80!black}{GE}
		\end{center}
	\end{enumerate}
	
	\section*{Aufgabe 6}
	\begin{enumerate}[label=(\alph*)]
		\item Betrachten wir zuerst den Gewinn von Volta, wenn $z$ bereits festgelegt wurde:
		\begin{align}
			\Pi_V &= (46-2x)x - (c+z)x \notag \\
			&= 46x - 2x^2 - (c+z)x \notag
		\end{align}
		Dieser soll maximiert werden:
		\begin{align}
			\frac{\partial\Pi_V}{\partial x} = 46-4x - (c+z) &= 0 \notag \\
			4x &= 46 - (c+z) \notag \\
			x &= \frac{46}{4} - \frac{c+z}{4} \notag
		\end{align}
		Mit $c=2$ ergibt sich
		\begin{align}
			x = \frac{46}{4} - \frac{2+z}{4} \notag
		\end{align}
		Wenn nun Photo Corp dies antizipiert, kann Photo Corp seinen Gewinn steigern. Dazu muss es sein $z$ in Abhängigkeit von $x$ bestimmen:
		\begin{align}
			x &= \frac{46}{4} - \frac{2+z}{4} \notag \\
			\frac{2+z}{4} &= \frac{46}{4} - x \notag \\
			2+z &= 46-4x \notag \\
			z &= 44-4x \notag
		\end{align}
		Der Gewinn ergibt sich nun
		\begin{align}
			\Pi_P &= z\cdot x - (10+4x) \notag \\
			&= (44-4x)x - (10+4x) \notag \\
			&= 44x-4x^2-10-4x \notag \\
			&= 40x - 4x^2 - 10 \notag
		\end{align}
		Dieser soll nun maximiert werden:
		\begin{align}
			\frac{\partial\Pi_P}{\partial x} = 40-8x &= 0 \notag \\
			40 &= 8x \notag \\
			x &= 5 \notag
		\end{align}
		Mit $x=5$ ergibt sich $z = 44-4\cdot 5 = 24$ und $p=46-2\cdot 5 = 36$.
		\item Der Gewinn des neuen Monopols ist
		\begin{align}
			\Pi &= (46-2x)x - (10+4x) \notag \\
			&= 46x-2x^2-10-4x \notag \\
			&= 42x-2x^2-10 \notag
		\end{align}
		Auch dieser soll maximiert werden:
		\begin{align}
			\frac{\partial\Pi}{\partial x} = 42-4x&=0 \notag \\
			42 &= 4x \notag \\
			x &= 10.5 \notag
		\end{align}
		Damit ergibt sich ein Preis von $p=46-2\cdot 10.5= 25$.
		\item Photo Corp sollte seine Gewinne im Fall des Doppelmonopols $\Pi_{Doppelm.}$ und im Fall der Übernahme ($\Pi_{Kauf}$) berechnen. Die Differenz dazwischen sind die maximalen Kosten, die Photo Corp bereit wäre zu bezahlen.
		\begin{align}
			\Pi_{Doppelm.} &= zx - K(x) \notag \\
			&= 24\cdot 5 - (10+4\cdot 5) \notag \\
			&= 90 \notag \\
			\Pi_{Kauf} &= px - K(x) \notag \\
			&= 25\cdot 10.5 - (10+4\cdot 10.5) \notag \\
			&= 210.5 \notag
		\end{align}
		Photo Corp sollte Volta nur dann übernehmen, wenn die Kosten $S$ kleiner als $210,5-90=120,5$ sind.
		\item Die gesamtwirtschaftlich optimale Länge, ist dann erreicht, wenn $GK=GZB$ ist. Es gilt $GK = c + \frac{\partial K}{\partial x} = c + 4$. Also
		\begin{align}
			2+4 &= 46-2x \notag \\
			2x &= 40 \notag \\
			x &= 20 \notag
		\end{align}
		Damit ergibt sich $p=46-2\cdot 20=6$.
		\item Gesamtwirtschaftlich am besten ist es, wenn es vollständige Konkurrenz gibt, also möglichst wenig Marktmacht. Das heißt auch, dass ein Monopol besser ist als 2 hintereinandergeschaltete Monopole. Die Marktmacht kann mittels Zerschlagung von Monopolen, Subvention der Monopole oder Preisgrenzen begrenzt werden.
	\end{enumerate}

\end{document}