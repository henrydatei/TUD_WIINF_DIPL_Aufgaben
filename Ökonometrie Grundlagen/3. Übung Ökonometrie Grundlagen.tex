\documentclass{article}

\usepackage{amsmath,amssymb}
\usepackage{tikz}
\usepackage{pgfplots}
\usepackage{xcolor}
\usepackage[left=2.1cm,right=3.1cm,bottom=3cm,footskip=0.75cm,headsep=0.5cm]{geometry}
\usepackage{enumerate}
\usepackage{enumitem}
\usepackage{marvosym}
\usepackage{tabularx}

\usepackage{listings}
\definecolor{lightlightgray}{rgb}{0.95,0.95,0.95}
\definecolor{lila}{rgb}{0.8,0,0.8}
\definecolor{mygray}{rgb}{0.5,0.5,0.5}
\definecolor{mygreen}{rgb}{0,0.8,0.26}
\lstdefinestyle{R} {language=R,morekeywords={confint,head}}
\lstset{language=R,
	basicstyle=\ttfamily,
	keywordstyle=\color{lila},
	commentstyle=\color{lightgray},
	stringstyle=\color{mygreen}\ttfamily,
	backgroundcolor=\color{white},
	showstringspaces=false,
	numbers=left,
	numbersep=10pt,
	numberstyle=\color{mygray}\ttfamily,
	identifierstyle=\color{blue},
	xleftmargin=.1\textwidth, 
	%xrightmargin=.1\textwidth,
	escapechar=§,
}

\usepackage[utf8]{inputenc}

\renewcommand*{\arraystretch}{1.4}

\newcolumntype{L}[1]{>{\raggedright\arraybackslash}p{#1}}
\newcolumntype{R}[1]{>{\raggedleft\arraybackslash}p{#1}}
\newcolumntype{C}[1]{>{\centering\let\newline\\\arraybackslash\hspace{0pt}}m{#1}}

\title{\textbf{Ökonometrie Grundlagen, Übung 3}}
\author{\textsc{Henry Haustein}}
\date{}

\begin{document}
	\maketitle
	
	\section*{Aufgabe 1}
	Wir diesen Beweis brauchen wir noch mal die Formeln zur Schätzung der Parameter $\beta_0$ und $\beta_1$ für lineare Modelle:
	\begin{align}
		\beta_0 &= \bar{y} - \beta_1\bar{x} \notag \\
		\beta_1 &= \frac{\sum_{i=1}^{n}(y_i-\bar{y})(x_i-\bar{x})}{\sum_{i=1}^n (x_i-\bar{x})^2} \notag
	\end{align}
	Ferner brauchen wir die Definition von $\hat{y}=\beta_0+\beta_1x$. Jetzt wenden wir uns dem eigentlichen Beweis zu. Wir werden zeigen, dass $R^2=r^2_{xy}$, was gleichbedeutend mit $r_{xy}=\sqrt{R^2}$ ist.
	\begin{align}
		R^2 &= \frac{\sum_{i=1}^n (\hat{y}_i-\bar{y})^2}{\sum_{i=1}^n (y_i-\bar{y})^2} \notag \\
		&= \frac{\sum_{i=1}^n (\beta_0+\beta_1x_i - \bar{y})^2}{\sum_{i=1}^n (y_i-\bar{y})^2} \notag \\
		&= \frac{\sum_{i=1}^n (\bar{y}-\beta_1\bar{x}+\beta_1x_i-\bar{y})^2}{\sum_{i=1}^n (y_i-\bar{y})^2} \notag \\
		&= \beta_1^2\frac{\sum_{i=1}^n (x_i-\bar{x})^2}{\sum_{i=1}^n (y_i-\bar{y})^2} \notag \\
		&=\left(\frac{\sum_{i=1}^n (y_i-\bar{y})(x_i-\bar{x})}{\sum_{i=1}^n (x_i-\bar{x})^2}\right)^2\frac{\sum_{i=1}^n (x_i-\bar{x})}{\sum_{i=1}^n (y_i-\bar{y})^2} \notag \\
		&= \frac{\left[\sum_{i=1}^n (y_i-\bar{y})(x_i-\bar{x})\right]^2}{\left[\sum_{i=1}^n (x_i-\bar{x})^2\right]^2}\frac{\sum_{i=1}^n (x_i-\bar{x})}{\sum_{i=1}^n (y_i-\bar{y})^2} \notag \\
		&= \frac{\left[\sum_{i=1}^n (y_i-\bar{y})(x_i-\bar{x})\right]^2}{\sum_{i=1}^n (x_i-\bar{x})^2\sum_{i=1}^n (y_i-\bar{y})^2} \notag \\
		&= r_{xy}^2 \notag
	\end{align}
	
	\section*{Aufgabe 2}
	Interpretation des theoretischen KI's: Der wahre Parameter der GG liegt mit einer Wahrscheinlichkeit von $1-\alpha$ innerhalb des Intervalls.
	
	Interpretation des numerischen KI's: Da das numerische bzw. realisierte KI eine Ausprägung der Zufallsvariable Konfidenzintervallschätzer ist (und damit aus 2 Zahlen, der linken und der rechten Grenze, besteht), liegt der Parameter innerhalb dieses Intervalls oder nicht. Die Wahrscheinlichkeit ist demnach 1 (innerhalb) oder 0 (außerhalb). Je höher der Stichprobenumfang $T$, desto näher liegt die Überdeckungwahrscheinlichkeit der Konfidenzintervalle beim Konfidenzniveau $(1-\alpha)100$.
	
	\section*{Aufgabe 3}
	\begin{enumerate}[label=(\alph*)]
		\item Bei einer Lageverschiebung verschiebt sich das Konfidenzintervall von $\beta_i$ mit; das Konfidenzintervall von $\sigma^2_u$ sollte sich nicht verändern, da sich die Varianz der Daten nicht ändert
		\item Das Konfidenzintervall von $\beta_i$ wird größer, wenn $\alpha$ kleiner wird. Selbiges gilt auch für $\sigma^2_u$.
		\item Das Konfidenzintervall von $\beta_i$ wird größer, wenn der standard error größer wird. Auch dies gilt für $\sigma^2_u$.
	\end{enumerate}

	\section*{Aufgabe 4}
	In allen 3 Fällen muss zuerst die Teststatistik berechnet werden. Wenn $\sigma^2_u$ unbekannt ist, macht man einen sogenannten $t$-Test mit der folgenden Teststatistik:
	\begin{align}
		t = \frac{\beta - \beta^\ast}{\text{se}(\beta)} \sim t_{T-2} \notag
	\end{align}
	Die Berechnung der kritischen Werte und Interpretation ist dann fallabhängig:
	\begin{enumerate}[label=(\alph*)]
		\item zweiseitiger Test: Man berechnet $t_{T-2;1-\frac{\alpha}{2}}$ und wenn $\vert t\vert > t_{T-2;1-\frac{\alpha}{2}}$ gilt, dann lehnt man die Nullhypothese ab.
		\item[(b) + (c)] einseitiger Test: Man berechnet $t_{T-2;1-\alpha}$ und wenn $\vert t\vert > t_{T-2;1-\alpha}$ gilt, dann lehnt man die Nullhypothese ab.
	\end{enumerate}

	\section*{Aufgabe 5}
	Man berechnet das Konfidenzintervall zum gewünschten Signifikanzniveau und wenn $\beta^\ast$ außerhalb dieses Intervalls liegt, lehnt man die Nullhypothese ab.
	
	\section*{Aufgabe 6}
	Der $p$-value ist die minimale Irrtumswahrscheinlichkeit $\alpha^\ast$ bei gegebenem Wert der Teststatistik, bei der $H_0$ gerade noch abgelehnt werden kann.
	
	\section*{Aufgabe 7}
	\begin{enumerate}[label=(\alph*)]
		\item ökonomisches Modell: $\log(y) = \beta_0 + \beta_1\log(x)$ \\
		ökonometrisches Modell: $\log(y_t) = \beta_0 + \beta_1\log(x_t) + u_t$
		\item in R
		\begin{lstlisting}[style=R]
datensatz = read.csv2("Zigaretten.csv")
head(datensatz)
modell = lm(log(Zigarettenkonsum) ~ log(Realpreis), data = datensatz)
modell$coefficients
confint(modell)
		\end{lstlisting}
		$\beta_0\in [4.975343, 5.2289130]$, $\hat{\beta}_0=5.102128$ \\
		$\log(\beta_1)\in [-1.796936,-0.6622263]$, $\hat{\log(\beta)}_1=-1.229581$
		\item Die Konfidenzintervalle werden größer
		\begin{lstlisting}[style=R]
confint(modell,level=0.99)
		\end{lstlisting}
		$\beta_0\in [4.932759, 5.2714968]$ \\
		$\log(\beta_1)\in [-1.987496,-0.4716662]$
		\item In R
		\begin{lstlisting}[style=R]
summary(modell)
		\end{lstlisting}
		Der $p$-Value für $\beta_0$ ist $< 2\cdot 10^{-16}$ und für $\beta_1$ $7.53\cdot 10^{-5}$. Das bedeutet, dass die zugehörigen Nullhypothesen $\beta_0=0$ und $\beta_1=0$ abgelehnt werden. $R^2=0.3024$ was auf kein gutes Modell für diese Daten hindeutet.
	\end{enumerate}

	\section*{Aufgabe 8}
	\begin{enumerate}[label=(\alph*)]
		\item ökonomisches Modell: $\text{Mineralölkonsum} = \beta_0 + \beta_1\cdot\text{Realeinkommen}$ \\
		ökonometrisches Modell: $\text{Mineralölkonsum}_t = \beta_0 + \beta_1\cdot
		\text{Realeinkommen}_t + u_t$
		\item In R
		\begin{lstlisting}[style=R]
datensatz2 = read.csv2("Mineral.csv")
head(datensatz2)
plot(datensatz2$Realeinkommen,datensatz2$Mineraloelkonsum)
modell2 = lm(Mineraloelkonsum ~ Realeinkommen, data = datensatz2)
summary(modell2)
confint(modell2)
		\end{lstlisting}
		$\beta_0\in [13.29841484,18.58500742]$, $\hat{\beta}_0=15.94$ \\
		$\beta_1\in [0.01987696,0.02198638]$, $\hat{\beta}_1=2.093\cdot 10^{-2}$
	\end{enumerate}
	
\end{document}