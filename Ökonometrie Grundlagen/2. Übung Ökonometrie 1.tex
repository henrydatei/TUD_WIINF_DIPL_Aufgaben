\documentclass{article}

\usepackage{amsmath,amssymb}
\usepackage{tikz}
\usepackage{xcolor}
\usepackage[left=2.1cm,right=3.1cm,bottom=3cm,footskip=0.75cm,headsep=0.5cm]{geometry}
\usepackage{enumerate}
\usepackage{enumitem}
\usepackage{marvosym}
\usepackage{tabularx}
\usepackage{listings}

\usepackage[utf8]{inputenc}

\renewcommand*{\arraystretch}{1.4}

\newcolumntype{L}[1]{>{\raggedright\arraybackslash}p{#1}}
\newcolumntype{R}[1]{>{\raggedleft\arraybackslash}p{#1}}
\newcolumntype{C}[1]{>{\centering\let\newline\\\arraybackslash\hspace{0pt}}m{#1}}

\newcommand{\E}{\mathbb{E}}
\DeclareMathOperator{\Var}{Var}
\DeclareMathOperator{\Cov}{Cov}

\definecolor{lightlightgray}{rgb}{0.95,0.95,0.95}
\definecolor{lila}{rgb}{0.8,0,0.8}
\definecolor{mygray}{rgb}{0.5,0.5,0.5}
\definecolor{mygreen}{rgb}{0,0.8,0.26}
\lstdefinestyle{R} {language=R}
\lstset{language=R,
	basicstyle=\ttfamily,
	keywordstyle=\color{lila},
	commentstyle=\color{lightgray},
	stringstyle=\color{mygreen}\ttfamily,
	backgroundcolor=\color{white},
	showstringspaces=false,
	numbers=left,
	numbersep=10pt,
	numberstyle=\color{mygray}\ttfamily,
	identifierstyle=\color{blue},
	xleftmargin=.1\textwidth, 
	%xrightmargin=.1\textwidth,
	escapechar=§
}

\title{\textbf{Ökonometrie Grundlagen, Übung 2}}
\author{\textsc{Henry Haustein}}
\date{}

\begin{document}
	\maketitle
	
	\section*{Aufgabe 1}
	\begin{itemize}
		\item Schätzung: näherungsweise Bestimmung eines unbekannten Parameters
		\item Schätzer: der Parameter, der bestimmt werden soll
	\end{itemize}
	
	\section*{Aufgabe 2}
	\begin{itemize}
		\item Fehler: Abweichung des gemessenen Wertes von der PRF
		\item Residuum: Schätzung des Fehlers
	\end{itemize}

	\section*{Aufgabe 3}
	\begin{enumerate}[label=(\alph*)]
		\item erster Beweis:
		\begin{align}
			\bar{x} = \frac{1}{T}\sum_{i=1}^{T} x_i  \Rightarrow T\bar{x} = \sum_{i=1}^T x_i \notag
		\end{align}
		\item zweiter Beweis:
		\begin{align}
			\sum_{i=1}^T (x_i-\bar{x})^2 &= \sum_{i=1}^T(x_i^2 - 2x_i\bar{x} + \bar{x}^2) \notag \\
			&= \sum_{i=1} x_i^2 - 2\bar{x}\underbrace{\sum_{i=1}^T x_i}_{T\bar{x}} + \sum_{i=1}^T \bar{x}^2  \notag \\
			&= \sum_{i=1}^T x_i^2 - 2T\bar{x}^2 + T\bar{x} \notag \\
			&= \sum_{i=1}^T x_i^2 - T\bar{x}^2 \notag
		\end{align}
		\item dritter Beweis:
		\begin{align}
			\sum_{i=1}^T (x_i-\bar{x})(y_i-\bar{y}) &= \sum_{i=1}^T (x_iy_i - x_i\bar{y} - \bar{x}y_i + \bar{x}\bar{y}) \notag \\
			&= \sum_{i=1}^T x_iy_i - \sum_{i=1}^T x_i\bar{y} - \sum_{i=1}^T \bar{x}y_i + \sum_{i=1}^T \bar{x}\bar{y} \notag \\
			&= \sum_{i=1}^T x_iy_i - \bar{y}\underbrace{\sum_{i=1}^T x_i}_{T\bar{x}} - \bar{x}\underbrace{\sum_{i=1}^T y_i}_{T\bar{y}} + T\bar{x}\bar{y} \notag \\
			&= \sum_{i=1}^T x_iy_i - \bar{y}T\bar{x} - \bar{x}T\bar{y} + T\bar{x}\bar{y} \notag \\
			&= \sum_{i=1}^n x_iy_i - T\bar{x}\bar{y} \notag
		\end{align}
	\end{enumerate}

	\section*{Aufgabe 4}
	Das Ziel der KQ-Schätzung ist, die Summe der Fehlerquadrate zu minimieren, also
	\begin{align}
		\sum_{i=1}^T u_t^2 = \sum_{i=1}^T (y_i-\beta_0 - \beta_1x_i)^2 \to\min \notag
	\end{align}
	Dazu müssen wir die partiellen Ableitungen nach $\beta_0$ und $\beta_1$ Null gesetzt werden. Wir wenden uns zunächst $\beta_0$ zu:
	\begin{align}
		\frac{\partial}{\partial\beta_0} \sum_{i=1}^T (y_i-\beta_0-\beta_1x_i)^2 &= 0 \notag \\
		-2\sum_{i=1}^T (y_i-\beta_0-\beta_1x_i) &= 0 \notag \\
		\sum_{i=1}^T y_i - T\beta_0 - \beta_1\sum_{i=1}^T x_i &= 0 \notag \\
		T\beta_0 &= \underbrace{\sum_{i=1}^T y_i}_{T\bar{y}} - \beta_1\underbrace{\sum_{i=1}^T x_i}_{T\bar{x}} \notag \\
		\beta_0 &= \bar{y} - \beta_1\bar{x} \notag
	\end{align}
	Jetzt kommt $\beta_1$:
	\begin{align}
		\frac{\partial}{\partial\beta_1} \sum_{i=1}^T(y_i-\beta_0-\beta_1x_i) &= 0 \notag \\
		2\sum_{i=1}^T (y_i-\beta_0-\beta_1x_i)(-x_i) &= 0 \notag \\
		\sum_{i=1}^T (x_iy_i - \beta_0x_i - \beta_1x_i^2) &= 0 \notag \\
		\sum_{i=1}^T x_iy_i - \beta_0\sum_{i=1}^T x_i - \beta_1\sum_{i=1}^T x_i^2 &= 0 \notag \\
		\sum_{i=1}^T x_iy_i - \bar{y}\sum_{i=1}^T x_i + \beta_1\bar{x}\sum_{i=1}^T x_i - \beta_1\sum_{i=1}^T x_i^2 &= 0 \notag \\
		\beta_1\left(-\bar{x}\sum_{i=1}^T x_i + \sum_{i=1}^T x_i^2\right) &= \sum_{i=1}^T x_iy_i - \bar{y}\sum_{i=1}^T x_i \notag \\
		\beta_1 &= \frac{\sum_{i=1}^T x_iy_i - \bar{y}T\bar{x}}{-\bar{x}T\bar{x} + \sum_{i=1}^T x_i^2} \notag \\
		\beta_1 &= \frac{\sum_{i=1}^T (y_i-\bar{y})(x_i-\bar{x})}{\sum_{i=1}^T (x_i-\bar{x})^2} \notag
	\end{align}
	
	\section*{Aufgabe 5}
	Das Ziel der KQ-Schätzung ist, die Summe der Fehlerquadrate zu minimieren, also
	\begin{align}
		\sum_{i=1}^T u_t^2 = \sum_{i=1}^T (y_i - \beta x_i)^2 \to\min \notag
	\end{align}
	Wir lösen also
	\begin{align}
		\frac{\partial}{\partial\beta} \sum_{i=1}^T (y-\beta x_i)^2 &= 0 \notag \\
		-2\sum_{i=1}^T (y_i-\beta x_i) &=0 \notag \\
		\underbrace{\sum_{i=1}^T y_i}_{T\bar{y}} - \beta\underbrace{\sum_{i=1}^T x_i}_{T\bar{x}} &=0 \notag \\
		\beta T\bar{x} &= T\bar{y} \notag \\
		\beta\bar{x} &= \bar{y} \notag \\
		\beta &= \frac{\bar{y}}{\bar{x}} \notag
	\end{align}
	
	\section*{Aufgabe 6}
	\begin{itemize}
		\item $\beta_0$ und $\beta_1$ sind normalverteilt
		\item $\beta_0$ und $\beta_1$ korrelieren miteinander
		\item $\beta_0$ und $\beta_1$ sind innerhalb der Klasse der linearen unverzerrten Schätzer diejenigen mit der höchsten Präzision sind (BLUE-Eigenschaft)
	\end{itemize}

	\section*{Aufgabe 7}
	\begin{enumerate}[label=(\alph*)]
		\item $\text{Fiscal\_Indicator} = \beta_0 + \beta_1\cdot\text{net\_Budget}$ \\
		$\text{Fiscal\_Indicator}_i = \beta_0 + \beta_1\cdot\text{net\_Budget}_i + u_i$ \\
		\item[(b) + (c)] Die Annahmen sind
		\begin{itemize}
			\item linear in den Parametern: offensichtlich richtig
			\item der Regressor ist nichtstochastisch: offensichtlich richtig
			\item kein systematischer Restfehler: $\E(u_i) = -2.417369\cdot 10^{-17} \approx 0$
			\item keine Heteroskedastie: $\Var(u_i) = 0.5193321 < \infty$
			\item keine Autokorrelation: $\Cov(u_i,u_j) = \E(u_i\cdot u_j) = 1.039646\cdot 10^{-19} \approx 0$
			\item kein Zusammenhang zwischen Fehler und Regressor: $\Cov(u_i,\text{net\_Budget}_i) = 3.828068\cdot 10^{-17} \approx 0$
			\item mehr Beobachtungen als unbekannte Parameter: offensichtlich richtig
			\item die Regressorvariable muss eine positive (endliche) Varianz aufweisen: offensichtlich richtig
			\item das Regressionsmodell ist korrekt spezifiziert: Gibt es noch andere Größen, die den Fiscal Indicator beeinflussen?
			\item keine Multikollinearität/Die Matrix $X$ der Regressoren soll gut konditioniert sein: nicht wichtig, da es sich um ein Einfach-Regressionsmodell handelt
			\item Fehler werden als normalverteilt angenommen: der Shapiro-Wilk-Test auf Normalverteilung ergibt, dass es sich wahrscheinlich um eine Normalverteilung handelt.
		\end{itemize}
		\begin{lstlisting}[style=R]
# Datensatz einlesen
datensatz = read.csv2("EU_Fiskaldaten.csv")

# Lineares Modell berechnen
modell = lm(datensatz$Fiscal_Indicator ~ datensatz$net_Budget)

# Ueberpruefung, ob Modellannahmen gelten
mean(modell$residuals)
sd(modell$residuals)^2
products = 0
for (i in modell$residuals) {
 for (j in modell$residuals) {
  products = c(products, i*j)
 }
}
mean(products)
cov(datensatz$net_Budget,modell$residuals)
shapiro.test(modell$residuals)
		\end{lstlisting}
		\item[(d)] Anschauen und plotten der Daten
		\begin{lstlisting}[style=R]
datensatz
plot(datensatz$Fiscal_Indicator,datensatz$net_Budget)
		\end{lstlisting}
		\item[(e)] $\hat{\beta}_0 = 0.45143$, $\hat{\beta}_1 = 0.17752$
		\item[(f)] $R^2 = 0.3296587$ $\Rightarrow$ Das Modell ist keine wirklich gute Approximation für diese Daten.
		\begin{lstlisting}[style=R]
summary(modell)$r.squared
		\end{lstlisting}
	\end{enumerate}
	
\end{document}