\documentclass{article}

\usepackage{amsmath,amssymb}
\usepackage{tikz}
\usepackage{pgfplots}
\usepackage{xcolor}
\usepackage[left=2.1cm,right=3.1cm,bottom=3cm,footskip=0.75cm,headsep=0.5cm]{geometry}
\usepackage{enumerate}
\usepackage{enumitem}
\usepackage{marvosym}
\usepackage{tabularx}

\usepackage{listings}
\definecolor{lightlightgray}{rgb}{0.95,0.95,0.95}
\definecolor{lila}{rgb}{0.8,0,0.8}
\definecolor{mygray}{rgb}{0.5,0.5,0.5}
\definecolor{mygreen}{rgb}{0,0.8,0.26}
\lstdefinestyle{R} {language=R,morekeywords={confint,head}}
\lstset{language=R,
	basicstyle=\ttfamily,
	keywordstyle=\color{lila},
	commentstyle=\color{lightgray},
	stringstyle=\color{mygreen}\ttfamily,
	backgroundcolor=\color{white},
	showstringspaces=false,
	numbers=left,
	numbersep=10pt,
	numberstyle=\color{mygray}\ttfamily,
	identifierstyle=\color{blue},
	xleftmargin=.1\textwidth, 
	%xrightmargin=.1\textwidth,
	escapechar=§,
}

\usepackage[utf8]{inputenc}

\renewcommand*{\arraystretch}{1.4}

\newcolumntype{L}[1]{>{\raggedright\arraybackslash}p{#1}}
\newcolumntype{R}[1]{>{\raggedleft\arraybackslash}p{#1}}
\newcolumntype{C}[1]{>{\centering\let\newline\\\arraybackslash\hspace{0pt}}m{#1}}

\title{\textbf{Ökonometrie Grundlagen, Übung 4}}
\author{\textsc{Henry Haustein}}
\date{}

\begin{document}
	\maketitle
	
	Wir definieren uns die Matrizen
	\begin{lstlisting}[style=R]
X = matrix(c(2,3,7,-1,3,5),3,2, byrow = TRUE)
Y = matrix(c(5,3,6,-2,2,-5,1,0),2,4, byrow = TRUE)
Z = matrix(c(2,5,-3,1,-5,0,0,1,6,7,2,3),4,3, byrow = TRUE)
A = matrix(c(1,1,1,6,3,2,6,1,1),3,3, byrow = TRUE)
	\end{lstlisting}
	
	\section*{Aufgabe 1}
	Wir berechnen einfach beide Ausdrücke und vergleichen das Ergebnis
	\begin{lstlisting}[style=R]
(X %*% Y) %*% Z
X %*% (Y %*% Z)
	\end{lstlisting}
	ergibt
	\begin{align}
		\begin{pmatrix}
			-5 & 132 & 30 \\ -6 & 48 & 105 \\ -8 & 216 & 45
		\end{pmatrix} \notag
	\end{align}
	
	\section*{Aufgabe 2}
	Wir berechnen einfach beide Ausdrücke und vergleichen das Ergebnis
	\begin{lstlisting}[style=R]
t(X %*% Y)
t(Y) %*% t(X)
	\end{lstlisting}
	ergibt
	\begin{align}
		\begin{pmatrix}
			16 & 33 & 25 \\ -9 & 26 & -16 \\ 15 & 41 & 23 \\ -4 & -14 & -6
		\end{pmatrix} \notag
	\end{align}
	
	\section*{Aufgabe 3}
	Eine Matrix $A$ ist genau dann symmetrisch, wenn $A'=A$ gilt. Wir berechnen einfach beide Ausdrücke und vergleichen das Ergebnis
	\begin{lstlisting}[style=R]
t(X) %*% X
t(t(X) %*% X)
	\end{lstlisting}
	ergibt
	\begin{align}
		\begin{pmatrix}
			62 & 14 \\ 14 & 35
		\end{pmatrix} \notag
	\end{align}

	\section*{Aufgabe 4}
	Wir berechnen einfach beide Ausdrücke und vergleichen das Ergebnis
	\begin{lstlisting}[style=R]
solve(t(A))
t(solve(A))
	\end{lstlisting}
	ergibt
	\begin{align}
		\begin{pmatrix}
			-0.2 & -1.2 & 2.4 \\ 0 & 1 & -1 \\ 0.2 & -0.8 & 0.6
		\end{pmatrix} \notag
	\end{align}
	Wenn $A$ symmetrisch ist, dann gilt $A=A'$ und damit $A = A^{-1}$.

	\section*{Aufgabe 5}
	Eine Matrix $A$ ist genau dann idempotent, wenn $A=A^2$ gilt. Wir berechnen einfach beide Ausdrücke und vergleichen das Ergebnis
	\begin{enumerate}[label=(\alph*)]
		\item für $P$:
		\begin{lstlisting}[style=R]
P = X %*% solve(t(X) %*% X) %*% t(X)
P
P %*% P
		\end{lstlisting}
		ergibt
		\begin{align}
			\begin{pmatrix}
				0.2685 & 0.0193 & .04428 \\ 0.0193 & 0.9995 & -0.0117 \\ 0.4428 & -0.0117 & 0.7320
			\end{pmatrix} \notag
		\end{align}
		\item für $Q$: 
		\begin{lstlisting}[style=R]
eins = c(1,1,1)
I = diag(1,3,3)
Q = I - 1/3*eins %*% t(eins)
Q
Q %*% Q
		\end{lstlisting}
		ergibt
		\begin{align}
			\begin{pmatrix}
				0.6667 & -0.3333 & -0.3333 \\ -0.3333 & 0.6667 & -0.3333 \\ -0.3333 & -0.3333 & 0.6667
			\end{pmatrix} \notag
		\end{align}
		\item für $I-P$:
		\begin{lstlisting}[style=R]
I - P
(I - P) %*% (I - P)
		\end{lstlisting}
		ergibt
		\begin{align}
			\begin{pmatrix}
				0.7315 & -0.0193 & -0.4428 \\ -0.0193 & 0.0005 &  0.0117 \\ -0.4428 & 0.0117 & 0.2680
			\end{pmatrix} \notag
		\end{align}
		
	\end{enumerate}
	
	\section*{Aufgabe 6}
	\begin{enumerate}[label=(\alph*)]
		\item Um den Rank einer Matrix zu bestimmen, gibt es das Paket \texttt{Matrix}, welches ich installieren werde
		\begin{lstlisting}[style=R]
install.packages("Matrix")
library(Matrix)
rankMatrix(X)\\
rankMatrix(t(X) %*% X)
		\end{lstlisting}
		Die Matrizen haben den Rang 2.
		\item Eine Matrix $A$ heißt positiv definit, wenn für alle Vektoren $x\in\mathbb{R}^n$ gilt: $x^TAx > 0$. Der Nachweis davon ist mit dieser Definition sehr schwierig, weswegen ich eine äquivalente Definition mittels Eigenwerten (die sich leichter berechnen lassen) zurückgreife: Eine Matrix $A$ ist genau dann positiv definit, wenn alle Eigenwerte\footnote{Die zu einer Matrix $A$ gehörenden Eigenwerte $\lambda$ und Eigenvektoren $v$ sind dadurch bestimmt, dass sie folgende Gleichung erfüllen: $Av=\lambda v$. Die Lösung einer solchen Gleichung ist nicht leicht, aber machbar, wenn man ein paar Tricks kennt. Insbesondere kann R die Eigenwerte und Eigenvektoren mittels der Funktion \texttt{eigen()} berechnen.} echt größer als 0 sind.
		\begin{lstlisting}[style=R]
eigen(t(X) %*% X)
		\end{lstlisting}
		ergibt die Eigenwerte $\lambda_1=67.94865$ und $\lambda_2=29.05135$. Beide sind echt größer als Null und damit ist $X'X$ positiv definit.
 	\end{enumerate}
	
\end{document}