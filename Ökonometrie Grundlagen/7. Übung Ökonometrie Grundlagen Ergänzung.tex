\documentclass{article}

\usepackage{amsmath,amssymb}
\usepackage{tikz}
\usepackage{pgfplots}
\usepackage{xcolor}
\usepackage[left=2.1cm,right=3.1cm,bottom=3cm,footskip=0.75cm,headsep=0.5cm]{geometry}
\usepackage{enumerate}
\usepackage{enumitem}
\usepackage{marvosym}
\usepackage{tabularx}

\usepackage{listings}
\definecolor{lightlightgray}{rgb}{0.95,0.95,0.95}
\definecolor{lila}{rgb}{0.8,0,0.8}
\definecolor{mygray}{rgb}{0.5,0.5,0.5}
\definecolor{mygreen}{rgb}{0,0.8,0.26}
\lstdefinestyle{R} {language=R,morekeywords={confint,head}}
\lstset{language=R,
	basicstyle=\ttfamily,
	keywordstyle=\color{lila},
	commentstyle=\color{lightgray},
	stringstyle=\color{mygreen}\ttfamily,
	backgroundcolor=\color{white},
	showstringspaces=false,
	numbers=left,
	numbersep=10pt,
	numberstyle=\color{mygray}\ttfamily,
	identifierstyle=\color{blue},
	xleftmargin=.1\textwidth, 
	%xrightmargin=.1\textwidth,
	escapechar=§,
}

\usepackage[utf8]{inputenc}

\renewcommand*{\arraystretch}{1.4}

\newcolumntype{L}[1]{>{\raggedright\arraybackslash}p{#1}}
\newcolumntype{R}[1]{>{\raggedleft\arraybackslash}p{#1}}
\newcolumntype{C}[1]{>{\centering\let\newline\\\arraybackslash\hspace{0pt}}m{#1}}

\newcommand{\E}{\mathbb{E}}
\DeclareMathOperator{\rk}{rk}
\DeclareMathOperator{\Var}{Var}
\DeclareMathOperator{\Cov}{Cov}

\title{\textbf{Ökonometrie Grundlagen, Übung 7, Ergänzung zu Aufgabe 1}}
\author{\textsc{Henry Haustein}}
\date{}

\begin{document}
	\maketitle
	
	\section*{Aufgabe 1}
	Wir bilden die Lagrangefunktion $L=(y-X\hat{\beta})'(y-X\hat{\beta}) - \lambda(R\hat{\beta}-r)$. Die Ableitungen sind
	\begin{align}
		\frac{\partial L}{\partial \hat{\beta}} &= -X'y - X'y + 2(X'X)'\hat{\beta} - (\lambda R)' = 0 \notag \\
		\label{ast}
		&= -2X'y + 2X'X\hat{\beta} - R'\lambda' = 0 \tag{\textasteriskcentered} \\
		\frac{\partial L}{\partial\lambda} &= -(R\hat{\beta}-r)=0 \notag \\
		R\hat{\beta} &= r \notag
	\end{align}
	Multiplikation von \eqref{ast} mit $R(X'X)^{-1}$ liefert
	\begin{align}
		0 &= -2R(X'X)^{-1}X'y + 2R\underbrace{(X'X)^{-1}X'X}_{I}\hat{\beta} - R(X'X)^{-1}R'\lambda' \notag \\
		&= -2R(X'X)^{-1}X'y + 2R\hat{\beta} - R(X'X)^{-1}R'\lambda' \notag
	\end{align}
	Einsetzen von $\beta_{UR} = (X'X)^{-1}X'y$ und auflösen nach $\lambda'$
	\begin{align}
		0 &= -2R\beta_{UR} + 2R\hat{\beta} - R(X'X)^{-1}R'\lambda' \notag \\
		R(X'X)^{-1}R'\lambda' &= -2R\beta_{UR} + 2R\hat{\beta} \notag \\
		\lambda' &= \left[R(X'X)^{-1}R'\right]^{-1}(-2R\beta_{UR} + 2\underbrace{R\hat{\beta}}_{r}) \notag \\
		&= -2\left[R(X'X)^{-1}R'\right]^{-1}(R\beta_{UR} - r) \notag
	\end{align}
	Einsetzen in \eqref{ast} und Auflösen nach $\hat{\beta}$
	\begin{align}
		0 &= -2X'y + 2X'X\hat{\beta} - R'\left[-2\left[R(X'X)^{-1}R'\right]^{-1}(R\beta_{UR} - r)\right] \notag \\
		&= X'y - X'X\hat{\beta} - R'\left[R(X'X)^{-1}R'\right]^{-1}(R\beta_{UR} - r) \notag \\
		X'X\hat{\beta} &= X'y - R'\left[R(X'X)^{-1}R'\right]^{-1}(R\beta_{UR} - r) \notag \\
		\hat{\beta} &= \underbrace{(X'X)^{-1}X'y}_{\beta_{UR}} - (X'X)^{-1}R'\left[R(X'X)^{-1}R'\right]^{-1}(R\beta_{UR} - r) \notag \\
		&= \beta_{UR} - (X'X)^{-1}R'\left[R(X'X)^{-1}R'\right]^{-1}(R\beta_{UR} - r) \notag
	\end{align}
	
\end{document}