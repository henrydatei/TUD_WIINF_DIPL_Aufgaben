\documentclass{article}

\usepackage{amsmath,amssymb}
\usepackage{tikz}
\usepackage{pgfplots}
\usepackage{xcolor}
\usepackage[left=2.1cm,right=3.1cm,bottom=3cm,footskip=0.75cm,headsep=0.5cm]{geometry}
\usepackage{enumerate}
\usepackage{enumitem}
\usepackage{marvosym}
\usepackage{tabularx}
\usepackage{hyperref}

\usepackage{listings}
\definecolor{lightlightgray}{rgb}{0.95,0.95,0.95}
\definecolor{lila}{rgb}{0.8,0,0.8}
\definecolor{mygray}{rgb}{0.5,0.5,0.5}
\definecolor{mygreen}{rgb}{0,0.8,0.26}
\lstdefinestyle{R} {language=R,morekeywords={confint,head,fitdist,ks,test}}
\lstset{language=R,
	basicstyle=\ttfamily,
	keywordstyle=\color{lila},
	commentstyle=\color{lightgray},
	stringstyle=\color{mygreen}\ttfamily,
	backgroundcolor=\color{white},
	showstringspaces=false,
	numbers=left,
	numbersep=10pt,
	numberstyle=\color{mygray}\ttfamily,
	identifierstyle=\color{blue},
	xleftmargin=.1\textwidth, 
	%xrightmargin=.1\textwidth,
	escapechar=§,
}

\usepackage[utf8]{inputenc}

\renewcommand*{\arraystretch}{1.4}

\newcolumntype{L}[1]{>{\raggedright\arraybackslash}p{#1}}
\newcolumntype{R}[1]{>{\raggedleft\arraybackslash}p{#1}}
\newcolumntype{C}[1]{>{\centering\let\newline\\\arraybackslash\hspace{0pt}}m{#1}}

\newcommand{\E}{\mathbb{E}}
\DeclareMathOperator{\Var}{Var}
\DeclareMathOperator{\CDF}{CDF}

\title{\textbf{Statistik 2, Wiederholungsübung}}
\author{\textsc{Henry Haustein}}
\date{}

\begin{document}
	\maketitle
	
	\section*{Aufgabe 1}
	\begin{enumerate}[label=(\alph*)]
		\item Es gibt genau $\binom{49}{6}$ verschiedene Möglichkeiten beim 6-aus-49-Lotto. Wir wollen jetzt die Anzahl der für uns \textit{günstigen} Möglichkeiten (alle Zahlenkombinationen, wo es genau 5 Richtige gibt) bestimmen. Bei 5 Richtigen müssen von den 6 gezogenen Zahlen 5 richtig sein, es gibt also $\binom{6}{5}$ Möglichkeiten dafür. Da auf einem Tippschein 6 Zahlen angekreuzt werden müssen, muss eine dieser Zahlen falsch sein, also aus der Menge der nicht gezogenen Zahlen kommen. Dafür gibt es $\binom{43}{1}$ Möglichkeiten. Die Wahrscheinlichkeit für das Ereignis \textit{5 Richtige} liegt also bei
		\begin{align}
			\frac{\binom{6}{5}\cdot\binom{43}{1}}{\binom{49}{6}} = \frac{258}{13983816} = \frac{43}{2330636} \notag
		\end{align}
		\item Selbiges Vorgehen wie oben, nur hier können wir das Ereignis in zwei Ereignisse \textit{kein Richtiger} und \textit{1 Richtiger} aufteilen. Die gesuchte Wahrscheinlichkeit ist also
		\begin{align}
			\frac{\binom{6}{0}\cdot\binom{43}{6} + \binom{6}{1}\cdot\binom{43}{5}}{\binom{49}{6}} = \frac{11872042}{13983816} = \frac{848003}{998844} \notag
		\end{align}
	\end{enumerate}
	
	\section*{Aufgabe 2}
	\begin{enumerate}[label=(\alph*)]
		\item Die Ungleichung von Tschebyscheff gibt die untere Schranke für die Wahrscheinlichkeit an, dass $X\in [\mu-\varepsilon,\mu+\varepsilon]$. Hier ist $\mu=40$ und damit $\varepsilon=10$. Es folgt dann
		\begin{align}
			\mathbb{P}(\vert X-\mu\vert \le \varepsilon) &\ge 1\frac{\Var(X)}{\varepsilon^2} \notag \\
			&\ge 1- \frac{5^2}{10^2} \notag \\
			&\ge 0.75 \notag
		\end{align}
		\item Wir führen eine neue Zufallsvariable $\bar{X}_n$ ein, die dem Mittelwert von $n$ Messungen entspricht. Für diese Zufallsvariable gilt nun bezüglich Erwartungswert und Varianz:
		\begin{align}
			\E(\bar{X}_n) &= \E(X_i) = 40 \notag \\
			\Var(\bar{X}_n) &= \frac{1}{n}\cdot\Var(X_i) = \frac{25}{n} \notag
		\end{align}
		Mit der Ungleichung von Tschebyscheff (und $\varepsilon=10$) folgt damit
		\begin{align}
			\mathbb{P}(\vert \bar{X}_n-\mu\vert \le \varepsilon) &\ge 1-\frac{\Var(\bar{X}_n)}{\varepsilon^2} \notag \\
			&\ge 1- \frac{\frac{25}{n}}{10^2} \notag \\
			&\ge 1- \frac{\frac{25}{9}}{10^2} \notag \\
			&\ge \frac{35}{36} \notag
		\end{align}
		\item Standardisieren wir die Zufallsvariable zuerst: $\frac{X-40}{5}\sim \Phi$. Die Grenzen von 50 bzw. 30 müssen wir derselben Transformation unterziehen, sodass sich die die benötigte Wahrscheinlichkeit ergibt:
		\begin{align}
			\mathbb{P}(30 \le X \le 50) &= \Phi\left(\frac{50-40}{5}\right) - \Phi\left(\frac{30-40}{5}\right) \notag \\
			&= \Phi(2) - \Phi(-2) \notag \\
			&= 0.9545 \notag
		\end{align}
		Interessant ist, dass die konkreten Zahlen in dieser Aufgabe gar nicht so wichtig sind, denn es gilt allgemein:
		\begin{itemize}
			\item Im Intervall der Abweichung $\pm\sigma$ vom Erwartungswert sind 68,27 \% aller Messwerte zu finden
			\item Im Intervall der Abweichung $\pm2\sigma$ vom Erwartungswert sind 95,45 \% aller Messwerte zu finden
			\item Im Intervall der Abweichung $\pm3\sigma$ vom Erwartungswert sind 99,73 \% aller Messwerte zu finden
		\end{itemize}
		siehe dazu auch \url{https://de.wikipedia.org/wiki/Normalverteilung}
		\item Bringen wir die Ideen aus (b) und (c) zusammen und standardisieren wieder zuerst: $\frac{\bar{X}_n-\E(\bar{X}_n)}{\sqrt{\Var(\bar{X}_n)}} = \frac{\bar{X}_n-40}{\frac{5}{n}}$ und dann ergibt sich:
		\begin{align}
			\mathbb{P}(30 \le \bar{X}_n \le 50) &= \Phi\left(\frac{50-40}{\frac{5}{n}}\right) - \Phi\left(\frac{30-40}{\frac{5}{n}}\right) \notag \\
			&= \Phi(2n) - \Phi(-2n) \notag \\
			&= \Phi(18) - \Phi(-18) \notag \\
			&\approx 1 \notag
		\end{align}
		\item Die Breite eines Konfidenzintervalls ist
		\begin{align}
			2\cdot z_{1-\frac{\alpha}{2}}\frac{\sigma}{\sqrt{n}} \notag
		\end{align}
		Damit gilt
		\begin{align}
			2\cdot z_{1-\frac{\alpha}{2}}\frac{\sigma}{\sqrt{n}} &= 10 \notag \\
			\sqrt{n} &= z_{1-\frac{\alpha}{2}}\frac{\sigma}{5} \notag \\
			n &= \left(z_{1-\frac{\alpha}{2}}\frac{\sigma}{5}\right)^2 \notag \\
			&= \left(1.95996\frac{5}{5}\right)^2 \notag \\
			&= 3.8414 \notag \\
			&\approx 4 \notag
		\end{align}
		\item Das Konfidenzintervall für unbekannte Varianz ist gegeben durch
		\begin{align}
			KI &= \left[\bar{X} - t_{n-1;1-\frac{\alpha}{2}}\frac{S}{\sqrt{n}}; \bar{X} + t_{n-1;1-\frac{\alpha}{2}}\frac{S}{\sqrt{n}}\right] \notag \\
			&= \left[40 - 2.306\frac{\sqrt{60.5}}{\sqrt{9}}; 40 + 2.306\frac{\sqrt{60.5}}{\sqrt{9}}\right] \notag \\
			&= [34.0212;45.9788] \notag
		\end{align}
	\end{enumerate}

	\section*{Aufgabe 3}
	\begin{enumerate}[label=(\alph*)]
		\item Ableiten der Verteilungsfunktion liefert die Dichtefunktion:
		\begin{align}
			f(x) = \begin{cases}
				2a\cdot \exp(-2ax) & \text{für } x\ge 0 \\
				0 &\text{sonst}
			\end{cases} \notag
		\end{align}
		\item Aufstellen der Likelihood-Funktion:
		\begin{align}
			L &= \prod_{i=1}^{n} f(x_i) \notag \\
			&= \prod_{i=1}^n 2a\cdot\exp(-2ax) \notag \\
			&= (2a)^n\cdot\prod_{i=1}^n \exp(-2ax) \notag
		\end{align}
		Berechnen der log-Likelihood-Funktion, um das Produkt in eine Summe umzuwandeln:
		\begin{align}
			l &= n\cdot\log(2a) + \sum_{i=1}^n\log(\exp(-2ax)) \notag \\
			&= n\cdot\log(2a) - \sum_{i=1}^n 2ax \notag \\
			&= n\cdot\log(2a) - 2a\sum_{i=1}^{n} x_i \notag
		\end{align}
		Ableiten nach $a$ und Nullsetzen liefert den Likelihood-Schätzer für $a$:
		\begin{align}
			\frac{\partial l}{\partial a} &= 2n\frac{1}{2a} - 2\sum_{i=1}^n x_i =0 \notag \\
			\frac{n}{a} &= 2\sum_{i=1}^n x_i \notag \\
			a &= \frac{n}{2\sum_{i=1}^n x_i} \notag \\
			&= \frac{1}{2\bar{x}} \notag
		\end{align}
		\item Der Mittelwert $\bar{x}$ ist 5, also ist $a=\frac{1}{2\bar{x}}=\frac{1}{2\cdot 5} = \frac{1}{10}$
	\end{enumerate}

	\section*{Aufgabe 4}
	\begin{enumerate}[label=(\alph*)]
		\item Wir führen einen rechtsseitigen Test durch: \\
		$H_0: \mu \le 7.6$ \\
		$H_1: \mu > 7.6$ \\
		Die Teststatistik ergibt sich zu
		\begin{align}
			T &= \frac{\bar{X} - 7.6}{\sigma}\sqrt{n} \notag \\
			&= \frac{7.8-7.6}{0.5}\sqrt{12} \notag \\
			&= 1.3856 \notag
		\end{align}
		Der kritische Wert ist $z_{1-\alpha}= z_{0.95}=1.6449$ und damit kann die Nullhypothese nicht abgelehnt werden.
		\item Das Konfidenzintervall für unbekannte Varianz ist gegeben durch
		\begin{align}
			KI &= \left[\bar{X} - t_{n-1;1-\frac{\alpha}{2}}\frac{S}{\sqrt{n}}; \bar{X} + t_{n-1;1-\frac{\alpha}{2}}\frac{S}{\sqrt{n}}\right] \notag \\
			&= \left[7.8 - 2.1788\frac{0.6}{\sqrt{12}}; 7.8 + 2.1788\frac{0.6}{\sqrt{12}}\right] \notag \\
			&= [7.4226;8.1774] \notag
		\end{align}
		\item Beschäftigen wir uns zuerst mit der Operationscharakteristik. Das ist die Wahrscheinlichkeit, dass wir $H_0$ nicht ablehnen. Wir lehnen $H_0$ genau dann nicht ab, wenn $T < z_{1-\frac{\alpha}{2}}$ ist, also
		\begin{align}
			L(\mu) &= \mathbb{P}(T \le z_{1-\frac{\alpha}{2}}) \notag \\
			&= \mathbb{P}\left(\frac{\bar{X}-\mu_0}{\sigma}\sqrt{n} \le z_{1-\frac{\alpha}{2}}\right) \notag \\
			&= \mathbb{P}\left(\frac{\bar{X}-\mu_0}{\sigma}\sqrt{n} \le z_{1-\frac{\alpha}{2}} + \frac{\mu_0-\mu}{\sigma}\sqrt{n}\right) \notag \\
			&= \Phi\left(z_{1-\frac{\alpha}{2}} + \frac{\mu_0-\mu}{\sigma}\sqrt{n}\right) \notag \\
			&= \Phi\left(1.6449 + \frac{7.6-\mu}{0.5}\sqrt{12}\right) \notag
		\end{align}
		Die Gütefunktion ist dann $1-L(\mu)$.
		\item Dafür müssen wir die Gütefunktion an der Stelle $\mu=7.5$ berechnen:
		\begin{align}
			G(7.5) &= 1- L(7.5) \notag \\
			&= 1- \Phi\left(1.6449 + \frac{7.6-7.5}{0.5}\sqrt{12}\right) \notag \\
			&= 1- \Phi(2.3328) \notag \\
			&= 0.0098 \notag
		\end{align}
	\end{enumerate}

	\section*{Aufgabe 5}
	\begin{enumerate}[label=(\alph*)]
		\item Wir führen hier einen $\chi^2$-Anpassungstest durch:
		\begin{center}
			\begin{tabular}{c|ccc|c}
				Ereignis & 0 & 1 & 2 & $\Sigma$ \\
				\hline
				$S_i$ & 320 & 48 & 32 & 400 \\
				$p_i$ & 0.64 & 0.32 & 0.04 & 1 \\
				$np_i$ & 256 & 128 & 16 & 400
			\end{tabular}
		\end{center}
		Die Teststatistik ergibt sich zu $T=82$, der kritische Wert ist $\chi^2_{3-1;1-\alpha}=\chi^2_{2,0.9}=4.6052$, also wird die Nullhypothese abgelehnt.
		\item Wir führen hier einen $\chi^2$-Anpassungstest durch:
		\begin{center}
			\begin{tabular}{c|ccc|c}
				Ereignis & 0 & 1 & 2 & $\Sigma$ \\
				\hline
				$S_i$ & 320 & 48 & 32 & 400 \\
				$p_i$ & 0.7408 & 0.2222 & 0.0333 & $\approx$ 1 \\
				$np_i$ & 296.32 & 88.90 & 13.33 & $\approx$ 400
			\end{tabular}
		\end{center}
		Die Teststatistik ergibt sich zu $T=46.8333$, der kritische Wert ist $\chi^2_{3-1;1-\alpha}=\chi^2_{2,0.9}=4.6052$, also wird die Nullhypothese abgelehnt.
	\end{enumerate}

	\section*{Aufgabe 6}
	Die Variable \textit{Arbeitsgeschwindigkeit} hat metrisches Skalenniveau, die Variable \textit{Qualität} hat nur ordinales Skalenniveau. Wir können hier also einen Unabhängigkeitstest mittels Rangkorrelationskoeffizient machen. Die Ränge sind
	\begin{center}
		\begin{tabular}{c|cccccccc}
			Arbeiterin & A & B & C & D & E & F & F & H \\
			\hline
			R(Geschwindigkeit) & 1 & 2 & 3 & 4 & 5 & 6 & 7 & 8 \\
			R(Qualität) & 4 & 5 & 3 & 2 & 8 & 6 & 1 & 7
		\end{tabular}
	\end{center}
	Die Teststatistik ergibt sich zu $T=0.1819$ und der kritische Wert für den zweiseitigen Test $H_0: \rho=0$ vs. $H_1: \rho\neq 0$ ist $z_{1-\frac{\alpha}{2}}=1.9599$. Die Nullhypothese kann also nicht abgelehnt werden (hier ist $\rho = 0.1905$).
	
	\section*{Aufgabe 7}
	\begin{enumerate}[label=(\alph*)]
		\item Wir führen hier einen Zweistichproben-$t$-Test durch. Der Mittelwert der ersten Waage ist $\bar{X}_1=1000$, der Mittelwert der zweiten Waage ist $\bar{X}_2=1000.01$. Die Varianz der ersten Waage ist $S_1^2=5\cdot 10^{-5}$, die der zweiten Waage $S_2=0.00035$. Da die Varianzen bei beiden Waagen gleich sein sollen, muss diese erst geschätzt werden:
		\begin{align}
			\hat{\sigma}^2 &= \frac{(n_1-1)S_1^2 + (n_2-1)S_2^2}{n_1+n_2-2} \notag \\
			&= \frac{(5-1)\cdot 5\cdot 10^{-5} + (5-1)\cdot 0.00035}{5+5-2} \notag \\
			&= 2\cdot 10^{-4} \notag
		\end{align}
		Die Teststatistik ergibt sich dann zu
		\begin{align}
			T &= \frac{\bar{X}_1-\bar{X}_2}{\sqrt{\hat{\sigma}^2\left(\frac{1}{n_1}+\frac{1}{n_2}\right)}} \notag \\
			&= \frac{1000-1000.01}{\sqrt{2\cdot 10^{-4}\left(\frac{1}{5}+\frac{1}{5}\right)}} \notag \\
			&= -1.1180 \notag
		\end{align}
		Der kritische Wert ist $t_{n_1+n_2-2;1-\frac{\alpha}{2}}=t_{8;0.95}=1.8595$, die Nullhypothese kann damit nicht abgelehnt werden.
		\item Die Nullhypothese wird dann abgelehnt, wenn $\vert T\vert > t_{krit}$, also dann, wenn $t_{krit}$ 1.180 unterschreitet.
	\end{enumerate}
\end{document}