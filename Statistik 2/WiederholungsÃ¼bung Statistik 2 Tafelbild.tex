\documentclass{article}

\usepackage{amsmath,amssymb}
\usepackage{tikz}
\usepackage{xcolor}
\usepackage[left=2.1cm,right=3.1cm,bottom=3cm,footskip=0.75cm,headsep=0.5cm]{geometry}
\usepackage{enumerate}
\usepackage{enumitem}
\usepackage{marvosym}
\usepackage{tabularx}
\usepackage{pgfplots}
\pgfplotsset{compat=1.10}
\usepgfplotslibrary{fillbetween}
\usepackage{parskip}

\usepackage[utf8]{inputenc}

\renewcommand*{\arraystretch}{1.4}

\newcolumntype{L}[1]{>{\raggedright\arraybackslash}p{#1}}
\newcolumntype{R}[1]{>{\raggedleft\arraybackslash}p{#1}}
\newcolumntype{C}[1]{>{\centering\let\newline\\\arraybackslash\hspace{0pt}}m{#1}}

\DeclareMathOperator{\tr}{tr}
\DeclareMathOperator{\Var}{Var}
\DeclareMathOperator{\Cov}{Cov}
\DeclareMathOperator{\Cor}{Cor}
\newcommand{\E}{\mathbb{E}}

\title{\textbf{Statistik 2, Wiederholungsübung, Tafelbild}}
\author{\textsc{Henry Haustein}}
\date{}

\begin{document}
	\maketitle
	
	\section*{Aufgabe 1}
	Hypergeometrische Verteilung: Urne mit $N$ Kugeln, wovon $M$ günstig für uns sind, es werden $n$ Kugeln gezogen. $X$... Anzahl der gezogenen günstigen Kugeln
	\begin{align}
		\mathbb{P}(X=m) = \frac{\binom{M}{m}\cdot\binom{N-M}{n-m}}{\binom{N}{n}} \notag
	\end{align}
	
	\section*{Aufgabe 2}
	Ungleichung von Tschebyscheff:
	\begin{align}
		\mathbb{P}(\vert X-\E(X)\vert \le \varepsilon) \ge 1- \frac{\Var(X)}{\varepsilon^2} \notag
	\end{align}
	Konfidenzintervall für $\mu$, wenn $X$ normalverteilt und $\sigma$ bekannt
	\begin{align}
		KI = \bar{x} \pm \frac{\sigma}{\sqrt{n}}\cdot z_{1-\alpha/2} \notag
	\end{align}
	Konfidenzintervall für $\mu$, wenn $X$ normalverteilt und $\sigma$ nicht bekannt
	\begin{align}
		KI = \bar{x} \pm \frac{s}{\sqrt{n}}\cdot t_{n-1,1-\alpha/2} \notag
	\end{align}
	
	\section*{Aufgabe 3}
	Dichte $f(x)$ ist Ableitung der Verteilungsfunktion $F(x)$
	
	Likelihood-Funktion aufstellen, [Logarithmus bilden], ableiten und nullsetzen. Dann überprüfen, ob es sich wirklich um ein Maximum handelt:
	\begin{align}
		L = \prod_{i=1}^n f(x_i) \notag 
	\end{align}

	\section*{Aufgabe 4}
	Teststatistik T-Test:
	\begin{align}
		T = \frac{\bar{x} - \mu_0}{\sigma}\sqrt{n} \sim \mathcal{N}(0,1) \notag
	\end{align}

	Gütefunktion: Wahrscheinlichkeit, $H_0$ abzulehnen, in Abhängigkeit von $\mu$
	
	\section*{Aufgabe 5}
	$\chi^2$-Anpassungstest
	\begin{align}
		Q = \sum_{i=1}^{r} \frac{(S_i - np_i)^2}{np_i} \notag
	\end{align}

	\section*{Aufgabe 6}
	Test auf monotonen Zusammenhang
	\begin{align}
		T = \frac{1}{2}\log\left(\frac{1+ \hat{R}}{1-\hat{R}}\right)\sqrt{\frac{n-3}{1.06}} \notag
	\end{align}
	mit $\log(\cdot)$ ist hier $\ln(\cdot)$ gemeint!
	
	\section*{Aufgabe 7}
	verbundene Stichproben: Berechne $D = X - Y$ und teste $D=0$ vs. $D\neq 0$
	
\end{document}