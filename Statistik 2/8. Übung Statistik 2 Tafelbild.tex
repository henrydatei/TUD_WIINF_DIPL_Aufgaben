\documentclass{article}

\usepackage{amsmath,amssymb}
\usepackage{tikz}
\usepackage{xcolor}
\usepackage[left=2.1cm,right=3.1cm,bottom=3cm,footskip=0.75cm,headsep=0.5cm]{geometry}
\usepackage{enumerate}
\usepackage{enumitem}
\usepackage{marvosym}
\usepackage{tabularx}
\usepackage{pgfplots}
\pgfplotsset{compat=1.10}
\usepgfplotslibrary{fillbetween}
\usepackage{parskip}

\usepackage[utf8]{inputenc}

\renewcommand*{\arraystretch}{1.4}

\newcolumntype{L}[1]{>{\raggedright\arraybackslash}p{#1}}
\newcolumntype{R}[1]{>{\raggedleft\arraybackslash}p{#1}}
\newcolumntype{C}[1]{>{\centering\let\newline\\\arraybackslash\hspace{0pt}}m{#1}}

\DeclareMathOperator{\tr}{tr}
\DeclareMathOperator{\Var}{Var}
\DeclareMathOperator{\Cov}{Cov}
\DeclareMathOperator{\Cor}{Cor}
\newcommand{\E}{\mathbb{E}}

\title{\textbf{Statistik 2, Übung 8, Tafelbild}}
\author{\textsc{Henry Haustein}}
\date{}

\begin{document}
	\maketitle
	
	\section*{Aufgabe 1}
	Zweiseitige Tests für den Mittelwert (häufig $t$-Test genannt) ($\nearrow$ Formelsammlung II, Seite 33):
	\begin{align}
		T &= \frac{\mu - \mu_0}{\sigma}\sqrt{n} \qquad z_{krit} = \pm z_{1-\alpha/2} \qquad\sigma\text{ bekannt} \notag \\
		T &= \frac{\mu - \mu_0}{s}\sqrt{n} \qquad t_{krit} = \pm t_{n-1,1-\alpha/2} \qquad\sigma\text{ unbekannt} \notag
	\end{align}
	Bei einseitigen Tests wird $1-\alpha/2$ durch $1-\alpha$ ersetzt und einer der kritischen Werte verschwindet. Für $n\ge 100$ ist die $t$-Verteilung sehr ähnlich zur Normalverteilung, wir werden deswegen häufig die Quantile der Standardnormalverteilung nehmen.
	
	Fehler und $\alpha$-Niveau
	\begin{itemize}
		\item Fehler 1. Art: Entscheide mich für $H_1$, aber $H_0$ ist richtig $\to \alpha$ (Signifikanzniveau/Irrtumswahrscheinlichkeit ist die obere Schranke für den Fehler 1. Art)
		\item Fehler 2. Art: Entscheide mich für $H_0$, aber $H_1$ ist richtig $\to\beta$
	\end{itemize}
	Die Gütefunktion $G$ gibt die Wahrscheinlichkeit an, dass man $H_0$ ablehnt. Für einen rechtsseitigen Test ($H_0: \mu \le 83$, $H_1: \mu > 83$):
	\begin{align}
		G(\mu) &= \mathbb{P}(T > z_{krit}) \notag
	\end{align}
	

	\section*{Aufgabe 2}	
	Berechnung von $p$-values:
	\begin{itemize}
		\item $T<0\Rightarrow p$-value $= \Phi(T)$
		\item $T>0\Rightarrow p$-value $= 1-\Phi(T)$
	\end{itemize}
	
\end{document}