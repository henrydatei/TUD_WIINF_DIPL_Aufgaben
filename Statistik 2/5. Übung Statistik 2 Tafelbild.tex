\documentclass{article}

\usepackage{amsmath,amssymb}
\usepackage{tikz}
\usepackage{xcolor}
\usepackage[left=2.1cm,right=3.1cm,bottom=3cm,footskip=0.75cm,headsep=0.5cm]{geometry}
\usepackage{enumerate}
\usepackage{enumitem}
\usepackage{marvosym}
\usepackage{tabularx}
\usepackage{pgfplots}
\pgfplotsset{compat=1.10}
\usepgfplotslibrary{fillbetween}
\usepackage{parskip}

\usepackage[utf8]{inputenc}

\renewcommand*{\arraystretch}{1.4}

\newcolumntype{L}[1]{>{\raggedright\arraybackslash}p{#1}}
\newcolumntype{R}[1]{>{\raggedleft\arraybackslash}p{#1}}
\newcolumntype{C}[1]{>{\centering\let\newline\\\arraybackslash\hspace{0pt}}m{#1}}

\DeclareMathOperator{\tr}{tr}
\DeclareMathOperator{\Var}{Var}
\DeclareMathOperator{\Cov}{Cov}
\DeclareMathOperator{\Cor}{Cor}
\newcommand{\E}{\mathbb{E}}

\title{\textbf{Statistik 2, Übung 5, Tafelbild}}
\author{\textsc{Henry Haustein}}
\date{}

\begin{document}
	\maketitle
	
	\section*{Aufgabe 1}
	Kriterien der Güte:
	\begin{itemize}
		\item erwartungstreu: $\E(\hat{\theta}) = \theta$ $\to$ Bias: $\E(\hat{\theta}) - \theta$
		\item $MSE(\hat{\theta}) = \Var(\hat{\theta}) + Bias(\hat{\theta})^2$
		\item konsistent: $MSE(\hat{\theta})\to 0$ für $n\to\infty$
	\end{itemize}

	\section*{Aufgabe 2}
	Übersicht über Konfidenzintervalle in FS Seite 32
	\begin{align}
		KI(\mu) &= \bar{x} \pm z_{1-\alpha/2}\frac{\sigma}{\sqrt{n}} &\sigma^2\text{ bekannt} \notag \\
		KI(\mu) &= \bar{x} \pm t_{n-1,1-\alpha/2}\frac{S}{\sqrt{n}} &\sigma^2\text{ unbekannt} \notag \\
		KI(\sigma^2) &= \left[\frac{n\tilde{S}}{\chi^2_{n,1-\alpha/2}}; \frac{n\tilde{S}}{\chi^2_{n,\alpha/2}}\right] &\mu\text{ bekannt} \notag \\
		KI(\sigma^2) &= \left[\frac{(n-1)S}{\chi^2_{n,1-\alpha/2}}; \frac{(n-1)S}{\chi^2_{n,\alpha/2}}\right] &\mu\text{ unbekannt} \notag
	\end{align}
	Die $t$-Verteilung nähert sich asymptotisch an die Standardnormalverteilung an, d.h. asymptotisches KI $\Rightarrow$ $t_{n-1,1-\alpha/2}$ mit $z_{1-\alpha/2}$ ersetzen
	
	Die Werte von $z_\alpha$ stehen auf Seite 23, die Werte von $t_{n,\alpha}$ auf Seite 24 und $\chi^2_{n,\alpha}$ auf Seite 25f.
	
	Es gilt (Seite 21):
	\begin{align}
		z_\alpha = \Phi^{-1}(\alpha) \notag
	\end{align}
	
\end{document}