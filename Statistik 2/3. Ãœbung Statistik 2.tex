\documentclass{article}

\usepackage{amsmath,amssymb}
\usepackage{tikz}
\usepackage{pgfplots}
\usepackage{xcolor}
\usepackage[left=2.1cm,right=3.1cm,bottom=3cm,footskip=0.75cm,headsep=0.5cm]{geometry}
\usepackage{enumerate}
\usepackage{enumitem}
\usepackage{marvosym}
\usepackage{tabularx}

\usepackage[utf8]{inputenc}

\renewcommand*{\arraystretch}{1.4}

\newcolumntype{L}[1]{>{\raggedright\arraybackslash}p{#1}}
\newcolumntype{R}[1]{>{\raggedleft\arraybackslash}p{#1}}
\newcolumntype{C}[1]{>{\centering\let\newline\\\arraybackslash\hspace{0pt}}m{#1}}

\title{\textbf{Statistik 2, Übung 3}}
\author{\textsc{Henry Haustein}}
\date{}

\begin{document}
	\maketitle
	
	\section*{Aufgabe 1}
	Es gilt
	\begin{align}
		f(\mu) &= \frac{1}{\sqrt{2\pi\sigma^2}}\cdot\exp\left(-\frac{(x-\mu)^2}{2\sigma^2}\right) \notag\\
		L(\mu) &= \prod_{i=1}^{n} f(\mu) \notag \\
		&= \frac{1}{\sqrt{2\pi\sigma^2}}\cdot\prod_{i=1}^n \exp\left(-\frac{(x_i-\mu)^2}{2\sigma^2}\right) \notag \\
		&= \frac{1}{\sqrt{2\pi\sigma^2}}\cdot \exp\left(-\frac{\sum_{i=1}^{n}(x_i-\mu)^2}{2\sigma^2}\right) \notag \\
		&= \frac{1}{\sqrt{2\pi\sigma^2}}\cdot \exp\left(-\frac{1}{2\sigma^2}\cdot\sum_{i=1}^{n}(x_i-\mu)^2\right) \notag \\
		l(\mu) &= \log\left(\frac{1}{\sqrt{2\pi\sigma^2}}\right) - \frac{1}{2\sigma^2}\sum_{i=1}^n \underbrace{(x_i-\mu)^2}_{x_i^2-2\mu x_i + \mu^2} \notag \\
		&= \log\left(\frac{1}{\sqrt{2\pi\sigma^2}}\right) - \frac{1}{2\sigma^2}\left(\sum_{i=1}^n x_i^2 - 2\mu\sum_{i=1}^n x_i + n\mu^2\right) \notag \\
		&= \log\left(\frac{1}{\sqrt{2\pi\sigma^2}}\right) - \frac{1}{2\sigma^2}\sum_{i=1}^n x_i^2 + \frac{1}{2\sigma^2}2\mu\sum_{i=1}^n - \frac{1}{2\sigma^2}n\mu^2 \notag \\
		\frac{\partial l(\mu)}{\partial\mu} &= \frac{1}{2\sigma}2\sum_{i=1}^n x_i - \frac{1}{2\sigma^2}2n\mu \notag \\
		&= \underbrace{\frac{1}{\sigma^2}}_{\neq 0}\left(\sum_{i=1}^n - n\mu\right) =0 \notag \\
		0&= \sum_{i=1}^n x_i - n\mu \notag \\
		\mu &= \frac{\sum_{i=1}^n x_i}{n} = \bar{x} \notag
	\end{align}
	
	\section*{Aufgabe 2}
	\begin{enumerate}[label=(\alph*)]
		\item Am Fenster: 0.4; am Gang: 0.5; in der Mitte: 0.1
		\item Wir stellen gleich die Likelihood-Funktion auf, die als Produkt der Wahrscheinlichkeiten der einzelnen Ausprägungen definiert ist. Also
		\begin{align}
			L(p) &= \underbrace{p\cdot p\cdots p}_{9\text{ mal}} \cdot (1-2p) \notag \\
			&= p^9\cdot (1-2p) \notag \\
			l(p) &= 9\log(p) \cdot \log(1-2p) \notag \\
			\frac{\partial l(p)}{\partial p} &= \frac{9}{p} - \frac{2}{1-2p} =0 \notag \\
			\frac{9}{p} &= \frac{2}{1-2p} \notag \\
			p &= \left(\frac{1}{2}-p\right)\cdot 9 \notag \\
			&= 4.5-9p \notag \\
			10p &= 4.5 \notag \\
			p &= 0.45 \notag
		\end{align}
	\end{enumerate}

	\section*{Aufgabe 3}
	\begin{enumerate}[label=(\alph*)]
		\item Es gilt
		\begin{align}
			f(\lambda) &= \frac{\lambda^k}{k!}\cdot\exp(-\lambda) \notag \\
			L(\lambda) &= \prod_{i=1}^n f(\lambda) \notag \\
			&= \prod_{i=1}^n \frac{\lambda^{k_i}}{k_i!} \cdot\prod_{i=1}^n \exp(-\lambda) \notag \\
			l(\lambda) &= \sum_{i=1}^n \log\left(\frac{\lambda^{k_i}}{k_i!}\right) + \sum_{i=1}^n -\lambda \notag \\
			&= \log(\lambda)\sum_{i=1}^n k_i - \sum_{i=1}^n\log(k_i!) - n\lambda \notag \\
			\frac{\partial l(\lambda)}{\partial\lambda} &= \frac{1}{\lambda}\sum_{i=1}^n k_i - n=0 \notag \\
			\lambda &= \frac{\sum_{i=1}^n k_i}{n} = \bar{k} \notag
		\end{align}
		\item Wir berechnen zuerst $\lambda$
		\begin{align}
			\lambda &= \frac{1}{15}(3\cdot 1 + 2\cdot 4 + 8\cdot 5 + 2\cdot 7) \notag \\
			&= \frac{13}{3} \notag
		\end{align}
		Dann gilt
		\begin{align}
			\mathbb{P}(X=0) &= \frac{\left(\frac{13}{3}\right)^0}{0!} \exp\left(-\frac{13}{3}\right) \notag \\
			&= 0.013 \notag
		\end{align}
	\end{enumerate}

	\section*{Aufgabe 4}
	Wir schätzen den Parameter $\lambda$ wieder über die Maximum-Likelihood-Methode:
	\begin{align}
		f(\lambda) &= \lambda\exp(-\lambda x) \notag \\
		L(\lambda) &= \prod_{i=1}^n f(\lambda) \notag \\
		&= \prod_{i=1}^n \lambda \cdot \prod_{i=1}^n \exp(-\lambda x_i) \notag \\
		&= \lambda^n \cdot\exp\left(-\lambda\sum_{i=1}^n x_i\right) \notag \\
		l(\lambda) &= n\log(\lambda) -\lambda\sum_{i=1}^n x_i \notag \\
		\frac{\partial l(\lambda)}{\partial\lambda} &= \frac{n}{\lambda} - \sum_{i=1}^n x_i =0 \notag \\
		\frac{n}{\lambda} &= \sum_{i=1}^n x_i \notag \\
		\frac{\lambda}{n} &= \frac{1}{\sum_{i=1}^n x_i} \notag \\
		\lambda &= \frac{n}{\sum_{i=1}^n x_i} = \frac{1}{\bar{x}} \notag
	\end{align}
	Mit den gegebenen Daten können wir $\lambda$ nun ermitteln:
	\begin{align}
		\lambda &= \frac{1}{\bar{x}} \notag \\
		&= \frac{1}{\frac{1}{8}(10+8+9+5+4+7+9+5)} \notag \\
		&= \frac{8}{57} = 0.14 \notag
	\end{align}
	
\end{document}