\documentclass{article}

\usepackage{amsmath,amssymb}
\usepackage{tikz}
\usepackage{xcolor}
\usepackage[left=2.1cm,right=3.1cm,bottom=3cm,footskip=0.75cm,headsep=0.5cm]{geometry}
\usepackage{enumerate}
\usepackage{enumitem}
\usepackage{marvosym}
\usepackage{tabularx}
\usepackage{pgfplots}
\pgfplotsset{compat=1.10}
\usepgfplotslibrary{fillbetween}
\usepackage{parskip}

\usepackage[utf8]{inputenc}

\renewcommand*{\arraystretch}{1.4}

\newcolumntype{L}[1]{>{\raggedright\arraybackslash}p{#1}}
\newcolumntype{R}[1]{>{\raggedleft\arraybackslash}p{#1}}
\newcolumntype{C}[1]{>{\centering\let\newline\\\arraybackslash\hspace{0pt}}m{#1}}

\DeclareMathOperator{\tr}{tr}
\DeclareMathOperator{\Var}{Var}
\DeclareMathOperator{\Cov}{Cov}
\DeclareMathOperator{\Cor}{Cor}
\newcommand{\E}{\mathbb{E}}

\title{\textbf{Statistik 2, Übung 11, Tafelbild}}
\author{\textsc{Henry Haustein}}
\date{}

\begin{document}
	\maketitle
	
	\section*{Aufgabe 1}
	ML-Schätzer der Poisson-Verteilung: $\hat{\lambda} = \bar{x}$. Wahrscheinlichkeitsfunktion (Poisson-Verteilung ist diskret)
	\begin{align}
		P(X = x) = \frac{\lambda^x}{x!}\exp(-\lambda) \notag
	\end{align}
	
	$\chi^2$-Anpassungstest
	\begin{align}
		H_0: X \sim F_0 \qquad&\text{vs.}\qquad H_1: X \not\sim F_0 \notag \\
		Q &= \sum_{i=1}^r \frac{(S_i - np_i)^2}{np_i} \notag
	\end{align}
	kritischer Wert $\chi^2_{r-l-1,1-\alpha}$. Sollte in einer Klasse $S_i < 5$ sein, so müssen Klassen zusammengefasst werden!
	
	\section*{Aufgabe 2}
	KS-Test verlangt Supremum einer Menge: Supremum ist die kleinste obere Schranke einer Menge, bei endlichen Mengen Supremum = Maximum. Bei unendlichen Mengen: Supremum = "Grenzwert" des Maximums
	
	KS-Test:
	\begin{align}
		H_0: X \sim F_0 \qquad&\text{vs.}\qquad H_1: X \not\sim F_0 \notag \\
		D &= \sup_{x\in\mathbb{R}} \vert \hat{F}(x) - F_0(x)\vert \notag
	\end{align}
	kritischer Wert: $c_{1-\alpha}$ (tabelliert)
	
	Für Aufgabe (b): $\mathbb{P}(a < X < b) = F(b) - F(a)$
	
\end{document}