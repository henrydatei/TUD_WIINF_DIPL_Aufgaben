\documentclass{article}

\usepackage{amsmath,amssymb}
\usepackage{tikz}
\usepackage{xcolor}
\usepackage[left=2.1cm,right=3.1cm,bottom=3cm,footskip=0.75cm,headsep=0.5cm]{geometry}
\usepackage{enumerate}
\usepackage{enumitem}
\usepackage{marvosym}
\usepackage{tabularx}
\usepackage{pgfplots}
\pgfplotsset{compat=1.10}
\usepgfplotslibrary{fillbetween}

\usepackage[utf8]{inputenc}

\renewcommand*{\arraystretch}{1.4}

\newcolumntype{L}[1]{>{\raggedright\arraybackslash}p{#1}}
\newcolumntype{R}[1]{>{\raggedleft\arraybackslash}p{#1}}
\newcolumntype{C}[1]{>{\centering\let\newline\\\arraybackslash\hspace{0pt}}m{#1}}

\DeclareMathOperator{\tr}{tr}
\DeclareMathOperator{\Var}{Var}
\DeclareMathOperator{\Cov}{Cov}

\title{\textbf{Statistik 2, Übung 1}}
\author{\textsc{Henry Haustein}}
\date{}

\begin{document}
	\maketitle
	
	\section*{Aufgabe 1}
	\begin{enumerate}[label=(\alph*)]
		\item Sei $X$ die Anzahl der erwischten Schwarzfahrer. Es ist sinnvoller, zuerst die Wahrscheinlichkeit für das Gegenereignis \textit{niemand wird beim Schwarzfahren erwischt} $(X=0)$ zu berechnen, um dann $\mathbb{P}(X\ge 1) = 1 - \mathbb{P}(X=0)$ zu berechnen:
		\begin{align}
			\mathbb{P}(X=0) = 0.97^{10} = 0.7374 \notag
		\end{align}
		Also ist $\mathbb{P}(X\ge 1) = 1 - 0.7374 = 0.2626$.
		\item Damit der Kontrolleur erst bei der 10. Kontrolle einen Schwarzfahrer findet, muss er neunmal keinen erwischen und beim 10. mal einen Schwarzfahrer finden. Die Wahrscheinlichkeit dafür ist
		\begin{align}
			0.97^9 \cdot 0.03 = 0.0228 \notag
		\end{align}
	\end{enumerate}

	\section{Aufgabe 2}
	\begin{enumerate}[label=(\alph*)]
		\item Die Dichte $f$ und Verteilungsfunktion $F$ einer Exponentialverteilung sind definiert als
		\begin{align}
			f(x) &= \lambda\cdot e^{-\lambda x} \notag \\
			F(x) &= \int_{0}^{x} f(t)\,\mathrm{d}t = 1- e^{-\lambda x} \notag
		\end{align}
		\begin{center}
			\begin{tikzpicture}[scale=0.9]
				\begin{axis}[
					xmin=0, xmax=20, xlabel=$x$,
					ymin=0, ymax=0.2, ylabel=$f(x)$,
					samples=400,
					axis x line=middle,
					axis y line=middle,
					domain=0:20,
					]
					\addplot[mark=none,smooth,blue] {0.2 * exp(-0.2*x)};
					
				\end{axis}
			\end{tikzpicture}
			\begin{tikzpicture}[scale=0.9]
				\begin{axis}[
					xmin=0, xmax=20, xlabel=$x$,
					ymin=0, ymax=1, ylabel=$F(x)$,
					samples=400,
					axis x line=middle,
					axis y line=middle,
					domain=0:20,
					]
					\addplot[mark=none,smooth,red] {1-exp(-0.2*x)};
					
				\end{axis}
			\end{tikzpicture}
		\end{center}
		\item $F(3) = 0.4512$
		\item $1-F(3) = 0.5488$
		\item $F(3) - F(2) = 0.1215$
		\item Erwartungswert und Varianz der Exponentialverteilung sind definiert als
		\begin{align}
			\text{Erwartungswert} &= \frac{1}{\lambda} = 5 \notag \\
			\text{Varianz} &= \frac{1}{\lambda^2} = 25 \notag
		\end{align}
	\end{enumerate}

	\section*{Aufgabe 3}
	\begin{enumerate}[label=(\alph*)]
		\item Die Dichte $f$ einer Normalverteilung ist
		\begin{align}
			f(x) = \frac{1}{\sqrt{2\pi\sigma}}\cdot\exp\left(-\frac{(x-\mu)^2}{2\sigma^2}\right) \notag
		\end{align}
		\begin{center}
			\begin{tikzpicture}
				\begin{axis}[
					xmin=-1, xmax=4, xlabel=$x$,
					ymin=0, ymax=0.7, ylabel=$f(x)$,
					samples=400,
					axis x line=middle,
					axis y line=middle,
					domain=-1:4,
					]
					\addplot[mark=none,smooth,blue] {1/sqrt(2*pi*0.36) * exp(-(x-1.5)^2/(2*0.36^2))};
					
				\end{axis}
			\end{tikzpicture}
		\end{center}
		\item Das Einzeichnen der Fläche sollte eigentlich ziemlich logisch sein, ich lasse es deswegen weg
		\begin{enumerate}[label=(\roman*)]
			\item $\int_{0}^{1.6} f(x)\,\mathrm{d}x = 0.6094$
			\item $\int_{0}^{1.4} f(x)\,\mathrm{d}x = 0.3906$
			\item $\int_{1.4}^{\infty} f(x)\,\mathrm{d}x = 0.6094$
			\item $1-\int_{1.14}^{1.9} f(x)\,\mathrm{d}x = 0.2919$
			\item $\int_{1.75}^{\infty} f(x)\,\mathrm{d}x = 0.2437$
			\item $\int_{1.3}^{2} f(x)\,\mathrm{d}x = 0.6283$
		\end{enumerate}
		\item Der 10\%-Quantilswert sagt aus, bis zu welchem $x$ 10\% aller Beobachtungen liegen werden.
		\begin{align}
			\int_{-\infty}^{x} f(t)\,\mathrm{d}t = 0.1 \notag
		\end{align}
		Für solche Berechnungen gibt es die Quantilsfunktion (hier für die Normalverteilung: \textcolor{blue}{https://en.wikipedia.org\\/wiki/Normal\_distribution}). Setzt man die Werte ein und berechnet erf$^{-1}(-0.8)$, so ergibt sich
		\begin{align}
			x = \mu-1.2816\sigma = 1.5 - 1.2816\cdot 0.36 = 1.0386 \notag
		\end{align}
		Das heißt, 10\% der beobachteten Preise von Erdbeeren wird unter 1.0386 Euro liegen.
	\end{enumerate}
	
\end{document}