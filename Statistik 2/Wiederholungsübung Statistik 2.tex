\documentclass{article}

\usepackage{amsmath,amssymb}
\usepackage{tikz}
\usepackage{pgfplots}
\usepackage{xcolor}
\usepackage[left=2.1cm,right=3.1cm,bottom=3cm,footskip=0.75cm,headsep=0.5cm]{geometry}
\usepackage{enumerate}
\usepackage{enumitem}
\usepackage{marvosym}
\usepackage{tabularx}
\usepackage{hyperref}

\usepackage{listings}
\definecolor{lightlightgray}{rgb}{0.95,0.95,0.95}
\definecolor{lila}{rgb}{0.8,0,0.8}
\definecolor{mygray}{rgb}{0.5,0.5,0.5}
\definecolor{mygreen}{rgb}{0,0.8,0.26}
\lstdefinestyle{R} {language=R,morekeywords={confint,head,fitdist,ks,test}}
\lstset{language=R,
	basicstyle=\ttfamily,
	keywordstyle=\color{lila},
	commentstyle=\color{lightgray},
	stringstyle=\color{mygreen}\ttfamily,
	backgroundcolor=\color{white},
	showstringspaces=false,
	numbers=left,
	numbersep=10pt,
	numberstyle=\color{mygray}\ttfamily,
	identifierstyle=\color{blue},
	xleftmargin=.1\textwidth, 
	%xrightmargin=.1\textwidth,
	escapechar=§,
}

\usepackage[utf8]{inputenc}

\renewcommand*{\arraystretch}{1.4}

\newcolumntype{L}[1]{>{\raggedright\arraybackslash}p{#1}}
\newcolumntype{R}[1]{>{\raggedleft\arraybackslash}p{#1}}
\newcolumntype{C}[1]{>{\centering\let\newline\\\arraybackslash\hspace{0pt}}m{#1}}

\newcommand{\E}{\mathbb{E}}
\DeclareMathOperator{\Var}{Var}
\DeclareMathOperator{\CDF}{CDF}

\title{\textbf{Statistik 2, Wiederholungsübung}}
\author{\textsc{Henry Haustein}}
\date{}

\begin{document}
	\maketitle
	
	\section*{Aufgabe 1}
	\begin{enumerate}[label=(\alph*)]
		\item Es gibt genau $\binom{49}{6}$ verschiedene Möglichkeiten beim 6-aus-49-Lotto. Wir wollen jetzt die Anzahl der für uns \textit{günstigen} Möglichkeiten (alle Zahlenkombinationen, wo es genau 5 Richtige gibt) bestimmen. Bei 5 Richtigen müssen von den 6 gezogenen Zahlen 5 richtig sein, es gibt also $\binom{6}{5}$ Möglichkeiten dafür. Da auf einem Tippschein 6 Zahlen angekreuzt werden müssen, muss eine dieser Zahlen falsch sein, also aus der Menge der nicht gezogenen Zahlen kommen. Dafür gibt es $\binom{43}{1}$ Möglichkeiten. Die Wahrscheinlichkeit für das Ereignis \textit{5 Richtige} liegt also bei
		\begin{align}
			\frac{\binom{6}{5}\cdot\binom{43}{1}}{\binom{49}{6}} = \frac{258}{13983816} = \frac{43}{2330636} \notag
		\end{align}
		\item Selbiges Vorgehen wie oben, nur hier können wir das Ereignis in zwei Ereignisse \textit{kein Richtiger} und \textit{1 Richtiger} aufteilen. Die gesuchte Wahrscheinlichkeit ist also
		\begin{align}
			\frac{\binom{6}{0}\cdot\binom{43}{6} + \binom{6}{1}\cdot\binom{43}{5}}{\binom{49}{6}} = \frac{11872042}{13983816} = \frac{848003}{998844} \notag
		\end{align}
	\end{enumerate}
	
	\section*{Aufgabe 2}
	\begin{enumerate}[label=(\alph*)]
		\item Die Ungleichung von Tschebyscheff gibt die untere Schranke für die Wahrscheinlichkeit an, dass $X\in [\mu-\varepsilon,\mu+\varepsilon]$. Hier ist $\mu=40$ und damit $\varepsilon=10$. Es folgt dann
		\begin{align}
			\mathbb{P}(\vert X-\mu\vert \le \varepsilon) &\ge 1\frac{\Var(X)}{\varepsilon^2} \notag \\
			&\ge 1- \frac{5^2}{10^2} \notag \\
			&\ge 0.75 \notag
		\end{align}
		\item 
	\end{enumerate}

	\section*{Aufgabe 3}
\end{document}