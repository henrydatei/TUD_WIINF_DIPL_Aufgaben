\documentclass{article}

\usepackage{amsmath,amssymb}
\usepackage{tikz}
\usepackage{pgfplots}
\usepackage{xcolor}
\usepackage[left=2.1cm,right=3.1cm,bottom=3cm,footskip=0.75cm,headsep=0.5cm]{geometry}
\usepackage{enumerate}
\usepackage{enumitem}
\usepackage{marvosym}
\usepackage{tabularx}
\usepackage{hyperref}

\usepackage{listings}
\definecolor{lightlightgray}{rgb}{0.95,0.95,0.95}
\definecolor{lila}{rgb}{0.8,0,0.8}
\definecolor{mygray}{rgb}{0.5,0.5,0.5}
\definecolor{mygreen}{rgb}{0,0.8,0.26}
\lstdefinestyle{R} {language=R,morekeywords={confint,head,fitdist,ks,test}}
\lstset{language=R,
	basicstyle=\ttfamily,
	keywordstyle=\color{lila},
	commentstyle=\color{lightgray},
	stringstyle=\color{mygreen}\ttfamily,
	backgroundcolor=\color{white},
	showstringspaces=false,
	numbers=left,
	numbersep=10pt,
	numberstyle=\color{mygray}\ttfamily,
	identifierstyle=\color{blue},
	xleftmargin=.1\textwidth, 
	%xrightmargin=.1\textwidth,
	escapechar=§,
}

\usepackage[utf8]{inputenc}

\renewcommand*{\arraystretch}{1.4}

\newcolumntype{L}[1]{>{\raggedright\arraybackslash}p{#1}}
\newcolumntype{R}[1]{>{\raggedleft\arraybackslash}p{#1}}
\newcolumntype{C}[1]{>{\centering\let\newline\\\arraybackslash\hspace{0pt}}m{#1}}

\newcommand{\E}{\mathbb{E}}
\DeclareMathOperator{\Var}{Var}
\DeclareMathOperator{\CDF}{CDF}

\title{\textbf{Statistik 2, Übung 12}}
\author{\textsc{Henry Haustein}}
\date{}

\begin{document}
	\maketitle
	
	\section*{Aufgabe 1}
	\begin{enumerate}[label=(\alph*)]
		\item Der MLE-Schätzer für $\hat{\mu}$ ist der Mittelwert $\bar{x}$ der Stichprobe
		\begin{align}
			\hat{\mu} &= \bar{x} = \frac{1}{20}(1.45+1.30+1.31+\dots+1.44) \notag \\
			&= \frac{1}{20}\cdot 29.15 \notag \\
			&= 1.4575 \notag
		\end{align}
		Der MLE-Schätzer für $\hat{\sigma}$ ist gegeben durch
		\begin{align}
			\hat{\sigma} &= \sqrt{\frac{\sum_{i=}^{n} (x_i-\bar{x})^2}{n}} \notag \\
			&= 0.7848941 \notag
		\end{align}
		Wenn man statt durch $n$ durch $n-1$ teilt, ergibt sich eine andere Schätzung für $\hat{\sigma} = 0.8052844$. Siehe \url{https://en.wikipedia.org/wiki/Unbiased_estimation_of_standard_deviation} für mehr Informationen.
		\item Eine Messung in der Grafik ergibt, dass der Abstand zwischen 0 und 1 auf der $y$-Achse genau 4.49 cm ist. Der maximale Abstand zwischen der empirischen Verteilungsfunktion (= Teststatistik $D$) ist 1.66 cm. Es gilt:
		\begin{align}
			\frac{1}{4.49\text{ cm}} &= \frac{D}{1.66\text{ cm}} \notag \\
			D &= \frac{1.66\text{ cm}}{4.49\text{ cm}} \notag \\
			&= 0.3697105 \notag
		\end{align}
		Der kritische Wert kann in der Tabelle abgelesen werden und ist 0.294. Die Nullhypothese wird also abgelehnt. Eine Durchführung dieses Test mittels R ergibt ein genaueres Ergebnis:
		\begin{lstlisting}[style=R]
x = c(1.45,1.30,1.31,1.18,1.20,1.70,1.22,1.30,1.26,1.23,0.95,
4.80,1.28,1.22,1.22,1.12,1.46,1.00,1.51,1.44)
# eventuell install.packages("fitdistrplus")
library(fitdistrplus)
fitdistrplus::fitdist(x,pnorm,method = "mle")
			
ks.test(x,'pnorm',1.4575,0.7848941)
		\end{lstlisting}
		\item Der Ausreißer war ein Fußgänger, der sehr schnell über die Ampel gegangen ist. Wenn man allerdings nach einer Mindest-Freigabezeit fragt, interessiert man sich tendenziell für den langsamsten Fußgänger.
	\end{enumerate}
	
	\section*{Aufgabe 2}
	\begin{enumerate}[label=(\alph*)]
		\item Es gilt für die theoretischen Quantile $v_i = \Phi^{-1}\left(\frac{i-0.5}{n}\right)$, also
		\begin{center}
			\begin{tabular}{c|ccccccc}
				$i$ & 1 & 2 & 3 & 4 & 5 & 6 & 7 \\
				\hline
				gemessene Quantile & 10 & 24 & 30 & 30 & 35 & 78 & 81 \\
				theoretische Quantile & $-1.4652$ & $-0.7916$ & $-0.3661$ & 0 & $0.3661$ & $0.7916$ & $1.4652$
			\end{tabular}
		\end{center}
		\begin{center}
			\begin{tikzpicture}
			\begin{axis}[
			xmin=-1.4652, xmax=1.4652, xlabel={theoretische Quantile},
			ymin=10, ymax=81, ylabel={gemessene Quantile},
			samples=400,
			axis x line=bottom,
			axis y line=left,
			domain=-1.4652:1.4652,
			]
			\addplot[mark=x,blue, only marks] coordinates {
				(-1.4652,10)
				(-0.7916,24)
				(-0.3661,30)
				(0,30)
				(0.3661,35)
				(0.7916,78)
				(1.4652,81)
			};
			\addplot[smooth,red,mark=none] {41.14 + 27.38*x};
			\end{axis}
			\end{tikzpicture} \\
			\textcolor{red}{$41.14 + 27.38\cdot x$}
		\end{center}
		\item Die Teststatistik ergibt sich zu $W=0.83588$, der kritische Wert ist 0.803. Die Nullhypothese wird nicht abgelehnt.
		\item Der $p$-Wert ist recht klein, was Zweifel an der Nullhypothese weckt. Die Nullhypothese wird also abgelehnt, was man auch gut im QQ-Plot sieht: Die Punkte liegen nicht auf einer Geraden.
	\end{enumerate}

	\section*{Aufgabe 3}
	Wir testen hier auf eine diskrete Verteilung, das heißt wir können nicht den Kolmogorov-Smirnov-Test anwenden, sondern müssen auf den $\chi^2$-Anpassungstest zurückgreifen.
	\begin{center}
		\begin{tabular}{c|ccccccccccc|c}
			 & 0 & 1 & 2 & 3 & 4 & 5 & 6 & 7 & 8 & 9 & $>9$ & $\Sigma$ \\
			 \hline
			$S_i$ & 18 & 24 & 56 & 63 & 61 & 39 & 26 & 6 & 5 & 2 & 0 & 300 \\
			$p_i$ & 0.0333 & 0.1135 & 0.1929 & 0.2186 & 0.1858 & 0.1264 & 0.0716 & 0.0348 & 0.0148 & 0.0056 & 0.0027 & 1 \\
			$np_i$ & 9.99
& 34.05
& 57.87
& 65.58
& 55.74
& 37.92
& 21.48
& 10.44
& 4.44
& 1.68
& 0.81 & 300
		\end{tabular}
	\end{center}
	Die Teststatistik ergibt sich zu 13.8588 und der kritische Wert ist 15.5073, also kann die Nullhypothese nicht abgelehnt werden.
\end{document}