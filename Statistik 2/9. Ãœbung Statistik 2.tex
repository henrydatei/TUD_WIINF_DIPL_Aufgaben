\documentclass{article}

\usepackage{amsmath,amssymb}
\usepackage{tikz}
\usepackage{pgfplots}
\usepackage{xcolor}
\usepackage[left=2.1cm,right=3.1cm,bottom=3cm,footskip=0.75cm,headsep=0.5cm]{geometry}
\usepackage{enumerate}
\usepackage{enumitem}
\usepackage{marvosym}
\usepackage{tabularx}

\usepackage[utf8]{inputenc}

\renewcommand*{\arraystretch}{1.4}

\newcolumntype{L}[1]{>{\raggedright\arraybackslash}p{#1}}
\newcolumntype{R}[1]{>{\raggedleft\arraybackslash}p{#1}}
\newcolumntype{C}[1]{>{\centering\let\newline\\\arraybackslash\hspace{0pt}}m{#1}}

\newcommand{\E}{\mathbb{E}}
\DeclareMathOperator{\Var}{Var}
\DeclareMathOperator{\CDF}{CDF}

\title{\textbf{Statistik 2, Übung 9}}
\author{\textsc{Henry Haustein}}
\date{}

\begin{document}
	\maketitle
	
	\section*{Aufgabe 1}
	\begin{enumerate}[label=(\alph*)]
		\item Wir haben hier unabhängige Stichproben und ungleiche Varianzen, wir führen also den Welch-Test durch: \\
		$H_0$: $\mu_A \le \mu_B$ \\
		$H_1$: $\mu_A > \mu_B$ \\
		Die Teststatistik ergibt sich zu
		\begin{align}
			T = \frac{140-134}{\sqrt{\frac{51}{16} + \frac{46}{11}}} = 2.2102 \notag
		\end{align}
		Die Berechnung des kritischen Wertes ist ein bisschen komplizierter:
		\begin{align}
			R &= \frac{11\cdot 51}{16\cdot 46} = 0.7622 \notag \\
			df &= \left\lfloor\frac{(1+R)^2}{\frac{R^2}{16-1} + \frac{1}{11-1}}\right\rfloor = \lfloor 22.3844\rfloor = 22 \notag
		\end{align}
		Der kritische Wert ist nun $t_{22;0.99}=2.50832$. Es gilt $T<t_{22;0.99}$ und damit kann $H_0$ nicht abgelehnt werden.
		\item Wir führen einen $F$-Test durch: \\
		$H_0$: $\sigma_A^2 = \sigma_B^2$ \\
		$H_1$: $\sigma_A^2 \neq \sigma_B^2$ \\
		Die Teststatistik ist
		\begin{align}
			T = \frac{51}{46} = 1.1087 \notag
		\end{align}
		Die kritischen Werte sind
		\begin{itemize}
			\item $F_{15,10;0.01}=0.262816$
			\item $F_{15,10;0.99}=4.55814$
		\end{itemize}
		Da die Teststatistik zwischen den beiden kritischen Werten liegt, kann $H_0$ nicht abgelehnt werden.
		\item Wir führen nun einen Zweistichprobentest durch: \\
		$H_0$: $\mu_A \le \mu_B$ \\
		$H_1$: $\mu_A > \mu_B$ \\
		Aber zuerst müssen wir die gemeinsame Varianz schätzen:
		\begin{align}
			\tilde{\sigma}^2 = \frac{(16-1)\cdot 51 + (11-1)\cdot 46}{16+11-2} = 49 \notag
		\end{align}
		Dann berechnet sich die Teststatistik zu
		\begin{align}
			T = \frac{140-134}{\sqrt{\tilde{\sigma}^2\left(\frac{1}{16}+\frac{1}{11}\right)}} = 2.1884 \notag
		\end{align}
		Der kritische Wert ist $t_{16+11-2;0.99}=2.48511$ und da $T < t_{25;0.99}$ kann $H_0$ nicht abgelehnt werden.
 	\end{enumerate}
	
	\section*{Aufgabe 2}
	\begin{enumerate}[label=(\alph*)]
		\item Wir haben immer noch unabhängige Stichproben und führen wieder den Welch-Test durch:
		$H_0$: $\mu_M \le \mu_P$ \\
		$H_1$: $\mu_M > \mu_P$ \\
		Die Teststatistik ergibt sich zu
		\begin{align}
			T = \frac{10.7-1.6}{\sqrt{\frac{9.8^2}{20} + \frac{11.1^2}{20}}} = 2.7484 \notag
		\end{align}
		Die Berechnung des kritischen Wertes ist ein bisschen komplizierter:
		\begin{align}
			R &= \frac{20\cdot 9.8^2}{20\cdot 11.1^2} = 0.7795 \notag \\
			df &= \left\lfloor\frac{(1+R)^2}{\frac{R^2}{20-1} + \frac{1}{20-1}}\right\rfloor = \lfloor 37.4253\rfloor = 37 \notag
		\end{align}
		Der kritische Wert ist nun $t_{37;0.99}=2.43145$. Es gilt $T>t_{37;0.99}$ und damit kann $H_0$ abgelehnt werden und $H_1$ wird angenommen. Wir führen jetzt noch einen $F$-Test durch: \\
		$H_0$: $\sigma_M^2 \ge \sigma_P^2$ \\
		$H_1$: $\sigma_M^2 < \sigma_P^2$ \\
		Die Teststatistik ist
		\begin{align}
			T = \frac{9.8^2}{11.1^2} = 0.7795 \notag
		\end{align}
		Der kritische Wert ist $F_{19,19;0.05}=0.461201$ und da $T>F_{19,19;0.05}$, kann $H_0$ nicht abgelehnt werden.
		\item Jetzt haben wir eine verbundene Stichprobe und testen: \\
		$H_0$: $\mu_{vor} \le \mu_{nach}$ \\
		$H_1$: $\mu_{vor} > \mu_{nach}$ \\
		Wir brauchen dazu die Differenzen der einzelnen Werte; diese sind
		\begin{center}
			\begin{tabular}{c|ccccc}
				$i$ & 1 & 2 & 3 & 4 & 5 \\
				\hline
				$d_i$ & 1.7 & -1.7 & 20.4 & 24.6 & -3.5
			\end{tabular}
		\end{center}
		Die Teststatistik ist dann
		\begin{align}
			T = \sqrt{5}\frac{8.3}{\sqrt{\frac{1}{5-1}\sum_{i=1}^{5}(d_i-\bar{d})^2}} = 1.4081 \notag
		\end{align}
		Der kritische Wert ist $t_{4;0.95}=2.1318$ und da $T< t_{4;0.95}$ kann $H_0$ nicht abgelehnt werden.
		\item Mit 12\%-iger Irrtumswahrscheinlichkeit behauptet der Test, dass das Medikament den Blutdruck senkt, obwohl das nicht so ist (Fehler 1. Art).
	\end{enumerate}
\end{document}