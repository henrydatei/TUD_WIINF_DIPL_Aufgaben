\documentclass{article}

\usepackage{amsmath,amssymb}
\usepackage{tikz}
\usepackage{xcolor}
\usepackage[left=2.1cm,right=3.1cm,bottom=3cm,footskip=0.75cm,headsep=0.5cm]{geometry}
\usepackage{enumerate}
\usepackage{enumitem}
\usepackage{marvosym}
\usepackage{tabularx}
\usepackage{pgfplots}
\pgfplotsset{compat=1.10}
\usepgfplotslibrary{fillbetween}
\usepackage{parskip}

\usepackage[utf8]{inputenc}

\renewcommand*{\arraystretch}{1.4}

\newcolumntype{L}[1]{>{\raggedright\arraybackslash}p{#1}}
\newcolumntype{R}[1]{>{\raggedleft\arraybackslash}p{#1}}
\newcolumntype{C}[1]{>{\centering\let\newline\\\arraybackslash\hspace{0pt}}m{#1}}

\DeclareMathOperator{\tr}{tr}
\DeclareMathOperator{\Var}{Var}
\DeclareMathOperator{\Cov}{Cov}
\DeclareMathOperator{\Cor}{Cor}
\newcommand{\E}{\mathbb{E}}

\title{\textbf{Statistik 2, Übung 9, Tafelbild}}
\author{\textsc{Henry Haustein}}
\date{}

\begin{document}
	\maketitle
	
	\section*{Aufgabe 1}
	\textbf{Zweistichprobentest, unbekannte und ungleiche Varianzen} ($\nearrow$ Formelsammlung 2, Seite 34)
	\begin{align}
		H_0: \mu_X = \mu_Y \qquad&\text{vs}\qquad H_1: \mu_X\neq \mu_Y \notag \\
		T &= \frac{\bar{X} - \bar{Y}}{\sqrt{\frac{S_X^2}{n_X} + \frac{S_Y^2}{n_Y}}} \notag
	\end{align}
	kritische Werte sind $\pm t_{df, 1-\alpha/2}$ mit
	\begin{align}
		df &= \frac{(1+R)^2}{\frac{R^2}{n_X-1} + \frac{1}{n_Y-1}} \notag \\
		R &= \frac{n_Y\cdot S_X^2}{n_X\cdot S_Y^2} \notag
	\end{align}
	\textbf{Test auf Varianzgleichheit}
	\begin{align}
		H_0: \sigma_X = \sigma_Y \qquad&\text{vs}\qquad H_1: \sigma_X\neq \sigma_Y \notag \\
		T &= \frac{S_X^2}{S_Y^2} \notag
	\end{align}
	kritische Werte sind $F_{n_X-1,n_Y-1,1-\alpha/2}$ und $F_{n_X-1,n_Y-1,\alpha/2}$, wobei
	\begin{align}
		F_{a,b,\alpha} = \frac{1}{F_{b,a,1-\alpha}} \notag
	\end{align}
	\textbf{Zweistichprobentest, unbekannte und gleiche Varianzen}
	\begin{align}
		H_0: \mu_X = \mu_Y \qquad&\text{vs}\qquad H_1: \mu_X\neq \mu_Y \notag \\
		T &= \frac{\bar{X} - \bar{Y}}{\sqrt{\hat{\sigma}^2\left(\frac{1}{n_X} + \frac{1}{n_Y}\right)}} \notag \\
		\hat{\sigma}^2 &= \frac{n_X-1}{n_X+n_Y-2}S_X^2 + \frac{n_Y-1}{n_X+n_Y-2}S_Y^2 \notag
	\end{align}
	kritische Werte sind $\pm t_{n_X+n_Y-2, 1-\alpha/2}$.
	
	\section*{Aufgabe 2}	
	Es geht in dieser Aufgabe um den Unterschied zwischen verbundenen und unverbundenen Stichproben. Bei verbundenen Stichproben hängen sind die Ergebnisse der Testobjekte nicht mehr unkorreliert; man berechnet hier die Differenz der beiden Ergebnisse und testet, ob $d=0$ mittels $t$-Test.
	
\end{document}