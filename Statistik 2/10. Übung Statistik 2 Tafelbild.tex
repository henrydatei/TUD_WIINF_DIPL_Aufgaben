\documentclass{article}

\usepackage{amsmath,amssymb}
\usepackage{tikz}
\usepackage{xcolor}
\usepackage[left=2.1cm,right=3.1cm,bottom=3cm,footskip=0.75cm,headsep=0.5cm]{geometry}
\usepackage{enumerate}
\usepackage{enumitem}
\usepackage{marvosym}
\usepackage{tabularx}
\usepackage{pgfplots}
\pgfplotsset{compat=1.10}
\usepgfplotslibrary{fillbetween}
\usepackage{parskip}

\usepackage[utf8]{inputenc}

\renewcommand*{\arraystretch}{1.4}

\newcolumntype{L}[1]{>{\raggedright\arraybackslash}p{#1}}
\newcolumntype{R}[1]{>{\raggedleft\arraybackslash}p{#1}}
\newcolumntype{C}[1]{>{\centering\let\newline\\\arraybackslash\hspace{0pt}}m{#1}}

\DeclareMathOperator{\tr}{tr}
\DeclareMathOperator{\Var}{Var}
\DeclareMathOperator{\Cov}{Cov}
\DeclareMathOperator{\Cor}{Cor}
\newcommand{\E}{\mathbb{E}}

\title{\textbf{Statistik 2, Übung 10, Tafelbild}}
\author{\textsc{Henry Haustein}}
\date{}

\begin{document}
	\maketitle
	
	\section*{Aufgabe 1}
	Test auf monotonen Zusammenhang
	\begin{align}
		H_0: R = 0 \qquad&\text{vs.}\qquad H_1: R \neq 0 \notag \\
		T &= \frac{1}{2}\log\left(\frac{1+R}{1-R}\right)\sqrt{\frac{n-3}{1.06}} \notag
	\end{align}
	kritische Werte: $\pm z_{1-\alpha/2}$
	
	Mit $\log(\cdot)$ ist hier der Logarithmus zur Basis $e$ gemeint, also $\ln(\cdot)$!
	
	\section*{Aufgabe 2}
	$\chi^2$-Test auf Unabhängigkeit
	\begin{align}
		H_0: \text{$X$ und $Y$ sind unabhängig} \qquad&\text{vs.}\qquad H_1:\text{$X$ und $Y$ sind abhängig} \notag \\
		\chi^2 &= n\left(\sum_{i=1}^{k}\sum_{j=1}^{l} \frac{n_{ij}^2}{n_{i\cdot}n_{\cdot j}} - 1\right) \notag
	\end{align}
	kritischer Wert: $\chi^2_{(l-1)(k-1),1-\alpha}$
	
	\section*{Aufgabe 3}	
	Test auf linearen Zusammenhang
	\begin{align}
		H_0: r = 0 \qquad&\text{vs.}\qquad H_1: r \neq 0 \notag \\
		T &= \frac{r\sqrt{n-2}}{\sqrt{1-r^2}} \notag
	\end{align}
	kritische Werte: $\pm t_{n-2,1-\alpha/2}$
	
\end{document}