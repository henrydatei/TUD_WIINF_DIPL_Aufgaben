\documentclass{article}

\usepackage{amsmath,amssymb}
\usepackage{tikz}
\usepackage{pgfplots}
\usepackage{xcolor}
\usepackage[left=2.1cm,right=3.1cm,bottom=3cm,footskip=0.75cm,headsep=0.5cm]{geometry}
\usepackage{enumerate}
\usepackage{enumitem}
\usepackage{marvosym}
\usepackage{tabularx}

\usepackage[utf8]{inputenc}

\renewcommand*{\arraystretch}{1.4}

\newcolumntype{L}[1]{>{\raggedright\arraybackslash}p{#1}}
\newcolumntype{R}[1]{>{\raggedleft\arraybackslash}p{#1}}
\newcolumntype{C}[1]{>{\centering\let\newline\\\arraybackslash\hspace{0pt}}m{#1}}

\title{\textbf{Statistik 2, Übung 4}}
\author{\textsc{Henry Haustein}}
\date{}

\begin{document}
	\maketitle
	
	\section*{Aufgabe 1}
	Wir wollen den Gesamtfehler $U = \sum_{i=1}^{3} u_i^2$ minimieren. Dabei sind die $u_i$ die Abweichung zwischen dem wahren Wert von $p$ und einer Schätzung $\hat{p}_i$. Die Schätzung für einen Platz am Fenster ist $\hat{p}_1 = \frac{4}{10}$, für einen Platz am Gang $\hat{p}_3 = \frac{5}{10}$ und für den Platz in der Mittel gilt $1-2\hat{p}_2 = \frac{1}{10} \Leftrightarrow \hat{p}_2 = \frac{9}{20}$. Mit diesen Informationen können wir eine Funktion für $U$ aufstellen:
	\begin{align}
		U = \sum_{i=1}^3 u_i &= \left(p-\frac{4}{10}\right)^2 + \left(p-\frac{5}{10}\right)^2 + \left(p-\frac{9}{20}\right)^2 \notag \\
		&= \left(p^2 - \frac{8}{10}p + \frac{16}{10}\right) + \left(p^2 - \frac{10}{10}p + \frac{25}{10}\right) + \left(p^2 - \frac{18}{20}p + \frac{81}{20}\right) \notag \\
		&= 3p^2 - \frac{27}{10}p + \frac{163}{20} \notag
	\end{align}
	Um ein Minimum von $U$ zu finden, müssen wir $U$ nach $p$ ableiten, also
	\begin{align}
		\frac{dU}{dp} = 6p - \frac{27}{10} &= 0 \notag \\
		6p &= \frac{27}{10} \notag \\
		p &= \frac{27}{60} = 0.45 \notag
	\end{align}
	Überprüfen wir noch schnell um, welche Art von Extremum es sich handelt: $\frac{d^2U}{dp^2} = 6 > 0 \Rightarrow$ Minimum. Das heißt $p=0.45$ minimiert den aufsummieren quadratischen Fehler.
	
	\section*{Aufgabe 2}
	\begin{enumerate}[label=(\alph*)]
		\item Wir wählen den Scatterplot als Darstellungsform:
		\begin{center}
			\begin{tikzpicture}
			\begin{axis}[
			xmin=160, xmax=190, xlabel=$x$,
			ymin=55, ymax=90, ylabel=$y$,
			samples=400,
			axis x line=middle,
			axis y line=middle,
			domain=160:190,
			]
			\node[blue] at (axis cs: 164,74) {$\times$};
			\node[blue] at (axis cs: 171,73) {$\times$};
			\node[blue] at (axis cs: 186,89) {$\times$};
			\node[blue] at (axis cs: 179,83) {$\times$};
			\node[blue] at (axis cs: 161,60) {$\times$};
			\node[blue] at (axis cs: 174,80) {$\times$};
			
			\end{axis}
			\end{tikzpicture}
		\end{center}
		\item Für ein Modell $y_i=\beta_0 + \beta_1x_i + \varepsilon_i$ sind die Koeffizienten gegeben durch
		\begin{align}
			\beta_0 &= \bar{y} - \beta_1\bar{x} \notag \\
			\beta_1 &= \frac{\frac{1}{n}\sum_{i=1}^n x_iy_i - \bar{x}\bar{y}}{\frac{1}{n}\sum_{i=1}^n x_i^2 - \bar{x}^2} \notag
		\end{align}
		Man kann weiterhin $\bar{x}=172.5$ und $\bar{y}=67.5$ berechnen und es ergibt sich
		\begin{align}
			y = -95.665 + 0.998x \notag
		\end{align}
		\item Die Fehler sind die Abweichung zwischen dem mittels Regression geschätzten $y$ und dem gemessenen $y$.
		\begin{center}
			\begin{tikzpicture}
			\begin{axis}[
			xmin=160, xmax=190, xlabel=$x$,
			ymin=55, ymax=90, ylabel=$y$,
			samples=400,
			axis x line=middle,
			axis y line=middle,
			domain=160:190,
			]
			\node[blue] at (axis cs: 164,74) {$\times$};
			\node[blue] at (axis cs: 171,73) {$\times$};
			\node[blue] at (axis cs: 186,89) {$\times$};
			\node[blue] at (axis cs: 179,83) {$\times$};
			\node[blue] at (axis cs: 161,60) {$\times$};
			\node[blue] at (axis cs: 174,80) {$\times$};
			
			\addplot[red,smooth,mark=none] {-95.665 + 0.998*x};
			
			\draw[<->,dotted] (axis cs: 164,74) -- (axis cs: 164,68);
			\draw[<->,dotted] (axis cs: 171,73) -- (axis cs: 171,75);
			\draw[<->,dotted] (axis cs: 186,89) -- (axis cs: 186,90);
			\draw[<->,dotted] (axis cs: 179,83) -- (axis cs: 179,83);
			\draw[<->,dotted] (axis cs: 161,60) -- (axis cs: 161,65);
			\draw[<->,dotted] (axis cs: 174,80) -- (axis cs: 174,78);
			
			\end{axis}
			\end{tikzpicture}
		\end{center}
		\item $-95.665 + 0.998 \cdot 170 = 74.005$
	\end{enumerate}

	\section*{Aufgabe 3}
	Um vernünftig mit dem Modell arbeiten zu können, sollten wir es erst in eine passende Form bringen: Wir definieren dazu $\tilde{x}_i = 16x_i - x_i^2$ und erhalten
	\begin{align}
		y_i = \kappa + \tilde{x}_i \notag
	\end{align}
	Dieses Modell hat starke Ähnlichkeit mit einem "normalen" linearen Modell, nur dass hier $\beta_0 = \kappa$ und $\beta_1=1$ ist. Wir können wieder unsere Formeln zur Bestimmung von $\beta_0=\kappa$ heranziehen und erhalten
	\begin{align}
		\kappa = \beta_0 = \bar{y} - 1\cdot\bar{\tilde{x}} \notag
	\end{align}
	Die transformierten $\tilde{x}_i$ lauten
	\begin{center}
		\begin{tabular}{c|ccccccc}
			$x_i$ & 6 & 7 & 7.5 & 7.75 & 8 & 8.75 & 9.25 \\
			\hline
			$\tilde{x}_i$ & 60 & 63 & 63.75 & 63.9375 & 64 & 63.4375 & 62.4375
		\end{tabular}
	\end{center}
	$\bar{y} = 8.2857$ und $\bar{\tilde{x}} = 62.9375$. Damit folgt $\kappa = -54.6518$.

	\section*{Aufgabe 4}
	Auch hier müssen wir das Modell erst in eine lineare Form bringen, damit wir damit arbeiten können. Wir wenden den natürlichen Logarithmus auf beide Seiten an:
	\begin{align}
		\ln(y_i) = \ln(\beta_0) + \beta_1x_i \notag
	\end{align}
	Mit ein bisschen Umbenennung der Variablen ($\tilde{y}_i = \ln(y_i)$ und $\tilde{\beta}_0 = \ln(\beta_0)$) ergibt sich
	\begin{align}
		\tilde{y}_i = \tilde{\beta}_0 + \beta_1x_i \notag
	\end{align}
	Die Formeln zur Bestimmung von $\tilde{\beta}_0$ und $\beta_1$ liefern:
	\begin{align}
		\tilde{\beta}_0 &= \bar{\tilde{y}} - \beta_1\bar{x} \notag \\
		\beta_1 &= \frac{\frac{1}{n}\sum_{i=1}^n x_i\tilde{y}_i - \bar{x}\bar{\tilde{y}}}{\frac{1}{n}\sum_{i=1}^n x_i^2 - \bar{x}^2} \notag
	\end{align}
	Es sind zum Glück schon alle dieser Summen in der Aufgabenstellung gegeben\footnote{Die Aufgabenstellung hat die Variablen nicht umbenannt, deswegen hier aufpassen: $\log(y_i) = \tilde{y}_i$}.
	\begin{align}
		\beta_1 &= \frac{\frac{1}{20}\cdot -249.635 - \bar{x}\bar{\tilde{y}}}{\frac{1}{20}930.76 - \bar{x}^2} \notag \\
		&= \frac{-12.48175 - 6.08\cdot (-1.60657)}{46.538 - 6.08^2} \notag \\
		&= \frac{-2.713804}{9.5716} \notag \\
		&= -0.2835267 \notag \\
		\tilde{\beta}_0 &= \bar{\tilde{y}} - (-0.2835267)\cdot\tilde{x} \notag \\
		&= -1.60657 - (-0.2835267) \cdot 6.08 \notag \\
		&= 0.1172723 \notag
	\end{align}
	mit $\bar{x} = \frac{1}{20}\sum_{i=1}^20 x_i = \frac{1}{20}121.6 = 6.08$ und $\bar{\tilde{y}} = \frac{1}{20}\sum_{i=1}^20 y_i = \frac{1}{20}\cdot -32.131 = -1.60657$. Da wir $\beta_0$ zu $\tilde{\beta}_0$ transformiert haben, müssen wir die Transformation rückgängig machen. Es ergibt sich für $\beta_0 = 1.124426$.
	
\end{document}