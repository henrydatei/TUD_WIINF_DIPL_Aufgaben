\documentclass{article}

\usepackage{amsmath,amssymb}
\usepackage{tikz}
\usepackage{xcolor}
\usepackage[left=2.1cm,right=3.1cm,bottom=3cm,footskip=0.75cm,headsep=0.5cm]{geometry}
\usepackage{enumerate}
\usepackage{enumitem}
\usepackage{marvosym}
\usepackage{tabularx}
\usepackage{pgfplots}
\pgfplotsset{compat=1.10}
\usepgfplotslibrary{fillbetween}
\usepackage{parskip}

\usepackage[utf8]{inputenc}

\renewcommand*{\arraystretch}{1.4}

\newcolumntype{L}[1]{>{\raggedright\arraybackslash}p{#1}}
\newcolumntype{R}[1]{>{\raggedleft\arraybackslash}p{#1}}
\newcolumntype{C}[1]{>{\centering\let\newline\\\arraybackslash\hspace{0pt}}m{#1}}

\DeclareMathOperator{\tr}{tr}
\DeclareMathOperator{\Var}{Var}
\DeclareMathOperator{\Cov}{Cov}
\DeclareMathOperator{\Cor}{Cor}
\newcommand{\E}{\mathbb{E}}

\title{\textbf{Statistik 2, Übung 7, Tafelbild}}
\author{\textsc{Henry Haustein}}
\date{}

\begin{document}
	\maketitle
	
	\section*{Aufgabe 1}
	Zweiseitige Tests für den Mittelwert (häufig $t$-Test genannt) ($\nearrow$ Formelsammlung II, Seite 33):
	\begin{align}
		T &= \frac{\mu - \mu_0}{\sigma}\sqrt{n} \qquad z_{krit} = \pm z_{1-\alpha/2} \qquad\sigma\text{ bekannt} \notag \\
		T &= \frac{\mu - \mu_0}{s}\sqrt{n} \qquad t_{krit} = \pm t_{n-1,1-\alpha/2} \qquad\sigma\text{ unbekannt} \notag
	\end{align}
	Bei einseitigen Tests wird $1-\alpha/2$ durch $1-\alpha$ ersetzt und einer der kritischen Werte verschwindet.
	
	Zweiseitiger Test für die Varianz
	\begin{align}
		T &= (n-1)\frac{s^2}{\sigma_0^2} \qquad\chi^2_{krit} = (\chi^2_{n-1,1-\alpha/2},\chi^2_{n-1,\alpha/2}) \notag
	\end{align}

	\section*{Aufgabe 2}
	Zweiseitiger Test für $p$
	\begin{align}
		T &= \frac{p-p_0}{\sqrt{p(1-p)}}\sqrt{n} \qquad z_{krit} = \pm z_{1-\alpha/2} \notag
	\end{align}
	Die kritischen Werte aus der Standardnormalverteilung kann man nur nehmen, wenn $n$ ausreichend groß ist ($n\ge 100$), ansonsten müsste man die Quantile der Binomialverteilung nutzen!
	
	Berechnung von $p$-values:
	\begin{itemize}
		\item $T<0\Rightarrow p$-value $= \Phi(T)$
		\item $T>0\Rightarrow p$-value $= 1-\Phi(T)$
	\end{itemize}

	\section*{Aufgabe 3}
	nichts neues
	
	\section*{Aufgabe 4}
	nichts neues
	
\end{document}