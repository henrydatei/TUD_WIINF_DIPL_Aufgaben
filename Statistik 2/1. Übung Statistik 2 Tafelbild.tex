\documentclass{article}

\usepackage{amsmath,amssymb}
\usepackage{tikz}
\usepackage{xcolor}
\usepackage[left=2.1cm,right=3.1cm,bottom=3cm,footskip=0.75cm,headsep=0.5cm]{geometry}
\usepackage{enumerate}
\usepackage{enumitem}
\usepackage{marvosym}
\usepackage{tabularx}
\usepackage{pgfplots}
\pgfplotsset{compat=1.10}
\usepgfplotslibrary{fillbetween}

\usepackage[utf8]{inputenc}

\renewcommand*{\arraystretch}{1.4}

\newcolumntype{L}[1]{>{\raggedright\arraybackslash}p{#1}}
\newcolumntype{R}[1]{>{\raggedleft\arraybackslash}p{#1}}
\newcolumntype{C}[1]{>{\centering\let\newline\\\arraybackslash\hspace{0pt}}m{#1}}

\DeclareMathOperator{\tr}{tr}
\DeclareMathOperator{\Var}{Var}
\DeclareMathOperator{\Cov}{Cov}

\title{\textbf{Statistik 2, Übung 1, Tafelbild}}
\author{\textsc{Henry Haustein}}
\date{}

\begin{document}
	\maketitle
	
	\section*{Aufgabe 1}
	Binomialverteilung $X\sim B(n,p)$: Ein Bernoulli-Experiment (Experiment mit 2 verschiedenen Ergebnissen $x_1$, $x_2$) mit $n$ mal wiederholt. Die Wahrscheinlichkeit, dass bei einem Bernoulli-Experiment $x_1$ eintritt, ist $p$ und $X$ gibt an, wie oft $x_1$ eintritt.
	\begin{align}
		\mathbb{P}(X = k) = \binom{n}{k}p^k(1-p)^{n-k} \qquad\qquad
		\mathbb{P}(X \le k) = \sum_{i=0}^{k} \mathbb{P}(X = i) \notag
	\end{align}
	Geometrische Verteilung $X\sim G(p)$: Ein Bernoulli-Experiment wird solange wiederholt, bis das erste mal $x_1$ eintritt. $X$ gibt an, wie viele Wiederholungen notwendig waren.
	\begin{align}
		\mathbb{P}(X = k) = p(1-p)^{n-1} \qquad\qquad
		\mathbb{P}(X\le k) = \sum_{i=0}^k \mathbb{P}(X = i) \notag
	\end{align}

	\section*{Aufgabe 2}
	Exponentialverteilung $X\sim Exp(\lambda)$ (Anmerkung: $\exp(x) = e^x$)
	\begin{align}
		f(x) = \begin{cases}
			\lambda\exp(-\lambda x) & x\ge 0 \\
			0 & x<0
		\end{cases} \qquad\qquad
		F(x) = \begin{cases}
			1-\exp(-\lambda x) &x\ge 0 \\
			0 & x<0
		\end{cases} \notag 
	\end{align}
	Die Exponentialverteilung ist eine stetige Verteilung, das heißt unter anderem $\mathbb{P}(X = k) = 0$ und
	\begin{align}
		\mathbb{P}(a < X < b) = \int_{a}^{b} f(x)\,dx = F(b) - F(a) \notag
	\end{align}

	\section*{Aufgabe 3}
	Verteilungsfunktion: 
	\begin{align}
		F(x) = \int_{-\infty}^x f(y)\, dy \notag
	\end{align}
	 die Funktion in dieser Aufgabe ist abschnittsweise definiert, insbesondere $\int_{-\infty}^1 f(y)\, dy = 0$ $\to$ untere Grenze sinnvoll wählen!
	
	\section*{Aufgabe 4}
	Es gilt:
	\begin{align}
		\int_{-\infty}^\infty f(x) \,dx = 1 \qquad\qquad \mathbb{E}(X) = \int_{-\infty}^\infty xf(x)\,dx \qquad\qquad \text{Var}(X) = \int_{-\infty}^\infty x^2f(x)\,dx - \mathbb{E}(X)^2 \notag
	\end{align}
	
\end{document}