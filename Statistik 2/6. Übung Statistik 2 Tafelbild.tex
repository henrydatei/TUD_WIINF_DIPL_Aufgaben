\documentclass{article}

\usepackage{amsmath,amssymb}
\usepackage{tikz}
\usepackage{xcolor}
\usepackage[left=2.1cm,right=3.1cm,bottom=3cm,footskip=0.75cm,headsep=0.5cm]{geometry}
\usepackage{enumerate}
\usepackage{enumitem}
\usepackage{marvosym}
\usepackage{tabularx}
\usepackage{pgfplots}
\pgfplotsset{compat=1.10}
\usepgfplotslibrary{fillbetween}
\usepackage{parskip}

\usepackage[utf8]{inputenc}

\renewcommand*{\arraystretch}{1.4}

\newcolumntype{L}[1]{>{\raggedright\arraybackslash}p{#1}}
\newcolumntype{R}[1]{>{\raggedleft\arraybackslash}p{#1}}
\newcolumntype{C}[1]{>{\centering\let\newline\\\arraybackslash\hspace{0pt}}m{#1}}

\DeclareMathOperator{\tr}{tr}
\DeclareMathOperator{\Var}{Var}
\DeclareMathOperator{\Cov}{Cov}
\DeclareMathOperator{\Cor}{Cor}
\newcommand{\E}{\mathbb{E}}

\title{\textbf{Statistik 2, Übung 6, Tafelbild}}
\author{\textsc{Henry Haustein}}
\date{}

\begin{document}
	\maketitle
	
	\section*{Aufgabe 1}
	Konfidenzintervall für $\mu$ bei unbekannter Varianz ($\to$ muss aus Stichprobe geschätzt werden)
	\begin{align}
		KI(\mu) &= \bar{x} \pm t_{n-1,1-\alpha/2}\frac{S}{\sqrt{n}} \notag
	\end{align}
	Die Musterlösung verwendet hier $z_{1-\alpha/2}$ statt $t_{n-1,1-\alpha/2}$, was nicht ganz korrekt ist, aber der Fehler ist sehr klein.
	\begin{center}
		\begin{tikzpicture}[
			declare function={gamma(\z)=
				2.506628274631*sqrt(1/\z)+ 0.20888568*(1/\z)^(1.5)+ 0.00870357*(1/\z)^(2.5)- (174.2106599*(1/\z)^(3.5))/25920- (715.6423511*(1/\z)^(4.5))/1244160)*exp((-ln(1/\z)-1)*\z;},
			declare function={student(\x,\n)= gamma((\n+1)/2.)/(sqrt(\n*pi) *gamma(\n/2.)) *((1+(\x*\x)/\n)^(-(\n+1)/2.));}
			]
			\begin{axis}[
				xmin=-3, xmax=3,
				ymin=0, ymax=0.5,
				samples=400,
				axis x line=middle,
				axis y line=middle,
				domain=-3:3,
				]
				\addplot [blue, smooth] {student(x,116)};
				\addplot [red, smooth] {1/sqrt(2*pi) * exp(-0.5*x^2)};
				
			\end{axis}
		\end{tikzpicture} \\
		\textcolor{red}{$t$-Verteilung mit 116 Freiheitsgraden}, \textcolor{blue}{Standardnormalverteilung}
	\end{center}

	\section*{Aufgabe 2}
	Konfidenzintervall bei Binomialverteilung für $p$ (Varianz einer binomialverteilten Zufallsvariable ist $p(1-p)$):
	\begin{align}
		KI(p) &= \hat{p} \pm t_{n-1,1-\alpha/2}\frac{\sqrt{\hat{p}(1-\hat{p})}}{\sqrt{n}} \notag
	\end{align}

	\section*{Aufgabe 3}
	Einseitige Konfidenzintervalle: ersetze $1-\alpha/2$ durch $1-\alpha$
	\begin{align}
		KI(p) &= \left[\hat{p} - t_{n-1,1-\alpha}\frac{\sqrt{\hat{p}(1-\hat{p})}}{\sqrt{n}},1\right] \qquad \text{linksseitig} \notag \\
		KI(p) &= \left[0,\hat{p} + t_{n-1,1-\alpha}\frac{\sqrt{\hat{p}(1-\hat{p})}}{\sqrt{n}}\right] \qquad \text{rechtsseitig} \notag
	\end{align}
	
\end{document}