\documentclass{article}

\usepackage{amsmath,amssymb}
\usepackage{tikz}
\usepackage{pgfplots}
\usepackage{xcolor}
\usepackage[left=2.1cm,right=3.1cm,bottom=3cm,footskip=0.75cm,headsep=0.5cm]{geometry}
\usepackage{enumerate}
\usepackage{enumitem}
\usepackage{marvosym}
\usepackage{tabularx}

\usepackage[utf8]{inputenc}

\renewcommand*{\arraystretch}{1.4}

\newcolumntype{L}[1]{>{\raggedright\arraybackslash}p{#1}}
\newcolumntype{R}[1]{>{\raggedleft\arraybackslash}p{#1}}
\newcolumntype{C}[1]{>{\centering\let\newline\\\arraybackslash\hspace{0pt}}m{#1}}

\newcommand{\E}{\mathbb{E}}
\DeclareMathOperator{\Var}{Var}
\DeclareMathOperator{\CDF}{CDF}

\title{\textbf{Statistik 2, Übung 8}}
\author{\textsc{Henry Haustein}}
\date{}

\begin{document}
	\maketitle
	
	\section*{Aufgabe 1}
	\begin{enumerate}[label=(\alph*)]
		\item Die Gütefunktion ist als Wahrscheinlichkeit, die Nullhypothese abzulehnen, definiert. Also
		\begin{align}
			G(\mu_0) &= \mathbb{P}(H_0\text{ ablehnen}) \notag \\
			&= \mathbb{P}(T > z_{1-\alpha}) \notag \\
			&= \mathbb{P}\left(\frac{\mu-\mu_0}{\sigma}\sqrt{n} > z_{1-\alpha}\right) \notag \\
			&= 1-\mathbb{P}\left(\frac{\mu-\mu_0}{\sigma}\sqrt{n} > -z_{1-\alpha}\right) \notag \\
			&= 1-\mathbb{P}\left(\frac{\mu-\mu_0}{\sigma}\sqrt{n} + z_{1-\alpha} > 0\right) \notag \\
			&= 1-\Phi\left(\frac{\mu-\mu_0}{\sigma}\sqrt{n} + z_{1-\alpha}\right) \notag
		\end{align}
		\item Wir testen $H:0: \bar{x} \le 83$ vs. $H_1: \bar{x} > 83$ und berechnen $G(84)=0.102936$. Wir machen keinen Fehler.
		\item Das ist das Gegenereignis zu (b), also $1-G(84)=0.897064$. Wir machen hier einen Fehler 2. Art.
		\item $G(82)=0.021953$. Wir machen hier einen Fehler 1. Art.
	\end{enumerate}
	
	\section*{Aufgabe 2}
	\begin{enumerate}[label=(\alph*)]
		\item Wir testen $H_0: \bar{x} \ge 175$ gegen $H_1:\bar{x} < 175$. Die Teststatistik berechnet sich zu
		\begin{align}
			T = \frac{175-174.4}{4}\sqrt{25} = 0.75 \notag
		\end{align}
		Der kritische Wert ist $z_{1-\alpha}=1.64$ und damit erfolgt keine Ablehnung von $H_0$.
		\item Wir müssen $\alpha^\ast$ finden, so dass
		\begin{align}
			z_{1-\alpha^\ast} &= 0.75 \notag \\
			1-\alpha^\ast &=  \CDF(0.75) \notag \\
			\alpha^\ast &= 1-\CDF(0.75) \notag \\
			&= 0.226627 \notag
		\end{align}
		wobei $\CDF(\cdot)$ für die Verteilungsfunktion der Standardnormalverteilung steht.
		\item Läuft ähnlich wie Aufgabe 1 (a):
		\begin{align}
			G(\mu_0) &= \mathbb{P}(H_0\text{ ablehnen}) \notag \\
			&= \mathbb{P}(T > -z_{1-\alpha}) \notag \\
			&= \mathbb{P}\left(\frac{\mu-\mu_0}{\sigma}\sqrt{n} > -z_{1-\alpha}\right) \notag \\
			&= \mathbb{P}\left(\frac{\mu-\mu_0}{\sigma}\sqrt{n} - z_{1-\alpha} < 0\right) \notag \\
			&= \Phi\left(\frac{\mu-\mu_0}{\sigma}\sqrt{n} - z_{1-\alpha}\right) \notag
		\end{align}
		\item Hier ist nach der Operationscharakteristik gefragt, also $1-G(174)=\Phi(-0.39)=0.651732$.
		\item Wir müssen ein $n$ finden, sodass die Gütefunktion für $\bar{x}=174$ größer als 0.975 ist:
		\begin{align}
			\mathbb{P}(H_0\text{ ablehnen}) &\ge 0.975 \notag \\
			\Phi\left(\frac{175-174}{4}\sqrt{n} - 1.64\right) &\ge 0.975 \notag \\
			\frac{1}{4}\sqrt{n} - 1.64 &\ge z_{0.975} \notag \\
			\sqrt{n} &\ge 4(1.95996 + 1.64) \notag \\
			n &\ge 207.335 \notag \\
			 &\ge 208 \notag
		\end{align}
	\end{enumerate}
\end{document}