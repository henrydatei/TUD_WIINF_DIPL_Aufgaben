\documentclass{article}

\usepackage{amsmath,amssymb}
\usepackage{tikz}
\usepackage{xcolor}
\usepackage[left=2.1cm,right=3.1cm,bottom=3cm,footskip=0.75cm,headsep=0.5cm]{geometry}
\usepackage{enumerate}
\usepackage{enumitem}
\usepackage{marvosym}
\usepackage{tabularx}
\usepackage{pgfplots}
\pgfplotsset{compat=1.10}
\usepgfplotslibrary{fillbetween}

\usepackage[utf8]{inputenc}

\renewcommand*{\arraystretch}{1.4}

\newcolumntype{L}[1]{>{\raggedright\arraybackslash}p{#1}}
\newcolumntype{R}[1]{>{\raggedleft\arraybackslash}p{#1}}
\newcolumntype{C}[1]{>{\centering\let\newline\\\arraybackslash\hspace{0pt}}m{#1}}

\DeclareMathOperator{\tr}{tr}
\DeclareMathOperator{\Var}{Var}
\DeclareMathOperator{\Cov}{Cov}
\newcommand{\E}{\mathbb{E}}

\title{\textbf{Statistik 2, Übung 2, Tafelbild}}
\author{\textsc{Henry Haustein}}
\date{}

\begin{document}
	\maketitle
	
	\section*{Aufgabe 1}
	Ungleichung von Tschebyscheff:
	\begin{align}
		\mathbb{P}(\vert X-\E(X)\vert < \varepsilon) &\ge 1 - \frac{\Var(X)}{\varepsilon^2} \notag \\
		\mathbb{P}(\E(X) - \varepsilon < X < \E(X) + \varepsilon) &\ge 1 - \frac{\Var(X)}{\varepsilon^2} \notag
	\end{align}

	\section*{Aufgabe 2}
	Wenn $X_i\sim \mathcal{N}(\mu, \sigma^2)$, dann $X = \sum X_i \sim\mathcal{N}(\mu\cdot n, \sigma^2\cdot n)$

	\section*{Aufgabe 3}
	Für \textit{große} $n$ kann man die Binomialverteilung mit der Normalverteilung approximieren. Damit die Approximation aber gut ist, sollte
	\begin{align}
		np(1-p) \ge 9 \notag
	\end{align}
	gelten. Falls dem nicht so ist, dann sollte zumindest
	\begin{align}
		np\ge 5\qquad\text{und} \qquad n(1-p)\ge 5 \notag
	\end{align}
	gelten. Ist $X\sim B(n,p)$, dann $\E(X) = np$ und $\Var(X) = np(1-p)$. Weitere wichtige Identitäten/Eigenschaften:
	\begin{itemize}
		\item $\Phi(1-x) = -\Phi(x)$
		\item $\frac{X-\mu}{\sigma}\sim \mathcal{N}(0,1)$, wenn $X\sim\mathcal{N}(\mu, \sigma^2)$
	\end{itemize}
	
	\section*{Aufgabe 4}
	Wenn $X_i\sim \mathcal{N}(\mu, \sigma^2)$, dann $\bar{X} = \frac{1}{n}\sum X_i \sim\mathcal{N}(\mu, \frac{\sigma^2}{n})$.
	
\end{document}