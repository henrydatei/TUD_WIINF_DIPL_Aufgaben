\documentclass[ngerman,t]{beamer}
\usepackage[ngerman]{babel}
\usetheme{Boadilla}

\usepackage{amsmath,amssymb}
\usepackage{tikz}
\usepackage{xcolor}
\usepackage{xfrac}
\usepackage{enumerate}
\usepackage{enumitem}
\usepackage{marvosym}
\usepackage{tabularx}
\usepackage{pgfplots}
\pgfplotsset{compat=1.10}
\usepgfplotslibrary{fillbetween}

\usepackage[utf8]{inputenc}

\renewcommand*{\arraystretch}{1.4}

\newcolumntype{L}[1]{>{\raggedright\arraybackslash}p{#1}}
\newcolumntype{R}[1]{>{\raggedleft\arraybackslash}p{#1}}
\newcolumntype{C}[1]{>{\centering\let\newline\\\arraybackslash\hspace{0pt}}m{#1}}

\DeclareMathOperator{\tr}{tr}
\DeclareMathOperator{\Var}{Var}
\DeclareMathOperator{\Cov}{Cov}
\newcommand{\E}{\mathbb{E}}

\title{Statistik 2, Übung 1}
\author{\textsc{Henry Haustein}}
\institute{TU Dresden}
\date{\today}

\begin{document}
	\begin{frame}
		\titlepage
	\end{frame}

	\begin{frame}
		\frametitle{Aufgabe 1.1}
		Nur 3\% der kontrollierten Fahrgäste der Dresdner Straßenbahnen besitzen keinen gültigen Fahrschein. Die Ergebnisse der Kontrolle seien unabhängig, d. h. es wird vereinfacht angenommen, dass nur Einzelpersonen kontrolliert werden.
		\begin{enumerate}[label=(\alph*)]
			\item Mit welcher Wahrscheinlichkeit wird bei der Kontrolle von 10 Personen wenigstens ein Schwarzfahrer erwischt?
			\item Wie groß ist die Wahrscheinlichkeit, dass der Kontrolleur erst bei der zehnten Kontrolle den ersten Schwarzfahrer aufspürt?
		\end{enumerate}
	\end{frame}
	\begin{frame}
		\frametitle{Aufgabe 1.1}
		\begin{enumerate}[label=(\alph*)]
			\item Sei $X$ die Anzahl der erwischten Schwarzfahrer. Dann $X\sim B(10, 0.03)$ und für die Binomialverteilung gilt:
			\begin{align}
				\mathbb{P}(X = k) &= \binom{n}{k}p^k(1-p)^{n-k} \notag
			\end{align}
			\pause
			Wir suchen
			\begin{align}
				\mathbb{P}(X \ge 1) &= 1 - \mathbb{P}(X = 0) \notag \\
				&= 1 - \binom{10}{0}\cdot 0.03^0\cdot 0.97^{10} \notag \\
				&= 0.263 \notag
			\end{align}
		\end{enumerate}
	\end{frame}
	\begin{frame}
		\frametitle{Aufgabe 1.1}
		\begin{itemize}
			\item[(b)] $Y$ sei die Anzahl der kontrollierten Fahrgäste, bis der erste Schwarzfahrer erwischt wird. Daher folgt $Y$ einer geometrischen Verteilung $Y \sim G(0.03)$:
			\begin{align}
				\mathbb{P}(Y = k) &= p(1-p)^{k-1} \notag \\
				\mathbb{P}(Y = 10) &= 0.03\cdot 0.97^9 \notag \\
				&= 0.023 \notag
			\end{align}
			\pause
			\item[$\Rightarrow$] Alternative Lösung, die vielleicht etwas intuitiver ist: Wenn man erst bei der 10. Kontrolle einen Schwarzfahrer finden will, darf man bei den 9 Kontrollen vorher keinen Schwarzfahrer finden ($0.97^9$), muss aber in der 10. Kontrolle ($0.03$):
			\begin{align}
				0.97^9\cdot 0.03 = 0.023 \notag
			\end{align}
		\end{itemize}
	\end{frame}

	\begin{frame}
		\frametitle{Aufgabe 1.2}
		Die Dauer von Telefongesprächen an der Information am Hauptbahnhof Dresden sei exponentialverteilt mit dem Parameter $\lambda = 0.2 min^{-1}$.
		\begin{enumerate}[label=(\alph*)]
			\item Skizzieren Sie die Dichte und die Verteilungsfunktion der Dauer der Telefongespräche.
			\item Wie groß ist die Wahrscheinlichkeit, dass ein Gespräch kürzer als 3 Minuten dauert?
			\item Wie groß ist die Wahrscheinlichkeit, dass ein Gespräch länger als 3 Minuten dauert?
			\item Wie groß ist die Wahrscheinlichkeit, dass ein Gespräch länger als 2 Minuten und kürzer als 3 Minuten dauert?
			\item Berechnen Sie den Erwartungswert und die Varianz für die Dauer der Telefongespräche.
		\end{enumerate}
	\end{frame}
	\begin{frame}
		\frametitle{Aufgabe 1.2}
		\begin{itemize}
			\item[(a)] Die Dichte $f$ und Verteilungsfunktion $F$ einer Exponentialverteilung sind definiert als
			\begin{align}
				f(x) &= \lambda\cdot e^{-\lambda x} \notag \\
				F(x) &= \int_{0}^{x} f(t)\,dt = 1- e^{-\lambda x} \notag
			\end{align}
			\pause
			\begin{center}
				\begin{tikzpicture}[scale=0.6]
					\begin{axis}[
						xmin=0, xmax=20, xlabel=$x$,
						ymin=0, ymax=0.2, ylabel=$f(x)$,
						samples=400,
						axis x line=middle,
						axis y line=middle,
						domain=0:20,
						title={Dichte}
						]
						\addplot[mark=none,smooth,blue] {0.2 * exp(-0.2*x)};
						
					\end{axis}
				\end{tikzpicture}
				\begin{tikzpicture}[scale=0.6]
					\begin{axis}[
						xmin=0, xmax=20, xlabel=$x$,
						ymin=0, ymax=1, ylabel=$F(x)$,
						samples=400,
						axis x line=middle,
						axis y line=middle,
						domain=0:20,
						title={Verteilungsfunktion}
						]
						\addplot[mark=none,smooth,red] {1-exp(-0.2*x)};
						
					\end{axis}
				\end{tikzpicture}
			\end{center}
		\end{itemize}
	\end{frame}
	\begin{frame}
		\frametitle{Aufgabe 1.2}
		\begin{itemize}
			\item[(b)] $\mathbb{P}(X < 3) = F(3) = 1 - e^{-0.2\cdot 3} = 0.451$
			\pause
			\item[(c)] $\mathbb{P}(X > 3) = 1 - F(3) = 0.549$ \\ \textcolor{gray}{Bemerkung: $\mathbb{P}(X = 3) = 0$, weil $X$ eine stetige Zufallsvariable ist.}
			\pause
			\item[(d)] $\mathbb{P}(2 < X < 3) = F(3) - F(2) = 0.121$
			\pause
			\item[(e)] In der Formelsammlung 2 ($\nearrow$ Ordner \textit{Extras} in OPAL) finden sich folgende Formeln für den Erwartungswert und die Varianz einer exponentialverteilten Zufallsvariable:
			\begin{align}
				\E(X) &= \frac{1}{\lambda} = 5 \notag \\
				\Var(X) &= \frac{1}{\lambda^2} = 25 \notag
			\end{align}
		\end{itemize}
	\end{frame}

	\begin{frame}
		\frametitle{Aufgabe 1.3}
		Gegeben ist folgende Dichtefunktion einer Zufallsvariablen $X$:
		\begin{align}
			f(x) &= \begin{cases}
				\frac{1}{12}x & 1\le x\le 5 \\
				0 & \text{sonst}
			\end{cases} \notag
		\end{align}
		\begin{enumerate}[label=(\alph*)]
			\item Bestimmen Sie die Verteilungsfunktion dieser Zufallsvariablen.
			\item Bestimmen Sie die Wahrscheinlichkeit des Ereignisses ($2 < X < 3$).
		\end{enumerate}
		\pause
		\begin{center}
			\begin{tikzpicture}[scale=0.6]
				\begin{axis}[
					xmin=0, xmax=6, xlabel=$x$,
					ymin=0, ymax=0.5, ylabel=$f(x)$,
					samples=400,
					axis x line=middle,
					axis y line=middle,
					domain=1:5,
					]
					\addplot[mark=none,smooth,blue, ultra thick] {1/12 * x};
					\draw[blue, ultra thick] (axis cs: 0,0) -- (axis cs: 1,0);
					\draw[blue, ultra thick] (axis cs: 5,0) -- (axis cs: 6,0);
				\end{axis}
			\end{tikzpicture}
		\end{center}
	\end{frame}
	\begin{frame}
		\frametitle{Aufgabe 1.3}
		\begin{itemize}
			\item[(a)] Die Verteilungsfunktion ist wie folgt definiert:
			\begin{align}
				F(x) = \int_{-\infty}^x f(t)\, dt \notag
			\end{align}
			\begin{center}
				\begin{tikzpicture}[scale=0.6]
					\begin{axis}[
						xmin=0, xmax=6, xlabel=$x$,
						ymin=0, ymax=0.5, ylabel=$f(x)$,
						samples=400,
						axis x line=middle,
						axis y line=middle,
						domain=1:5,
						]
						\addplot[mark=none,smooth,blue, ultra thick] {1/12 * x};
						\draw[blue, ultra thick] (axis cs: 0,0) -- (axis cs: 1,0);
						\draw[blue, ultra thick] (axis cs: 5,0) -- (axis cs: 6,0);
						\draw[red, dashed, thick] (axis cs: 1,0) -- (axis cs: 1,0.5);
						\draw[red, dashed, thick] (axis cs: 5,0) -- (axis cs: 5,0.5);
						
						\draw[fill=blue, opacity = 0.2] (axis cs: 1,0) -- (axis cs: 1,1/12) -- (axis cs: 5,5/12) -- (axis cs: 5,0) -- (axis cs: 1,0);
					\end{axis}
				\end{tikzpicture}
			\end{center}
			\pause
			Offensichtlich gilt für alle $x<1: F(x) = 0$ und für $x>5: F(x) = 1$. 
			\pause
			Da $f$ eine Dichte ist, hat die \textcolor{blue}{blaue} Fläche den Inhalt 1.
		\end{itemize}
	\end{frame}
	\begin{frame}
		\frametitle{Aufgabe 1.3}
		\begin{itemize}
			\item[] Der interessante Fall ist $x\in [1,5]$:
			\begin{align}
				F(x) &= \int_{-\infty}^x f(t)\, dt = \int_{1} ^x \frac{1}{12}t\, dt \notag \\
				&= \left.\frac{1}{24}t^2\right\vert_1^x = \frac{x^2 - 1}{24} \notag
			\end{align}
			\pause
			Damit können wir die komplette Verteilungsfunktion zusammensetzen:
			\begin{align}
				F(x) &= \begin{cases}
					0 & x<1 \\
					\frac{x^2-1}{24} &x\in [1,5] \\
					1 & x>5
				\end{cases} \notag
			\end{align}
			\pause
			\item[(b)] $\mathbb{P}(2<X<3) = F(3) - F(2) = \frac{3^2-1}{24} - \frac{2^2-1}{24} = \frac{5}{24}$
		\end{itemize}
	\end{frame}

	\begin{frame}
		\frametitle{Aufgabe 1.4}
		$X$ sei eine stetige Zufallsvariable mit der Dichte
		\begin{align}
			f(x) = \begin{cases}
				cx(1-x) & x\in [0,1] \\
				0 & \text{sonst}
			\end{cases} \notag
		\end{align}
		\begin{enumerate}[label=(\alph*)]
			\item Bestimmen Sie die Konstante $c$.
			\item Wie lautet die Verteilungsfunktion $F$ der Zufallsvariablen $X$?
			\item Berechnen Sie $\mathbb{P}(X\le \frac{1}{2})$, $\mathbb{P}(X> \frac{2}{3})$, $\mathbb{P}(\frac{1}{2} < X \le \frac{2}{3})$, $\mathbb{P}(2 < X\le 3)$.
			\item Berechnen Sie $\E(X)$ und $\Var(X)$.
		\end{enumerate}
		\begin{center}
			\begin{tikzpicture}[scale=0.6]
				\begin{axis}[
					xmin=0, xmax=1, xlabel=$x$,
					ymin=0, ymax=0.5, ylabel=$f(x)$,
					samples=400,
					axis x line=middle,
					axis y line=middle,
					domain=0:1,
					]
					\addplot[mark=none,smooth,blue!25!green, ultra thick] {0.25*x*(1-x)};
					\addplot[mark=none,smooth,blue!50!green, ultra thick] {0.5*x*(1-x)};
					\addplot[mark=none,smooth,blue!75!green, ultra thick] {0.75*x*(1-x)};
					\addplot[mark=none,smooth,blue, ultra thick] {x*(1-x)};
					\addplot[mark=none,smooth,blue!75!red, ultra thick] {1.25*x*(1-x)};
					\addplot[mark=none,smooth,blue!50!red, ultra thick] {1.5*x*(1-x)};
					\addplot[mark=none,smooth,blue!25!red, ultra thick] {1.75*x*(1-x)};
					\addplot[mark=none,smooth,red, ultra thick] {2*x*(1-x)};
				\end{axis}
			\end{tikzpicture}
		\end{center}
	\end{frame}
	\begin{frame}
		\frametitle{Aufgabe 1.4}
		\begin{itemize}
			\item[(a)] Da $f$ eine Dichte sein soll, gilt:
			\begin{align}
				\int_{-\infty}^\infty f(x)\, dx = \int_{0}^1 cx(1-x)\,dx \overset{!}{=} 1 \notag
			\end{align}
			\pause
			Daraus ergibt sich
			\begin{align}
				\int_{0}^1 cx(1-x)\,dx &= c\left.\left(\frac{x^2}{2} - \frac{x^3}{3}\right)\right\vert_0^1 \notag \\
				&= \frac{c}{6} \overset{!}{=} 1 \notag \\
				\Rightarrow c &= 6 \notag
			\end{align}
			\pause
			\item[(b)] Genau wie bei Aufgabe 1.3 ist auch hier für alle $x<0$ $F(x) = 0$ und für alle $x>1$ $F(x) = 1$. Für den interessanten Teil dazwischen ergibt sich:
		\end{itemize}
	\end{frame}
	\begin{frame}
		\frametitle{Aufgabe 1.4}
		\begin{itemize}
			\item[] $x\in [0,1]$
			\begin{align}
				F(x) &= \int_{-\infty}^x f(t)\,dt = \int_{0}^x 6t(1-t)\,dt \notag \\
				&= \left.6\left(\frac{t^2}{2} - \frac{t^3}{3}\right)\right\vert_0^x \notag \\
				&= 6\left(\frac{x^2}{2} - \frac{x^3}{3}\right) \notag
			\end{align}
			\pause
			Damit ist die Verteilungsfunktion
			\begin{align}
				F(x) &= \begin{cases}
					0 & x<0 \\
					6\left(\frac{x^2}{2} - \frac{x^3}{3}\right) &x\in [0,1] \\
					1 & x>1
				\end{cases} \notag
			\end{align}
		\end{itemize}
	\end{frame}
	\begin{frame}
		\frametitle{Aufgabe 1.4}
		\begin{itemize}
			\item[(c)] $\mathbb{P}(X\le \frac{1}{2}) = F(\frac{1}{2}) = \frac{1}{2}$ \pause \\
			$\mathbb{P}(X> \frac{2}{3}) = 1 - F(\frac{2}{3}) = \frac{7}{27}$ \pause \\
			$\mathbb{P}(\frac{1}{2} < X \le \frac{2}{3}) = F(\frac{2}{3}) - F(\frac{1}{2}) = \frac{13}{54}$ \pause \\
			$\mathbb{P}(2 < X\le 3) = F(3) - F(2) = 1 - 1 = 0$
			\pause
			\item[(d)] Für den Erwartungswert gilt:
			\begin{align}
				\E(X) &= \int_{-\infty}^\infty x\cdot f(x)\,dx = \int_{0}^1 x\cdot 6x(1-x)\,dx \notag \\
				&= \frac{1}{2} \notag
			\end{align}
			\pause
			und die Varianz:
			\begin{align}
				\Var(X) &= \int_{-\infty}^\infty x^2\cdot f(x)\,dx - \E(X)^2 \notag \\
				&= \int_{0}^1 x^2\cdot 6x(1-x)\,dx - \left(\frac{1}{2}\right)^2 \notag \\
				&= \frac{1}{20} \notag
			\end{align}
		\end{itemize}
	\end{frame}
\end{document}