\documentclass{article}

\usepackage{amsmath,amssymb}
\usepackage{tikz}
\usepackage{xcolor}
\usepackage[left=2.1cm,right=3.1cm,bottom=3cm,footskip=0.75cm,headsep=0.5cm]{geometry}
\usepackage{enumerate}
\usepackage{enumitem}
\usepackage{marvosym}
\usepackage{tabularx}
\usepackage{pgfplots}
\pgfplotsset{compat=1.10}
\usepgfplotslibrary{fillbetween}
\usepackage{hyperref}
\usepackage{parskip}

\usepackage[utf8]{inputenc}

\renewcommand*{\arraystretch}{1.4}

\newcolumntype{L}[1]{>{\raggedright\arraybackslash}p{#1}}
\newcolumntype{R}[1]{>{\raggedleft\arraybackslash}p{#1}}
\newcolumntype{C}[1]{>{\centering\let\newline\\\arraybackslash\hspace{0pt}}m{#1}}

\DeclareMathOperator{\tr}{tr}
\DeclareMathOperator{\Var}{Var}
\DeclareMathOperator{\Cov}{Cov}
\newcommand{\E}{\mathbb{E}}

\title{\textbf{Statistik 1, Test 5}}
\date{}

\begin{document}
	\maketitle
	
	\section*{Aufgabe 1}
	\begin{enumerate}[label=(\alph*)]
		\item Die 4-Felder-Tafel ist
		\begin{center}
			\begin{tabular}{c|cc|c}
				& nicht infiziert & infiziert & $\Sigma$ \\
				\hline
				Test positiv & 0.0198 & \textcolor{blue}{0.008} & 0.0278 \\
				Test negativ & \textcolor{red}{0.9702} & 0.002 & 0.9722 \\
				\hline
				$\Sigma$ & 0.99 & 0.01 & 1
			\end{tabular}
		\end{center}
		Wobei \textcolor{red}{$0.99\cdot 0.98 = 0.9702$} und \textcolor{blue}{$0.01\cdot 0.8 = 0.008$}.
		\item Die Wahrscheinlichkeit dafür ist $\frac{0.008}{0.0278}=0.2878$.
		\item Die Wahrscheinlichkeit dafür ist $\frac{0.9702}{0.9722}=0.9979$.
		\item Die neue 4-Felder-Tafel ist nun
		\begin{center}
			\begin{tabular}{c|cc|c}
				& nicht infiziert & infiziert & $\Sigma$ \\
				\hline
				Test positiv & 0.018  & 0.08  & 0.098  \\
				Test negativ & 0.882  & 0.02 & 0.902  \\
				\hline
				$\Sigma$ & 0.90 & 0.1 & 1
			\end{tabular}
		\end{center}
		Die Wahrscheinlichkeiten sind dann $\frac{0.08}{0.098}=0.8163$ bzw. $\frac{0.882}{0.902}=0.9778$.
	\end{enumerate}

	\section*{Aufgabe 2}
	\begin{enumerate}[label=(\alph*)]
		\item Damit es sich um eine Dichtefunktion handelt, muss gelten
		\begin{align}
			\int_1^2 f(x) \,dx &= 1 \notag \\
			\int_1^2 a\cdot x\cdot\exp(-x)\,dx &= 1 \notag \\
			\frac{(2e-3)a}{e^2} &= 1 \notag \\
			a &= \frac{e^2}{2e-3} \notag \\
			&= 3.0326 \notag
		\end{align}
		\item Für die Verteilungsfunktion $F(t) = \int_{-\infty}^t f(x)\, dx$ gilt
		\begin{align}
			F(x) = \begin{cases}
				\int_1^x f(t)\, dt & x\in [1,2] \\
				0 & x\notin [1,2]
			\end{cases} = \begin{cases}
				a\left(\frac{2}{\exp(1)}-\exp(-x)(x+1)\right) & x\in [1,2] \\
				0 & x\notin [1,2]
			\end{cases} \notag
		\end{align}
		\item Der Erwartungswert ist
		\begin{align}
			\E(X) &= \int_1^2 x\cdot(3.0326\cdot x\cdot\exp(-x)) \,dx \notag \\
			&= 1.4740 \notag
		\end{align}
		\item Der Erwartungswert ist
		\begin{align}
			\E\left(\frac{1}{3}X^3 + 2X^2\right) &= \int_1^2 \left(\frac{1}{3}X^3 + 2X^2\right)\cdot (3.0326\cdot x\cdot\exp(-x)) \,dx \notag \\
			&= 5.6971 \notag
		\end{align}
		\item Die Varianz ist
		\begin{align}
			\Var(X) &= \int_1^2 x^2\cdot (3.0326\cdot x\cdot\exp(-x))\, dx - 1.4740^2 \notag \\
			&= 0.0815 \notag
		\end{align}
	\end{enumerate}

	\section*{Aufgabe 3}
	Die Aussagen sind
	\begin{itemize}
		\item Die Ereignisse $A$ und $B$ heißen genau dann stochastisch unabhängig, ... RICHTIG
		\item Die Varianz ist linear. FALSCH, es gilt $\Var(a\cdot X) = a^2\cdot\Var(X)$.
		\item Der Erwartungswert ist linear. RICHTIG
		\item Es gilt $\Var(X)=\E(X^2)-\E(X)^2$. RICHTIG
		\item Die standardisierte Zufallsvariable hat Erwartungswert 1 und Varianz 0. FALSCH, der Erwartungswert ist 0 und die Varianz ist 1.
	\end{itemize}

	\section*{Aufgabe 4}
	Die richtigen Eigenschaften sind
	\begin{itemize}
		\item $\mathbb{P}(X>a)=\int_a^{\infty} f(t)\,dt$
		\item $\int_{-\infty}^{\infty} f(t)\, dt=1$
	\end{itemize}

	\section*{Aufgabe 5}
	Die richtigen Eigenschaften sind
	\begin{itemize}
		\item $F$ ist rechtsseitig stetig.
		\item $F(\infty)=1$ und $F(-\infty)=0$.
		\item $F$ ist monoton wachsend.
	\end{itemize}
	
\end{document}