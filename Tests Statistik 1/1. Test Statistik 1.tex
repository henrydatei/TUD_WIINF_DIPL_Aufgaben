\documentclass{article}

\usepackage{amsmath,amssymb}
\usepackage{tikz}
\usepackage{xcolor}
\usepackage[left=2.1cm,right=3.1cm,bottom=3cm,footskip=0.75cm,headsep=0.5cm]{geometry}
\usepackage{enumerate}
\usepackage{enumitem}
\usepackage{marvosym}
\usepackage{tabularx}
\usepackage{pgfplots}
\pgfplotsset{compat=1.10}
\usepgfplotslibrary{fillbetween}
\usepackage{hyperref}
\usepackage{parskip}

\usepackage[utf8]{inputenc}

\renewcommand*{\arraystretch}{1.4}

\newcolumntype{L}[1]{>{\raggedright\arraybackslash}p{#1}}
\newcolumntype{R}[1]{>{\raggedleft\arraybackslash}p{#1}}
\newcolumntype{C}[1]{>{\centering\let\newline\\\arraybackslash\hspace{0pt}}m{#1}}

\DeclareMathOperator{\tr}{tr}
\DeclareMathOperator{\Var}{Var}
\DeclareMathOperator{\Cov}{Cov}
\newcommand{\E}{\mathbb{E}}

\title{\textbf{Statistik 1, Test 1}}
\date{}

\begin{document}
	\maketitle
	
	\section*{Aufgabe 1}
	\begin{enumerate}[label=(\alph*)]
		\item Der Mittelwert ist
		\begin{align}
			\mu &= \frac{16\cdot 180 + 17\cdot 170 + 29\cdot 160 + 37\cdot 150 + 22\cdot 140 + 31\cdot 130 + 37\cdot 120 + 48\cdot 110 + 31\cdot 100 + 27\cdot 90 + 70\cdot 80}{365} \notag \\
			&= 120.3288 \notag
		\end{align}
		\item Der Median ist das mittlere Element der Messreihe, also das $\lceil\frac{365}{2}\rceil=183$. Element. Das ist 120.
		\item Die empirische Varianz ist
		\begin{align}
			\tilde{s}^2 &= \frac{16(180-120.3288)^2 + 17(170-120.3288) + \dots + 70(80-120.3288)}{365} \notag \\
			&= 945.6453 \notag
		\end{align}
		\item Die Stichprobenvarianz ist
		\begin{align}
			s^2 &= \frac{16(180-120.3288)^2 + 17(170-120.3288) + \dots + 70(80-120.3288)}{364} \notag \\
			&= 948.2433 \notag
		\end{align}
		\item Das 25\%-Quantil ist das $\lceil 0.25\cdot 365\rceil = 92$. Element, also 90. \textcolor{red}{Im Forum steht, dass der Test aber 150 als richtigen Wert akzeptiert.}
		\item Das 75\%-Quantil ist das $\lceil 0.75\cdot 365\rceil = 274$. Element, also 150. \textcolor{red}{Im Forum steht, dass der Test aber 90 als richtigen Wert akzeptiert.}
	\end{enumerate}
	
	 In $K_1$ sind 16 + 17 + 29 + 37 + 22 + 31 = 152 Studenten, in $K_2$ sind 37 + 48 + 31 = 116 Studenten, in $K_3$ sind 27 + 70 = 97 Studenten. Die Klassenmitten sind 155, 110, 85, damit ist der Mittelwert der klassierten Daten
	 \begin{align}
	 	\mu &= \frac{152\cdot 155 + 116\cdot 110 + 97\cdot 85}{365} \notag \\
	 	&= 122.0959 \notag
	 \end{align}
	 
	 Mindestens 150 Punkte haben $\frac{16+17+29+37}{365}=0.2712329$ der Studenten erreicht.
	 
	 Nur 90 oder weniger Punkte haben $\frac{29+70}{365}=0.2657 = \hat{F}(90)$ der Studenten erreicht. \textcolor{red}{Der Test nimmt aber nur 0.808219.}
	 
	 Übrigens scheinen das die Prüfungsergebnisse der letzten Statistik 1 Klausur im SS2020 gewesen zu sein: \url{https://wwilq.wiwi.tu-dresden.de/ects_v2/index.php?action=output\&jahr=2020\&status=SS\&typ=PF\&prn=14530}

	\section*{Aufgabe 2}
	Der Mittelwert ist
	\begin{align}
		\mu &= \frac{24+21+29+14+27+13+28+13}{8} \notag \\
		&= 21.125 \notag
	\end{align}
	Die Tabelle sieht so aus
	\begin{center}
		\begin{tabular}{c|c|c|c}
			 & $K_1$ & $K_2$ & $K_3$ \\
			 \hline
			 $n(K_i)$ & 3 & 2 & 3 \\
			 \hline
			 $h(K_i)$ & 0.375 & 0.25 & 0.375
		\end{tabular}
	\end{center}
	Und der Mittelwert der klassierten Daten ist 20.

	\section*{Aufgabe 3}
	Die Aussagen sind:
	\begin{itemize}
		\item Das Merkmal Lieblingsprimzahl ist stetig, da es unendlich viele Primzahlen gibt. FALSCH, stetig wäre das Merkmal nur, wenn es überabzählbar viele Primzahlen gibt, aber die gibt es nicht.
		\item Die entscheidende Information bei einem Histogramm ist nicht die Höhe der Balken, sondern die Fläche! RICHTIG
		\item Im Gegensatz zur induktiven Statistik beziehen sich die Ergebnisse der deskriptiven Statistik nur auf die untersuchte Datenmenge. RICHTIG
		\item Ein Merkmalsvektor $X$ ist eine Abbildung $\Omega\to S$, wobei $S$ der Merkmalsraum ist. RICHTIG
		\item Die empirische Verteilungsfunktion $\hat{F}$ ist linksseitig stetig und monoton steigend. FALSCH, die Funktion ist rechtsseitig stetig.
	\end{itemize}

	\section*{Aufgabe 4}
	Richtig ist
	\begin{itemize}
		\item Sie können das arithmetische Mittel verwenden, wenn Sie die Vermögensangabe von Beate Heister und Karl Albrecht Junior aus Ihrem Datensatz entfernen.
		\item Sie können den Median verwenden, da dieser robust in Bezug auf Ausreißer ist.
	\end{itemize}

	\section*{Aufgabe 5}
	Die Merkmale sind:
	\begin{itemize}
		\item Adresse eines Studenten: nominal, qualitativ, diskret
		\item Preis eines Mensa-Essens: metrisch, quantitativ, quasistetig
		\item Hobby eines Studenten: nominal, qualitativ, diskret
		\item durchschnittliche Länge eines Stückes Kreide im Hörsaal: metrisch, quantitativ, stetig
		\item Datum der Klausur Statistik I: metrisch, quantitativ, diskret
	\end{itemize}
	
	\section*{Aufgabe 6}
	Term D, Term B, Term B, $\lambda = 0.25$, $L=6.58\cdot 10^{-6}$
	
\end{document}