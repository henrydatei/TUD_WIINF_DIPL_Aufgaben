\documentclass{article}

\usepackage{amsmath,amssymb}
\usepackage{tikz}
\usepackage{xcolor}
\usepackage[left=2.1cm,right=3.1cm,bottom=3cm,footskip=0.75cm,headsep=0.5cm]{geometry}
\usepackage{enumerate}
\usepackage{enumitem}
\usepackage{marvosym}
\usepackage{tabularx}
\usepackage{pgfplots}
\pgfplotsset{compat=1.10}
\usepgfplotslibrary{fillbetween}
\usepackage{hyperref}
\usepackage{parskip}

\usepackage[utf8]{inputenc}

\renewcommand*{\arraystretch}{1.4}

\newcolumntype{L}[1]{>{\raggedright\arraybackslash}p{#1}}
\newcolumntype{R}[1]{>{\raggedleft\arraybackslash}p{#1}}
\newcolumntype{C}[1]{>{\centering\let\newline\\\arraybackslash\hspace{0pt}}m{#1}}

\DeclareMathOperator{\tr}{tr}
\DeclareMathOperator{\Var}{Var}
\DeclareMathOperator{\Cov}{Cov}
\newcommand{\E}{\mathbb{E}}

\title{\textbf{Statistik 1, Test 6}}
\date{}

\begin{document}
	\maketitle
	
	\section*{Aufgabe 1}
	Die Aussagen sind
	\begin{itemize}
		\item Eine stetige Verteilung liegt genau dann vor, wenn es unabzählbar viele Ereignisse gibt. FALSCH, es sind überabzählbar viele Ereignisse (unabzählbar ist kein Wort!)
		\item Ist die Zahl $X$ der Ereignisse in einem Intervall der Länge $T$ poissonverteilt, dann sind die Abstände $Y$ zwischen den Ereignissen exponentialverteilt. RICHTIG
		\item Die Dichtefunktion der F-Verteilung ist linksschief. FALSCH, sie ist rechtsschief
		\item Für eine steigende Anzahl an Freiheitsgraden strebt sowohl die Dichte der Chi-Quadrat, als auch die Dichte der t-Verteilung gegen die Dichte der Normalverteilung. RICHTIG
		\item Es sei $X\sim N(\mu_1,\sigma_1^2)$ und $Y\sim N(\mu_2,\sigma_2^2)$. Die Summe  $X+Y\sim N(\mu_1+\mu_2,\sigma_1^2+\sigma_2^2)$. FALSCH, ich weiß aber nicht warum. Ich würde sagen, dass die Aussage richtig ist.
	\end{itemize}

	\section*{Aufgabe 2}
	\begin{enumerate}[label=(\alph*)]
		\item Anzahl der Vögel, die sich auf einem Baum niederlassen, innerhalb eines Jahres: Poissonverteilung
		\item Anzahl der Sechsen bei mehrmaligem Werfen eines Würfels: Binomialverteilung
		\item Anzahl der richtigen Zahlen beim 6-aus-49 Lotto: hypergeometrische Verteilung
		\item Anzahl der Gewinn-Lose bei einer Lotterie mit Nieten und Gewinnen bei mehrfacher Ziehung: Hypergeometrische Verteilung
		\item Anzahl der Würfe mit einem Würfel, bis man eine Sechs bekommt: geometrische Verteilung
		\item Anzahl der erhaltenen Emails an einem Tag: Poissonverteilung
		\item Anzahl der Jungen bei nacheinanderfolgenden Geburten: Binomialverteilung
		\item Anzahl der Versuche, bis man die Führerscheinprüfung bestanden hat: geometrische Verteilung
	\end{enumerate}

\end{document}