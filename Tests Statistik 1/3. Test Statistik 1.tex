\documentclass{article}

\usepackage{amsmath,amssymb}
\usepackage{tikz}
\usepackage{xcolor}
\usepackage[left=2.1cm,right=3.1cm,bottom=3cm,footskip=0.75cm,headsep=0.5cm]{geometry}
\usepackage{enumerate}
\usepackage{enumitem}
\usepackage{marvosym}
\usepackage{tabularx}
\usepackage{pgfplots}
\pgfplotsset{compat=1.10}
\usepgfplotslibrary{fillbetween}
\usepackage{hyperref}
\usepackage{parskip}

\usepackage[utf8]{inputenc}

\renewcommand*{\arraystretch}{1.4}

\newcolumntype{L}[1]{>{\raggedright\arraybackslash}p{#1}}
\newcolumntype{R}[1]{>{\raggedleft\arraybackslash}p{#1}}
\newcolumntype{C}[1]{>{\centering\let\newline\\\arraybackslash\hspace{0pt}}m{#1}}

\DeclareMathOperator{\tr}{tr}
\DeclareMathOperator{\Var}{Var}
\DeclareMathOperator{\Cov}{Cov}
\newcommand{\E}{\mathbb{E}}

\title{\textbf{Statistik 1, Test 3}}
\date{}

\begin{document}
	\maketitle
	
	\section*{Aufgabe 1}
	\begin{enumerate}[label=(\alph*)]
		\item Der Mittelwert der Länge ist
		\begin{align}
			\mu_l &= \frac{5.1 + 4.9 + 4.7 + 4.6 + 5.0 + 5.4 + 4.6 + 5.0 + 4.4 + 4.9}{10} \notag \\
			&= 4.86 \notag
		\end{align}
		\item Der Mittelwert der Breite ist
		\begin{align}
			\mu_b &= \frac{3.5 + 3.0 + 3.2 + 3.1 + 3.6 + 3.9 + 3.4 + 3.4 + 2.9 + 3.1}{10} \notag \\
			&= 3.31 \notag
		\end{align}
		\item Die Stichprobenvarianz der Länge ist
		\begin{align}
			s_l^2 &= \frac{(5.1-4.86)^2 + (4.9-4.86)^2 + \dots + (4.9-4.86)^2}{9} \notag \\
			&= 0.0849 \notag
		\end{align}
		\item Die Stichprobenvarianz der Breite ist
		\begin{align}
			s_b^2 &= \frac{(3.5-3.31)^2 + (3.0-3.31)^2 + \dots + (3.1-3.31)^2}{9} \notag \\
			&= 0.0943 \notag
		\end{align}
		\item Die Stichprobenkovarianz ist dann
		\begin{align}
			s_{lb} &= \frac{(5.1-4.86)(3.5-3.31) + (4.9-4.86)(3.0-3.31) + \dots + (4.9-4.86)(3.1-3.31)}{9} \notag \\
			&= 0.0704 \notag
		\end{align}
		\item Der Korrelationskoeffizient ist dann
		\begin{align}
			r &= \frac{s_{lb}}{\sqrt{s^2_l}\cdot \sqrt{s_b^2}} \notag \\
			&= \frac{0.0704}{\sqrt{0.0849} \cdot \sqrt{0.0943}} \notag \\
			&= 0.7868 \notag
		\end{align}
		$\Rightarrow$ Das lässt auf einen mittleren, linearen, positiven Zusammenhang schließen.
	\end{enumerate}

	\section*{Aufgabe 2}
	\begin{enumerate}[label=(\alph*)]
		\item Zuerst müssen wir den Daten einen Rang zuordnen. Für die Michelin-Sterne gilt
		\begin{center}
			\begin{tabular}{c|cccccccccccc}
				$x_i$ & 2 & 2 & 3 & 3 & 3 & 3 & 3 & 4 & 4 & 4 & 4 & 4 \\
				\hline
				$R(x_i)$ & 1.5 & 1.5 & 5 & 5 & 5 & 5 & 5 & 10 & 10 & 10 & 10 & 10
			\end{tabular}
		\end{center}
		Für die Kreativität
		\begin{center}
			\begin{tabular}{c|cccccccccccc}
				$y_i$ & 0 & 0 & 0 & 0 & 0 & 1 & 1 & 0 & 0 & 1 & 1 & 1 \\
				\hline
				$R(y_i)$ & 4 & 4 & 4 & 4 & 4 & 10 & 10 & 4 & 4 & 10 & 10 & 10
			\end{tabular}
		\end{center}
		Der durchschnittliche Rang ist damit
		\begin{align}
			\bar{R} &= \frac{1.5 + 1.5 + 5 + \dots + 10 + 4 + 4 + \dots + 10}{24} \notag \\
			&= 6.5 \notag
		\end{align}
		\item Der Korrelationskoeffizient nach Spearman ist dann
		\begin{align}
			R &= \frac{1.5\cdot 4 + 1.5\cdot 4 + \dots + 10\cdot 4 + 10\cdot 10 - 12\cdot 6.5^2}{\sqrt{1.5^2 + 1.5^2 + \dots + 10^2 - 12\cdot 6.5^2}\cdot\sqrt{4^2 + 4^2 + \dots + 10^2 - 12\cdot 6.5^2}} \notag \\
			&= 0.396 \notag
		\end{align}
		Es handelt sich um einen mittleren monotonen positiven Zusammenhang.
	\end{enumerate}

	\section*{Aufgabe 3}
	\begin{enumerate}[label=(\alph*)]
		\item Die Kontingenztafel der absoluten Häufigkeiten ist
		\begin{center}
			\begin{tabular}{c|cc|c}
				& männlich & weiblich & $\Sigma$ \\
				\hline
				Informatik & 15 & 5 & 20 \\
				Wiwi & 35 & 35 & 70 \\
				Geisteswissenschaften & 0 & 10 & 10 \\
				\hline
				$\Sigma$ & 50 & 50 & 100 
			\end{tabular}
		\end{center}
		\item Die Kontingenztafel der relativen Häufigkeiten ist
		\begin{center}
			\begin{tabular}{c|cc|c}
				& männlich & weiblich & $\Sigma$ \\
				\hline
				Informatik & 0.15 & 0.05 & 0.2 \\
				Wiwi & 0.35 & 0.35 & 0.7 \\
				Geisteswissenschaften & 0 & 0.1 & 0.1 \\
				\hline
				$\Sigma$ & 0.5 & 0.5 & 1 
			\end{tabular}
		\end{center}
		\item $\chi^2$ ist dann
		\begin{align}
			\chi^2 &= 100\left(\frac{15^2}{20\cdot 50} + \frac{5^2}{20\cdot 50} + \frac{35^2}{70\cdot 50} + \frac{35^2}{70\cdot 50} + \frac{0^2}{10\cdot 50} + \frac{10^2}{10\cdot 50}-1\right) \notag \\
			&= 15 \notag
		\end{align}
		\item Für $C_{max}$ gilt
		\begin{align}
			C_{max} &= \sqrt{\frac{2-1}{2}} \notag \\
			&= 0.7071 \notag
		\end{align}
		\item Für $C$ gilt
		\begin{align}
			C &= \sqrt{\frac{15}{15+100}} \notag \\
			&= 0.3612 \notag
		\end{align}
		\item Für $C_{Korr}$ gilt
		\begin{align}
			C_{Korr} &= \frac{0.3612}{0.7071} \notag \\
			&= 0.5108 \notag
		\end{align}
		Es besteht also ein mittlerer Zusammenhang.
	\end{enumerate}

	\section*{Aufgabe 4}
	\begin{enumerate}[label=(\alph*)]
		\item Alter und Haarfarbe: Kontingenzkoeffizient
		\item Alter und Einkommen: Bravis-Pearson, Spearman, Kontingenzkoeffizient
		\item Abiturnote und Bachelornote: Spearman, Kontingenzkoeffizient
		\item Preis ein Kunstwerkes und Names des Künstlers: Kontingenzkoeffizient
		\item Körpergröße und Körpergewicht: Bravis-Pearson, Spearman, Kontingenzkoeffizient
		\item Zufriedenheit des Kunden und Rabatt: Spearman, Kontingenzkoeffizient
	\end{enumerate}

	\section*{Aufgabe 5}
	Die Aussagen sind
	\begin{itemize}
		\item Wenn der Korrelationskoeffizient nach Pearson zwischen zwei Stichproben klein ist, so sind die Merkmale unabhängig voneinander. FALSCH, die Merkmale könnten z.B. auch quadratisch voneinander abhängen.
		\item $R_{XY}=\text{Cor}(R(X),R(Y))$. RICHTIG
		\item Die Zuordnung $x\mapsto R(x)$ ist eineindeutig. FALSCH, da man aus $R(x)$ nicht auf x schließen kann.
		\item Der Korrelationskoeffizient von Spearman liegt immer im Intervall $[-1,1]$. RICHTIG
		\item Genau dann wenn $X$ und $Y$ unabhängig sind, folgt $C_{Korr}=0$. RICHTIG
	\end{itemize}
	
\end{document}