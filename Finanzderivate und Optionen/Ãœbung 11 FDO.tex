\documentclass{article}

\usepackage{amsmath,amssymb}
\usepackage{tikz}
\usepackage{pgfplots}
\usepackage{xcolor}
\usepackage[left=2.1cm,right=3.1cm,bottom=3cm,footskip=0.75cm,headsep=0.5cm]{geometry}
\usepackage{enumerate}
\usepackage{enumitem}
\usepackage{marvosym}
\usepackage{tabularx}
\usepackage{parskip}

\usepackage{listings}
\definecolor{lightlightgray}{rgb}{0.95,0.95,0.95}
\definecolor{lila}{rgb}{0.8,0,0.8}
\definecolor{mygray}{rgb}{0.5,0.5,0.5}
\definecolor{mygreen}{rgb}{0,0.8,0.26}
%\lstdefinestyle{java} {language=java}
\lstset{language=R,
	basicstyle=\ttfamily,
	keywordstyle=\color{lila},
	commentstyle=\color{lightgray},
	stringstyle=\color{mygreen}\ttfamily,
	backgroundcolor=\color{white},
	showstringspaces=false,
	numbers=left,
	numbersep=10pt,
	numberstyle=\color{mygray}\ttfamily,
	identifierstyle=\color{blue},
	xleftmargin=.1\textwidth, 
	%xrightmargin=.1\textwidth,
	escapechar=§,
	%literate={\t}{{\ }}1
	breaklines=true,
	postbreak=\mbox{\space}
}

\usepackage[colorlinks = true, linkcolor = blue, urlcolor  = blue, citecolor = blue, anchorcolor = blue]{hyperref}
\usepackage[utf8]{inputenc}

\renewcommand*{\arraystretch}{1.4}

\newcolumntype{L}[1]{>{\raggedright\arraybackslash}p{#1}}
\newcolumntype{R}[1]{>{\raggedleft\arraybackslash}p{#1}}
\newcolumntype{C}[1]{>{\centering\let\newline\\\arraybackslash\hspace{0pt}}m{#1}}

\newcommand{\E}{\mathbb{E}}
\DeclareMathOperator{\rk}{rk}
\DeclareMathOperator{\Var}{Var}
\DeclareMathOperator{\Cov}{Cov}

\title{\textbf{Finanzderivate und Optionen, Übung 11}}
\author{\textsc{Henry Haustein}}
\date{}

\begin{document}
	\maketitle
	
	\section*{Aufgabe 1}
	Falsch, EMIR setzt sich für Stärkung von Transparenz im OTC-Handel ein (zentrale Gegenparteien, Transaktionsregister)	

	\section*{Aufgabe 2}
	Richtig, es geht z.B. darum wie Firewalls bei Banken eingerichtet werden oder wie Gelder von Anlegern getrennt werden müssen. Weitere Punkte: Tick-Größe, Positions-Größen, HFT
	
	\section*{Aufgabe 3}
	PRA und FCA
	
	\section*{Aufgabe 4}
	Falsch
	
	\section*{Aufgabe 5}
	Pro-Rata Methode
	
	\section*{Aufgabe 6}
	Zeit-Pro-Rata Methode
	
	\section*{Aufgabe 7}
	Es wird die Order 30 @ 3125 (für 3125) und die Order 30 @ 3124 (für 3124) ausgeführt. Auf der Ask-Seite verbleibt der Rest 40 @ 3124.
	
	\section*{Aufgabe 8}
	Da bisher nichts auf der Bid-Seite steht, wird solange gewartet, bis dort eine weitere Order reinkommt, damit man weiß, wie die Market-Order-Matching-Range beginnt und endet.
	
	\section*{Aufgabe 9}
	Matching mit der Market-Order 20 @ 3.16. 
	
	\section*{Aufgabe 10}
	\begin{center}
		\begin{tabular}{c|c|c|c|c}
			\textbf{Order} & \textbf{Quantity} & \textbf{Anteil} & \textbf{zugeteilt} & \textbf{noch offen} \\
			\hline
			1 & 20 & $\frac{20}{90}\cdot 25 = 5.55$ & 6 & 14 \\
			2 & 20 & $\frac{20}{70}\cdot 19 = 5.42$ & 5 & 15 \\
			3 & 50 & $\frac{50}{50}\cdot 14 = 14$ & 14 & 36
		\end{tabular}
	\end{center}
	
	
\end{document}