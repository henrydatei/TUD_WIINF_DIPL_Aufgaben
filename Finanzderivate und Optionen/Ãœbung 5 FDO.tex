\documentclass{article}

\usepackage{amsmath,amssymb}
\usepackage{tikz}
\usepackage{pgfplots}
\usepackage{xcolor}
\usepackage[left=2.1cm,right=3.1cm,bottom=3cm,footskip=0.75cm,headsep=0.5cm]{geometry}
\usepackage{enumerate}
\usepackage{enumitem}
\usepackage{marvosym}
\usepackage{tabularx}
\usepackage{parskip}

\usepackage{listings}
\definecolor{lightlightgray}{rgb}{0.95,0.95,0.95}
\definecolor{lila}{rgb}{0.8,0,0.8}
\definecolor{mygray}{rgb}{0.5,0.5,0.5}
\definecolor{mygreen}{rgb}{0,0.8,0.26}
%\lstdefinestyle{java} {language=java}
\lstset{language=R,
	basicstyle=\ttfamily,
	keywordstyle=\color{lila},
	commentstyle=\color{lightgray},
	stringstyle=\color{mygreen}\ttfamily,
	backgroundcolor=\color{white},
	showstringspaces=false,
	numbers=left,
	numbersep=10pt,
	numberstyle=\color{mygray}\ttfamily,
	identifierstyle=\color{blue},
	xleftmargin=.1\textwidth, 
	%xrightmargin=.1\textwidth,
	escapechar=§,
	%literate={\t}{{\ }}1
	breaklines=true,
	postbreak=\mbox{\space}
}

\usepackage[colorlinks = true, linkcolor = blue, urlcolor  = blue, citecolor = blue, anchorcolor = blue]{hyperref}
\usepackage[utf8]{inputenc}

\renewcommand*{\arraystretch}{1.4}

\newcolumntype{L}[1]{>{\raggedright\arraybackslash}p{#1}}
\newcolumntype{R}[1]{>{\raggedleft\arraybackslash}p{#1}}
\newcolumntype{C}[1]{>{\centering\let\newline\\\arraybackslash\hspace{0pt}}m{#1}}

\newcommand{\E}{\mathbb{E}}
\DeclareMathOperator{\rk}{rk}
\DeclareMathOperator{\Var}{Var}
\DeclareMathOperator{\Cov}{Cov}

\title{\textbf{Finanzderivate und Optionen, Übung 5}}
\author{\textsc{Henry Haustein}}
\date{}

\begin{document}
	\maketitle
	
	\section*{Aufgabe 1}
	\begin{enumerate}[label=(\alph*)]
		\item\textcolor{green}{zahlt den variablen Zinssatz}
		\item\textcolor{red}{empfängt den variablen Zinssatz}
		\item\textcolor{red}{zahlt den festen Zinssatz}
		\item\textcolor{green}{empfängt den festen Zinssatz}
	\end{enumerate}

	\section*{Aufgabe 2}
	\begin{enumerate}[label=(\alph*)]
		\item \textcolor{green}{Kreditnehmer mit variabel verzinstem Kredit können sich durch den Kauf eines FRA gegen steigende Zinsen schützen.}
		\item \textcolor{red}{Kreditnehmer mit variabel verzinstem Kredit können sich durch den Verkauf eines FRA gegen steigende Zinsen schützen.}
		\item \textcolor{green}{Ein Händler kann bei Erwartung steigender Zinsen FRAs kaufen, wenn er der Meinung ist, dass der Zins zum Zeitpunkt der zukünftigen Zinsfeststellung oberhalb des im FRA vereinbarten Zinssatzes liegen wird.}
		\item \textcolor{red}{Ein Händler kann bei Erwartung fallender Zinsen FRAs kaufen, wenn er der Meinung ist, dass der Zins zum Zeitpunkt der zukünftigen Zinsfeststellung unterhalb des im FRA vereinbarten Zinssatzes liegen wird.}
	\end{enumerate}
	
	\section*{Aufgabe 3}
	Richtig
	
	\section*{Aufgabe 4}
	Falsch, die Wette war, dass die Zinsen steigen, aber das hat sich nicht bestätigt.
	
	\section*{Aufgabe 5}
	Unternehmen A nimmt Kredit zu 5\% auf und macht Swap mit Bank (zahlt LIBOR, erhält 5.3\%). Insgesamt: LIBOR - 0.3\% (statt LIBOR + 0.1\%)
	
	Unternehmen B nimmt Kredit zu LIBOR + 0.6\% auf und macht Swap mit Bank (zahlt 5.4\%, erhält LIBOR). Insgesamt: 6\% (statt 6.4\%)
	
	Die Bank zahlt einmal LIBOR und erhält ihn, außerdem erhält sie 5.4\% und zahlt 5.3\%. Es bleiben 0.1\% Gewinn.
	
	\section*{Aufgabe 6}
	Die Seite, die in Pfund bezahlt, muss jedes Jahr 2 Millionen Pfund zahlen und erhält 1.8 Millionen Dollar. Die erste Zahlung in 3 Monaten, es handelt sich um ein ganzjähriges Geschäft.
	
	Am Ende des Swaps Austausch der Nominalbeträge (bei Währungsswaps wird immer der Nominalbetrag getauscht):
	\begin{align}
		\frac{2}{1.07^{1/4}} + \frac{22}{1.07^{5/4}} &= 22.182 \notag \\
		\frac{1.8}{1.04^{1/4}} + \frac{31.8}{1.04^{5/4}} &= 32.061 \notag
	\end{align}
	Die Seite die Leistungen in GBP leistet: $32.061 - (22.182\cdot 185) = -8.976$ Dollar \\
	Die Seite die Leistungen in USD leistet: $-32.061 + (22.182\cdot 185) = 8.976$ Dollar
	
	\section*{Aufgabe 7}
	Gewinn: $(3.5\% - 2.5\%)\cdot 15000000 \cdot \frac{90}{360} = 37500$
	
	\section*{Aufgabe 8}
	Am Ende des Kontraktes
	
	\section*{Aufgabe 9}
	Ich erhalte $10000000\cdot 0.4\%\cdot\frac{90}{360} = 10000$
	
	FRA 3/6 (oder FRA 3x6) bedeutet: In 3 Monaten beginnt die Zinsperiode, die Zahlung in 6 Monaten. Damit ist die Zinsperiode 3 Monate lang.
	
	\section*{Aufgabe 10}
	\begin{enumerate}[label=(\alph*)]
		\item\textcolor{red}{Laufzeit des Futures}
		\item\textcolor{red}{Nominalwert}
		\item\textcolor{green}{Basispunktwert}
		\item\textcolor{green}{Modified Duration}
	\end{enumerate}
	
	\section*{Aufgabe 11}
	Lieferpreis: $((155\cdot 0.63) + 0.45)\cdot 1000 = 98100$ \\
	Einstandskosten: $(97.50 + 0.45)\cdot 1000 = 97950$ \\
	Gewinn: 150
	
	\section*{Aufgabe 12}
	Ja, Futures kaufen ist wie den DAX auf Kredit kaufen. Wenn das Portfolio nur deutsch ist, dann erhöht sich damit das $\beta$.
	
	\section*{Aufgabe 13}
	Die Hedge Ratio ist
	\begin{align}
		HR &= \frac{1450000\cdot 0.87}{4211\cdot 10} \notag \\
		&= 29.9573 \notag
	\end{align}
	Mit dem Kauf von 30 Kontrakten ist das Portfolio etwas überabgesichert.
	
	\section*{Aufgabe 14}
	Die Hedge Ratio ist
	\begin{align}
		HR &= \frac{4400}{0.02933\cdot 1000} \notag \\
		&= 150.02 \notag
	\end{align}
	Also sollten 150 Kontrakte verkauft werden.
	
	\section*{Aufgabe 15}
	\begin{enumerate}[label=(\alph*)]
		\item\textcolor{green}{Basisrisiko}
		\item\textcolor{green}{Unterschied zwischen Laufzeit des Absicherungsinstrumentes und Halteperiode des Basiswertes}
		\item\textcolor{green}{Keine Übereinstimmung zwischen abzusicherndem Basiswert und dem Absicherungsinstrument}
		\item\textcolor{red}{Ein Cross Hedge ist immer ein perfekter Hedge}
	\end{enumerate}

	\section*{Aufgabe 16}
	Der Verlust ist $(152.50 - 154.40)\cdot 1000 = -1900$
	
	\section*{Aufgabe 17}
	Es gilt $(4817 - x)\cdot 25 = -375$, damit $x = 4832$
	
	\section*{Aufgabe 18}
	Es gilt $(155.40 - 156.70)\cdot n\cdot 1000 = -84500$, damit $n=65$
	
\end{document}