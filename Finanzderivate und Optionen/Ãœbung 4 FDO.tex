\documentclass{article}

\usepackage{amsmath,amssymb}
\usepackage{tikz}
\usepackage{pgfplots}
\usepackage{xcolor}
\usepackage[left=2.1cm,right=3.1cm,bottom=3cm,footskip=0.75cm,headsep=0.5cm]{geometry}
\usepackage{enumerate}
\usepackage{enumitem}
\usepackage{marvosym}
\usepackage{tabularx}
\usepackage{parskip}

\usepackage{listings}
\definecolor{lightlightgray}{rgb}{0.95,0.95,0.95}
\definecolor{lila}{rgb}{0.8,0,0.8}
\definecolor{mygray}{rgb}{0.5,0.5,0.5}
\definecolor{mygreen}{rgb}{0,0.8,0.26}
%\lstdefinestyle{java} {language=java}
\lstset{language=R,
	basicstyle=\ttfamily,
	keywordstyle=\color{lila},
	commentstyle=\color{lightgray},
	stringstyle=\color{mygreen}\ttfamily,
	backgroundcolor=\color{white},
	showstringspaces=false,
	numbers=left,
	numbersep=10pt,
	numberstyle=\color{mygray}\ttfamily,
	identifierstyle=\color{blue},
	xleftmargin=.1\textwidth, 
	%xrightmargin=.1\textwidth,
	escapechar=§,
	%literate={\t}{{\ }}1
	breaklines=true,
	postbreak=\mbox{\space}
}

\usepackage[colorlinks = true, linkcolor = blue, urlcolor  = blue, citecolor = blue, anchorcolor = blue]{hyperref}
\usepackage[utf8]{inputenc}

\renewcommand*{\arraystretch}{1.4}

\newcolumntype{L}[1]{>{\raggedright\arraybackslash}p{#1}}
\newcolumntype{R}[1]{>{\raggedleft\arraybackslash}p{#1}}
\newcolumntype{C}[1]{>{\centering\let\newline\\\arraybackslash\hspace{0pt}}m{#1}}

\newcommand{\E}{\mathbb{E}}
\DeclareMathOperator{\rk}{rk}
\DeclareMathOperator{\Var}{Var}
\DeclareMathOperator{\Cov}{Cov}

\title{\textbf{Finanzderivate und Optionen, Übung 4}}
\author{\textsc{Henry Haustein}}
\date{}

\begin{document}
	\maketitle
	
	\section*{Aufgabe 1}
	Reverse-Cash-and-Carry: Verkauf des Underlyings, Kauf des Futures $\to$ festgelegter Preis ist der Verkaufspreis des Underlyings

	\section*{Aufgabe 2}
	Richtig ist \textit{Gewinn (Verlust) der Future-Position/Initial Margin}.
	
	\section*{Aufgabe 3}
	\begin{enumerate}[label=(\alph*)]
		\item\textcolor{green}{Finanzierungskosten sind höher als die Erlöse.}
		\item\textcolor{red}{Erlöse sind höher als die Finanzierungskosten.}
		\item\textcolor{green}{Erlöse sind geringer als die Lagerhaltungskosten.}
		\item\textcolor{red}{Lagerhaltungskosten sind größer als die Versicherungskosten.}
	\end{enumerate}
	
	\section*{Aufgabe 4}
	Es gilt $\text{Basis} = \text{Kassa} - \text{Termin}$, also ist der Futurepreis hier 490 Euro.
	
	\section*{Aufgabe 5}
	Richtig
	
	\section*{Aufgabe 6}
	Es gilt ($r_1, r_2$ Rendite für Geldanlage 1 oder 2 Jahre, $f_1$ Forward Rate für 1 Jahr)
	\begin{align}
		(1+r_1)(1+r_1f_1) &= (1+r_2)^2 \notag \\
		r_1f_1 &= \frac{(1+r_2)^2}{1+r_1}-1 \notag \\
		&= \frac{1.07^2}{1.05}-1 \notag \\
		&= 0.0904 \notag
	\end{align}
	
	\section*{Aufgabe 7}
	Es gilt
	\begin{align}
		(1+r_2)^2(1+r_1f_2) &= (1+r_3)^3 \notag \\
		r_1f_2 &= \frac{(1+r_3)^3}{(1+r_2)^2}-1 \notag \\
		&= \frac{1.095^3}{1.07^2}-1 \notag \\
		&= 0.1468 \notag
	\end{align}
	
	\section*{Aufgabe 8}
	Es gilt
	\begin{align}
		(1+r_1)(1+r_1f_1) &= (1+r_2)^2 \notag \\
		r_1f_1 &= \frac{(1+r_2)^2}{1+r_1}-1 \notag \\
		&= \frac{1.07^2}{1.06}-1 \notag \\
		&= 0.0801 \notag
	\end{align}
	
	\section*{Aufgabe 9}
	\begin{itemize}
		\item Hohe Nachfrage (und damit niedrige Rendite) weil T-Bills zur Erfüllung regulatorischer Anforderungen gebraucht werden
		\item Steuerlicher Vorteil, da man sie nicht auf Bundesstaat-Ebene versteuern muss
		\item Deposit Requirements sind bei T-Bills geringer als bei Assets mit ähnlich geringem Risiko
	\end{itemize}
	
	\section*{Aufgabe 10}
	\begin{itemize}
		\item LIBOR: London Interbank Offered Rate ist ein in London an allen Bankarbeitstagen ermittelter Referenzzinssatz, der unter anderem als Grundlage für die Berechnung des Kreditzinses herangezogen wird.
		\item EONIA: Der EONIA (Euro OverNight Index Average) war der Zinssatz, zu dem auf dem Interbankenmarkt im Euroraum unbesicherte Ausleihungen in Euro von einem Target-Tag auf den nächsten gewährt wurden. Ein Target-Tag ist jeder Tag, an dem das Target-2-System Zahlungen abwickelt.
		\item \EUR STR: Die Euro Short-Term Rate (\EUR STR) ist ein Referenzzinssatz für die Währung Euro. Die €STR wird von der Europäischen Zentralbank (EZB) ermittelt und basiert auf der Geldmarktstatistik des Eurosystems. Nachfolger von EONIA
		\item EURIBOR: Die Abkürzung Euribor steht für Euro Interbank Offered Rate. Euribor bezeichnet die durchschnittlichen Zinssätze, zu denen viele europäische Banken einander Anleihen in Euro gewähren. Dabei gelten verschiedene Laufzeiten: von einer Woche bis 12 Monate. Nachfolger von LIBOR
		\item SOFR: Secured Overnight Financing Rate (SOFR) ist ein Referenzzinssatz für die Währung US-Dollar. Der SOFR basiert auf den Transaktionen des US-Dollar Repo-Marktes.
	\end{itemize}
	
	\section*{Aufgabe 11}
	Weil die Geschäfte mit Gütern besichert sind und die Laufzeiten kurz sind.
	
	\section*{Aufgabe 12}
	Forward-Preis ist der heutige Preis zu dem ein Asset an einem zukünftigen Zeitpunkt ge- oder verkauft werden kann. Wer des Kontaktes ist zum Abschluss 0, ändert sich aber mit der Zeit durch Preisänderungen des Underlyings.
	
	\section*{Aufgabe 13}
	Richtig, da Gewinn: $1000\cdot (105.10 - 102.70) = 2400$ (die Kontraktgröße findet man im Handbuch)
	
	\section*{Aufgabe 14}
	Richtig ist \textit{Short einjährige Anleihe und Long zweijährige Anleihe}. Der Future sorgt dafür, dass man von Jahr 1-2 eine Anleihe hat, im Jahr 0-1 keine. Durch den Kauf der zweijährigen Anleihe habe ich schon die Anleihe für die Jahre 1-2, die Anleihe im Jahr 0-1 bekomme ich durch die Short-Position weg.
	
	\section*{Aufgabe 15}
	Falsch, da Gewinn: $4\cdot 10\cdot (3533 - 3407) = 5040$
	
\end{document}