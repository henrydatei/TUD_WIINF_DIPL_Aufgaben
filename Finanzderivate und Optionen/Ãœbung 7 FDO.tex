\documentclass{article}

\usepackage{amsmath,amssymb}
\usepackage{tikz}
\usepackage{pgfplots}
\usepackage{xcolor}
\usepackage[left=2.1cm,right=3.1cm,bottom=3cm,footskip=0.75cm,headsep=0.5cm]{geometry}
\usepackage{enumerate}
\usepackage{enumitem}
\usepackage{marvosym}
\usepackage{tabularx}
\usepackage{parskip}

\usepackage{listings}
\definecolor{lightlightgray}{rgb}{0.95,0.95,0.95}
\definecolor{lila}{rgb}{0.8,0,0.8}
\definecolor{mygray}{rgb}{0.5,0.5,0.5}
\definecolor{mygreen}{rgb}{0,0.8,0.26}
%\lstdefinestyle{java} {language=java}
\lstset{language=R,
	basicstyle=\ttfamily,
	keywordstyle=\color{lila},
	commentstyle=\color{lightgray},
	stringstyle=\color{mygreen}\ttfamily,
	backgroundcolor=\color{white},
	showstringspaces=false,
	numbers=left,
	numbersep=10pt,
	numberstyle=\color{mygray}\ttfamily,
	identifierstyle=\color{blue},
	xleftmargin=.1\textwidth, 
	%xrightmargin=.1\textwidth,
	escapechar=§,
	%literate={\t}{{\ }}1
	breaklines=true,
	postbreak=\mbox{\space}
}

\usepackage[colorlinks = true, linkcolor = blue, urlcolor  = blue, citecolor = blue, anchorcolor = blue]{hyperref}
\usepackage[utf8]{inputenc}

\renewcommand*{\arraystretch}{1.4}

\newcolumntype{L}[1]{>{\raggedright\arraybackslash}p{#1}}
\newcolumntype{R}[1]{>{\raggedleft\arraybackslash}p{#1}}
\newcolumntype{C}[1]{>{\centering\let\newline\\\arraybackslash\hspace{0pt}}m{#1}}

\newcommand{\E}{\mathbb{E}}
\DeclareMathOperator{\rk}{rk}
\DeclareMathOperator{\Var}{Var}
\DeclareMathOperator{\Cov}{Cov}

\title{\textbf{Finanzderivate und Optionen, Übung 7}}
\author{\textsc{Henry Haustein}}
\date{}

\begin{document}
	\maketitle
	
	\section*{Aufgabe 1}
	Ja, in Summe schon. Zwar ist bei einer Conversion und einem Reversal immer noch ein Long Aktie und Short Aktie dabei, aber das hebt sich auf.

	\section*{Aufgabe 2}
	Falsch, der Wert einer Box ist am Ausübungstag die Differenz der Strike-Preise. Für den Wert von heute muss also abgezinst werden. Macht man eine Box auf FI-Produkte, so muss man gar nicht mit den Zinsen arbeiten, da die Zinsen direkt im Preis eingepreist sind.
	
	\section*{Aufgabe 3}
	Wert der Box
	\begin{align}
		P = \frac{550-500}{1.1} = 45.45 \notag
	\end{align}
	
	\section*{Aufgabe 4}
	Es gilt ($r$ für 3 Monate)
	\begin{align}
		490 &= \frac{9500-9000}{1 + r} \notag \\
		r &= 0.0204 \notag
	\end{align}
	
	\section*{Aufgabe 5}
	Kauf eines Calls $K_1$, Verkauf eines Calls mit $K_2$ und $K_1 < K_2$ \\
	max. Verlust: Nettoprämie \\
	max. Gewinn: $K_2 - K_1 - \text{Nettoprämie}$ \\
	BE-Point: $K_1 + \text{Nettoprämie}$
	
	\section*{Aufgabe 6}
	Kauf eines Calls $K_2$, Verkauf eines Calls mit $K_1$ und $K_1 < K_2$ \\
	max. Verlust: $K_2 - K_1 - \text{Nettoprämie}$ \\
	max. Gewinn: Nettoprämie \\
	BE-Point: $K_1 + \text{Nettoprämie}$
	
	\section*{Aufgabe 7}
	Kauf eines Put $K_1$, Verkauf eines Put mit $K_2$ und $K_1 < K_2$ \\
	max. Verlust: $K_2 - K_1 - \text{Nettoprämie}$ \\
	max. Gewinn: Nettoprämie \\
	BE-Point: $K_2 - \text{Nettoprämie}$
	
	\section*{Aufgabe 8}
	Kauf eines Put $K_2$, Verkauf eines Put mit $K_1$ und $K_1 < K_2$ \\
	max. Verlust: Nettoprämie \\
	max. Gewinn: $K_2 - K_1 - \text{Nettoprämie}$ \\
	BE-Point: $K_2 - \text{Nettoprämie}$
	
	\section*{Aufgabe 9}
	Kauf eines Calls und eines Puts zum selben Strike $K$ \\
	max. Verlust: gezahlte Prämien \\
	max. Gewinn: unbegrenzt \\
	BE-Point: $K \pm \text{gezahlte Prämien}$
	
	\section*{Aufgabe 10}
	Verkauf eines Calls und eines Puts zum selben Strike $K$ \\
	max. Verlust: unbegrenzt \\
	max. Gewinn: erhaltene Prämien \\
	BE-Point: $K \pm \text{erhaltene Prämien}$
	
	\section*{Aufgabe 11}
	Kauf eines Puts $K_1$ und eines Calls $K_2$ mit $K_1 < K_2$ \\
	max. Verlust: gezahlte Prämien \\
	max. Gewinn: unbegrenzt \\
	BE-Point: $K_2 + \text{gezahlte Prämien}$ oder $K_1 - \text{gezahlte Prämien}$
	
	\section*{Aufgabe 12}
	Verkauf eines Puts $K_1$ und eines Calls $K_2$ mit $K_1 < K_2$ \\
	max. Verlust: unbegrenzt \\
	max. Gewinn: erhaltene Prämien \\
	BE-Point: $K_2 + \text{erhaltene Prämien}$ oder $K_1 - \text{erhalteneß Prämien}$
	
	\section*{Aufgabe 13}
	Kauf Call $K_1$, 2 mal Verkauf Call $K_2$, Kauf Call $K_3$
	max. Verlust: gezahlte Prämien \\
	max. Gewinn: $K_2 - K_1 - \text{gezahlte Prämien} $ \\
	BE-Point: $K_1 + \text{gezahlte Prämien}$ oder $K_3 - \text{gezahlte Prämien}$
	
	\section*{Aufgabe 14}
	Verkauf Call $K_1$, 2 mal Kauf Call $K_2$, Verkauf Call $K_3$
	max. Verlust: $K_2 - K_1 - \text{erhaltene Prämien} $ \\
	max. Gewinn:  erhaltene Prämien \\
	BE-Point: $K_3 - \text{erhaltene Prämien}$ oder $K_1 + \text{erhaltene Prämien}$
	
	\section*{Aufgabe 15}
	Richtig. Die verkaufte, kürzer laufende Option verliert schneller an Wert als die gekaufte, länger laufenden Option, d.h. Sie haben mehr Profit durch die verkaufte Option als Verluste durch die gekauften Option.

	\section*{Aufgabe 16}
	Richtig
	
	\section*{Aufgabe 17}
	Richtig
	
	\section*{Aufgabe 18}
	\begin{enumerate}[label=(\alph*)]
		\item\textcolor{green}{Kombination aus Long Put-Option und Short Call-Option auf DAX}
		\item\textcolor{green}{Long DAX-Put-Option} $\to$ mit (a) und (d) könnte man mit weniger Prämie wetten. Aber grundsätzlich funktioniert das auch.
		\item\textcolor{red}{Long Straddle DAX-Option}
		\item\textcolor{green}{Short DAX-Future}
	\end{enumerate}
	
\end{document}