\documentclass{article}

\usepackage{amsmath,amssymb}
\usepackage{tikz}
\usepackage{pgfplots}
\usepackage{xcolor}
\usepackage[left=2.1cm,right=3.1cm,bottom=3cm,footskip=0.75cm,headsep=0.5cm]{geometry}
\usepackage{enumerate}
\usepackage{enumitem}
\usepackage{marvosym}
\usepackage{tabularx}
\usepackage{parskip}

\usepackage{listings}
\definecolor{lightlightgray}{rgb}{0.95,0.95,0.95}
\definecolor{lila}{rgb}{0.8,0,0.8}
\definecolor{mygray}{rgb}{0.5,0.5,0.5}
\definecolor{mygreen}{rgb}{0,0.8,0.26}
%\lstdefinestyle{java} {language=java}
\lstset{language=R,
	basicstyle=\ttfamily,
	keywordstyle=\color{lila},
	commentstyle=\color{lightgray},
	stringstyle=\color{mygreen}\ttfamily,
	backgroundcolor=\color{white},
	showstringspaces=false,
	numbers=left,
	numbersep=10pt,
	numberstyle=\color{mygray}\ttfamily,
	identifierstyle=\color{blue},
	xleftmargin=.1\textwidth, 
	%xrightmargin=.1\textwidth,
	escapechar=§,
	%literate={\t}{{\ }}1
	breaklines=true,
	postbreak=\mbox{\space}
}

\usepackage[colorlinks = true, linkcolor = blue, urlcolor  = blue, citecolor = blue, anchorcolor = blue]{hyperref}
\usepackage[utf8]{inputenc}

\renewcommand*{\arraystretch}{1.4}

\newcolumntype{L}[1]{>{\raggedright\arraybackslash}p{#1}}
\newcolumntype{R}[1]{>{\raggedleft\arraybackslash}p{#1}}
\newcolumntype{C}[1]{>{\centering\let\newline\\\arraybackslash\hspace{0pt}}m{#1}}

\newcommand{\E}{\mathbb{E}}
\DeclareMathOperator{\rk}{rk}
\DeclareMathOperator{\Var}{Var}
\DeclareMathOperator{\Cov}{Cov}

\title{\textbf{Datensicherheit, Übung 4}}
\author{\textsc{Henry Haustein}}
\date{}

\begin{document}
	\maketitle
	
	\section*{Aufgabe 1}
	Wenn ich mich nicht verguckt habe, dann hat Kode 1 ein $d_{min}=2$ und Kode 2 ein $d_{min}=1$. Damit kann nur Kode 1 Fehler erkennen. Korrektur ist keine möglich.

	\section*{Aufgabe 2}
	\begin{enumerate}[label=(\alph*)]
		\item Es gilt $k=3$, damit $n = 2^3-1 = 7$, $l=n-k = 4$ und $d_{min} = 3$
		\item Maximal 2 Bitfehler oder maximal 3 Bündelfehler
		\item Multiplikationsverfahren: $(x^3+x+1)\cdot (x^2+x+1) = x^5 + x^4 + 1\mod 2\Rightarrow 110001$ \\
		Divisionsverfahren: $(x^2+x+1)\cdot x^3 = x^5+x^4+x^3$, Ermittlung Rest:
		\begin{align}
			\frac{x^5+x^4+x^3}{x^3+x+1} = x^2+x + \frac{-2x^2-x}{x^3+x+1} \notag
		\end{align}
		Im $GF(2)$ ist der Rest dann $-x = x$. Damit würde kodiert werden $x^5+x^4+x^3 + x \Rightarrow$ 111010.
		\item Es gilt:
		\begin{align}
			\frac{x^6+x^4+x^2+x+1}{x^3+x+1} = x^3-1 + \frac{x^2+2x+1}{x^3+x+1} \notag
		\end{align}
		also wurde die Bitfolge nicht richtig übertragen.
		\begin{align}
			\frac{x^6+x^3+x^2+x}{x^3+x+1} = x^3-x + \frac{2x^2+2x}{x^3+x+1} \notag
		\end{align}
		allerdings ist in $GF(2)$ der Rest äquivalent zu 0, damit wurde die Bitfolge richtig übertragen.
		\item $b_1$ ist nicht richtig übertragen worden, damit ist keine Dekodierung möglich. \\
		Falls $b_2$ mit dem Multiplikationsverfahren kodiert wurde, so ist die dekodierte Folge 1010. Divisionsverfahren: Die ersten 4 Stellen sind die dekodierte Folge, also 1001.
		\item Beim Empfänger kommt an: $1101001 \oplus 0011101 = 1110100$. Dekodierung (Multiplikationsverfahren):
		\begin{align}
			\frac{x^6+x^5+x^4+x^2}{x^3+x+1} = x^3+x^2-2 + \frac{2x+2}{x^3+x+1} \notag
		\end{align}
		wobei der Rest äquivalent zu 0 ist. Es wird kein Fehler erkannt, obwohl es einen Fehler gab.
	\end{enumerate}
	
	\section*{Aufgabe 3}
	\begin{enumerate}[label=(\alph*)]
		\item $k_1=4$ und $k=5$, damit $n = 2^4-1 = 15$, $l = 10$ und $d_{min} = 4$
		\item Maximal 3 Bitfehler oder maximal 5 Bündelfehler oder ungeradzahlige Fehlermuster
		\item $b_2$ hat 7 Einsen und $b_4$ hat 9 Einsen. Bei $b_1$ sind 5 Bits hintereinander falsch, diese können erkannt werden. Bei $b_3$ gibt es 6 Einzelfehler, damit ist keine Erkennung möglich.
	\end{enumerate}
	
	\section*{Aufgabe 4}
	Nein, Kodierung schützt nicht vor Angreifern. Angreifer können die Leitung abhören und die Nachricht dekodieren ($\nearrow$ Vertraulichkeit), sie können sogar die Nachricht abfangen, verändern, neu kodieren und über die Leitung schicken! ($\nearrow$ Integrität)

\end{document}