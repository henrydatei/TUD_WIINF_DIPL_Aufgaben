\documentclass{article}

\usepackage{amsmath,amssymb}
\usepackage{tikz}
\usepackage{pgfplots}
\usepackage{xcolor}
\usepackage[left=2.1cm,right=3.1cm,bottom=3cm,footskip=0.75cm,headsep=0.5cm]{geometry}
\usepackage{enumerate}
\usepackage{enumitem}
\usepackage{marvosym}
\usepackage{tabularx}
\usepackage{parskip}

\usepackage{listings}
\definecolor{lightlightgray}{rgb}{0.95,0.95,0.95}
\definecolor{lila}{rgb}{0.8,0,0.8}
\definecolor{mygray}{rgb}{0.5,0.5,0.5}
\definecolor{mygreen}{rgb}{0,0.8,0.26}
%\lstdefinestyle{java} {language=java}
\lstset{language=R,
	basicstyle=\ttfamily,
	keywordstyle=\color{lila},
	commentstyle=\color{lightgray},
	stringstyle=\color{mygreen}\ttfamily,
	backgroundcolor=\color{white},
	showstringspaces=false,
	numbers=left,
	numbersep=10pt,
	numberstyle=\color{mygray}\ttfamily,
	identifierstyle=\color{blue},
	xleftmargin=.1\textwidth, 
	%xrightmargin=.1\textwidth,
	escapechar=§,
	%literate={\t}{{\ }}1
	breaklines=true,
	postbreak=\mbox{\space}
}

\usepackage[colorlinks = true, linkcolor = blue, urlcolor  = blue, citecolor = blue, anchorcolor = blue]{hyperref}
\usepackage[utf8]{inputenc}

\renewcommand*{\arraystretch}{1.4}

\newcolumntype{L}[1]{>{\raggedright\arraybackslash}p{#1}}
\newcolumntype{R}[1]{>{\raggedleft\arraybackslash}p{#1}}
\newcolumntype{C}[1]{>{\centering\let\newline\\\arraybackslash\hspace{0pt}}m{#1}}

\newcommand{\E}{\mathbb{E}}
\DeclareMathOperator{\rk}{rk}
\DeclareMathOperator{\Var}{Var}
\DeclareMathOperator{\Cov}{Cov}

\title{\textbf{Datensicherheit, Übung 10}}
\author{\textsc{Henry Haustein}}
\date{}

\begin{document}
	\maketitle
	
	\section*{Aufgabe 1}
	\begin{enumerate}[label=(\alph*)]
		\item Ergebnis
		\begin{center}
			\begin{tabular}{|c|c|c|c|c|}
				\hline
				1 & 3 & 2 & 2 & 9 \\
				\hline
				0 & 0 & 7 & 6 & 9 \\
				\hline
				4 & 2 & 4 & 5 & 5 \\
				\hline
				3 & 4 & 6 & 6 & 7 \\
				\hline
				1 & 3 & 5 & 5 & 6 \\
				\hline
			\end{tabular}
		\end{center}
		\item Ergebnis (Werte von 255 müssen übersprungen werden)
		\begin{center}
			\begin{tabular}{|c|c|c|c|c|}
				\hline
				1 & 3 & 2 & 4 & 9 \\
				\hline
				2 & 2 & 7 & 6 & 9 \\
				\hline
				6 & 2 & 4 & 5 & 5 \\
				\hline
				3 & 4 & 6 & 6 & 7 \\
				\hline
				1 & 3 & 5 & 5 & 6 \\
				\hline
			\end{tabular}
		\end{center}
		\item Ergebnis (Werte von 0 müssen übersprungen werden)
		\begin{center}
			\begin{tabular}{|c|c|c|c|c|}
				\hline
				0 & 3 & 1 & 2 & 8 \\
				\hline
				1 & 0 & 6 & 6 & 7 \\
				\hline
				4 & 1 & 4 & 5 & 5 \\
				\hline
				3 & 4 & 6 & 6 & 7 \\
				\hline
				1 & 3 & 5 & 5 & 6 \\
				\hline
			\end{tabular}
		\end{center}
	\end{enumerate}

	\section*{Aufgabe 2}
	\begin{enumerate}[label=(\alph*)]
		\item $\frac{800\cdot 1000}{2500\cdot 8} = 40$
		\item 50\%
	\end{enumerate}
	
	\section*{Aufgabe 3}
	\begin{enumerate}[label=(\alph*)]
		\item alle Abstände probieren, Nachricht sinnvoll?
		\item alle Abstände probieren, Histogramm der Buchstabenverteilung entspricht natürlicher Sprache?
	\end{enumerate}
	
	\section*{Aufgabe 4}
	Dekrementieren, LSB-Ersetzung

\end{document}