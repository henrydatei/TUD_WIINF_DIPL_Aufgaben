\documentclass{article}

\usepackage{amsmath,amssymb}
\usepackage{tikz}
\usepackage{pgfplots}
\usepackage{xcolor}
\usepackage[left=2.1cm,right=3.1cm,bottom=3cm,footskip=0.75cm,headsep=0.5cm]{geometry}
\usepackage{enumerate}
\usepackage{enumitem}
\usepackage{marvosym}
\usepackage{tabularx}
\usepackage{parskip}

\usepackage{listings}
\definecolor{lightlightgray}{rgb}{0.95,0.95,0.95}
\definecolor{lila}{rgb}{0.8,0,0.8}
\definecolor{mygray}{rgb}{0.5,0.5,0.5}
\definecolor{mygreen}{rgb}{0,0.8,0.26}
%\lstdefinestyle{java} {language=java}
\lstset{language=R,
	basicstyle=\ttfamily,
	keywordstyle=\color{lila},
	commentstyle=\color{lightgray},
	stringstyle=\color{mygreen}\ttfamily,
	backgroundcolor=\color{white},
	showstringspaces=false,
	numbers=left,
	numbersep=10pt,
	numberstyle=\color{mygray}\ttfamily,
	identifierstyle=\color{blue},
	xleftmargin=.1\textwidth, 
	%xrightmargin=.1\textwidth,
	escapechar=§,
	%literate={\t}{{\ }}1
	breaklines=true,
	postbreak=\mbox{\space}
}

\usepackage[colorlinks = true, linkcolor = blue, urlcolor  = blue, citecolor = blue, anchorcolor = blue]{hyperref}
\usepackage[utf8]{inputenc}

\renewcommand*{\arraystretch}{1.4}

\newcolumntype{L}[1]{>{\raggedright\arraybackslash}p{#1}}
\newcolumntype{R}[1]{>{\raggedleft\arraybackslash}p{#1}}
\newcolumntype{C}[1]{>{\centering\let\newline\\\arraybackslash\hspace{0pt}}m{#1}}

\newcommand{\E}{\mathbb{E}}
\DeclareMathOperator{\rk}{rk}
\DeclareMathOperator{\Var}{Var}
\DeclareMathOperator{\Cov}{Cov}

\title{\textbf{Datensicherheit, Übung 8}}
\author{\textsc{Henry Haustein}}
\date{}

\begin{document}
	\maketitle
	
	\section*{Aufgabe 1}
	\begin{enumerate}[label=(\alph*)]
		\item Insgesamt 10 Runden $\Rightarrow$ 11 Rundenschlüssel
		\item Jeder Schlüssel besteht aus 4 $w$'s $\Rightarrow$ 44 $w$'s
		\item Rcon[2] = [$x^1$, 00, 00, 00] = [02, 00, 00, 00] wird für $w_8$ bis $w_{11}$ gebraucht; \\ Rcon[3] = [$x^2$, 00, 00, 00] = [04, 00, 00, 00] für $w_{12}$ bis $w_{15}$
	\end{enumerate}

	\section*{Aufgabe 2}
	$k_0$ = AB 8F 20 D3 74 E9 5C 37 32 C8 52 30 1F C6 7F 3E \\
	$w_4 = SubWord(Rot(w_3)) \oplus Rcon[1] \oplus w_0 = SubWord([C6,7F,3E,1F]) \oplus [01,00,00,00] = [B4, D2, B2, C0] \oplus [01,00,00,00] \oplus [AB, 8F, 20, D3] = [B5, D2, B2, C0] \oplus [AB, 8F, 20, D3] = [1E, 5D, 92, 13]$ \\
	$w_5 = w_4 \oplus w_1 = [1E, 5D, 92, 13] \oplus [74, E9, 5C, 37] = [6A, B4, CE, 24]$ \\
	$w_6 = w_5 \oplus w_2 = [6A, B4, CE, 24] \oplus [32, C8, 52, 30] = [58, 7C, 9C, 14]$ \\
	$w_7 = w_6 \oplus w_3 = [58, 7C, 9C, 14] \oplus [1F, C6, 7F, 3E] = [47, BA, E3, 2A]$
	
	\section*{Aufgabe 3}
	letzte Runde: Shift$^{-1}$, Sub$^{-1}$ vertauschen \\
	vorletzte Runden: $\oplus k_{r-1}$, MC$^{-1}$ vertauschen ($k_{r-1} \to k'_{r-1}$) und Shift$^{-1}$, Sub$^{-1}$ vertauschen, ... \\
	$\Rightarrow$ neue Reihenfolge: Sub$^{-1}$, Shift$^{-1}$, MC$^{-1}$, $\oplus k'_{r-1}$, Sub$^{-1}$, Shift$^{-1}$, ... $\Rightarrow$ selbe Reihenfolge wie bei der Verschlüsselung
	
	\section*{Aufgabe 4}
	Bis $j = i-1$ kann alles korrekt entschlüsselt werden. Da $c_i$ fehlt, entschlüsselt der Empfänger $m_i = dec(k, c_{c+1}) \oplus c_{i-1}$, was schief geht. Dann $m_{i+1} = dec(k, c_{i+1}) \oplus c_{i-1}$, was fehlerhaft ist. Ab $m_{i+2} = dec(k, c_{i+2}) \oplus c_{i+1}$ geht wieder alles. 
	
	\section*{Aufgabe 5}
	Original: $c_{i+1} = enc(k, (m_{i+1} \oplus c_i))$ \\
	Überprüfen: $c_{i+1}' = enc(k, (m' \oplus c_i))$ \\
	$c_{i+2}' = enc(k, (m_{i+1} \oplus c_{i+1}'))$ $\Rightarrow$ Fehlerfortpflanzung
	
	\section*{Aufgabe 6}
	ECB: parallelisierbar, CBC: nicht parallelisierbar in enc(), parallelisierbar in dec(), CTR: parallelisierbar

\end{document}