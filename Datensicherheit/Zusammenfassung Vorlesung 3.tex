\documentclass{article}

\usepackage{amsmath,amssymb}
\usepackage{tikz}
\usepackage{pgfplots}
\usepackage{xcolor}
\usepackage[left=2.1cm,right=3.1cm,bottom=3cm,footskip=0.75cm,headsep=0.5cm]{geometry}
\usepackage{enumerate}
\usepackage{enumitem}
\usepackage{marvosym}
\usepackage{tabularx}
\usepackage{parskip}

\usepackage{listings}
\definecolor{lightlightgray}{rgb}{0.95,0.95,0.95}
\definecolor{lila}{rgb}{0.8,0,0.8}
\definecolor{mygray}{rgb}{0.5,0.5,0.5}
\definecolor{mygreen}{rgb}{0,0.8,0.26}
%\lstdefinestyle{java} {language=java}
\lstset{language=R,
	basicstyle=\ttfamily,
	keywordstyle=\color{lila},
	commentstyle=\color{lightgray},
	stringstyle=\color{mygreen}\ttfamily,
	backgroundcolor=\color{white},
	showstringspaces=false,
	numbers=left,
	numbersep=10pt,
	numberstyle=\color{mygray}\ttfamily,
	identifierstyle=\color{blue},
	xleftmargin=.1\textwidth, 
	%xrightmargin=.1\textwidth,
	escapechar=§,
	%literate={\t}{{\ }}1
	breaklines=true,
	postbreak=\mbox{\space}
}

\usepackage[colorlinks = true, linkcolor = blue, urlcolor  = blue, citecolor = blue, anchorcolor = blue]{hyperref}
\usepackage[utf8]{inputenc}

\renewcommand*{\arraystretch}{1.4}

\newcolumntype{L}[1]{>{\raggedright\arraybackslash}p{#1}}
\newcolumntype{R}[1]{>{\raggedleft\arraybackslash}p{#1}}
\newcolumntype{C}[1]{>{\centering\let\newline\\\arraybackslash\hspace{0pt}}m{#1}}

\newcommand{\E}{\mathbb{E}}
\DeclareMathOperator{\rk}{rk}
\DeclareMathOperator{\Var}{Var}
\DeclareMathOperator{\Cov}{Cov}

\title{\textbf{Datensicherheit, Zusammenfassung Vorlesung 3}}
\author{\textsc{Henry Haustein}, \textsc{Dennis Rössel}}
\date{}

\begin{document}
	\maketitle
	
	\section*{Welche Angriffsmethoden werden beispielsweise angewendet?}
	Ausnutzung von Schwachstellen in Software, Malware, Spam, Drive-by-Exploits, Botnetze, Denial-of-Service, Social Engineering
	
	\section*{Was zeichnet Computerviren aus?}
	selbstreproduzierend, keine selbstständige Software - benötigt Wirt

	\section*{Was sind Beispiele für Verschleierungsmethoden für Viren?}
	Virus wird verschlüsselt in infizierter Datei gespeichert, Polymorphe Viren, Metamorphe Viren, Stealth-Viren

	\section*{Was sind grundlegende Antivirentechniken, welche Vor- und Nachteile haben sie?}
	Antiviren-Techniken:
	\begin{itemize}
		\item statische Techniken: Scanner mit Signaturerkennung, Heuristik, Integritätsprüfungen $\to$ teilweise nur bekannte Viren erkennbar, Identifizierung als Virus
		\item dynamische Techniken: Monitoring von Aktivitäten, Emulation $\to$ auch unbekannte Viren erkennbar, keine Identifizierung
	\end{itemize}
	
	\section*{Was sind Beispiele für empfohlene Maßnahmen gegen die Gefährdung durch Malware?}
	Empfohlene Maßnahmen:
	\begin{itemize}
		\item Regeln (Policies)
		\item Problembewusstsein
		\item Verringerung möglicher Schwachstellen
		\item Verringerung möglicher Bedrohungen
	\end{itemize}
	
	\section*{Was verstehen Sie unter dem Prinzip \textit{least privilege}?}
	nur die Rechte vergeben, die unbedingt benötigt werden, arbeiten mit verschiedenen Nutzeraccounts
	
	\section*{Was beschreibt ein Angreifermodell, was sind wesentliche Inhalte?}
	Angreifermodell: Angabe der maximal berücksichtigten Stärke eines Angreifers, das heißt Stärke des Angreifers, gegen die ein bestimmter Schutzmechanismus gerade noch sicher ist
	
	Inhalte:
	\begin{itemize}
		\item Rolle des Angreifers
		\item Verbreitung des Angreifers
		\item Verhalten des Angreifers
		\item Rechenkapazität des Angreifers
		\item verfügbare Mittel des Angreifers
	\end{itemize}
	
	\section*{Was besagt das Prinzip der Angemessenheit?}
	ausgewogenes Verhältnis zwischen Sicherheitsanforderungen und Aufwand für Realisierung der Maßnahmen, Reduzierung der Risiken

\end{document}