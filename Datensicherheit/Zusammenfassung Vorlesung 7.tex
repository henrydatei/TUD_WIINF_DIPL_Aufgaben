\documentclass{article}

\usepackage{amsmath,amssymb}
\usepackage{tikz}
\usepackage{pgfplots}
\usepackage{xcolor}
\usepackage[left=2.1cm,right=3.1cm,bottom=3cm,footskip=0.75cm,headsep=0.5cm]{geometry}
\usepackage{enumerate}
\usepackage{enumitem}
\usepackage{marvosym}
\usepackage{tabularx}
\usepackage{parskip}

\usepackage{listings}
\definecolor{lightlightgray}{rgb}{0.95,0.95,0.95}
\definecolor{lila}{rgb}{0.8,0,0.8}
\definecolor{mygray}{rgb}{0.5,0.5,0.5}
\definecolor{mygreen}{rgb}{0,0.8,0.26}
%\lstdefinestyle{java} {language=java}
\lstset{language=R,
	basicstyle=\ttfamily,
	keywordstyle=\color{lila},
	commentstyle=\color{lightgray},
	stringstyle=\color{mygreen}\ttfamily,
	backgroundcolor=\color{white},
	showstringspaces=false,
	numbers=left,
	numbersep=10pt,
	numberstyle=\color{mygray}\ttfamily,
	identifierstyle=\color{blue},
	xleftmargin=.1\textwidth, 
	%xrightmargin=.1\textwidth,
	escapechar=§,
	%literate={\t}{{\ }}1
	breaklines=true,
	postbreak=\mbox{\space}
}

\usepackage[colorlinks = true, linkcolor = blue, urlcolor  = blue, citecolor = blue, anchorcolor = blue]{hyperref}
\usepackage[utf8]{inputenc}

\renewcommand*{\arraystretch}{1.4}

\newcolumntype{L}[1]{>{\raggedright\arraybackslash}p{#1}}
\newcolumntype{R}[1]{>{\raggedleft\arraybackslash}p{#1}}
\newcolumntype{C}[1]{>{\centering\let\newline\\\arraybackslash\hspace{0pt}}m{#1}}

\newcommand{\E}{\mathbb{E}}
\DeclareMathOperator{\rk}{rk}
\DeclareMathOperator{\Var}{Var}
\DeclareMathOperator{\Cov}{Cov}

\title{\textbf{Datensicherheit, Zusammenfassung Vorlesung 7}}
\author{\textsc{Henry Haustein}, \textsc{Dennis Rössel}}
\date{}

\begin{document}
	\maketitle
	
	\section*{Wie wird beim Multiplikations- und Divisionsverfahren kodiert bzw. dekodiert?}
	Multiplikationsverfahren:
	\begin{itemize}
		\item Kodierung: $a(x) = a^\ast(x) \cdot g(x)$
		\item Dekodierung: $a^\ast(x) = \frac{a(x)}{g(x)}$
	\end{itemize}
	Divisionsverfahren:
	\begin{itemize}
		\item Kodierung: $a(x) = a^\ast(x) \cdot x^r + r(x)$, sodass $a(x)$ ein Vielfaches von $g(x)$ ist
		\item Dekodierung: $a^\ast(x) = \frac{a(x)}{g(x)}$
	\end{itemize}
	
	\section*{Wie erfolgt die Fehlerprüfung?}
	Wenn es bei der Division zu keinem Rest kommt, so war die Übertragung fehlerfrei (oder der Fehler wurde nicht erkannt)
	
	\section*{Was versteht man unter den Eigenschaften zyklisch und systematisch?}
	zyklisch: Ein Kode heißt zyklisch, wenn für jedes Kanalkodewort durch zyklische Verschiebung der Elemente wieder ein Kanalkodewort entsteht. \\
	systematisch: Position der Informationstsellen ist bekannt
	
	\section*{Welche Kodeparameter und Fehlererkennungseigenschaften hat der zyklische \textsc{Hamming}-Kode?}
	Parameter: $(n,l,d_{min}) = (2^k-1, n-k, 3)$ mit $k=\deg(g(x))$ \\
	Fehlererkennung: $f_e=2$, Erkennung von Bündelfehlern bis zu $f_b\le k$
	
	\section*{Welche Kodeparameter und Fehlererkennungseigenschaften hat der \textsc{Abramson}-Kode?}
	Parameter: $(n,l,d_{min}) = (2^{k_1}-1, n-k, 4)$ mit $k=\deg(g(x))$, $k_1 = \deg(M(x))$ \\
	Fehlererkennung: $f_e=3$, Erkennung von Bündelfehlern bis zu $f_b\le k$, Erkennung ungradzahliger Fehlermuster
	
\end{document}