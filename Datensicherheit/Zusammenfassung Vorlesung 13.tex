\documentclass{article}

\usepackage{amsmath,amssymb}
\usepackage{tikz}
\usepackage{pgfplots}
\usepackage{xcolor}
\usepackage[left=2.1cm,right=3.1cm,bottom=3cm,footskip=0.75cm,headsep=0.5cm]{geometry}
\usepackage{enumerate}
\usepackage{enumitem}
\usepackage{marvosym}
\usepackage{tabularx}
\usepackage{parskip}

\usepackage{listings}
\definecolor{lightlightgray}{rgb}{0.95,0.95,0.95}
\definecolor{lila}{rgb}{0.8,0,0.8}
\definecolor{mygray}{rgb}{0.5,0.5,0.5}
\definecolor{mygreen}{rgb}{0,0.8,0.26}
%\lstdefinestyle{java} {language=java}
\lstset{language=R,
	basicstyle=\ttfamily,
	keywordstyle=\color{lila},
	commentstyle=\color{lightgray},
	stringstyle=\color{mygreen}\ttfamily,
	backgroundcolor=\color{white},
	showstringspaces=false,
	numbers=left,
	numbersep=10pt,
	numberstyle=\color{mygray}\ttfamily,
	identifierstyle=\color{blue},
	xleftmargin=.1\textwidth, 
	%xrightmargin=.1\textwidth,
	escapechar=§,
	%literate={\t}{{\ }}1
	breaklines=true,
	postbreak=\mbox{\space}
}

\usepackage[colorlinks = true, linkcolor = blue, urlcolor  = blue, citecolor = blue, anchorcolor = blue]{hyperref}
\usepackage[utf8]{inputenc}

\renewcommand*{\arraystretch}{1.4}

\newcolumntype{L}[1]{>{\raggedright\arraybackslash}p{#1}}
\newcolumntype{R}[1]{>{\raggedleft\arraybackslash}p{#1}}
\newcolumntype{C}[1]{>{\centering\let\newline\\\arraybackslash\hspace{0pt}}m{#1}}

\newcommand{\E}{\mathbb{E}}
\DeclareMathOperator{\rk}{rk}
\DeclareMathOperator{\Var}{Var}
\DeclareMathOperator{\Cov}{Cov}

\title{\textbf{Datensicherheit, Zusammenfassung Vorlesung 13}}
\author{\textsc{Henry Haustein}, \textsc{Dennis Rössel}}
\date{}

\begin{document}
	\maketitle

	\section*{Wie können Primzahlen erzeugt werden?}
	Probabilistischer Test nach Rabin-Miller: Falls $p$ prim, dann $\forall a\in\mathbb{Z}_p^\ast: a^{\frac{p-1}{2}} \equiv \pm 1 \mod p$. Falls $p$ nicht prim, dann gilt dies höchstens für $\frac{1}{4}$ der möglichen $a$.
	
	\section*{Wie werden die öffentlichen und geheimen Parameter für RSA bestimmt?}
	Wahl von $k_e$ mit $1<k_e<\Phi(n)$ und $ggT(k_e,\Phi(n))=1$, $k_d = k_e^{-1} \mod \Phi(n)$
	
	\section*{Wie erfolgt die Ver- bzw. Entschlüsselung?}
	Verschlüsselung: $c = m^{k_e} \mod n$
	
	Entschlüsselung: $m = c^{k_d} \mod n$
	
	\section*{Wie erfolgt das Signieren und Testen?}
	Signieren: $s = m^{k_s}\mod n$
	
	Testen: $m = s^{k_t} \mod n$?
	
	\section*{Worauf ist bei der Parameterwahl bzgl. Sicherheit zu achten?}
	Wahl von $p$ und $q$ als große Primzahlen, die nicht dicht beieinander liegen, aber auch nicht zu weit auseinander
	
	\section*{Welche Angriffsmöglichkeiten bestehen bei der einfachen, unsicheren Variante von RSA?}
	passive Angriffe: RSA arbeitet deterministisch, man kann also verschlüsseln und vergleichen
	
	aktive Angriffe: RSA ist ein Homomorphismus bezüglich Multiplikation: Angreifer beobachtet Signaturen $s_1$, $s_2$ für Nachrichten $m_1$, $m_2$. Dann ist $s_3 = s_1\cdot s_2$ eine Signatur für $m_3 = m_1\cdot m_2$
	
	\section*{Wie werden die passiven Angriffe verhindert?}
	Zufallszahl $r$ hinzufügen: $c = (r,m, h(r, m))^{k_e}$ $\to$ indeterministische Verschlüsselung
	
	\section*{Wie werden die aktiven Angriffe verhindert?}
	Redundanz
	
\end{document}