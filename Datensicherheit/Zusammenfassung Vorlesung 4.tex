\documentclass{article}

\usepackage{amsmath,amssymb}
\usepackage{tikz}
\usepackage{pgfplots}
\usepackage{xcolor}
\usepackage[left=2.1cm,right=3.1cm,bottom=3cm,footskip=0.75cm,headsep=0.5cm]{geometry}
\usepackage{enumerate}
\usepackage{enumitem}
\usepackage{marvosym}
\usepackage{tabularx}
\usepackage{parskip}

\usepackage{listings}
\definecolor{lightlightgray}{rgb}{0.95,0.95,0.95}
\definecolor{lila}{rgb}{0.8,0,0.8}
\definecolor{mygray}{rgb}{0.5,0.5,0.5}
\definecolor{mygreen}{rgb}{0,0.8,0.26}
%\lstdefinestyle{java} {language=java}
\lstset{language=R,
	basicstyle=\ttfamily,
	keywordstyle=\color{lila},
	commentstyle=\color{lightgray},
	stringstyle=\color{mygreen}\ttfamily,
	backgroundcolor=\color{white},
	showstringspaces=false,
	numbers=left,
	numbersep=10pt,
	numberstyle=\color{mygray}\ttfamily,
	identifierstyle=\color{blue},
	xleftmargin=.1\textwidth, 
	%xrightmargin=.1\textwidth,
	escapechar=§,
	%literate={\t}{{\ }}1
	breaklines=true,
	postbreak=\mbox{\space}
}

\usepackage[colorlinks = true, linkcolor = blue, urlcolor  = blue, citecolor = blue, anchorcolor = blue]{hyperref}
\usepackage[utf8]{inputenc}

\renewcommand*{\arraystretch}{1.4}

\newcolumntype{L}[1]{>{\raggedright\arraybackslash}p{#1}}
\newcolumntype{R}[1]{>{\raggedleft\arraybackslash}p{#1}}
\newcolumntype{C}[1]{>{\centering\let\newline\\\arraybackslash\hspace{0pt}}m{#1}}

\newcommand{\E}{\mathbb{E}}
\DeclareMathOperator{\rk}{rk}
\DeclareMathOperator{\Var}{Var}
\DeclareMathOperator{\Cov}{Cov}

\title{\textbf{Datensicherheit, Zusammenfassung Vorlesung 4}}
\author{\textsc{Henry Haustein}, \textsc{Dennis Rössel}}
\date{}

\begin{document}
	\maketitle
	
	\section*{Was kann bzgl. der erreichbaren Sicherheit ausgesagt werden?}
	keine 100\%ige Sicherheit möglich
	
	\section*{Warum ist Sicherheit kein Zustand, sondern ein Prozess?}
	erreichtes Sicherheitsniveau ist nicht dauerhaft
	
	\section*{Welche Aufgaben sind der Awareness bzgl. Sicherheit zuzuordnen?}
	Sensibilisierung, Schulung, Training
	
	\section*{Welche Prinzipien sind beim Sicherheitsmanagement zu beachten?}
	Kombi aus technischen, organisatorischen, personellen und infrastrukturellen Maßnahmen
	
	\section*{Was beinhalten IT-Sicherheitspolitik und IT-Sicherheitskonzept?}
	IT-Sicherheitspolitik:
	\begin{itemize}
		\item Charakterisierung des Unternehmens
		\item Geltungsbereich der Sicherheitspolitik
		\item Bedeutung der Sicherheit
		\item Gefährdungslage
		\item weitere Vorgaben
		\item Organisationsbeschluss und Verpflichtungserklärung
	\end{itemize}

	IT-Sicherheitskonzept:
	\begin{itemize}
		\item Anforderungsanalyse (Was?)
		\item Risikoanalyse (Wovor?)
		\item Festlegung der Maßnahmen (Wie?)
	\end{itemize}
	
	\section*{Was sind die wesentlichen drei Aufgaben bei der Erstellung eines IT-Sicherheitskonzepts?}
	IT-Sicherheitskonzept:
	\begin{itemize}
		\item Anforderungsanalyse (Was?)
		\item Risikoanalyse (Wovor?)
		\item Festlegung der Maßnahmen (Wie?)
	\end{itemize}
	
	\section*{Welche Schritte beinhaltet die Anforderungsanalyse?}
	Anforderungsanalyse
	\begin{itemize}
		\item Bestandsaufnahme: relevante Objekte
		\item Schutzbedarfsfeststellung
		\item Gesetze, Verträge und unternehmensinterne Regelungen
	\end{itemize}
	
	\section*{Wie entstehen allgemein Risiken (Risikobildungsmodell)?}
	Bedrohungen/Schwachstellen $\to$ gefährdende Ereignisse $\to$ Risiken
	
	\section*{Mit welchen zwei Faktoren werden Risiken bewertet?}
	Eintrittswahrscheinlichkeit, möglicher Schaden
	
	\section*{Wie können Maßnahmen zur Risikobewältigung nach der Zielrichtung bzw. dem Zeitpunkt ihrer Wirkung eingeteilt werden?}
	Maßnahmenklassifikation nach Zielrichtung und Zeitpunkt:
	\begin{itemize}
		\item vermeiden
		\item vermindern
		\item überwälzen
		\item selbst tragen
	\end{itemize}
	
	\section*{Was beinhaltet die Validierung der Maßnahmen?}
	Validierung der Maßnahmen
	\begin{itemize}
		\item Eignung
		\item Wirksamkeit
		\item Zusammenwirken
		\item Praktikabilität
		\item Akzeptanz
		\item Wirtschaftlichkeit
		\item Angemessenheit
	\end{itemize}
	
	\section*{Welche Fragen sind bei der Definition von Zugriffsberechtigungen zu klären?}
	Wer? Wann? Wo? Welche? Was? Warum?
	
	\section*{Wie werden die Zugriffskontrollinformationen grundsätzlich verwaltet, welche vereinfachten Varianten (ACL, CL) gibt es?}
	allgemein: Zugriffskontrollmatrix, ACL (Access Control List), CL (Capability List)
	
	\section*{Was ist das Prinzip bei RBAC, welche Aufgaben sind zu lösen?}
	Grundidee: Repräsentation einer bestimmten Aufgabe und der damit einhergehenden Zugriffsrechte durch eine Rolle
	
	Aufgaben:
	\begin{itemize}
		\item Definition der Rollen und der zugehörigen Zugriffsberechtigungen
		\item Zuweisung der Rollen zu den Objekten
	\end{itemize}
	
	\section*{Welche prinzipiellen Möglichkeiten der Identifikation von Menschen durch IT-Systeme gibt es?}
	Was man ist (Biometrie), Was man hat (Dokument), Was man weiß (Passwort)
	
\end{document}