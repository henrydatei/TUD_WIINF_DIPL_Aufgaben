\documentclass{article}

\usepackage{amsmath,amssymb}
\usepackage{tikz}
\usepackage{pgfplots}
\usepackage{xcolor}
\usepackage[left=2.1cm,right=3.1cm,bottom=3cm,footskip=0.75cm,headsep=0.5cm]{geometry}
\usepackage{enumerate}
\usepackage{enumitem}
\usepackage{marvosym}
\usepackage{tabularx}
\usepackage{parskip}

\usepackage{listings}
\definecolor{lightlightgray}{rgb}{0.95,0.95,0.95}
\definecolor{lila}{rgb}{0.8,0,0.8}
\definecolor{mygray}{rgb}{0.5,0.5,0.5}
\definecolor{mygreen}{rgb}{0,0.8,0.26}
%\lstdefinestyle{java} {language=java}
\lstset{language=R,
	basicstyle=\ttfamily,
	keywordstyle=\color{lila},
	commentstyle=\color{lightgray},
	stringstyle=\color{mygreen}\ttfamily,
	backgroundcolor=\color{white},
	showstringspaces=false,
	numbers=left,
	numbersep=10pt,
	numberstyle=\color{mygray}\ttfamily,
	identifierstyle=\color{blue},
	xleftmargin=.1\textwidth, 
	%xrightmargin=.1\textwidth,
	escapechar=§,
	%literate={\t}{{\ }}1
	breaklines=true,
	postbreak=\mbox{\space}
}

\usepackage[colorlinks = true, linkcolor = blue, urlcolor  = blue, citecolor = blue, anchorcolor = blue]{hyperref}
\usepackage[utf8]{inputenc}

\renewcommand*{\arraystretch}{1.4}

\newcolumntype{L}[1]{>{\raggedright\arraybackslash}p{#1}}
\newcolumntype{R}[1]{>{\raggedleft\arraybackslash}p{#1}}
\newcolumntype{C}[1]{>{\centering\let\newline\\\arraybackslash\hspace{0pt}}m{#1}}

\newcommand{\E}{\mathbb{E}}
\DeclareMathOperator{\rk}{rk}
\DeclareMathOperator{\Var}{Var}
\DeclareMathOperator{\Cov}{Cov}

\title{\textbf{Datensicherheit, Zusammenfassung Vorlesung 9}}
\author{\textsc{Henry Haustein}, \textsc{Dennis Rössel}}
\date{}

\begin{document}
	\maketitle

	\section*{Welche Angriffsarten auf Kryptosysteme werden unterschieden?}
	Passiver Angreifer nutzt Wissen über System (Algorithmen, Protokolle), Öffentliche Schlüssel/Parameter und Beobachtung (unsicherer Kanal)
	\begin{itemize}
		\item Reiner Schlüsseltext-Angriff (ciphertext-only attack)
		\item Klartext-Schlüsseltext-Angriff (known-plaintext attack)
	\end{itemize}
	
	Aktiver Angreifer: bringt Inhaber der geheimen bzw. privaten Schlüssel dazu, die entsprechenden Operationen für selbst gewählte Daten auszuführen
	\begin{itemize}
		\item Gewählter Klartext-Schlüsseltext-Angriff (chosen-plaintext attack CPA; \textit{Verschlüsselungsorakel})
		\item Gewählter Schlüsseltext-Klartext-Angriff (chosen-ciphertext attack; \textit{Entschlüsselungsorakel})
	\end{itemize}
	
	\section*{Was bedeutet informationstheoretische (perfekte) Sicherheit?}
	Auch einem unbeschränkten Angreifer gelingt es nicht, das System zu brechen.
	
	\section*{Was sind relevante Anforderungen an die Schlüssel bei einer informationstheoretisch sicheren Chiffre?}
	Ein System heißt informationstheoretisch sicher, wenn für alle Nachrichten und Schlüsseltexte gilt, dass die a posteriori Wahrscheinlichkeiten $p(m\mid c)$ der möglichen Nachrichten nach Beobachtung eines gesendeten Geheimtextes gleich der a priori Wahrscheinlichkeiten $p(m)$ dieser Nachrichten sind:
	\begin{align}
		\forall m\in M, \forall c\in C:\quad p(m\mid c)=p(m) \notag
	\end{align}
	
	\section*{Wie funktioniert die Vernam-Chiffre?}
	Jeder Schlüssel wird nur einmal verwendet, Schlüssellänge und Länge des Klartextes gleich, Schlüssel zufällig $\Rightarrow$ XOR
	
	\section*{Warum kann es bei asymmetrischen Verfahren keine informationstheoretische Sicherheit geben?}
	Schlüsselmanagement problematisch
	
	\section*{Wie funktionieren Transpositionen und Substitutionen?}
	Transposition = Vertauschen der Zeichen des Klartextes \\
	MM-Substitutionen (monoalphabetisch, monographisch): ein Buchstabe des Klartextes wird mit einem Buchstaben ersetzt. Die Buchstaben zu denen ersetzt wird kommen aus einem Alphabet $\Rightarrow$ eineindeutige Zuordnung \\
	PM-Substitutionen (polyalphabetisch, monographisch): wie MM-Substitutionen, nur dass die Buchstaben zu denen ersetzt wird, aus mehreren Alphabeten kommen $\Rightarrow$ eindeutige Zuordnung
	
	\section*{Wie kann das verwendete historische Verschlüsselungsverfahren anhand eines vorliegenden Schlüsseltextes identifiziert werden?}
	mittels Histogramm
	
	\section*{Wie kann die Analyse von MM-Substitutionen bzw. PM-Substitu\-tionen erfolgen?}
	MM-Substitution: Analyse von Buchstaben, Bi- und Trigrammen, Nutzung Redundanz bei fehlenden Zeichen
	
	PM-Substitution: Schlüssellänge mit Kasiski-Test $\to$ MM-Analyse
	
\end{document}