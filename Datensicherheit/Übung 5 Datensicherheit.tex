\documentclass{article}

\usepackage{amsmath,amssymb}
\usepackage{tikz}
\usepackage{pgfplots}
\usepackage{xcolor}
\usepackage[left=2.1cm,right=3.1cm,bottom=3cm,footskip=0.75cm,headsep=0.5cm]{geometry}
\usepackage{enumerate}
\usepackage{enumitem}
\usepackage{marvosym}
\usepackage{tabularx}
\usepackage{parskip}

\usepackage{listings}
\definecolor{lightlightgray}{rgb}{0.95,0.95,0.95}
\definecolor{lila}{rgb}{0.8,0,0.8}
\definecolor{mygray}{rgb}{0.5,0.5,0.5}
\definecolor{mygreen}{rgb}{0,0.8,0.26}
%\lstdefinestyle{java} {language=java}
\lstset{language=R,
	basicstyle=\ttfamily,
	keywordstyle=\color{lila},
	commentstyle=\color{lightgray},
	stringstyle=\color{mygreen}\ttfamily,
	backgroundcolor=\color{white},
	showstringspaces=false,
	numbers=left,
	numbersep=10pt,
	numberstyle=\color{mygray}\ttfamily,
	identifierstyle=\color{blue},
	xleftmargin=.1\textwidth, 
	%xrightmargin=.1\textwidth,
	escapechar=§,
	%literate={\t}{{\ }}1
	breaklines=true,
	postbreak=\mbox{\space}
}

\usepackage[colorlinks = true, linkcolor = blue, urlcolor  = blue, citecolor = blue, anchorcolor = blue]{hyperref}
\usepackage[utf8]{inputenc}

\renewcommand*{\arraystretch}{1.4}

\newcolumntype{L}[1]{>{\raggedright\arraybackslash}p{#1}}
\newcolumntype{R}[1]{>{\raggedleft\arraybackslash}p{#1}}
\newcolumntype{C}[1]{>{\centering\let\newline\\\arraybackslash\hspace{0pt}}m{#1}}

\newcommand{\E}{\mathbb{E}}
\DeclareMathOperator{\rk}{rk}
\DeclareMathOperator{\Var}{Var}
\DeclareMathOperator{\Cov}{Cov}

\title{\textbf{Datensicherheit, Übung 5}}
\author{\textsc{Henry Haustein}}
\date{}

\begin{document}
	\maketitle
	
	\section*{Aufgabe 1}
	\begin{enumerate}[label=(\alph*)]
		\item Verschlüsselung und Kodierung
		\item Zuerst verschlüsseln, dann kodieren
	\end{enumerate}

	\section*{Aufgabe 2}
	Die Sicherheit eines Verfahrens darf nicht von der Geheimhaltung des Verfahrens abhängen, sondern nur von der Geheimhaltung des Schlüssels. Schlüssel sind leichter geheimzuhalten als das Verfahren, weil das Verfahren ja auch dem Kommunikationspartner bekannt sein muss.
	
	\section*{Aufgabe 3}
	\begin{enumerate}[label=(\alph*)]
		\item Für jede Kommunikation einen Schlüssel: $\frac{20\cdot 19}{2} = 190$
		\item Jeder braucht 2 Schlüssel: $2\cdot 20 = 40$
	\end{enumerate}
	
	\section*{Aufgabe 4}
	\begin{enumerate}[label=(\alph*)]
		\item alle Verschlüsselungssysteme hintereinander schalten
		\item Parallele Verwendung von allen Verschlüsselungssystemen $\Rightarrow$ mehr Prüfsummen
	\end{enumerate}
	
	\section*{Aufgabe 5}
	Folgende Möglichkeiten gibt es:
	\begin{enumerate}[label=(\arabic*)]
		\item Zuerst verschlüsseln, dann die MAC berechnen
		\item MAC und Verschlüsselung parallel aus der Nachricht berechnen
		\item MAC berechnen, dann MAC + Klartext verschlüsseln
	\end{enumerate}
	\begin{center}
		\begin{tabular}{l|c|c|c}
			& \textbf{(1)} & \textbf{(2)} & \textbf{(3)} \\
			\hline
			\textbf{Integrität Klartext} & $\checkmark$ & $\checkmark$ & $\checkmark$ \\
			\hline
			\textbf{Integrität Schlüsseltext} & $\checkmark$ & & \\
			\hline
			\textbf{Parallelität Verschlüsselung} & & $\checkmark$ & \\
			\hline
			\textbf{Parallelität Entschlüsselung} & $\checkmark$ & & 
		\end{tabular}
	\end{center}
	
	\section*{Aufgabe 6}
	\begin{enumerate}[label=(\alph*)]
		\item Vertraulichkeit: nicht gegeben, weil Signatur keine Verschlüsselung ist, Zurechenbarkeit: hängt vom Vertrauen in die Instanz ab, Integrität: gegeben
		\item Vertraulichkeit: nicht gegeben, weil Signatur keine Verschlüsselung ist, Zurechenbarkeit: gegeben, Integrität: gegeben
	\end{enumerate}
	
	\section*{Aufgabe 7}
	\begin{enumerate}[label=(\alph*)]
		\item asymmetrisches Authentifikationssystem: Wichtig ist, von wem die Nachricht kommt, der Inhalt ist nicht ganz so wichtig, Verschlüsselung wäre trotzdem schön.
		\item symmetrisches Authentifikationssystem: es muss schnell gehen, noch besser wäre allerdings Verfahren zur Prüfung der Integrität der Daten zu nutzen wie z.B. CRC.
		\item asymmetrisches Konzelations- und Authentifikationssystem: ähnlich wie (a), allerdings sind Preisinformationen deutlich schützenswerter als Vor- und Nachteile einer Wohnung.
	\end{enumerate}

\end{document}