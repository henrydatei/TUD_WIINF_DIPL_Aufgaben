\documentclass{article}

\usepackage{amsmath,amssymb}
\usepackage{tikz}
\usepackage{pgfplots}
\usepackage{xcolor}
\usepackage[left=2.1cm,right=3.1cm,bottom=3cm,footskip=0.75cm,headsep=0.5cm]{geometry}
\usepackage{enumerate}
\usepackage{enumitem}
\usepackage{marvosym}
\usepackage{tabularx}
\usepackage{parskip}

\usepackage{listings}
\definecolor{lightlightgray}{rgb}{0.95,0.95,0.95}
\definecolor{lila}{rgb}{0.8,0,0.8}
\definecolor{mygray}{rgb}{0.5,0.5,0.5}
\definecolor{mygreen}{rgb}{0,0.8,0.26}
%\lstdefinestyle{java} {language=java}
\lstset{language=R,
	basicstyle=\ttfamily,
	keywordstyle=\color{lila},
	commentstyle=\color{lightgray},
	stringstyle=\color{mygreen}\ttfamily,
	backgroundcolor=\color{white},
	showstringspaces=false,
	numbers=left,
	numbersep=10pt,
	numberstyle=\color{mygray}\ttfamily,
	identifierstyle=\color{blue},
	xleftmargin=.1\textwidth, 
	%xrightmargin=.1\textwidth,
	escapechar=§,
	%literate={\t}{{\ }}1
	breaklines=true,
	postbreak=\mbox{\space}
}

\usepackage[colorlinks = true, linkcolor = blue, urlcolor  = blue, citecolor = blue, anchorcolor = blue]{hyperref}
\usepackage[utf8]{inputenc}

\renewcommand*{\arraystretch}{1.4}

\newcolumntype{L}[1]{>{\raggedright\arraybackslash}p{#1}}
\newcolumntype{R}[1]{>{\raggedleft\arraybackslash}p{#1}}
\newcolumntype{C}[1]{>{\centering\let\newline\\\arraybackslash\hspace{0pt}}m{#1}}

\newcommand{\E}{\mathbb{E}}
\DeclareMathOperator{\rk}{rk}
\DeclareMathOperator{\Var}{Var}
\DeclareMathOperator{\Cov}{Cov}

\title{\textbf{Datensicherheit, Zusammenfassung Vorlesung 2}}
\author{\textsc{Henry Haustein}, \textsc{Dennis Rössel}}
\date{}

\begin{document}
	\maketitle
	
	\section*{Welche Rechte stehen den Betroffenen bzgl. ihrer personenbezogenen Daten zu (insbesondere bzgl. Transparenz und Richtigkeit; Widerspruch)?}
	Rechte der betroffenen Person:
	\begin{itemize}
		\item Transparenz: Informationspflichten und Auskunftsrecht
		\item Berichtigung, Löschung, Einschränkung (Korrekturrechte)
		\item Datenübertragbarkeit
		\item Widerspruch
		\item automatisierte Entscheidungsfindung im Einzelfall
	\end{itemize}
	
	\section*{Welche Risiken ergeben sich für den Datenschutz durch IKT?}
	Risiken:
	\begin{itemize}
		\item Aktionen in der Online-Welt hinterlassen unbemerkt Datenspuren
		\item Entstehung großer Mengen von Daten
		\item Speicherung und Auswertung großer Datenmengen unproblematisch
		\item Kontrolle für den Nutzer schwierig
	\end{itemize}

	\section*{Welche Klassen von Bedrohungen kann man unterscheiden?}
	Bedrohungen durch unerwünschte Ereignisse
	\begin{itemize}
		\item unbeabsichtigte Ereignisse
		\item beabsichtigte Angriffe
	\end{itemize}

	\section*{Was sind (beispielsweise) Ursachen für Sicherheitsprobleme?}
	Vernetzung, Komplexität, Allgegenwärtigkeit, Verfügbarkeit von Angriffswerkzeugen und Angriffsmethoden, fehlendes Sicherheitsbewusstsein und digitale Sorglosigkeit, hohe Anzahl kritischer Schachstellen

\end{document}