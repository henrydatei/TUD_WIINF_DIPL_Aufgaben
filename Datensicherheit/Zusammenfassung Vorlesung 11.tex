\documentclass{article}

\usepackage{amsmath,amssymb}
\usepackage{tikz}
\usepackage{pgfplots}
\usepackage{xcolor}
\usepackage[left=2.1cm,right=3.1cm,bottom=3cm,footskip=0.75cm,headsep=0.5cm]{geometry}
\usepackage{enumerate}
\usepackage{enumitem}
\usepackage{marvosym}
\usepackage{tabularx}
\usepackage{parskip}

\usepackage{listings}
\definecolor{lightlightgray}{rgb}{0.95,0.95,0.95}
\definecolor{lila}{rgb}{0.8,0,0.8}
\definecolor{mygray}{rgb}{0.5,0.5,0.5}
\definecolor{mygreen}{rgb}{0,0.8,0.26}
%\lstdefinestyle{java} {language=java}
\lstset{language=R,
	basicstyle=\ttfamily,
	keywordstyle=\color{lila},
	commentstyle=\color{lightgray},
	stringstyle=\color{mygreen}\ttfamily,
	backgroundcolor=\color{white},
	showstringspaces=false,
	numbers=left,
	numbersep=10pt,
	numberstyle=\color{mygray}\ttfamily,
	identifierstyle=\color{blue},
	xleftmargin=.1\textwidth, 
	%xrightmargin=.1\textwidth,
	escapechar=§,
	%literate={\t}{{\ }}1
	breaklines=true,
	postbreak=\mbox{\space}
}

\usepackage[colorlinks = true, linkcolor = blue, urlcolor  = blue, citecolor = blue, anchorcolor = blue]{hyperref}
\usepackage[utf8]{inputenc}

\renewcommand*{\arraystretch}{1.4}

\newcolumntype{L}[1]{>{\raggedright\arraybackslash}p{#1}}
\newcolumntype{R}[1]{>{\raggedleft\arraybackslash}p{#1}}
\newcolumntype{C}[1]{>{\centering\let\newline\\\arraybackslash\hspace{0pt}}m{#1}}

\newcommand{\E}{\mathbb{E}}
\DeclareMathOperator{\rk}{rk}
\DeclareMathOperator{\Var}{Var}
\DeclareMathOperator{\Cov}{Cov}

\title{\textbf{Datensicherheit, Zusammenfassung Vorlesung 11}}
\author{\textsc{Henry Haustein}, \textsc{Dennis Rössel}}
\date{}

\begin{document}
	\maketitle

	\section*{Was charakterisiert den Algorithmus AES?}
	Substitutions-Permutations-Netzwerk mit wahlweise 10, 12 oder 14 Runden
	
	\section*{Wie erfolgt bei AES die Verschlüsselung?}
	Verschlüsselung von Klartextblöcken der Länge 128 Bit (vorgeschlagene Längen von 192 und 256 Bits nicht standardisiert), Schlüssellänge wahlweise 128, 192 oder 256 Bits, Rundenanzahl $r$ hängt von Schlüssel- und Klartextlänge ab:
	\begin{itemize}
		\item Schlüssellänge 128 Bit: 10 Runden
		\item Schlüssellänge 192 Bit: 12 Runden
		\item Schlüssellänge 256 Bit: 14 Runden
	\end{itemize}
	
	Runde 0: $\oplus k_0$ \\
	Struktur der ersten $r-1$ Runden: SubBytes, ShiftRow, MixColumn, $\oplus k_i$ \\
	Struktur der $r$-ten Runde: SubBytes, ShiftRow, $\oplus k_r$
	
	\section*{Wie werden die Teilschlüssel erzeugt?}
	Schlüsselexpansion mit Rot (zyklische Verschiebung), SubWord (Substitution mit $S_8$) und Rcon (Rundenkonstante) 
	
	\section*{Wie erfolgt die Entschlüsselung?}
	Runde $r$: $\oplus k_r$, ShiftRow$^{-1}$, SubBytes$^{-1}$ \\
	Runde $1,...,r-1$: $\oplus k_i$, MixColumn$^{-1}$, ShiftRow$^{-1}$, SubBytes$^{-1}$ \\
	Runde 0: $\oplus k_0$
	
	\section*{Was versteht man unter synchronen/selbstsynchronisierenden Chiffren?}
	Synchrone Stromchiffre: Verschlüsselung eines Zeichens ist abhängig von der Position bzw. von vorhergehenden Klartext- oder Schlüsselzeichen
	
	Selbstsynchronisierende Stromchiffre: Verschlüsselung ist nur von begrenzter Anzahl vorhergehender Zeichen abhängig
	
	\section*{Wie erfolgen Ver- und Entschlüsselung bei den Betriebsarten ECB und CBC?}
	ECB: Jeder Block wird einzeln verschlüsselt/entschlüsselt
	
	CBC: $c_i = enc(k, m_i \oplus c_{i-1})$, $m_i = dec(k, c_i) \oplus c_{i-1}$
	
\end{document}