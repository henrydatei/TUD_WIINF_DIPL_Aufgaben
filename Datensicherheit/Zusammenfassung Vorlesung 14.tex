\documentclass{article}

\usepackage{amsmath,amssymb}
\usepackage{tikz}
\usepackage{pgfplots}
\usepackage{xcolor}
\usepackage[left=2.1cm,right=3.1cm,bottom=3cm,footskip=0.75cm,headsep=0.5cm]{geometry}
\usepackage{enumerate}
\usepackage{enumitem}
\usepackage{marvosym}
\usepackage{tabularx}
\usepackage{parskip}

\usepackage{listings}
\definecolor{lightlightgray}{rgb}{0.95,0.95,0.95}
\definecolor{lila}{rgb}{0.8,0,0.8}
\definecolor{mygray}{rgb}{0.5,0.5,0.5}
\definecolor{mygreen}{rgb}{0,0.8,0.26}
%\lstdefinestyle{java} {language=java}
\lstset{language=R,
	basicstyle=\ttfamily,
	keywordstyle=\color{lila},
	commentstyle=\color{lightgray},
	stringstyle=\color{mygreen}\ttfamily,
	backgroundcolor=\color{white},
	showstringspaces=false,
	numbers=left,
	numbersep=10pt,
	numberstyle=\color{mygray}\ttfamily,
	identifierstyle=\color{blue},
	xleftmargin=.1\textwidth, 
	%xrightmargin=.1\textwidth,
	escapechar=§,
	%literate={\t}{{\ }}1
	breaklines=true,
	postbreak=\mbox{\space}
}

\usepackage[colorlinks = true, linkcolor = blue, urlcolor  = blue, citecolor = blue, anchorcolor = blue]{hyperref}
\usepackage[utf8]{inputenc}

\renewcommand*{\arraystretch}{1.4}

\newcolumntype{L}[1]{>{\raggedright\arraybackslash}p{#1}}
\newcolumntype{R}[1]{>{\raggedleft\arraybackslash}p{#1}}
\newcolumntype{C}[1]{>{\centering\let\newline\\\arraybackslash\hspace{0pt}}m{#1}}

\newcommand{\E}{\mathbb{E}}
\DeclareMathOperator{\rk}{rk}
\DeclareMathOperator{\Var}{Var}
\DeclareMathOperator{\Cov}{Cov}

\title{\textbf{Datensicherheit, Zusammenfassung Vorlesung 14}}
\author{\textsc{Henry Haustein}, \textsc{Dennis Rössel}}
\date{}

\begin{document}
	\maketitle

	\section*{Was ist Multimedia-Sicherheit?}
	Forschungsgebiet, das sich mit der Durchsetzung von Schutzzielen an und mit digitalisierten Signalen als Abbild von Ausschnitten der Realität beschäftigt.
	
	\section*{Welche Schutzziele sind für Multimedia-Sicherheit relevant, und was bedeuten sie in diesem Kontext?}
	Integrität, Zurechenbarkeit, Verdecktheit, Vertraulichkeit
	
	\section*{Mit welchen Schutzmechanismen können diese Ziele durchgesetzt werden?}
	Multimedia-Forensik, digitale Wasserzeichen, Steganografie
	
	\section*{Wie grenzt sich Steganographie von Kryptographie ab?}
	Steganographie verdeckt die Existenz von vertraulicher Kommunikation
	
	\section*{Wie ist ein steganographisches System prinzipiell aufgebaut (Funktionen mit Ein- und Ausgabewerten)?}
	Einbetten der Nachricht in Coverdaten unter Verwendung eines Schlüssels
	
	Extrahieren der Daten unter Verwendung des Schlüssels
	
	\section*{Was sind relevante Anforderungen an steganographische Systeme?}
	Unentdeckbarkeit: Entdeckung nicht besser als Raten
	
	Hohe Einbettungsrate
	
	\section*{Welche Möglichkeiten für die Verwendung von Schlüsseln in der Steganographie gibt es?}
	Schlüssel kann Abstände zwischen Einbettungen oder Schlüssel ist Startwert für Zufallszahlengenerator
	
	\section*{Welche Klassen von Einbettungstechniken gibt es?}
	LSB-Ersetzung, Inkrementieren und Dekrementieren
	
	\section*{Wie funktionieren LSB-Ersetzung, Inkrementieren und Dekrementieren?}
	LSB-Ersetzung: letztes Bit jedes Pixels mit Nachricht ersetzen
	
	Inkrementieren: Wenn $cover\mod 2 \neq emb$, dann $stego = cover + 1$
	
	Dekrementieren: Wenn $cover\mod 2 \neq emb$, dann $stego = cover - 1$
	
	\section*{Wie ist die Sicherheit der LSB-Ersetzung zu bewerten?}
	unsicher, da gut untersucht, verschiedene Ansätze zur Analyse: visueller Angriff, Histogrammangriff, Analyse der Bildstruktur
	
\end{document}