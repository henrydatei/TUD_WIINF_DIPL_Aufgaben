\documentclass{article}

\usepackage{amsmath,amssymb}
\usepackage{tikz}
\usepackage{pgfplots}
\usepackage{xcolor}
\usepackage[left=2.1cm,right=3.1cm,bottom=3cm,footskip=0.75cm,headsep=0.5cm]{geometry}
\usepackage{enumerate}
\usepackage{enumitem}
\usepackage{marvosym}
\usepackage{tabularx}
\usepackage{parskip}

\usepackage{listings}
\definecolor{lightlightgray}{rgb}{0.95,0.95,0.95}
\definecolor{lila}{rgb}{0.8,0,0.8}
\definecolor{mygray}{rgb}{0.5,0.5,0.5}
\definecolor{mygreen}{rgb}{0,0.8,0.26}
%\lstdefinestyle{java} {language=java}
\lstset{language=R,
	basicstyle=\ttfamily,
	keywordstyle=\color{lila},
	commentstyle=\color{lightgray},
	stringstyle=\color{mygreen}\ttfamily,
	backgroundcolor=\color{white},
	showstringspaces=false,
	numbers=left,
	numbersep=10pt,
	numberstyle=\color{mygray}\ttfamily,
	identifierstyle=\color{blue},
	xleftmargin=.1\textwidth, 
	%xrightmargin=.1\textwidth,
	escapechar=§,
	%literate={\t}{{\ }}1
	breaklines=true,
	postbreak=\mbox{\space}
}

\usepackage[colorlinks = true, linkcolor = blue, urlcolor  = blue, citecolor = blue, anchorcolor = blue]{hyperref}
\usepackage[utf8]{inputenc}

\renewcommand*{\arraystretch}{1.4}

\newcolumntype{L}[1]{>{\raggedright\arraybackslash}p{#1}}
\newcolumntype{R}[1]{>{\raggedleft\arraybackslash}p{#1}}
\newcolumntype{C}[1]{>{\centering\let\newline\\\arraybackslash\hspace{0pt}}m{#1}}

\newcommand{\E}{\mathbb{E}}
\DeclareMathOperator{\rk}{rk}
\DeclareMathOperator{\Var}{Var}
\DeclareMathOperator{\Cov}{Cov}

\title{\textbf{Datensicherheit, Übung 1}}
\author{\textsc{Henry Haustein}}
\date{}

\begin{document}
	\maketitle
	
	\section*{Aufgabe 1}
	Die drei Schutzziele sind
	\begin{itemize}
		\item Vertraulichkeit
		\item Integrität
		\item Verfügbarkeit
	\end{itemize}
	\begin{enumerate}[label=(\alph*)]
		\item Alle 3 Schutzziele sind wichtig, Patientendaten sind sehr persönliche Daten und gehen damit nur den Arzt und den Patienten etwas an. Zudem sollten die Informationen auch sicher vor Manipulation sein, sonst kann es zu Fehlbehandlungen kommen. Und da auch immer Notfälle eintreten können, muss eine ständige Verfügbarkeit gewährleistet sein.
		\item Unter der Annahme, dass man für so einen Zugang zu dieser Datenbank bezahlen muss, dürfen nur Berechtigte (= zahlende Kunden) auf die Informationen zugreifen. Diese erwarten dann auch eine ständige Verfügbarkeit und die Daten dürfen auch nicht unbemerkt verändert werden, man verlässt sich auf die Richtigkeit der Daten.
		\item analog zu (a)
		\item analog zu (b)
	\end{enumerate}

	\section*{Aufgabe 2}
	Die Bedingungen für die Einwilligung sind
	\begin{itemize}
		\item Verantwortlicher muss Einwilligung nachweisen können
		\item Betroffener muss informiert sein
		\item Einwilligung muss freiwillig erfolgen
		\item Einwilligung durch eindeutige bestätigende Handlung
		\item Widerrufbarkeit der Einwilligung
	\end{itemize}
	
	\section*{Aufgabe 3}
	Damit der Betroffene seine Recht geltend machen kann.
	
	\section*{Aufgabe 4}
	Grundsatz der Datenminimierung. Familienstand ist nicht notwendig, um einen Kaufvertrag abzuschließen und das Produkt an die richtige Adresse zu liefern.
	
	\section*{Aufgabe 5}
	\begin{itemize}
		\item Intimsphäre: Einkommen, Krankendaten
		\item Privatsphäre: Adresse, Telefonnummer
		\item öffentlicher Bereich: Name, Freizeitaktivitäten
	\end{itemize}
	
	\section*{Aufgabe 6}
	Wohnort (wo ist das Handy normalerweise nachts?), Arbeitsort, Freizeitaktivitäten, Schätzung des Lebensstils (Luxusläden vs. Billigläden), Urlaubsregionen, Überschreitet derjenige das Tempolimit?, ...
	
	\section*{Aufgabe 7}
	Klassen von Bedrohungen
	\begin{itemize}
		\item Unbeabsichtigte Ereignisse
		\begin{itemize}
			\item Nichtabsehbare Folgen von Handlungen und Ereignissen (Höhere Gewalt (z.B. Blitzschlag, Brand, Wasser), Störungen der Energieversorgung, indirekte Blitzschäden)
			\item Ausfall oder Fehlverhalten technischer Mittel
			\item Fahrlässige Handlungen/Unterlassungen (Unterlassung notwendiger Handlungen, Leichtsinn, Unvermögen, Unachtsamkeit, Spieltrieb)
		\end{itemize}
		\item Beabsichtigte Angriffe
		\begin{itemize}
			\item Malware
		\end{itemize}
	\end{itemize}
	
	\section*{Aufgabe 8}
	Empfohlene Maßnahmen
	\begin{itemize}
		\item Organisationen: Regeln (Policies)
		\item Problembewusstsein (Awareness)
		\item Verringerung möglicher Schwachstellen
		\item Verringerung möglicher Bedrohungen
	\end{itemize}
	Für Beteiligte am Entwicklungsprozess: Security by Design, Privacy  by Design, Software einsetzen, die auch noch Updates bekommt, Best Practices beim Programmieren/Aufbau des Systems

\end{document}