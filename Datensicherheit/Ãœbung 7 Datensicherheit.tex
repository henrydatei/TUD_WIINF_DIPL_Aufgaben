\documentclass{article}

\usepackage{amsmath,amssymb}
\usepackage{tikz}
\usepackage{pgfplots}
\usepackage{xcolor}
\usepackage[left=2.1cm,right=3.1cm,bottom=3cm,footskip=0.75cm,headsep=0.5cm]{geometry}
\usepackage{enumerate}
\usepackage{enumitem}
\usepackage{marvosym}
\usepackage{tabularx}
\usepackage{parskip}

\usepackage{listings}
\definecolor{lightlightgray}{rgb}{0.95,0.95,0.95}
\definecolor{lila}{rgb}{0.8,0,0.8}
\definecolor{mygray}{rgb}{0.5,0.5,0.5}
\definecolor{mygreen}{rgb}{0,0.8,0.26}
%\lstdefinestyle{java} {language=java}
\lstset{language=R,
	basicstyle=\ttfamily,
	keywordstyle=\color{lila},
	commentstyle=\color{lightgray},
	stringstyle=\color{mygreen}\ttfamily,
	backgroundcolor=\color{white},
	showstringspaces=false,
	numbers=left,
	numbersep=10pt,
	numberstyle=\color{mygray}\ttfamily,
	identifierstyle=\color{blue},
	xleftmargin=.1\textwidth, 
	%xrightmargin=.1\textwidth,
	escapechar=§,
	%literate={\t}{{\ }}1
	breaklines=true,
	postbreak=\mbox{\space}
}

\usepackage[colorlinks = true, linkcolor = blue, urlcolor  = blue, citecolor = blue, anchorcolor = blue]{hyperref}
\usepackage[utf8]{inputenc}

\renewcommand*{\arraystretch}{1.4}

\newcolumntype{L}[1]{>{\raggedright\arraybackslash}p{#1}}
\newcolumntype{R}[1]{>{\raggedleft\arraybackslash}p{#1}}
\newcolumntype{C}[1]{>{\centering\let\newline\\\arraybackslash\hspace{0pt}}m{#1}}

\newcommand{\E}{\mathbb{E}}
\DeclareMathOperator{\rk}{rk}
\DeclareMathOperator{\Var}{Var}
\DeclareMathOperator{\Cov}{Cov}

\title{\textbf{Datensicherheit, Übung 7}}
\author{\textsc{Henry Haustein}}
\date{}

\begin{document}
	\maketitle
	
	\section*{Aufgabe 1}
	Erste Runde: 11000100, damit $L_0 = 1100$ und $R_0 = 0100$, damit $L_1 = R_0 = 0100$ und $R_1 = S(R_0 \oplus k_1) \oplus L_0 = S(0100 \oplus 0110) \oplus 1100 = 0001 \oplus 1100 = 1101 \Rightarrow 01001101$ \\
	Zweite Runde: $L_1 = 0100$, $R_1 = 1101$, $L_2 = R_1 = 1101$, $R_2 = S(R_1 \oplus k_2) \oplus L_1 = S(1101 \oplus 1010) \oplus 0100 = S(0111) \oplus 0100 = 1111 \oplus 0100 = 1011 \Rightarrow 11011011$
	
	Entschlüsselung erste Runde: $L_2 = 1101$, $R_2 = 1011$, $R_1 = 1101$, $L_1 = S(L_2\oplus k_2) \oplus R_2 = S(1101 \oplus 1010) \oplus 1011 = S(0111) \oplus 1011 = 1111 \oplus 1011 = 0100 \Rightarrow 01001101$ \\
	Entschlüsselung zweite Runde: $L_1 = 0100$ und $R_1 = 1101$, damit $R_0 = 0100$ und $L_0 = S(L_1 \oplus k_1) \oplus R_1 = S(0100 \oplus 0110) \oplus 1101 = S(0010) \oplus 1101 = 1100 \Rightarrow 11000100$

	\section*{Aufgabe 2}
	\begin{enumerate}[label=(\alph*)]
		\item $S(m) = 0111$, abgespeicherte Werte sind unterstrichen
		\begin{center}
			\begin{tabular}{c|c|c|c|c|c}
				\textbf{$k_i$ bei $t=1$} & \textbf{enc()} & \textbf{T()} & \textbf{enc()} & \textbf{T()} & \textbf{enc()} \\
				\hline
				\underline{1010} & 1101 & 1011 & 1100 & 1001 & \underline{1110} \\
				\hline
				\underline{0101} & 0010 & 0100 & 0011 & 0110 & \underline{0001}
			\end{tabular}
		\end{center}
		Wir finden $c = 0011$ nicht in der Tabelle, deswegen $c'=enc(T(c),m) = 0110 \oplus 0111 = 0001$. Dieser Wert findet sich, der Startschlüssel war 0101. Neuberechnung dieser Kette liefert nach der zweiten Iteration den gesuchten Ciphertext von 0011, damit ist der gesuchte Schlüssel $k = 0100$.
		\item $\frac{2\cdot 3}{2^4} = \frac{3}{8}$
	\end{enumerate}
	
	\section*{Aufgabe 3}
	Nachricht bleibt gleich: Eine Runde in DES ist eine Feistel-Chiffre, und damit selbstinvers. Entschlüsselung = Verschlüsselung, wenn der Schlüssel gleich ist, was hier gegeben ist. 
	
	\section*{Aufgabe 4}
	\begin{enumerate}[label=(\alph*)]
		\item $2^{56}$ Bits zu knacken, mit 2 Millionen Schlüssel pro Sekunde $\frac{2^{56}}{2000000} = 3.6\cdot 10^{10}$ Sekunden $\approx 1142$ Jahre.
		\item Im Schnitt wird der Angreifer nur die Hälfte der Schlüssel testen müssen, er ist also nach 571 Jahren fertig.
		\item $\frac{2^{56}}{65280000000} = 1.1\cdot 10^6$ Sekunden $\approx 12.78$ Tage
		\item Wieder nur die Hälfte, also weniger als eine Woche.
	\end{enumerate}
	siehe: \url{https://de.wikipedia.org/wiki/Copacobana}
	
	\section*{Aufgabe 5}
	Jedes $(k,c)$-Paar abspeichern, jedes Paar ist $56 + 64 = 120$ Bit, es gibt $2^{56}$ Paare, damit 1.08 EB.
	
	Zeitaufwand: $2^{56}$ Verschlüsselungen und sortieren nach $c$ für schnelleres Suchen mittels binärer Suche
	
	\section*{Aufgabe 6}
	Meet-in-the-middle-Angriff, zweifache Verschlüsselung (Aufwand $2^{56}\cdot 2^{56} = 2^{112}$) und einfache Entschlüsselung (Aufwand $2^{56}$)
	
	Speicheraufwand: $2^{56}$ nur für die einfache Entschlüsselung reicht.

\end{document}