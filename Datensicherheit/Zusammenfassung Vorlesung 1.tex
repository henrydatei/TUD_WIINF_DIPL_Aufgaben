\documentclass{article}

\usepackage{amsmath,amssymb}
\usepackage{tikz}
\usepackage{pgfplots}
\usepackage{xcolor}
\usepackage[left=2.1cm,right=3.1cm,bottom=3cm,footskip=0.75cm,headsep=0.5cm]{geometry}
\usepackage{enumerate}
\usepackage{enumitem}
\usepackage{marvosym}
\usepackage{tabularx}
\usepackage{parskip}

\usepackage{listings}
\definecolor{lightlightgray}{rgb}{0.95,0.95,0.95}
\definecolor{lila}{rgb}{0.8,0,0.8}
\definecolor{mygray}{rgb}{0.5,0.5,0.5}
\definecolor{mygreen}{rgb}{0,0.8,0.26}
%\lstdefinestyle{java} {language=java}
\lstset{language=R,
	basicstyle=\ttfamily,
	keywordstyle=\color{lila},
	commentstyle=\color{lightgray},
	stringstyle=\color{mygreen}\ttfamily,
	backgroundcolor=\color{white},
	showstringspaces=false,
	numbers=left,
	numbersep=10pt,
	numberstyle=\color{mygray}\ttfamily,
	identifierstyle=\color{blue},
	xleftmargin=.1\textwidth, 
	%xrightmargin=.1\textwidth,
	escapechar=§,
	%literate={\t}{{\ }}1
	breaklines=true,
	postbreak=\mbox{\space}
}

\usepackage[colorlinks = true, linkcolor = blue, urlcolor  = blue, citecolor = blue, anchorcolor = blue]{hyperref}
\usepackage[utf8]{inputenc}

\renewcommand*{\arraystretch}{1.4}

\newcolumntype{L}[1]{>{\raggedright\arraybackslash}p{#1}}
\newcolumntype{R}[1]{>{\raggedleft\arraybackslash}p{#1}}
\newcolumntype{C}[1]{>{\centering\let\newline\\\arraybackslash\hspace{0pt}}m{#1}}

\newcommand{\E}{\mathbb{E}}
\DeclareMathOperator{\rk}{rk}
\DeclareMathOperator{\Var}{Var}
\DeclareMathOperator{\Cov}{Cov}

\title{\textbf{Datensicherheit, Zusammenfassung Vorlesung 1}}
\author{\textsc{Henry Haustein}, \textsc{Dennis Rössel}}
\date{}

\begin{document}
	\maketitle
	
	\section*{Was sind besondere Eigenschaften von Informationen, aus denen sich Missbrauchsmöglichkeiten und die Notwendigkeit von Datensicherheit ergeben?}
	Informationen haben Wert durch alleinigen Besitz, Informationen sind duplizierbar, Wertzuwachs durch Akkumulation möglich
	
	\section*{Was sind die grundlegenden Schutzziele und was bedeuten sie?}
	Schutzziele
	\begin{itemize}
		\item Vertraulichkeit: Informationen werden nur Berechtigten bekannt.
		\item Integrität: Informationen können nicht unerkannt modifiziert werden.
		\item Zurechenbarkeit (spezielles Integritätsziel): Es kann gegenüber Dritten nachgewiesen werden, wer die Information erzeugt hat.
	\end{itemize}
	Der Schutz der Verfügbarkeit erfordert andere Maßnahmen, z.B. Redundanz oder Kontrolle der Ressourcen\-nutzung.

	\section*{Ist ein Verlust dieser Schutzziele erkennbar/verhinderbar/kann er rückgängig gemacht werden?}
	Vertraulichkeit: verhinderbar, Integrität: erkennbar und kann rückgängig gemacht werden, Zurechenbarkeit: erkennbar und kann rückgängig gemacht werden, Verfügbarkeit: erkennbar und kann rückgängig gemacht werden
	
	\section*{Wie grenzen sich Datenschutz und Datensicherheit ab?}
	Datenschutz: Schutz der Betroffenen, Schutz vor Daten
	
	Datensicherheit: Schutz der Daten vor Missbrauch, Verfälschung und Nichtverfügbarkeit
	
	\section*{Welche Grundsätze der Datenverarbeitung sind einzuhalten?}
	Art. 5 DSGVO
	\begin{itemize}
		\item Rechtmäßigkeit, Verarbeitung nach Treu und Glauben, Transparenz
		\item Zweckbindung
		\item Datenminimierung
		\item Richtigkeit
		\item Speicherbegrenzung
		\item Integrität und Vertraulichkeit
		\item Rechenschaftspflicht
	\end{itemize}
	
	\section*{Was verstehen Sie unter dem \textit{Verbot mit Erlaubnisvorbehalt}? Wann ist die Verarbeitung personenbezogener Daten rechtmäßig?}
	Verarbeitung ist grundsätzlich verboten und eine begründungsbedürftige Ausnahme
	
	Rechtmäßigkeit: Einwilligung der betroffenen Person oder sonstige Rechtsgrundlage

\end{document}