\documentclass{article}

\usepackage{amsmath,amssymb}
\usepackage{tikz}
\usepackage{pgfplots}
\usepackage{xcolor}
\usepackage[left=2.1cm,right=3.1cm,bottom=3cm,footskip=0.75cm,headsep=0.5cm]{geometry}
\usepackage{enumerate}
\usepackage{enumitem}
\usepackage{marvosym}
\usepackage{tabularx}
\usepackage{parskip}

\usepackage{listings}
\definecolor{lightlightgray}{rgb}{0.95,0.95,0.95}
\definecolor{lila}{rgb}{0.8,0,0.8}
\definecolor{mygray}{rgb}{0.5,0.5,0.5}
\definecolor{mygreen}{rgb}{0,0.8,0.26}
%\lstdefinestyle{java} {language=java}
\lstset{language=R,
	basicstyle=\ttfamily,
	keywordstyle=\color{lila},
	commentstyle=\color{lightgray},
	stringstyle=\color{mygreen}\ttfamily,
	backgroundcolor=\color{white},
	showstringspaces=false,
	numbers=left,
	numbersep=10pt,
	numberstyle=\color{mygray}\ttfamily,
	identifierstyle=\color{blue},
	xleftmargin=.1\textwidth, 
	%xrightmargin=.1\textwidth,
	escapechar=§,
	%literate={\t}{{\ }}1
	breaklines=true,
	postbreak=\mbox{\space}
}

\usepackage[colorlinks = true, linkcolor = blue, urlcolor  = blue, citecolor = blue, anchorcolor = blue]{hyperref}
\usepackage[utf8]{inputenc}

\renewcommand*{\arraystretch}{1.4}

\newcolumntype{L}[1]{>{\raggedright\arraybackslash}p{#1}}
\newcolumntype{R}[1]{>{\raggedleft\arraybackslash}p{#1}}
\newcolumntype{C}[1]{>{\centering\let\newline\\\arraybackslash\hspace{0pt}}m{#1}}

\newcommand{\E}{\mathbb{E}}
\DeclareMathOperator{\rk}{rk}
\DeclareMathOperator{\Var}{Var}
\DeclareMathOperator{\Cov}{Cov}

\title{\textbf{Datensicherheit, Zusammenfassung Vorlesung 10}}
\author{\textsc{Henry Haustein}, \textsc{Dennis Rössel}}
\date{}

\begin{document}
	\maketitle

	\section*{Was ist unter \textit{iterierter Blockchiffre} zu verstehen?}
	Verschlüsselung geschieht in mehreren Runden, Anzahl der Runden relevant für Sicherheit, Rundenfunktion mit Rundenschlüssel
	
	\section*{Was sind Beispiele für allgemeine Ansätze zur Analyse von Blockchiffren? Was ist jeweils das Ziel und der Ablauf?}
	vollständige Schlüsselsuche, vorab berechnete Tabelle, Time-Memory-Tradeoff
	
	\section*{Was sind die charakteristischen Eigenschaften der Feistel-Chiffre?}
	selbstinvers, Vertauschung linker und rechter Hälfte
	
	\section*{Was bedeutet Selbstinversität?}
	Verschlüsselung = Entschlüsselung, nur Reihenfolge der Rundenschlüssel wird umgekehrt
	
	\section*{Was charakterisiert den Algorithmus DES?}
	Feistelchiffre mit 16 Runden, 64 Bit Blöcke, Schlüssel 64 Bit mit 8 Paritätsbits
	
	\section*{Wie ist die Sicherheit des DES-Algorithmus zu bewerten?}
	nicht mehr sicher wegen Schlüssellänge
	
	\section*{Was ist das Ziel der Mehrfachverschlüsselung? Genügt eine doppelte Verschlüsselung?}
	mehrere Schlüssel zu knacken; nein wegen Meet-in-the-Middle-Angriff
	
	\section*{Wie ist der Ablauf des Meet-in-the-Middle-Angriffs?}
	alle Verschlüsselungen des Klartextes, alle Entschlüsselungen des Ciphertextes berechnen $\Rightarrow$ abgleichen
	
\end{document}