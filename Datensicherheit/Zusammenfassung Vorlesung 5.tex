\documentclass{article}

\usepackage{amsmath,amssymb}
\usepackage{tikz}
\usepackage{pgfplots}
\usepackage{xcolor}
\usepackage[left=2.1cm,right=3.1cm,bottom=3cm,footskip=0.75cm,headsep=0.5cm]{geometry}
\usepackage{enumerate}
\usepackage{enumitem}
\usepackage{marvosym}
\usepackage{tabularx}
\usepackage{parskip}

\usepackage{listings}
\definecolor{lightlightgray}{rgb}{0.95,0.95,0.95}
\definecolor{lila}{rgb}{0.8,0,0.8}
\definecolor{mygray}{rgb}{0.5,0.5,0.5}
\definecolor{mygreen}{rgb}{0,0.8,0.26}
%\lstdefinestyle{java} {language=java}
\lstset{language=R,
	basicstyle=\ttfamily,
	keywordstyle=\color{lila},
	commentstyle=\color{lightgray},
	stringstyle=\color{mygreen}\ttfamily,
	backgroundcolor=\color{white},
	showstringspaces=false,
	numbers=left,
	numbersep=10pt,
	numberstyle=\color{mygray}\ttfamily,
	identifierstyle=\color{blue},
	xleftmargin=.1\textwidth, 
	%xrightmargin=.1\textwidth,
	escapechar=§,
	%literate={\t}{{\ }}1
	breaklines=true,
	postbreak=\mbox{\space}
}

\usepackage[colorlinks = true, linkcolor = blue, urlcolor  = blue, citecolor = blue, anchorcolor = blue]{hyperref}
\usepackage[utf8]{inputenc}

\renewcommand*{\arraystretch}{1.4}

\newcolumntype{L}[1]{>{\raggedright\arraybackslash}p{#1}}
\newcolumntype{R}[1]{>{\raggedleft\arraybackslash}p{#1}}
\newcolumntype{C}[1]{>{\centering\let\newline\\\arraybackslash\hspace{0pt}}m{#1}}

\newcommand{\E}{\mathbb{E}}
\DeclareMathOperator{\rk}{rk}
\DeclareMathOperator{\Var}{Var}
\DeclareMathOperator{\Cov}{Cov}

\title{\textbf{Datensicherheit, Zusammenfassung Vorlesung 5}}
\author{\textsc{Henry Haustein}, \textsc{Dennis Rössel}}
\date{}

\begin{document}
	\maketitle
	
	\section*{Welche Vor- und Nachteile hat redundante Speicherung?}
	Vorteile
	\begin{itemize}
		\item gleiche Datenbestände auf zwei physisch getrennten Datenspeichern
		\item bei Nichtverfügbarkeit eines Gerätes kann sofort mit zweitem weitergearbeitet werden
	\end{itemize}
	
	Nachteile:
	\begin{itemize}
		\item räumliche Nähe beider Geräte $\Rightarrow$ gleiche Gefährdung beider durch mutwillige Zerstörung und höhere Gewalt
	\end{itemize}
	
	\section*{Wie funktionieren RAID 0, 1 und 5?}
	RAID 0: Aufteilung der Daten auf die vorhandenen Festplatten
	
	RAID 1: Spiegelung der Daten
	
	RAID 5: Aufteilung der Daten + Paritätsinformationen auf die vorhandenen Festplatten
	
	\section*{Welche Fragen sind bei der Organisation von Backups zu klären?}
	Welche Daten? Welche Abstände? Wie viele Kopien? Wo?
	
	\section*{Welche Backup-Strategien unterscheidet man, wie funktionieren diese und wie erfolgt jeweils die Wiederherstellung?}
	Backup-Strategien:
	\begin{itemize}
		\item Voll-Backup
		\item inkrementelles Backup: Sicherung aller Änderungen seit dem letzten Backup
		\item differentielles Backup: Sicherung aller Änderungen seit dem letzten Voll-Backup
	\end{itemize}
	
	\section*{Welche Schutzziele sind bzgl. der Aufbewahrung der Kopien relevant und wie können diese umgesetzt werden?}
	Verfügbarkeit: Kopien nicht am selben Ort wie Original aufbewahren
	
	Vertraulichkeit, Integrität: Verschlüsselung
	
	\section*{Was ist Zielstellung der Kanalkodierung?}
	Schutz vor Störungen im Kanal
	
\end{document}