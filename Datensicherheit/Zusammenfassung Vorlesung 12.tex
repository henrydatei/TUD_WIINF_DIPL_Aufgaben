\documentclass{article}

\usepackage{amsmath,amssymb}
\usepackage{tikz}
\usepackage{pgfplots}
\usepackage{xcolor}
\usepackage[left=2.1cm,right=3.1cm,bottom=3cm,footskip=0.75cm,headsep=0.5cm]{geometry}
\usepackage{enumerate}
\usepackage{enumitem}
\usepackage{marvosym}
\usepackage{tabularx}
\usepackage{parskip}

\usepackage{listings}
\definecolor{lightlightgray}{rgb}{0.95,0.95,0.95}
\definecolor{lila}{rgb}{0.8,0,0.8}
\definecolor{mygray}{rgb}{0.5,0.5,0.5}
\definecolor{mygreen}{rgb}{0,0.8,0.26}
%\lstdefinestyle{java} {language=java}
\lstset{language=R,
	basicstyle=\ttfamily,
	keywordstyle=\color{lila},
	commentstyle=\color{lightgray},
	stringstyle=\color{mygreen}\ttfamily,
	backgroundcolor=\color{white},
	showstringspaces=false,
	numbers=left,
	numbersep=10pt,
	numberstyle=\color{mygray}\ttfamily,
	identifierstyle=\color{blue},
	xleftmargin=.1\textwidth, 
	%xrightmargin=.1\textwidth,
	escapechar=§,
	%literate={\t}{{\ }}1
	breaklines=true,
	postbreak=\mbox{\space}
}

\usepackage[colorlinks = true, linkcolor = blue, urlcolor  = blue, citecolor = blue, anchorcolor = blue]{hyperref}
\usepackage[utf8]{inputenc}

\renewcommand*{\arraystretch}{1.4}

\newcolumntype{L}[1]{>{\raggedright\arraybackslash}p{#1}}
\newcolumntype{R}[1]{>{\raggedleft\arraybackslash}p{#1}}
\newcolumntype{C}[1]{>{\centering\let\newline\\\arraybackslash\hspace{0pt}}m{#1}}

\newcommand{\E}{\mathbb{E}}
\DeclareMathOperator{\rk}{rk}
\DeclareMathOperator{\Var}{Var}
\DeclareMathOperator{\Cov}{Cov}

\title{\textbf{Datensicherheit, Zusammenfassung Vorlesung 12}}
\author{\textsc{Henry Haustein}, \textsc{Dennis Rössel}}
\date{}

\begin{document}
	\maketitle

	\section*{Wie erfolgt Ver- und Entschlüsselung bei der Betriebsart CTR?}
	Zählvariable verschlüsseln/entschlüsseln und $\oplus$ mit Klartext
	
	\section*{Was sind Eigenschaften der Betriebsarten ECB, CBC und CTR?}
	ECB (Electronic Code Book)
	\begin{itemize}
		\item Selbstsynchronisierend (Abhängigkeit von 0 Blöcken)
		\item Länge der verarbeiteten Einheiten: entsprechend Blockgröße der Blockchiffre (AES: $l = 128$ Bit)
		\item Keine Abhängigkeiten zwischen den Blöcken
	\end{itemize}
	
	CBC (Cipher Block Chaining)
	\begin{itemize}
		\item Selbstsynchronisierend (Abhängigkeit von 1 Block)
		\item Länge der verarbeiteten Einheiten: entsprechend Blockgröße der Blockchiffre (AES: $l = 128$ Bit)
		\item Abhängigkeiten zwischen den Blöcken: gleiche Klartextblöcke liefern unterschiedliche Schlüsseltextblöcke
		\item Initialisierungsvektor IV muss nicht geheim sein, darf aber nicht vorhersagbar sein
	\end{itemize}

	CTR (Counter Mode)
	\begin{itemize}
		\item synchron
		\item Abhängigkeit von Position der verarbeiteten Einheit
		\item Direktzugriff auf einzelne Schlüsseltextblöcke möglich
	\end{itemize}
	
	\section*{Wie wirken sich additive bzw. Synchronisationsfehler bei diesen Betriebsarten aus?}
	ECB: keine Fehlerfortpflanzung bei additivem Fehler und Synchronisationfehler
	
	CBC: Fehlerfortpflanzung in den Folgeblock bei additivem Fehler, 2 Blöcke bei Synchronisationfehler bzgl. ganzem Block betroffen bzw. Entschlüsselung fehlerhaft, bei Synchronisationsfehler bzgl. Bits (bis Blockgrenzen neu festgelegt werden)
	
	CTR: keine Fehlerfortpflanzung bei additivem Fehler, anfällig gegen Synchronisationsfehler
	
	\section*{Welchen Vorteil bietet der Counter Mode?}
	Effizienz
	
	\section*{Welche dieser Betriebsarten eignet sich für die Berechnung eines MACs? Warum?}
	CBC-MAC
	
	\section*{Wie erfolgt die Berechnung bzw. das Testen des MACs?}
	letzten Schlüsseltextblock als MAC anhängen (Empfänger verschlüsselt ebenfalls und vergleicht dann)
	
	\section*{Was bedeutet \textit{multiplikatives Inverses}? Wie kann es bestimmt werden?}
	$x\cdot x^{-1} \equiv 1\mod n$, Bestimmung mittels erweiterter Euklidischer Algorithmus
	
\end{document}