\documentclass{article}

\usepackage{amsmath,amssymb}
\usepackage{tikz}
\usepackage{pgfplots}
\usepackage{xcolor}
\usepackage[left=2.1cm,right=3.1cm,bottom=3cm,footskip=0.75cm,headsep=0.5cm]{geometry}
\usepackage{enumerate}
\usepackage{enumitem}
\usepackage{marvosym}
\usepackage{tabularx}
\usepackage{parskip}

\usepackage{listings}
\definecolor{lightlightgray}{rgb}{0.95,0.95,0.95}
\definecolor{lila}{rgb}{0.8,0,0.8}
\definecolor{mygray}{rgb}{0.5,0.5,0.5}
\definecolor{mygreen}{rgb}{0,0.8,0.26}
%\lstdefinestyle{java} {language=java}
\lstset{language=R,
	basicstyle=\ttfamily,
	keywordstyle=\color{lila},
	commentstyle=\color{lightgray},
	stringstyle=\color{mygreen}\ttfamily,
	backgroundcolor=\color{white},
	showstringspaces=false,
	numbers=left,
	numbersep=10pt,
	numberstyle=\color{mygray}\ttfamily,
	identifierstyle=\color{blue},
	xleftmargin=.1\textwidth, 
	%xrightmargin=.1\textwidth,
	escapechar=§,
	%literate={\t}{{\ }}1
	breaklines=true,
	postbreak=\mbox{\space}
}

\usepackage[colorlinks = true, linkcolor = blue, urlcolor  = blue, citecolor = blue, anchorcolor = blue]{hyperref}
\usepackage[utf8]{inputenc}

\renewcommand*{\arraystretch}{1.4}

\newcolumntype{L}[1]{>{\raggedright\arraybackslash}p{#1}}
\newcolumntype{R}[1]{>{\raggedleft\arraybackslash}p{#1}}
\newcolumntype{C}[1]{>{\centering\let\newline\\\arraybackslash\hspace{0pt}}m{#1}}

\newcommand{\E}{\mathbb{E}}
\DeclareMathOperator{\rk}{rk}
\DeclareMathOperator{\Var}{Var}
\DeclareMathOperator{\Cov}{Cov}

\title{\textbf{Datensicherheit, Zusammenfassung Vorlesung 6}}
\author{\textsc{Henry Haustein}, \textsc{Dennis Rössel}}
\date{}

\begin{document}
	\maketitle
	
	\section*{Welche Eigenschaft sichert bei einem ungleichmäßigen Quellenkode die Dekodierbarkeit?}
	Präfixfreiheit: kein Kodewort darf Beginn eines anderen Kodewortes sein
	
	\section*{Fehlererkennung bei Kanalkodierung: Wie prüft der Empfänger eine empfangene Binärfolge auf Verfälschungen?}
	Redundanz überprüfen
	
	\section*{Was sind mögliche Ergebnisse der Fehlererkennung, und wie sind diese zu interpretieren?}
	Fehler und kann rekonstruiert werden, Fehler und kann nicht rekonstruiert werden, (Fehler wird nicht erkannt)
	
	\section*{Welche Möglichkeiten der Fehlerkorrektur gibt es?}
	durch Wiederholung (erneute Übertragung im Fehlerfall), durch Rekonstruktion (Redundanz)
	
	\section*{Was sind mögliche Ergebnisse der Rekonstruktion im Fehlerfall?}
	korrekte Rekonstruktion, falsche Rekonstruktion, Versagen der Rekonstruktion
	
	\section*{Was sagt die minimale Hammingdistanz $d_{min}$ über die Fehlererkennungs- bzw. Fehlerkorrektureigenschaften eines Kodes aus?}
	$f_e = d_{min} - 1$ von Null verschiedene Stellen erkennen, $f_k = \left\lfloor \frac{d_{min} - 1}{2}\right\rfloor$ von Null verschiedene Stellen korrigieren
	
	\section*{Wie funktioniert der Paritätskode? Welche Kodeparameter hat er? Wie erfolgt die Fehlererkennung?}
	Hinzufügen einer weiteren Stelle, geradzahliges Gewicht insgesamt. Parameter: $(n,l,d_{min}) = (n, n-1, 2)$, Fehlererkennung: Überprüfen des Paritätsbits
	
\end{document}