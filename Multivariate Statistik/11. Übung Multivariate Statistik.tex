\documentclass{article}

\usepackage{amsmath,amssymb}
\usepackage{tikz}
\usepackage{pgfplots}
\usepackage{xcolor}
\usepackage[left=2.1cm,right=3.1cm,bottom=3cm,footskip=0.75cm,headsep=0.5cm]{geometry}
\usepackage{enumerate}
\usepackage{enumitem}
\usepackage{marvosym}
\usepackage{tabularx}
\usepackage[amsmath,thmmarks,standard]{ntheorem}
\usepackage{parskip}

\usepackage{listings}
\definecolor{lightlightgray}{rgb}{0.95,0.95,0.95}
\definecolor{lila}{rgb}{0.8,0,0.8}
\definecolor{mygray}{rgb}{0.5,0.5,0.5}
\definecolor{mygreen}{rgb}{0,0.8,0.26}
\lstdefinestyle{R} {language=R,morekeywords={confint,head}}
\lstset{language=R,
	basicstyle=\ttfamily,
	keywordstyle=\color{lila},
	commentstyle=\color{lightgray},
	stringstyle=\color{mygreen}\ttfamily,
	backgroundcolor=\color{white},
	showstringspaces=false,
	numbers=left,
	numbersep=10pt,
	numberstyle=\color{mygray}\ttfamily,
	identifierstyle=\color{blue},
	xleftmargin=.1\textwidth, 
	%xrightmargin=.1\textwidth,
	escapechar=§,
}

\usepackage[utf8]{inputenc}

\renewcommand*{\arraystretch}{1.4}
\newcommand{\E}{\mathbb{E}}

\newcolumntype{L}[1]{>{\raggedright\arraybackslash}p{#1}}
\newcolumntype{R}[1]{>{\raggedleft\arraybackslash}p{#1}}
\newcolumntype{C}[1]{>{\centering\let\newline\\\arraybackslash\hspace{0pt}}m{#1}}

\DeclareMathOperator{\tr}{tr}
\DeclareMathOperator{\Var}{Var}
\DeclareMathOperator{\Cov}{Cov}
\DeclareMathOperator{\Cor}{Cor}
\renewcommand{\E}{\mathbb{E}}

\newtheorem{thm}{Theorem}
\newtheorem{lem}{Lemma}

\title{\textbf{Multivariate Statistik, Übung 11}}
\author{\textsc{Henry Haustein}}
\date{}

\begin{document}
	\maketitle
	
	\section*{Aufgabe 1}
	Die Tabellen ergeben sich für die Studenten zu (der Durchschnittsrang ist $\bar{p}=\frac{21}{6}=\frac{7}{2}$)
	\begin{center}
		\begin{tabular}{cc|cc|c|c}
			&& \multicolumn{2}{c|}{B} & & \\
			&& 4 Stunden & 5 Stunden & $\bar{p}_A$ & $\bar{p}_A-\bar{p} = \beta_{Ai}$ \\
			\hline
			& ohne Hilfsmittel & 3 & 4 & $\frac{7}{2}$ & 0 \\
			A & Formelsammlung & 5 & 6 & $\frac{11}{2}$ & 2 \\
			& alles benutzen & 1 & 2 & $\frac{3}{2}$ & $-2$ \\
			\hline
			& $\bar{p}_B$ & $\frac{10}{3}$ & $\frac{11}{3}$ & \multicolumn{2}{c}{ } \\
			& $\bar{p}_B-\bar{p} = \beta_{Bj}$ & $-\frac{1}{6}$ & $\frac{1}{6}$ & \multicolumn{2}{c}{ }
		\end{tabular}
	\end{center}
	\begin{center}
		\begin{tabular}{cc|cc|c|c}
			&& \multicolumn{2}{c|}{B} & & \\
			&& 4 Stunden & 5 Stunden & $\bar{p}_A$ & $\bar{p}_A-\bar{p} = \beta_{Ai}$ \\
			\hline
			& ohne Hilfsmittel & 6 & 1 & $\frac{7}{2}$ & 0 \\
			A & Formelsammlung & 5 & 2 & $\frac{7}{2}$ & 0 \\
			& alles benutzen & 4 & 3 & $\frac{7}{2}$ & 0 \\
			\hline
			& $\bar{p}_B$ & 5 & 2 & \multicolumn{2}{c}{ } \\
			& $\bar{p}_B-\bar{p} = \beta_{Bj}$ & $\frac{3}{2}$ & $-\frac{3}{2}$ & \multicolumn{2}{c}{ }
		\end{tabular}
	\end{center}
	Dem ersten Studenten sind die erlaubten Hilfsmittel wichtiger, dem zweiten Studenten die Zeit.
	
	Aggregation der Nutzenwerte und Bestimmung der relativen Wichtigkeit
	\begin{align}
		\beta_{jm}^\ast &= \beta_{jm} - \beta_j^{min} \notag \\
		\tilde{\beta}_{jm} &= \frac{\beta_{jm}^\ast}{\max\left\lbrace \beta_{A1}^\ast,\beta_{A2}^\ast,\beta_{A3}^\ast\right\rbrace  + \max\left\lbrace \beta_{B1}^\ast + \beta_{B2}^\ast\right\rbrace } \notag \\
		\bar{\beta}_{jm} &= \frac{1}{2}\cdot\sum_{i=1}^{N_j} \beta_{ji} \notag
	\end{align}
	Es ergibt sich
	\begin{center}
		\begin{tabular}{c|cc|cc|c}
			& \multicolumn{2}{c|}{$\beta^\ast_{jm}$} & \multicolumn{2}{c|}{$\tilde{\beta}_{jm}$} & $\bar{\beta}_{jm}$ \\
			& Student 1 & Student 2 & Student 1 & Student 2 & \\
			\hline
			ohne Hilfsmittel & 2 & 0 & $\frac{6}{13}$ & 0 & $\frac{3}{13}$ \\
			Formelsammlung & 4 & 0 & $\frac{12}{13}$ & 0 & $\frac{6}{13}$ \\
			alles erlaubt & 0 & 0 & 0 & 0 & 0 \\
			\hline
			4 Stunden & 0 & 3 & 0 & 1 & $\frac{1}{2}$ \\
			5 Stunden & $\frac{1}{3}$ & 0 & $\frac{1}{13}$ & 0 & $\frac{1}{26}$
		\end{tabular}
	\end{center}
	Die Wichtigkeiten sind dann
	\begin{align}
		W_j &= \frac{\max_j\left\lbrace \bar{\beta}_{jm}\right\rbrace }{\max\left\lbrace \beta_{A1}^\ast,\beta_{A2}^\ast,\beta_{A3}^\ast\right\rbrace  + \max\left\lbrace \beta_{B1}^\ast + \beta_{B2}^\ast\right\rbrace } \notag \\
		W_A &= \frac{\frac{6}{13}}{\frac{6}{13} + \frac{1}{2}} = \frac{12}{25} = 48\% \notag \\
		W_B &= \frac{\frac{1}{2}}{\frac{6}{13} + \frac{1}{2}} = \frac{13}{25} = 52\% \notag
	\end{align}

	\section*{Aufgabe 2}
	Die traditionelle Conjoint-Analyse ist nur dann möglich, wenn 1 das am wenigsten bevorzugte Produkt ist. In der Aufgabe ist aber 1 das beste Produkt. Die Rangfolge muss also umgedreht werden (der Durchschnittsrang ist $\bar{p}=\frac{45}{9}=5$):
	\begin{center}
		\begin{tabular}{cc|ccc|c|c}
			&& \multicolumn{3}{c|}{B} & & \\
			&& 250g & 500g & 750g & $\bar{p}_A$ & $\bar{p}_A-\bar{p} = \beta_{Ai}$ \\
			\hline
			& Bio-Flakes & 7 & 9 & 1 & $\frac{17}{3}$ & $\frac{2}{3}$ \\
			A & Bio-Pads & 4 & 3 & 2 & 3 & $-2$ \\
			& Bio-Balls & 5 & 8 & 6 & $\frac{19}{3}$ & $\frac{4}{3}$ \\
			\hline
			& $\bar{p}_B$ & $\frac{16}{3}$ & $\frac{20}{3}$ & 3 & \multicolumn{2}{c}{ } \\
			& $\bar{p}_B-\bar{p} = \beta_{Bj}$ & $\frac{1}{3}$ & $\frac{5}{3}$ & $-2$ & \multicolumn{2}{c}{ }
		\end{tabular}
	\end{center}
	Berechnung der relativen Wichtigkeit
	\begin{center}
		\begin{tabular}{c|c|c|c|c}
			& $\beta^\ast_{jm}$ & $\tilde{\beta}_{jm}$ & $\bar{\beta}_{jm}$ & $W_j$ \\
			\hline
			250g & $\frac{8}{3}$ & $\frac{8}{21}$ & $\frac{8}{21}$ & \\
			500g & 0 & 0 & 0 & $\frac{10}{21}=47.6\%$ \\
			750g & $\frac{10}{3}$ & $\frac{10}{21}$ & $\frac{10}{21}$ & \\
			\hline
			Bio-Flakes & $\frac{7}{3}$ & $\frac{1}{3}$ & $\frac{1}{3}$ & \\
			Bio-Pads & $\frac{11}{3}$ & $\frac{11}{21}$ & $\frac{11}{21}$ & $\frac{11}{21}=52.4\%$ \\
			Bio-Balls & 0 & 0 & 0 &
		\end{tabular}
	\end{center}
	Die Art ist wichtiger als die Größe der Packung. Bei mehreren befragten Personen muss zwischendurch noch der Mittelwert über die normierten Teilnutzen berechnet werden; bei mehr Eigenschaften müssen noch mehr $\beta$'s berechnet werden.
	
	\section*{Aufgabe 3}
	\begin{enumerate}[label=(\alph*)]
		\item Jede der drei Eigenschaften hat 3 Ausprägungen, es gibt also $3\cdot 3\cdot 3=27$ Stimuli.
		\item Es gibt $\binom{27}{3}=2925$ Möglichkeiten aus 27 Stimuli 3 auszuwählen.
		\item Es gibt 27 ($3\cdot 3\cdot 3$) mögliche Autos für Alternative 1. Da es keine Überlappungen gegen soll, hat man für Alternative 2 nur noch 2 Kaufpreise, 2 PS-Zahlen und 3 Kraftstoffarten zur Auswahl, also insgesamt $2\cdot 2\cdot 2=8$ Autos. In der dritten Alternative bleibt dann nur noch ein Auto mit dem letzten Kaufpreis, der letzten PS-Zahl und der letzten Kraftstoffart. Man hat also $27\cdot 8\cdot 1 = 216$ mögliche Choice-Sets ohne Überlappung.
		\item Wenn wir jeder Person nur 10 Choice-Sets vorlegen dürfen, brauchen wir $\frac{216}{10}=21.6$ Personen. Also müssen wir mindestens 22 Personen befragen.
		\item Um den zentrierten Nutzen zu bestimmen, müssen wir von jedem Nutzenwert den mittleren Nutzen einer Eigenschaft abziehen (bei der \textit{none}-Option muss die Summe der mittleren Nutzen aller Eigenschaften abgezogen werden). Die mittleren Nutzen sind:
		\begin{align}
			\bar{\beta}_1 &= \frac{21}{3} = 7 \notag \\
			\bar{\beta}_2 &= \frac{9}{3} = 3 \notag \\
			\bar{\beta}_3 &= \frac{6}{3} = 2 \notag \\
			\bar{\beta}_1 + \bar{\beta}_2 + \bar{\beta}_3 &= 7+3+2=12 \notag
		\end{align}
		Es ergibt sich
		\begin{center}
			\begin{tabular}{c|c|c}
				& zentrierter Teilnutzen & Wichtigkeit \\
				\hline
				15 T\EUR & 2.8 & \\
				20 T\EUR & 4.2 & 54.4 \% \\
				25 T\EUR & $-7$ & \\
				\hline
				86 PS & $-3$ & \\
				104 PS & 1.2 & 23.3 \% \\
				132 PS & 1.8 & \\
				\hline
				Benzin & 1.4 & \\
				Diesel & 1.6 & 22.3 \% \\
				Erdgas & $-3$ & \\
				\hline
				\textit{none} & $-14.8$ &
			\end{tabular}
		\end{center}
		Der Kaufpreis ist am wichtigsten. Die none-Option hat den geringsten Teilnutzen, jedes Auto hat einen höheren Teilnutzen (selbst ein 25.000 \EUR\, teures Erdgas-Fahrzeug mit nur 86 PS) und würde eher gekauft werden als gar kein Auto zu haben.
	\end{enumerate}
	
\end{document}