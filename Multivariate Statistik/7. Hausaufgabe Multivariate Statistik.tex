\documentclass{article}

\usepackage{amsmath,amssymb}
\usepackage{tikz}
\usepackage{xcolor}
\usepackage[left=2.1cm,right=3.1cm,bottom=3cm,footskip=0.75cm,headsep=0.5cm]{geometry}
\usepackage{enumerate}
\usepackage{enumitem}
\usepackage{marvosym}
\usepackage{tabularx}
\usepackage{multirow}

\usepackage[utf8]{inputenc}

\renewcommand*{\arraystretch}{1.4}

\newcolumntype{L}[1]{>{\raggedright\arraybackslash}p{#1}}
\newcolumntype{R}[1]{>{\raggedleft\arraybackslash}p{#1}}
\newcolumntype{C}[1]{>{\centering\let\newline\\\arraybackslash\hspace{0pt}}m{#1}}

\DeclareMathOperator{\tr}{tr}
\DeclareMathOperator{\Var}{Var}
\DeclareMathOperator{\Cov}{Cov}
\DeclareMathOperator{\Cor}{Cor}
\newcommand{\E}{\mathbb{E}}

\title{\textbf{Multivariate Statistik, Hausaufgabe 7}}
\author{\textsc{Henry Haustein}}
\date{}

\begin{document}
	\maketitle
	
	\section*{Aufgabe 1}
	\begin{enumerate}[label=(\alph*)]
		\item euklidische Distanz:
		\begin{itemize}
			\item $d(x_1,x_1) = 0$
			\item $d(x_1,x_2) = \sqrt{4^2 + 8^2} = \sqrt{80}$
			\item $d(x_1,x_3) = \sqrt{1^2 + 1^1} = \sqrt{2}$
			\item $d(x_2,x_2) = 0$
			\item $d(x_2,x_3) = \sqrt{3^2 + 7^2} = \sqrt{58}$
			\item $d(x_3,x_3) = 0$
		\end{itemize}
		Manhatten Distanz:
		\begin{itemize}
			\item $d(x_1,x_1) = 0$
			\item $d(x_1,x_2) = 4 + 8 = 12$
			\item $d(x_1,x_3) = 1 + 1 = 2$
			\item $d(x_2,x_2) = 0$
			\item $d(x_2,x_3) = 3 + 7 = 10$
			\item $d(x_3,x_3) = 0$
		\end{itemize}
		Für die Mahalanobisdistanz brauchen wir die Inverse der Varianz-Kovarianz-Matrix, die sich einfach mit R berechnen lässt. Es ergibt sich
		\begin{align}
			S^{-1} = \begin{pmatrix}
				14.25 & -6.75 \\
				-6.75 & 3.25
			\end{pmatrix} \notag
		\end{align}
		und damit
		\begin{itemize}
			\item $d(x_1,x_1) = 0$
			\item $d(x_1,x_2) = \sqrt{4} = 2$
			\item $d(x_1,x_3) = \sqrt{4} = 2$
			\item $d(x_2,x_2) = 0$
			\item $d(x_2,x_3) = \sqrt{4} = 2$
			\item $d(x_3,x_3) = 0$
		\end{itemize}
		\item Man sieht eigentlich recht schnell, dass zuerst $x_1$ und $x_2$ sowohl bei euklidischer als auch bei Manhatten Distanz zu einem Cluster zusammengefasst werden, danach wird $x_3$ diesem Cluster hinzugefügt. Bei der Mahalanobisdistanz haben alle Punkte den selben Abstand zueinander, man könnte sie also direkt in einem Cluster zusammenfassen.
	\end{enumerate}
	
\end{document}