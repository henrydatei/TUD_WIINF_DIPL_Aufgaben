\documentclass{article}

\usepackage{amsmath,amssymb}
\usepackage{tikz}
\usepackage{xcolor}
\usepackage[left=2.1cm,right=3.1cm,bottom=3cm,footskip=0.75cm,headsep=0.5cm]{geometry}
\usepackage{enumerate}
\usepackage{enumitem}
\usepackage{marvosym}
\usepackage{tabularx}
\usepackage[amsmath,thmmarks,standard]{ntheorem}

\usepackage[utf8]{inputenc}

\renewcommand*{\arraystretch}{1.4}
\newcommand{\E}{\mathbb{E}}

\newcolumntype{L}[1]{>{\raggedright\arraybackslash}p{#1}}
\newcolumntype{R}[1]{>{\raggedleft\arraybackslash}p{#1}}
\newcolumntype{C}[1]{>{\centering\let\newline\\\arraybackslash\hspace{0pt}}m{#1}}

\DeclareMathOperator{\tr}{tr}
\DeclareMathOperator{\Var}{Var}
\DeclareMathOperator{\Cov}{Cov}

\newtheorem{thm}{Theorem}

\title{\textbf{Multivariate Statistik, Übung 1}}
\author{\textsc{Henry Haustein}}
\date{}

\begin{document}
	\maketitle
	
	\section*{Aufgabe 1}
	\begin{enumerate}[label=(\alph*)]
		\item Determinante
		\begin{align}
			\det(D) &= (1\cdot 3) - (1\cdot 2) = 1 \notag \\
			\det(E) &= (1\cdot 2) - (2\cdot 1) = 0 \notag \\
			\det(F) &= \text{nicht definiert, da $F$ nicht quadratisch} \notag
		\end{align}
		\item Spur
		\begin{align}
			\tr(D) &= 1 + 3 = 4 \notag \\
			\tr(E) &= 1 + 2 = 3 \notag \\
			\tr(F) &= \text{nicht definiert, da $F$ nicht quadratisch} \notag
		\end{align}
		\item Eigenwerte
		\begin{align}
			\det\begin{pmatrix}
				1-\lambda & 2 \\ 1 & 3-\lambda
			\end{pmatrix} &= 0 \notag \\
			(1-\lambda)(3-\lambda)-2 &= 0 \notag \\
			\lambda^2 - 4\lambda + 1 &= 0 \notag \\
			\lambda_1 &= 0.2679 \notag \\
			\lambda_2 &= 3.7321 \notag \\
			\det\begin{pmatrix}
				1-\lambda & 2 \\ 1 & 2-\lambda
			\end{pmatrix} &= 0 \notag \\
			(1-\lambda)(2-\lambda)-2 &= 0 \notag \\
			\lambda^2 - 3\lambda &= 0 \notag \\
			\lambda_1 &= 0 \notag \\
			\lambda_2 &= 3 \notag
		\end{align}
		$F$ ist nicht quadratisch, deswegen existieren keine Eigenwerte.
		\item Eigenvektoren
		\begin{align}
			(D-\lambda_1I)x &= 0 \notag \\
			\begin{pmatrix}
				1-0.2679 & 2 \\ 1 & 3-0.2679
			\end{pmatrix} \cdot \begin{pmatrix}
				x_1 \\ x_2
			\end{pmatrix} &= 0 \notag \\
			0.7321 x_1 + 2 x_2 &= 0 \notag \\
			1 x_1 + 2.7321 x_2 &= 0 \notag
		\end{align}
		Das Gleichungssystem hat unendlich viele Lösungen\footnote{Diese Lösungen spannen den sogenannten Eigenraum auf.}. Wähle z.B. $x_1=1$, dann folgt $x_2=-0.36605$. Die Norm/der Betrag dieses Vektors ist $\sqrt{1^2+(-0.36605)^2}=1.0649$, also ist der erste Eigenvektor
		\begin{align}
			x= \frac{1}{1.0649}\begin{pmatrix}
				1 \\ -0.36605
			\end{pmatrix} \notag
		\end{align}
		Für den zweiten Eigenvektor ergibt sich analog
		\begin{align}
			(D-\lambda_2I)y &= 0 \notag \\
			\begin{pmatrix}
				1-3.7321 & 2 \\ 1 & 3-3.7321
			\end{pmatrix} \cdot \begin{pmatrix}
				y_1 \\ y_2
			\end{pmatrix} &= 0 \notag \\
			-2.7321 y_1 + 2 y_2 &= 0 \notag \\
			1 y_1 - 0.7321 y_2 &= 0 \notag
		\end{align}
		Wähle wieder $y_1=1$, dann folgt $y_2=1.36605$, Normierung $\sqrt{1^2+1.36605^2}=1.6930$, also ist der zweite Eigenvektor
		\begin{align}
			y= \frac{1}{1.6930}\begin{pmatrix}
				1 \\ 1.36605
			\end{pmatrix} \notag
		\end{align}
		\item keine der Matrizen ist symmetrisch, da gilt $D\neq D'$, $E\neq E'$ und $F\neq F'$.
		\item Es kann nur $D$ invertiert werden. Bei $E$ ist die Determinante 0 und $F$ ist nicht quadratisch.
		\item Varianz $\Var(e_{\cdot1}) = \frac{1}{1}[(1-1)^2 + (1-1)^2] = 0$ \\
		Varianz $\Var(e_{\cdot2}) = \frac{1}{1}[(2-2)^2 + (2-2)^2] = 0$ \\
		Kovarianz $\Cov(e_{\cdot1},e_{\cdot 2}) = \frac{1}{1}[(1-1)(2-2) + (1-1)(2-2)] = 0$
		\begin{align}
			\Cov(E) = \begin{pmatrix}
				0 & 0 \\ 0 & 0
			\end{pmatrix} \notag
		\end{align}
	\end{enumerate}
	
	\section*{Aufgabe 2}
	\begin{align}
		\E((X-\E(X))(Y-\E(Y))) &= \Cov(X,Y) \notag \\
		&= \Cov(X,Y) + \E(X)\E(Y) - \E(X)\E(Y) \notag \\
		&= \E(XY) - \E(X)\E(Y) \notag
	\end{align}
	
	\section*{Aufgabe 3}
	\begin{thm}[Spektralsatz]
		Für einen endlichdimensionalen unitären $\mathbb{K}$-Vektorraum existiert genau dann eine Orthonormalbasis von Eigenvektoren eines Endomorphismus, wenn dieser normal ist und alle Eigenwerte zu $\mathbb{K}$ gehören. In Matrixsprechweise bedeutet dies, dass eine Matrix genau dann unitär diagonalisierbar ist, wenn sie normal ist und nur Eigenwerte aus $\mathbb{K}$ hat.
	\end{thm}
	Wir wählen $\mathbb{K}=\mathbb{R}$ und da $AA' = A'A$ gilt, ist $A$ normal. Dass die Eigenwerte zu $\mathbb{R}$ gehören, kann man wie folgt beweisen: Für einen Eigenwert $\lambda$ und einen Eigenvektor $v$ gilt $Av=\lambda v$ und
	\begin{align}
		\lambda\langle v,v\rangle = \langle \lambda v,v\rangle = \langle Av,v\rangle = \langle v,A'v\rangle = \langle v,Av\rangle = \langle v,\lambda v\rangle = \bar{\lambda}\langle v,v\rangle \notag
	\end{align}
	Also $\lambda = \bar{\lambda}$ und damit ist $\lambda\in\mathbb{R}$. Der Spektralsatz garantiert uns nun, dass es eine Orthonomalbasis aus Eigenvektoren von $A$ gibt, damit ist $A$ diagonalisierbar, dass heißt es gibt eine Matrix $T$, so dass gilt
	\begin{align}
		A = T^{-1}\cdot \text{diag}(\lambda_1,...,\lambda_n) \cdot T \notag
	\end{align}
	Weiterhin gilt:
	\begin{align}
		\det(A) &= \det(T^{-1}\cdot \text{diag}(\lambda_1,...,\lambda_n) \cdot T) \notag \\
		&= \det(T^{-1}) \cdot \det(\text{diag}(\lambda_1,...,\lambda_n)) \cdot \det(T) \notag \\
		&= \det(\text{diag}(\lambda_1,...,\lambda_n)) \notag \\
		&= \prod_{i=1}^{n} \lambda_i \notag
	\end{align}
	
	\section*{Aufgabe 4}
	\begin{align}
		\lambda\text{ ist ein EW von } A &\Leftrightarrow Av = \lambda v \notag \\
		&\Leftrightarrow A^{-1}Av = A^{-1}\lambda v \notag \\
		&\Leftrightarrow v = A^{-1}\lambda v \notag \\
		&\Leftrightarrow v\lambda^{-1} = A^{-1}\lambda v\lambda^{-1} \notag \\
		&\Leftrightarrow \lambda^{-1}v = A^{-1}v \notag \\
		&\Leftrightarrow \lambda^{-1}\text{ ist ein EW von } A^{-1} \notag
	\end{align}
	
\end{document}