\documentclass{article}

\usepackage{amsmath,amssymb}
\usepackage{tikz}
\usepackage{xcolor}
\usepackage[left=2.1cm,right=3.1cm,bottom=3cm,footskip=0.75cm,headsep=0.5cm]{geometry}
\usepackage{enumerate}
\usepackage{enumitem}
\usepackage{marvosym}
\usepackage{tabularx}
\usepackage{multirow}

\usepackage[utf8]{inputenc}

\renewcommand*{\arraystretch}{1.4}

\newcolumntype{L}[1]{>{\raggedright\arraybackslash}p{#1}}
\newcolumntype{R}[1]{>{\raggedleft\arraybackslash}p{#1}}
\newcolumntype{C}[1]{>{\centering\let\newline\\\arraybackslash\hspace{0pt}}m{#1}}

\DeclareMathOperator{\tr}{tr}
\DeclareMathOperator{\Var}{Var}
\DeclareMathOperator{\Cov}{Cov}
\DeclareMathOperator{\Cor}{Cor}
\newcommand{\E}{\mathbb{E}}

\title{\textbf{Multivariate Statistik, Hausaufgabe 8}}
\author{\textsc{Henry Haustein}}
\date{}

\begin{document}
	\maketitle
	
	\section*{Aufgabe 1}
	\begin{enumerate}[label=(\alph*)]
		\item Die Klasse $R_1$ ist wie folgt definiert und vereinfacht sich in dieser Aufgabe zu
		\begin{align}
			R_1 &= \left\lbrace x\in \mathbb{R}^k\,\bigg\vert\, \frac{f_1(x)}{f_2(x)} \ge \frac{c(1\mid 2)\mathbb{P}(Y=2)}{c(2\mid 1)\mathbb{P}(Y=1)}\right\rbrace \notag \\
			&= \left\lbrace x\in \mathbb{R}^k\,\bigg\vert\, \frac{f_1(x)}{f_2(x)} \ge 1\right\rbrace \notag
		\end{align}
		wobei
		\begin{align}
			\frac{f_1(x)}{f_2(x)} = \exp\left(-\frac{1}{2}x'(\Sigma_1^{-1}-\Sigma_2^{-1})x + (\mu_1'\Sigma_1^{-1} - \mu_2'\Sigma_2^{-1})x\right) \cdot\frac{\sqrt{\det(\Sigma_2)}}{\sqrt{\det(\Sigma_1)}}\exp\left(-\frac{1}{2}(\mu_1'\Sigma_1^{-1}\mu_1 - \mu_2'\Sigma_2^{-1}\mu_2)\right) \notag
		\end{align}
		Die dazu benötigten inversen Matrizen lauten
		\begin{align}
			\Sigma_1^{-1} = \begin{pmatrix}
				\frac{3}{2} & -\frac{1}{2} \\ -\frac{1}{2} & \frac{1}{2}
			\end{pmatrix} \qquad \Sigma_1^{-1} = \begin{pmatrix}
				\frac{1}{2} & -\frac{1}{2} \\ -\frac{1}{2} & 1
			\end{pmatrix} \notag
		\end{align}
		Damit ist
		\begin{align}
			\frac{f_1(x)}{f_2(x)} = \exp\left(-\frac{1}{2}x'\begin{pmatrix}1 & 0 \\ 0 & -\frac{1}{2}\end{pmatrix}x + \begin{pmatrix}\frac{1}{2} & -\frac{5}{2}\end{pmatrix}x\right) \cdot\sqrt{2}\cdot\exp\left(\frac{21}{4}\right) \notag
		\end{align}
		\item Einsetzen von $x=(2,2)$ liefert $\frac{f_1(x)}{f_2(x)}=1.815886$, damit wird $x$ der Klasse 1 zugeordnet.
		\item Durch Einbeziehen der Fehlklassifikationskosten ändert sich $R_1$ auf
		\begin{align}
			R_1 &= \left\lbrace x\in \mathbb{R}^k\,\bigg\vert\, \frac{f_1(x)}{f_2(x)} \ge \frac{c(1\mid 2)\mathbb{P}(Y=2)}{c(2\mid 1)\mathbb{P}(Y=1)}\right\rbrace \notag \\
			&= \left\lbrace x\in \mathbb{R}^k\,\bigg\vert\, \frac{f_1(x)}{f_2(x)} \ge \frac{c(1\mid 2)}{c(2\mid 1)}\right\rbrace \notag \\
			&= \left\lbrace x\in \mathbb{R}^k\,\bigg\vert\, \frac{f_1(x)}{f_2(x)} \ge \frac{350}{100}\right\rbrace \notag \\
			&= \left\lbrace x\in \mathbb{R}^k\,\bigg\vert\, \frac{f_1(x)}{f_2(x)} \ge 3.5\right\rbrace \notag
		\end{align}
	\end{enumerate}
	
\end{document}