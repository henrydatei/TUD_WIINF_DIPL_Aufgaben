\documentclass{article}

\usepackage{amsmath,amssymb}
\usepackage{tikz}
\usepackage{xcolor}
\usepackage[left=2.1cm,right=3.1cm,bottom=3cm,footskip=0.75cm,headsep=0.5cm]{geometry}
\usepackage{enumerate}
\usepackage{enumitem}
\usepackage{marvosym}
\usepackage{tabularx}

\usepackage[utf8]{inputenc}

\renewcommand*{\arraystretch}{1.4}

\newcolumntype{L}[1]{>{\raggedright\arraybackslash}p{#1}}
\newcolumntype{R}[1]{>{\raggedleft\arraybackslash}p{#1}}
\newcolumntype{C}[1]{>{\centering\let\newline\\\arraybackslash\hspace{0pt}}m{#1}}

\DeclareMathOperator{\tr}{tr}
\DeclareMathOperator{\Var}{Var}
\DeclareMathOperator{\Cov}{Cov}
\newcommand{\E}{\mathbb{E}}

\title{\textbf{Multivariate Statistik, Hausaufgabe 2}}
\author{\textsc{Henry Haustein}}
\date{}

\begin{document}
	\maketitle
	
	\section*{Aufgabe 1}
	\begin{enumerate}[label=(\alph*)]
		\item Erwartungswert und Varianz
		\begin{align}
			\E(X) &= 1\cdot 0,1 + 1,3 \cdot 0,2 + 1,7 \cdot 0,4 + 2 \cdot 0,2 + 2,3 \cdot 0,1 = 1,67 \notag \\
			\Var(X) &= \E(X^2) - 1,67^2 = (1^2\cdot 0,1 + 1,3^2 \cdot 0,2 + 1,7^2 \cdot 0,4 + 2^2 \cdot 0,2 + 2,3^2 \cdot 0,1) - 1,67^2 = 0,1341 \notag
		\end{align}
		\item Erwartungswert und Varianz
		\begin{align}
			\E(X) &= 1\cdot 0,25 + 1,7 \cdot 0,25 + 1,3 \cdot 0,25 + 2 \cdot 0,25 = 1,5 \notag \\
			\Var(X) &= \E(X^2) - 1,5^2 = (1^2\cdot 0,25 + 1,7^2 \cdot 0,25 + 1,3^2 \cdot 0,25 + 2^2 \cdot 0,25) - 2,25 = 0,145 \notag
		\end{align}
	\end{enumerate}
	
	
\end{document}