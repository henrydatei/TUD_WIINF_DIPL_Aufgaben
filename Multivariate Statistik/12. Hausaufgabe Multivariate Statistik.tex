\documentclass{article}

\usepackage{amsmath,amssymb}
\usepackage{tikz}
\usepackage{xcolor}
\usepackage[left=2.1cm,right=3.1cm,bottom=3cm,footskip=0.75cm,headsep=0.5cm]{geometry}
\usepackage{enumerate}
\usepackage{enumitem}
\usepackage{marvosym}
\usepackage{tabularx}
\usepackage{multirow}

\usepackage[utf8]{inputenc}

\renewcommand*{\arraystretch}{1.4}

\newcolumntype{L}[1]{>{\raggedright\arraybackslash}p{#1}}
\newcolumntype{R}[1]{>{\raggedleft\arraybackslash}p{#1}}
\newcolumntype{C}[1]{>{\centering\let\newline\\\arraybackslash\hspace{0pt}}m{#1}}

\DeclareMathOperator{\tr}{tr}
\DeclareMathOperator{\Var}{Var}
\DeclareMathOperator{\Cov}{Cov}
\DeclareMathOperator{\Cor}{Cor}
\newcommand{\E}{\mathbb{E}}

\title{\textbf{Multivariate Statistik, Hausaufgabe 12}}
\author{\textsc{Henry Haustein}}
\date{}

\begin{document}
	\maketitle
	
	\section*{Aufgabe 1}
	Die richtige Zuordnung ist
	\begin{enumerate}[label=(\alph*)]
		\item Korrelation: Zusammenhänge zwischen zwei oder mehr Variablen quantifizieren
		\item Faktorenanalyse: Dimensionsreduktion durch Einführen weniger Faktoren
		\item multivariate Kovarianzanalyse: Erwartungswertunterschiede zwischen mindestens zwei Erwartungswertvektoren aufdecken, dabei den Einfluss von anderen Variablen bereinigen
		\item Korrespondenzanalyse: Abhängigkeit kategorieller Variablen darstellen
		\item Diskriminanzanalyse: Objekte unbekannter Klassenzugehörigkeit in vorhandene Klassen einsortieren
		\item Hauptkomponentenanalyse: Dimensionsreduktion durch Einführen weniger künstlicher Variablen
		\item Clusteranalyse: Objekte in homogene Gruppen unterteilen
		\item Conjoint-Analyse: Präferenzen bestimmen
		\item multivariate Varianzanalyse: Erwartungswertunterschiede zwischen mindestens zwei Erwartungswertvektoren aufdecken
	\end{enumerate}
\end{document}