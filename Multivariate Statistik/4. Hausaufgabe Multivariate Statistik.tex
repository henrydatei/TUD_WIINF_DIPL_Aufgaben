\documentclass{article}

\usepackage{amsmath,amssymb}
\usepackage{tikz}
\usepackage{xcolor}
\usepackage[left=2.1cm,right=3.1cm,bottom=3cm,footskip=0.75cm,headsep=0.5cm]{geometry}
\usepackage{enumerate}
\usepackage{enumitem}
\usepackage{marvosym}
\usepackage{tabularx}
\usepackage{multirow}

\usepackage[utf8]{inputenc}

\renewcommand*{\arraystretch}{1.4}

\newcolumntype{L}[1]{>{\raggedright\arraybackslash}p{#1}}
\newcolumntype{R}[1]{>{\raggedleft\arraybackslash}p{#1}}
\newcolumntype{C}[1]{>{\centering\let\newline\\\arraybackslash\hspace{0pt}}m{#1}}

\DeclareMathOperator{\tr}{tr}
\DeclareMathOperator{\Var}{Var}
\DeclareMathOperator{\Cov}{Cov}
\DeclareMathOperator{\Cor}{Cor}
\newcommand{\E}{\mathbb{E}}

\title{\textbf{Multivariate Statistik, Hausaufgabe 4}}
\author{\textsc{Henry Haustein}}
\date{}

\begin{document}
	\maketitle
	
	\section*{Aufgabe 1}
	Für die Teststatistik brauchen wir die Determinante von $R$:
	\begin{align}
		\det(R) &= 1+0.02a + 0.02a - a^2 - 0.04 - 0.01 \notag \\
		&= -a^2 + 0.04a + 0.95 \notag
	\end{align}
	Setzen wir das in die Teststatistik ein:
	\begin{align}
		W &= -c \cdot \ln(\det(R)) \notag \\
		&= -\left(200-3-\frac{2\cdot 3+5}{6}\right)\cdot \ln(-a^2 + 0.04a + 0.95) \notag
	\end{align}
	Dies muss kleiner als der kritische Wert $\chi_{f;1-\alpha}^2=\chi_{3,0.99}=11.3449$. Wir lösen also die Ungleichung
	\begin{align}
		11.3449 &> -\left(200-3-\frac{2\cdot 3+5}{6}\right)\cdot \ln(-a^2 + 0.04a + 0.95) \notag \\
		a &< 0.1029 \notag
	\end{align}
	
\end{document}