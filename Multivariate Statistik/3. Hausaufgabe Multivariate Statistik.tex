\documentclass{article}

\usepackage{amsmath,amssymb}
\usepackage{tikz}
\usepackage{xcolor}
\usepackage[left=2.1cm,right=3.1cm,bottom=3cm,footskip=0.75cm,headsep=0.5cm]{geometry}
\usepackage{enumerate}
\usepackage{enumitem}
\usepackage{marvosym}
\usepackage{tabularx}
\usepackage{multirow}

\usepackage[utf8]{inputenc}

\renewcommand*{\arraystretch}{1.4}

\newcolumntype{L}[1]{>{\raggedright\arraybackslash}p{#1}}
\newcolumntype{R}[1]{>{\raggedleft\arraybackslash}p{#1}}
\newcolumntype{C}[1]{>{\centering\let\newline\\\arraybackslash\hspace{0pt}}m{#1}}

\DeclareMathOperator{\tr}{tr}
\DeclareMathOperator{\Var}{Var}
\DeclareMathOperator{\Cov}{Cov}
\DeclareMathOperator{\Cor}{Cor}
\newcommand{\E}{\mathbb{E}}

\title{\textbf{Multivariate Statistik, Hausaufgabe 2}}
\author{\textsc{Henry Haustein}}
\date{}

\begin{document}
	\maketitle
	
	\section*{Aufgabe 1}
	Es gilt
	\begin{center}
		\begin{tabular}{c|c|c|c}
			& $x_1$ & $x_2$ & $x_3$ \\
			\hline
			Mittelwert & 37,6 & 4,8 & 4990 \\
			\hline
			Stichprobenvarianz & 175,3 & 12,7 & 105530000 \\
			\hline
			Kovarianz & \multicolumn{2}{c|}{35,4} & \\
			\cline{2-4}
			Kovarianz & & \multicolumn{2}{c}{10347,5} \\
			\hline
			Korrelation & \multicolumn{2}{c|}{0,7503} & \\
			\cline{2-4}
			Korrelation & & \multicolumn{2}{c}{0,8938}
		\end{tabular}
	\end{center}
	$\Cov(x_1,x_3)=25670$ und $\Cor(x_1,x_3)=0,5968$.	
	
\end{document}