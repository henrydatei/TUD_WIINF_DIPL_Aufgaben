\documentclass{article}

\usepackage{amsmath,amssymb}
\usepackage{tikz}
\usepackage{xcolor}
\usepackage[left=2.1cm,right=3.1cm,bottom=3cm,footskip=0.75cm,headsep=0.5cm]{geometry}
\usepackage{enumerate}
\usepackage{enumitem}
\usepackage{marvosym}
\usepackage{tabularx}
\usepackage{multirow}

\usepackage[utf8]{inputenc}

\renewcommand*{\arraystretch}{1.4}

\newcolumntype{L}[1]{>{\raggedright\arraybackslash}p{#1}}
\newcolumntype{R}[1]{>{\raggedleft\arraybackslash}p{#1}}
\newcolumntype{C}[1]{>{\centering\let\newline\\\arraybackslash\hspace{0pt}}m{#1}}

\DeclareMathOperator{\tr}{tr}
\DeclareMathOperator{\Var}{Var}
\DeclareMathOperator{\Cov}{Cov}
\DeclareMathOperator{\Cor}{Cor}
\newcommand{\E}{\mathbb{E}}

\title{\textbf{Multivariate Statistik, Hausaufgabe 11}}
\author{\textsc{Henry Haustein}}
\date{}

\begin{document}
	\maketitle
	
	\section*{Aufgabe 1}
	\begin{enumerate}[label=(\alph*)]
		\item Eine Varianz-Kovarianz-Matrix enthält Varianzen, eine Korrelationsmatrix enthält Korrelationen.
		\item Auf der Hauptdiagonalen einer Korrelationsmatrix steht die Korrelation einer Variable mit sich selber. Diese Korrelation ist logischerweise 1.
		\item Der Eintrag der $i$-ten Zeile und $j$-ten Spalte der Korrelationsmatrix gibt die Korrelation der Variable $X_i$ mit $X_j$ an.
		\item Varianzen kann man durch Wahl von Skalen der Variablen (z.B. von 1 bis 1 Million gegenüber einer Skala von 0 bis 1) beliebig verändern. Bei einer Hauptkomponentenanalyse der Varianz-Kovarianz-Matrix kommt dann heraus, dass die Variable mit der großen Skala viel wichtiger ist als die Variable mit der kleinen Skala. Bei der Korrelationsmatrix werden solche Effekte herausgerechnet.
		\item Ja, da sowohl $\Cov(X_i,X_j)=\Cov(X_j,X_i)$ als auch $\Cor(X_i,X_j)=\Cor(X_j,X_i)$ gilt, folgt daraus, dass die Matrizen symmetrisch und quadratisch sind.
	\end{enumerate}
\end{document}