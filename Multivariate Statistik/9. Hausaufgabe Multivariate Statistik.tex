\documentclass{article}

\usepackage{amsmath,amssymb}
\usepackage{tikz}
\usepackage{xcolor}
\usepackage[left=2.1cm,right=3.1cm,bottom=3cm,footskip=0.75cm,headsep=0.5cm]{geometry}
\usepackage{enumerate}
\usepackage{enumitem}
\usepackage{marvosym}
\usepackage{tabularx}
\usepackage{multirow}

\usepackage[utf8]{inputenc}

\renewcommand*{\arraystretch}{1.4}

\newcolumntype{L}[1]{>{\raggedright\arraybackslash}p{#1}}
\newcolumntype{R}[1]{>{\raggedleft\arraybackslash}p{#1}}
\newcolumntype{C}[1]{>{\centering\let\newline\\\arraybackslash\hspace{0pt}}m{#1}}

\DeclareMathOperator{\tr}{tr}
\DeclareMathOperator{\Var}{Var}
\DeclareMathOperator{\Cov}{Cov}
\DeclareMathOperator{\Cor}{Cor}
\newcommand{\E}{\mathbb{E}}

\title{\textbf{Multivariate Statistik, Hausaufgabe 9}}
\author{\textsc{Henry Haustein}}
\date{}

\begin{document}
	\maketitle
	
	\section*{Aufgabe 1}
	\begin{enumerate}[label=(\alph*)]
		\item Die Eigenwerte sind $\lambda_1=\frac{11}{10}$ und $\lambda_2=\frac{9}{10}$. Die zugehörigen normierten Eigenvektoren sind $v_1=\sqrt{2}\cdot (1,1)$ und $v_2=\sqrt{2}\cdot (-1,1)$. Damit wird
		\begin{align}
			\frac{\lambda_1}{\lambda_1 + \lambda_2} = \frac{\frac{11}{10}}{\frac{11}{10} + \frac{9}{10}} = 55\% \notag
		\end{align}
		der Streuung erklärt.
		\item Die erste Hauptkomponente ist $y_1=v_1\cdot x = \sqrt{2}\cdot 2 + \sqrt{2}\cdot 10 = \sqrt{2}\cdot 12 = 16.9706$.
	\end{enumerate}
	
\end{document}