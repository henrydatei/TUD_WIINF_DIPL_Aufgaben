\documentclass{article}

\usepackage{amsmath,amssymb}
\usepackage{tikz}
\usepackage{pgfplots}
\usepackage{xcolor}
\usepackage[left=2.1cm,right=3.1cm,bottom=3cm,footskip=0.75cm,headsep=0.5cm]{geometry}
\usepackage{enumerate}
\usepackage{enumitem}
\usepackage{marvosym}
\usepackage{tabularx}
\usepackage[amsmath,thmmarks,standard]{ntheorem}
\usepackage{parskip}

\usepackage{listings}
\definecolor{lightlightgray}{rgb}{0.95,0.95,0.95}
\definecolor{lila}{rgb}{0.8,0,0.8}
\definecolor{mygray}{rgb}{0.5,0.5,0.5}
\definecolor{mygreen}{rgb}{0,0.8,0.26}
\lstdefinestyle{R} {language=R,morekeywords={confint,head}}
\lstset{language=R,
	basicstyle=\ttfamily,
	keywordstyle=\color{lila},
	commentstyle=\color{lightgray},
	stringstyle=\color{mygreen}\ttfamily,
	backgroundcolor=\color{white},
	showstringspaces=false,
	numbers=left,
	numbersep=10pt,
	numberstyle=\color{mygray}\ttfamily,
	identifierstyle=\color{blue},
	xleftmargin=.1\textwidth, 
	%xrightmargin=.1\textwidth,
	escapechar=§,
}

\usepackage[utf8]{inputenc}

\renewcommand*{\arraystretch}{1.4}
\newcommand{\E}{\mathbb{E}}

\newcolumntype{L}[1]{>{\raggedright\arraybackslash}p{#1}}
\newcolumntype{R}[1]{>{\raggedleft\arraybackslash}p{#1}}
\newcolumntype{C}[1]{>{\centering\let\newline\\\arraybackslash\hspace{0pt}}m{#1}}

\DeclareMathOperator{\tr}{tr}
\DeclareMathOperator{\Var}{Var}
\DeclareMathOperator{\Cov}{Cov}
\DeclareMathOperator{\Cor}{Cor}
\renewcommand{\E}{\mathbb{E}}

\newtheorem{thm}{Theorem}
\newtheorem{lem}{Lemma}

\title{\textbf{Multivariate Statistik, Übung 10}}
\author{\textsc{Henry Haustein}}
\date{}

\begin{document}
	\maketitle
	
	\section*{Aufgabe 1}
	\begin{enumerate}[label=(\alph*)]
		\item Die Kommunalitäten sind die durch die Faktoren erklärte Varianz einer Variablen.
		\item Mit dem Fundamentaltheorem der FA stellen die Faktoren ein statisches Modell der Variablen dar.
		\begin{align}
			R = \Cov(\mathcal{Z}) &= \E(\mathcal{Z}\mathcal{Z}') \notag \\
			&= \E((L\mathcal{F}+\mathcal{U})(L\mathcal{F}+\mathcal{U})') \notag \\
			&= \E(L\mathcal{F}\mathcal{F}'L) + \E(L\mathcal{F}\mathcal{U}') + \E(\mathcal{U}\mathcal{F}'L') + \E(\mathcal{U}\mathcal{U}') \notag \\
			\label{fundamentaltheoremFA}
			&= L\E(\mathcal{F}\mathcal{F}')L' + L\E(\mathcal{F}\mathcal{U}') + \E(\mathcal{U}\mathcal{F}')L' + \E(\mathcal{U}\mathcal{U}') \notag \\
			&= LL' + \Cov(\mathcal{U}) \notag
		\end{align}
		\item Bei einer Einfachstruktur der Faktoren gilt:
		\begin{itemize}
			\item Jede Zeile von $L$ soll mindestens eine Null enthalten (Variable wird durch höchstens $k - 1$ Faktoren beschrieben)
			\item Jede Spalte enthält mindestens $q$ Nullladungen (Faktor beschreibt höchstens $k - q$ Variablen)
			\item Für jedes Spaltenpaar in $L$ sollten nur wenige Variablen in beiden Spalten hohe Ladungen haben. (Im Beispiel ist dies Variable $X_3$)
		\end{itemize}
		\item Es wird metrisches Skalenniveau benötigt, da wir auch die Differenzen zwischen Bewertungen interpretieren müssen.
	\end{enumerate}

	\section*{Aufgabe 2}
	Ich werde hier immer den auf 2 Nachkommastellen gerundeten Wert angegeben, aber zusätzlich auch den exakten Wert, da sich dieser zum Teil deutlich von dem gerundeten Wert unterscheidet.
	
	Zuerst testen wir mittels Bartletts-Test, ob wir überhaupt eine Faktorenanalyse machen sollten. Die Teststatistik ergibt sich zu
	\begin{align}
		\chi^2_{err} &= -\left(n-\frac{2\cdot k+11}{6}\right)\cdot\ln\left(\prod_{i=1}^{4} \lambda_i\right) \notag \\
		&= -\left(6-\frac{2\cdot 4+11}{6}\right) \cdot\ln(2.91\cdot 0.82\cdot 0.27 \cdot 0.01 \quad\textcolor{red}{2.906162898\cdot 0.815503170 \cdot 0.271634856 \cdot 0.006699077}) \notag \\
		&= -\left(6-\frac{2\cdot 4+11}{6}\right) \cdot\ln(0.01 \quad\textcolor{red}{0.004312668}) \notag \\
		&= 13.05 \quad\textcolor{red}{15.4309} \notag
	\end{align}
	Der kritische Wert ist $\chi^2_{1-\alpha;\frac{k(k-1)}{2}}=\chi^2_{0.95;6}=12.59$ \textcolor{red}{12.5916}, also wird die Nullhypothese abgelehnt und eine Faktorenanalyse ist sinnvoll.
	
	Wir schätzen nun die Kommunalitäten mittels
	\begin{align}
		h_j^2 = 1-\frac{1}{r^{jj}} \notag
	\end{align}
	Es ergibt sich
	\begin{center}
		\begin{tabular}{c|cc}
			& \textbf{Kommunalitäten ($h_j^2$)} & \textbf{Einzelrestvarianzen ($u_j^2$)} \\
			\hline
			Preis & 0.99 \textcolor{red}{0.9893378} & 0.01 \textcolor{red}{0.01066222} \\
			Nützlichkeit & 0.25 \textcolor{red}{0.25} & 0.75 \textcolor{red}{0.75} \\
			Aussehen & 0.97 \textcolor{red}{0.9706669} & 0.03 \textcolor{red}{0.02933313} \\
			Haltbarkeit & 0.96 \textcolor{red}{0.9604938} & 0.04 \textcolor{red}{0.03950617}
		\end{tabular}
	\end{center}
	
	Da genau ein Eigenwert der reduzierten Korrelationsmatrix größer als 1 ist, wählen wir genau einen Faktor. Der Eigenwert ist $\lambda_1 = 2.82$ \textcolor{red}{2.820235871} und der zugehörige Eigenvektor ist $v_1=(0.59,-0.25,0.56,0.52)$ \textcolor{red}{$(0.5943829, -0.2517280, 0.5558792, 0.5237750)$}. Damit ergibt sich die Ladungsmatrix zu $L=\sqrt{\lambda_1}\cdot v_1$:
	\begin{align}
		L = \begin{pmatrix}
			0.99 \\
			-0.42 \\
			0.94 \\
			0.87
		\end{pmatrix} \quad\textcolor{red}{\begin{pmatrix}
			0.9981804 \\
			-0.4227409 \\
			0.9335190 \\
			0.8796046
		\end{pmatrix}} \notag
	\end{align}
	Unser Faktor lädt also auf die Variablen Preis, Aussehen und Haltbarkeit hoch.¸
\end{document}