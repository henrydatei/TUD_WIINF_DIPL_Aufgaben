\documentclass{article}

\usepackage{amsmath,amssymb}
\usepackage{tikz}
\usepackage{pgfplots}
\usepackage{xcolor}
\usepackage[left=2.1cm,right=3.1cm,bottom=3cm,footskip=0.75cm,headsep=0.5cm]{geometry}
\usepackage{enumerate}
\usepackage{enumitem}
\usepackage{marvosym}
\usepackage{tabularx}
\usepackage{parskip}

\usepackage{listings}
\definecolor{lightlightgray}{rgb}{0.95,0.95,0.95}
\definecolor{lila}{rgb}{0.8,0,0.8}
\definecolor{mygray}{rgb}{0.5,0.5,0.5}
\definecolor{mygreen}{rgb}{0,0.8,0.26}
%\lstdefinestyle{java} {language=java}
\lstset{language=R,
	basicstyle=\ttfamily,
	keywordstyle=\color{lila},
	commentstyle=\color{lightgray},
	stringstyle=\color{mygreen}\ttfamily,
	backgroundcolor=\color{white},
	showstringspaces=false,
	numbers=left,
	numbersep=10pt,
	numberstyle=\color{mygray}\ttfamily,
	identifierstyle=\color{blue},
	xleftmargin=.1\textwidth, 
	%xrightmargin=.1\textwidth,
	escapechar=§,
	%literate={\t}{{\ }}1
	breaklines=true,
	postbreak=\mbox{\space}
}

\usepackage[colorlinks = true, linkcolor = blue, urlcolor  = blue, citecolor = blue, anchorcolor = blue]{hyperref}
\usepackage{nameref}
\usepackage[utf8]{inputenc}

\renewcommand*{\arraystretch}{1.4}

\newcolumntype{L}[1]{>{\raggedright\arraybackslash}p{#1}}
\newcolumntype{R}[1]{>{\raggedleft\arraybackslash}p{#1}}
\newcolumntype{C}[1]{>{\centering\let\newline\\\arraybackslash\hspace{0pt}}m{#1}}

\newcommand{\E}{\mathbb{E}}
\DeclareMathOperator{\rk}{rk}
\DeclareMathOperator{\Var}{Var}
\DeclareMathOperator{\Cov}{Cov}

\title{\textbf{Zusammenfassung Spanisch A1}}
\author{\textsc{Henry Haustein}}
\date{}

\begin{document}
	\maketitle
	
	\section{Grundvokabeln}
	
	\subsection{Los números}
	
	\begin{center}
		\begin{tabular}{llllll}
			uno & once & veintiuno & treinta y uno & & ciento uno  \\
			dos & doce & veintidós & treinta y dos & & doscientos \\
			tres & trece & veintitrés & treinta y tres & & trecientos \\
			cuatro & catorce & veinticuatro & treinta y cuadro & & cuatrocientos\\
			cinco & quince & veinticinco & treinta y cinco & cincuenta & quinientos \\
			seis & dieciséis & veintiséis & treinta y seis & sesenta & seiscientos \\
			siete & diecisiete & veintisiete & treinta y siete & setenta & setecientos \\
			ocho & dieciocho & veintiocho & treinta y ocho & ochenta & ochocientos \\
			nueve & diecinueve & veintinueve & treinta y nueve & noventa & novecientos \\
			diez & veinte & treinta & cuarenta & cien & mil
		\end{tabular}
	\end{center}
	
	\subsection{Los colores}
	Colores:
	\begin{itemize}
		\item blanco, blanca
		\item \textcolor{gray}{gris}
		\item \textcolor{yellow}{amarillo, amarilla}
		\item \textcolor{pink}{rosado, rosada}
		\item \textcolor{red}{rojo, rojo}
		\item \textcolor{green}{verde}
		\item \textcolor{blue}{azul}
		\item \textcolor{brown}{marrón, moreno}
		\item \textcolor{orange}{naranja}
		\item \textcolor{violet}{violeta}
		\item negro, negra
	\end{itemize}
	
	\subsection{La familia}
	
	\begin{center}
		\begin{tabular}{ll|ll}
			\textbf{Español} & \textbf{Alemán} & \textbf{Español} & \textbf{Alemán} \\
			\hline
			abuelo/abuela & Opa/Oma & hermano & Bruder \\
			padre/madre & Vater/Mutter & sobrino & Nichte \\
			tío/tía & Onkel/Tante & hijo & Sohn \\
			esposo & Ehemann & nieto & Enkel \\
			pareja & Partner & yerno & Schwiegersohn
		\end{tabular}
	\end{center}

	\subsection{Las horas}
	\begin{itemize}
		\item 12:00 $\to$ doce en punto
		\item 1:15 $\to$ una y cuarto
		\item 3:30 $\to$ tres y media
		\item 9:45 $\to$ diez menos cuarto
	\end{itemize}

	\subsection{Palabras de la pregunta}
	\begin{itemize}
		\item ¿Que? (Was?)
		\item ¿Quien? (Wer?)
		\item ¿Dónde? (Wo?)
		\item ¿Cómo? (Wie?)
		\item ¿Cuánto? (Wie viel?)
		\item ¿Cuál? (Welcher?, Welche?, Welches?)
	\end{itemize}

	\subsection{Los conectores}
	\begin{itemize}
		\item o (oder)
		\item y (und)
		\item sin embargo (obwohl)
		\item porque (weil)
		\item pero (aber)
		\item aunque (obwohl)
		\item por eso (deshalb)
		\item también (auch)
	\end{itemize}
	
	\section{Grammatik}
	
	\subsection{Pronombres personales}
	
	\begin{center}
		\begin{tabular}{l|l|l}
			& \textbf{singular} & \textbf{plural} \\
			\hline
			\textbf{primera persona} & yo & nosotros/-as \\
			\hline
			\textbf{segunda persona} & tú & vosotros/-as \\
			\hline
			\textbf{tercera persona} & él, ella, usted & ellos, ellas, ustedes
		\end{tabular}
	\end{center}
	
	\subsection{Pronombres demostrativos}
	
	\begin{center}
		\begin{tabular}{l|lll}
			& \textbf{masculino} & \textbf{femenino} & \textbf{grupo mixto} \\
			\hline
			\textbf{singular} & este & esta & \\
			\hline
			\textbf{plural} & estos & estas & estos
		\end{tabular}
	\end{center}
	
	\subsection{Artículos definidos/indefinidos}
	
	\begin{center}
		\begin{tabular}{l|ll|ll}
			& \multicolumn{2}{c|}{\textbf{definido}} & \multicolumn{2}{c}{\textbf{indefinido}} \\
			& \textbf{singular} & \textbf{plural} &  \textbf{singular} & \textbf{plural} \\
			\hline
			\textbf{masculino} & el & los & un & unos \\
			\hline
			\textbf{femenino} & la & las & una & unas
		\end{tabular}
	\end{center}

	\subsection{Pronombres del objeto directo}
	
	Me gusta tomar \underline{el café}. $\xrightarrow{\text{el} \to \text{to}}$ Yo \underline{lo} tomo con leche.
	
	Yo como \underline{carne} dos veces a la semana. $\xrightarrow{\text{ella} \to \text{la}}$ Yo \underline{la} como asada.
	
	\subsection{Verbo \textit{ser} (sein)}
	
	\begin{center}
		\begin{tabular}{l|l|l}
			& \textbf{singular} & \textbf{plural} \\
			\hline
			\textbf{primera persona} & yo soy & nosotros/-as sois \\
			\hline
			\textbf{segunda persona} & tú eres & vosotros/-as somos \\
			\hline
			\textbf{tercera persona} & él/ella/usted es &ellos/ellas/ustedes son
		\end{tabular}
	\end{center}
	
	\subsection{Verbo \textit{llamarse} (heißen) $\to$ \nameref{reflexivverben}}
	
	\begin{center}
		\begin{tabular}{l|l|l}
			& \textbf{singular} & \textbf{plural} \\
			\hline
			\textbf{primera persona} & yo me llamo & nosotros/-as nos llamamos \\
			\hline
			\textbf{segunda persona} & tú te llamas & vosotros/-as os llamáis \\
			\hline
			\textbf{tercera persona} & él/ella/usted se llama & ellos/ellas/ustedes se llaman
		\end{tabular}
	\end{center}
	
	\subsection{Verbo \textit{tener} (haben)}
	
	\begin{center}
		\begin{tabular}{l|l|l}
			& \textbf{singular} & \textbf{plural} \\
			\hline
			\textbf{primera persona} & yo tengo & nosotros/-as tenemos \\
			\hline
			\textbf{segunda persona} & tú tienes & vosotros/-as tenéis \\
			\hline
			\textbf{tercera persona} & él/ella/usted tiene &ellos/ellas/ustedes tienen
		\end{tabular}
	\end{center}
	
	\subsection{Conjugación de verbos}
	
	\begin{center}
		\begin{tabular}{l|lll}
			& \textbf{primera conjugación} (-ar) & \textbf{segunda conjugación} (-er) & \textbf{tercera conjugación} (-ir) \\
			\hline
			yo & canto & como & vivo \\
			tú & cantas & comes & vives \\
			él/ella/usted & canta & come & vive \\
			nosotros & cantamos & comemos & vivimos \\
			vosotros & cantáis & coméis & vivís \\
			ellos/ellas/ustedes & cantan & comen & viven
		\end{tabular}
	\end{center}
	
	\subsection{Adjetivos}
	Terminaciones adjetivas
	\begin{itemize}
		\item -e: neutro
		\item -a: femenino
		\item -o: masculino
		\item sin final: -ista o -[consonante]
	\end{itemize}
	
	\subsection{Los adjetivos posesivos}
	
	\begin{center}
		\begin{tabular}{l|l|l}
			& \textbf{singular} & \textbf{plural} \\
			\hline
			\textbf{primera persona} & mi & mis \\
			\hline
			\textbf{segunda persona} & tu & tus \\
			\hline
			\textbf{tercera persona} & su & sus \\
			\hline
			\textbf{primera persona} & nuestro & nuestros \\
			\hline
			\textbf{segunda persona} & vuestro & vuestros \\
			\hline
			\textbf{tercera persona} & su & sus
		\end{tabular}
	\end{center}

	\subsection{ser/estar/hay}
	
	\begin{center}
		\begin{tabular}{L{4.5cm}|L{4.5cm}|L{4.5cm}}
			\textbf{ser} & \textbf{estar} & \textbf{hay} \\
			\hline
			Nacionalidades, origen, presentación de una persona, definición o descripción más detallada de una persona o un lugar, ocupación, características de las personas y las cosas, fechas, horas y precios, posesiones y afiliaciones & Indicar el lugar, la condición o el estado personal del ser, describir los estados temporales, no permanentes, las acciones que se están llevando a cabo actualmente $\to$ \nameref{gerundium} & hay + articulo indefinido, hay + substantivo sin articulo, hay + números, hay + cantidad no especificada
		\end{tabular}
	\end{center}

	\subsection{Oraciones comparativos}
	\begin{itemize}
		\item de superioridad: Mi libro es \underline{más nuevo que} el tuyo.
		\item de inferioridad: Mi casa es \underline{menos grande que} de la María.
		\item igualdad:
		\begin{itemize}
			\item Mi libro es \underline{tan nuevo como} el tuyo.
			\item Yo \underline{leo tanto como} tú.
			\item Hay \underline{tantas cases como} en mi barrio.
		\end{itemize}
	\end{itemize}

	\subsection{Verbo \textit{ir} (gehen)}
	
	\begin{center}
		\begin{tabular}{l|l|l}
			& \textbf{singular} & \textbf{plural} \\
			\hline
			\textbf{primera persona} & yo voy & nosotros/-as vamos \\
			\hline
			\textbf{segunda persona} & tú vas & vosotros/-as vais \\
			\hline
			\textbf{tercera persona} & él/ella/usted va &ellos/ellas/ustedes van
		\end{tabular}
	\end{center}

	\subsection{El verbo \textit{ir} + preposiciones}
	
	\begin{center}
		\begin{tabular}{L{5cm}|L{5cm}}
			\textbf{a} & \textbf{en} \\
			\hline
			a pie & en autobús \\
			a caballo (reiten) & en moto \\
			& en coche (Auto) \\
			& en bici \\
			& en barco (Schiff) \\
			& en avíen (Flugzeug) \\
			& en tren (Zug)
		\end{tabular}
	\end{center}

	\subsection{Verbo \textit{poder} (können/dürfen)}
	
	\begin{center}
		\begin{tabular}{l|l|l}
			& \textbf{singular} & \textbf{plural} \\
			\hline
			\textbf{primera persona} & yo puedo & nosotros/-as podemos \\
			\hline
			\textbf{segunda persona} & tú puedes & vosotros/-as podéis \\
			\hline
			\textbf{tercera persona} & él/ella/usted puede &ellos/ellas/ustedes pueden
		\end{tabular}
	\end{center}

	\subsection{Verbo \textit{querer} (wollen)}
	
	\begin{center}
		\begin{tabular}{l|l|l}
			& \textbf{singular} & \textbf{plural} \\
			\hline
			\textbf{primera persona} & yo quiero & nosotros/-as queremos \\
			\hline
			\textbf{segunda persona} & tú quieres & vosotros/-as queréis \\
			\hline
			\textbf{tercera persona} & él/ella/usted quiere &ellos/ellas/ustedes quieren
		\end{tabular}
	\end{center}

	\subsection{Verbo \textit{saber} (wissen)}
	
	\begin{center}
		\begin{tabular}{l|l|l}
			& \textbf{singular} & \textbf{plural} \\
			\hline
			\textbf{primera persona} & yo sé & nosotros/-as sabemos \\
			\hline
			\textbf{segunda persona} & tú sabes & vosotros/-as sabéis \\
			\hline
			\textbf{tercera persona} & él/ella/usted sabe &ellos/ellas/ustedes saben
		\end{tabular}
	\end{center}

	\subsection{Verbos modales}
	Verbos modales + infinitivo
	\begin{itemize}
		\item deber (müssen/sollen): Yo debo estudiar.
		\item poder (können/dürfen): Tú puedes ir a un museo.
		\item querer (wollen): Nosotras queremos viajar.
		\item saber (wissen): Él sabe hablar portugués.
		\item soler (etwas, was man immer macht): Ellas suelen despertar temprano.
	\end{itemize}

	\subsection{Verbos reflexivos}\label{reflexivverben}
	
	Utilizando el ejemplo de \textit{ducharse}:
	\begin{center}
		\begin{tabular}{l|l|l}
			& \textbf{singular} & \textbf{plural} \\
			\hline
			\textbf{primera persona} & yo me ducho & nosotros/-as nos duchamos \\
			\hline
			\textbf{segunda persona} & tú te duchas & vosotros/-as os ducháis \\
			\hline
			\textbf{tercera persona} & él/ella/usted se ducha &ellos/ellas/ustedes se duchan
		\end{tabular}
	\end{center}

	\subsection{Verbo \textit{haber} (haben)}
	
	\begin{center}
		\begin{tabular}{l|l|l}
			& \textbf{singular} & \textbf{plural} \\
			\hline
			\textbf{primera persona} & yo he & nosotros/-as hemos \\
			\hline
			\textbf{segunda persona} & tú has & vosotros/-as habéis \\
			\hline
			\textbf{tercera persona} & él/ella/usted ha &ellos/ellas/ustedes han
		\end{tabular}
	\end{center}

	\subsection{Pretérito perfecto compuesto}
	
	haber + participio
	\begin{itemize}
		\item -ar: -ado
		\item -er: -ido
		\item -ir: -ido
	\end{itemize}

	irregulares:
	\begin{itemize}
		\item hacer $\to$ hecho
		\item decir $\to$ dicho
		\item escribir $\to$ escrito
		\item ir $\to$ ido
		\item ser $\to$ sido
	\end{itemize}

	con verbos reflexivos: Nosotros \underline{nos} hemos \underline{duchado}.
	
	\subsection{El gerundio}\label{gerundium}
	
	estar + gerundio
	\begin{itemize}
		\item -ar: -ando
		\item -er: -iendo
		\item -ir: iendo
	\end{itemize}

	irregulares:
	\begin{itemize}
		\item decir $\to$ diciendo
		\item pedir $\to$ pidiendo
		\item dormir $\to$ durmiendo
		\item ir $\to$ yendo
		\item leer $\to$ leyendo
		\item oír $\to$ oyendo
	\end{itemize}

	\subsection{El futuro próximo}
	
	ir + a + infinitivo

\end{document}