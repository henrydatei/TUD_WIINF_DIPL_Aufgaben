\documentclass{article}

\usepackage{amsmath,amssymb}
\usepackage{tikz}
\usepackage{pgfplots}
\usepackage{xcolor}
\usepackage[left=2.1cm,right=3.1cm,bottom=3cm,footskip=0.75cm,headsep=0.5cm]{geometry}
\usepackage{enumerate}
\usepackage{enumitem}
\usepackage{marvosym}
\usepackage{tabularx}
\usepackage{parskip}

\usepackage{listings}
\definecolor{lightlightgray}{rgb}{0.95,0.95,0.95}
\definecolor{lila}{rgb}{0.8,0,0.8}
\definecolor{mygray}{rgb}{0.5,0.5,0.5}
\definecolor{mygreen}{rgb}{0,0.8,0.26}
\lstdefinestyle{java} {language=java}
\lstset{language=java,
	basicstyle=\ttfamily,
	keywordstyle=\color{lila},
	commentstyle=\color{lightgray},
	stringstyle=\color{mygreen}\ttfamily,
	backgroundcolor=\color{white},
	showstringspaces=false,
	numbers=left,
	numbersep=10pt,
	numberstyle=\color{mygray}\ttfamily,
	identifierstyle=\color{blue},
	xleftmargin=.1\textwidth, 
	%xrightmargin=.1\textwidth,
	escapechar=§,
	%literate={\t}{{\ }}1
}

\usepackage[utf8]{inputenc}

\renewcommand*{\arraystretch}{1.4}

\newcolumntype{L}[1]{>{\raggedright\arraybackslash}p{#1}}
\newcolumntype{R}[1]{>{\raggedleft\arraybackslash}p{#1}}
\newcolumntype{C}[1]{>{\centering\let\newline\\\arraybackslash\hspace{0pt}}m{#1}}

\newcommand{\E}{\mathbb{E}}
\DeclareMathOperator{\rk}{rk}
\DeclareMathOperator{\Var}{Var}
\DeclareMathOperator{\Cov}{Cov}

\title{\textbf{INLOOP Softwaretechnologie, Collections III (Library with Map)}}
\author{\textsc{Henry Haustein}}
\date{}

\begin{document}
	\maketitle
	
	Datei \texttt{Book.java}
	\begin{lstlisting}[style=java,tabsize=2]
// Nahezu identischer Code wie fuer Collections II (Library with Set)
package collections3;

public class Book implements Comparable<Book> {
	private String isbn;
	private String author;
	private String title;

	public Book(String isbn, String author, String title) {
		this.isbn = isbn;
		this.author = author;
		this.title = title;
	}

	public Book(String isbn) {
		this.isbn = isbn;
		this.author = "";
		this.title = "";
	}

	public String getIsbn() {
		return this.isbn;
	}

	public String getAuthor() {
		return this.author;
	}

	public String getTitle() {
		return this.title;
	}

	public String toString() {
		return this.title + " by " + this.author + " (ISBN " + this.isbn + ")";
	}

	public int compareTo(Book book){
		return this.getIsbn().compareTo(book.getIsbn());
	}

	public boolean equals(Object o) {
		if (o instanceof Book) {
			Book book = (Book) o;
			return this.isbn.equals(book.getIsbn());
		}
		return false;
	}

	public int hashCode() {
		return this.isbn.hashCode();
	}
}
	\end{lstlisting}
	Datei \texttt{Library.java}
	\begin{lstlisting}[style=java,tabsize=2]
package collections3;

import java.util.*;

public class Library {
	private Map<String, Set<Book>> stock;

	public Library() {
		this.stock = new TreeMap<String, Set<Book>>();
	}

	public boolean insertBook(Book newBook) {
		if (stock.containsKey(newBook.getAuthor())) {
			// Autor gibt es bereits
			// Hinzufuegen des Buches zur bereits bestehenden Menge
			Set<Book> authorBooks = stock.get(newBook.getAuthor());
			boolean result = authorBooks.add(newBook);
			stock.put(newBook.getAuthor(), authorBooks);
			return result;
		}
		else {
			// Autor muss noch hinzugefuegt werden
			// Neuerstellung der Menge von Buechern des Autors
			Set<Book> authorBooks = new TreeSet<Book>();
			authorBooks.add(newBook);
			stock.put(newBook.getAuthor(), authorBooks);
			return true;
		}
	}

	public Book searchForIsbn(String isbn) {
		// Durchlaufen aller Autoren (= Keys der Map) und
		// anschliessendes Durchlaufen aller Buecher eines Autors
		for (String author : stock.keySet()) {
			Set<Book> authorBooks = stock.get(author);
			for (Book b : authorBooks) {
				if (b.getIsbn().equals(isbn)) {
					return b;
				}
			}
		}
		return null;
	}

	public Collection<Book> searchForAuthor(String author) {
		// Wenn es einen Autor nicht gibt, soll eine leere Menge 
		// zurueckgegeben werden
		if (stock.get(author) == null) {
			return new TreeSet<Book>();
		} 
		else {
			return stock.get(author);
		}
	}

	public Map<String, Set<Book>> listStockByAuthor() {
		return stock;
	}

	public Collection<Book> getStock() {
		Collection<Book> books = new TreeSet<Book>();
		// stock.values() gibt eine Collection der Values der Map 
		// zurueck, also eine Menge von Mengen von Buechern bzw.
		// Collection<Collection<Book>>
		for (Collection<Book> b : this.stock.values()) {
			// addAll fuegt eine ganze Collection an Buechern auf
			// einmal hinzu
			books.addAll(b);
		}
		return books;
	}
}
	\end{lstlisting}
	
\end{document}