\documentclass{article}

\usepackage{amsmath,amssymb}
\usepackage{tikz}
\usepackage{pgfplots}
\usepackage{xcolor}
\usepackage[left=2.1cm,right=3.1cm,bottom=3cm,footskip=0.75cm,headsep=0.5cm]{geometry}
\usepackage{enumerate}
\usepackage{enumitem}
\usepackage{marvosym}
\usepackage{tabularx}
\usepackage{parskip}

\usepackage{listings}
\definecolor{lightlightgray}{rgb}{0.95,0.95,0.95}
\definecolor{lila}{rgb}{0.8,0,0.8}
\definecolor{mygray}{rgb}{0.5,0.5,0.5}
\definecolor{mygreen}{rgb}{0,0.8,0.26}
\lstdefinestyle{java} {language=java}
\lstset{language=java,
	basicstyle=\ttfamily,
	keywordstyle=\color{lila},
	commentstyle=\color{lightgray},
	stringstyle=\color{mygreen}\ttfamily,
	backgroundcolor=\color{white},
	showstringspaces=false,
	numbers=left,
	numbersep=10pt,
	numberstyle=\color{mygray}\ttfamily,
	identifierstyle=\color{blue},
	xleftmargin=.1\textwidth, 
	%xrightmargin=.1\textwidth,
	escapechar=§,
	%literate={\t}{{\ }}1
	breaklines=true,
	postbreak=\mbox{\space}
}

\usepackage[utf8]{inputenc}

\renewcommand*{\arraystretch}{1.4}

\newcolumntype{L}[1]{>{\raggedright\arraybackslash}p{#1}}
\newcolumntype{R}[1]{>{\raggedleft\arraybackslash}p{#1}}
\newcolumntype{C}[1]{>{\centering\let\newline\\\arraybackslash\hspace{0pt}}m{#1}}

\newcommand{\E}{\mathbb{E}}
\DeclareMathOperator{\rk}{rk}
\DeclareMathOperator{\Var}{Var}
\DeclareMathOperator{\Cov}{Cov}

\title{\textbf{INLOOP Softwaretechnologie, Generics (Bottle)}}
\author{\textsc{Henry Haustein}}
\date{}

\begin{document}
	\maketitle
	
	\section*{vollständiger Code}
	Datei \texttt{Bottle.java}
	\begin{lstlisting}[style=java,tabsize=2]
public class Bottle<T extends Drink> {
	private T content;

	public Bottle() {
		this.content = null;
	}

	public boolean isEmpty() {
		if (content == null) {
			return true;
		}
		else {
			return false;
		}
	}

	public void fill(T con) {
		if (this.isEmpty()) {
			this.content = con;
		}
		else {
			throw new IllegalStateException("Bottle not empty!");
		}
	}

	public T empty() {
		if (this.isEmpty()) {
			throw new IllegalStateException("Bottle must be filled");
		}
		else {
			T oldcontent = content;
			content = null;
			return oldcontent;   
		}
	}
}
	\end{lstlisting}
	Datei \texttt{Drink.java}
	\begin{lstlisting}[style=java, tabsize=2]
public abstract class Drink {

}
	\end{lstlisting}
	Datei \texttt{Beer.java}
	\begin{lstlisting}[style=java, tabsize=2]
public class Beer extends Drink {
	private String brewery;

	public Beer(String brew) {
		this.brewery = brew;
	}

	public String getBrewery() {
		return brewery;
	}

	public String toString() {
		return brewery;
	}
}
	\end{lstlisting}
	Datei \texttt{Wine.java}
	\begin{lstlisting}[style=java,tabsize=2]
public abstract class Wine extends Drink {
	private String region;

	public Wine(String reg) {
		this.region = reg;
	}

	public String getRegion() {
		return region;
	}

	public String toString() {
		return region;
	}
}
	\end{lstlisting}
	Datei \texttt{WhiteWine.java}
	\begin{lstlisting}[style=java, tabsize=2]
public class WhiteWine extends Wine {
	public WhiteWine(String reg) {
		super(reg);
	}
}
	\end{lstlisting}
	Datei \texttt{RedWine.java}
	\begin{lstlisting}[style=java,tabsize=2]
public class RedWine extends Wine {
	public RedWine(String reg) {
		super(reg);
	}
}
	\end{lstlisting}

	\section*{Erklärung}
	Die Klassen der Drinks sind selbsterklärend.
	
	In der Klasse \texttt{Bottle} steht der Parameter \texttt{T} für einen Typ (bzw. Klasse) von Drink. \texttt{T} wird damit zum Datentyp und kann damit Werte wie \texttt{Beer}, \texttt{WhiteWine}, \texttt{RedWine}, ... annehmen, sodass im konkreten Fall statt \texttt{private T content;} eben \texttt{private Beer content;} steht. So zieht sich das dann durch die ganze Klasse.
	
\end{document}