\documentclass{article}

\usepackage{amsmath,amssymb}
\usepackage{tikz}
\usepackage{pgfplots}
\usepackage{xcolor}
\usepackage[left=2.1cm,right=3.1cm,bottom=3cm,footskip=0.75cm,headsep=0.5cm]{geometry}
\usepackage{enumerate}
\usepackage{enumitem}
\usepackage{marvosym}
\usepackage{tabularx}
\usepackage{parskip}

\usepackage{listings}
\definecolor{lightlightgray}{rgb}{0.95,0.95,0.95}
\definecolor{lila}{rgb}{0.8,0,0.8}
\definecolor{mygray}{rgb}{0.5,0.5,0.5}
\definecolor{mygreen}{rgb}{0,0.8,0.26}
\lstdefinestyle{java} {language=java}
\lstset{language=java,
	basicstyle=\ttfamily,
	keywordstyle=\color{lila},
	commentstyle=\color{lightgray},
	stringstyle=\color{mygreen}\ttfamily,
	backgroundcolor=\color{white},
	showstringspaces=false,
	numbers=left,
	numbersep=10pt,
	numberstyle=\color{mygray}\ttfamily,
	identifierstyle=\color{blue},
	xleftmargin=.1\textwidth, 
	%xrightmargin=.1\textwidth,
	escapechar=§,
	%literate={\t}{{\ }}1
}

\usepackage[utf8]{inputenc}

\renewcommand*{\arraystretch}{1.4}

\newcolumntype{L}[1]{>{\raggedright\arraybackslash}p{#1}}
\newcolumntype{R}[1]{>{\raggedleft\arraybackslash}p{#1}}
\newcolumntype{C}[1]{>{\centering\let\newline\\\arraybackslash\hspace{0pt}}m{#1}}

\newcommand{\E}{\mathbb{E}}
\DeclareMathOperator{\rk}{rk}
\DeclareMathOperator{\Var}{Var}
\DeclareMathOperator{\Cov}{Cov}

\title{\textbf{INLOOP Softwaretechnologie, Collections I (Library with List)}}
\author{\textsc{Henry Haustein}}
\date{}

\begin{document}
	\maketitle
	
	Datei \texttt{Book.java}
	\begin{lstlisting}[style=java,tabsize=2]
package collections1;

// Durch Nutzung des Interfaces Comparable<Book> ist die 
// compareTo()-Funktion verfuegbar
public class Book implements Comparable<Book> {
	private String isbn;
	private String author;
	private String title;

	public Book(String isbn) {
		this.isbn = isbn;
		this.title = "";
		this.author = "";
	}

	public Book(String isbn, String author, String title){
		this.isbn = isbn;
		this.author = author;
		this.title = title;
	}

	public String getTitle() {
		return title;
	}

	public String getIsbn() {
		return this.isbn;
	}

	public String getAuthor() {
		return this.author;
	}

	// Vergleich vn Buechern = Vergleich der ISBNs
	// fuer den Vergleich der ISBNs wird compareTo von Strings verwendet
	public int compareTo(Book book) {
		return this.getIsbn().compareTo(book.getIsbn());
	}
}
	\end{lstlisting}
	Datei \texttt{Library.java}
	\begin{lstlisting}[style=java,tabsize=2]
package collections1;

// wird benoetigt fuer Collections.sort() und Collections.binarySearch()
import java.util.*;

public class Library {
	private List<Book> stock;

	public Library() {
		this.stock = new LinkedList<Book>();
	}

	public boolean insertBook(Book newBook) {
		stock.add(newBook);
		Collections.sort(stock);
		return true;
	}

	public Book searchForIsbn(String isbn) {
		int index = Collections.binarySearch(stock, new Book(isbn));
		// wird -1 als index zurueckgegeben, ist das Buch nicht vorhanden
		if (index >= 0) {
			return stock.get(index);
		}
		else {
			return null;
		}
	}

	public Collection<Book> searchForAuthor(String author) {
		ArrayList<Book> authorList = new ArrayList<Book>();
		// for-Schleife, funktioniert genau so wie in Python
		// for listentry in list -> for (listentry : list)
		// jedes Element der Liste wird einmal in listentry 
		// abgespeichert und  innerhalb der for-Schleife
		// kann dann auf listentry zugegriffen werden
		for (Book b : stock) {
			if (b.getAuthor().equals(author)) {
				authorList.add(b);
			}
		}
		return authorList;
	}
}
	\end{lstlisting}
	
\end{document}