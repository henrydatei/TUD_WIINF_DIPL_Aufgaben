\documentclass{article}

\usepackage{amsmath,amssymb}
\usepackage{tikz}
\usepackage{pgfplots}
\usepackage{xcolor}
\usepackage[left=2.1cm,right=3.1cm,bottom=3cm,footskip=0.75cm,headsep=0.5cm]{geometry}
\usepackage{enumerate}
\usepackage{enumitem}
\usepackage{marvosym}
\usepackage{tabularx}
\usepackage{parskip}

\usepackage{listings}
\definecolor{lightlightgray}{rgb}{0.95,0.95,0.95}
\definecolor{lila}{rgb}{0.8,0,0.8}
\definecolor{mygray}{rgb}{0.5,0.5,0.5}
\definecolor{mygreen}{rgb}{0,0.8,0.26}
\lstdefinestyle{java} {language=java}
\lstset{language=java,
	basicstyle=\ttfamily,
	keywordstyle=\color{lila},
	commentstyle=\color{lightgray},
	stringstyle=\color{mygreen}\ttfamily,
	backgroundcolor=\color{white},
	showstringspaces=false,
	numbers=left,
	numbersep=10pt,
	numberstyle=\color{mygray}\ttfamily,
	identifierstyle=\color{blue},
	xleftmargin=.1\textwidth, 
	%xrightmargin=.1\textwidth,
	escapechar=§,
	%literate={\t}{{\ }}1
}

\usepackage[utf8]{inputenc}

\renewcommand*{\arraystretch}{1.4}

\newcolumntype{L}[1]{>{\raggedright\arraybackslash}p{#1}}
\newcolumntype{R}[1]{>{\raggedleft\arraybackslash}p{#1}}
\newcolumntype{C}[1]{>{\centering\let\newline\\\arraybackslash\hspace{0pt}}m{#1}}

\newcommand{\E}{\mathbb{E}}
\DeclareMathOperator{\rk}{rk}
\DeclareMathOperator{\Var}{Var}
\DeclareMathOperator{\Cov}{Cov}

\title{\textbf{INLOOP Softwaretechnologie, String Processing - Leetspeek}}
\author{\textsc{Henry Haustein}}
\date{}

\begin{document}
	\maketitle
	
	\section*{vollständiger Code}
	Datei \texttt{Leet.java}
	\begin{lstlisting}[style=java,tabsize=2]
class Leet {
	public static String toLeet(String normal) {
		String l1 = normal.replace("elite", "1337");
		String l2 = l1.replace("cool", "k3wl");
		String l3 = l2.replace("!", "!!!11");
		String l4 = l3.replace("ck", "xx");
		String l5 = l4.replace("ers", "0rz");
		String l6 = l5.replace("er", "0rz");
		String l7 = l6.replace("en", "n");
		String l8 = l7.replace("e", "3");
		String l9 = l8.replace("t", "7");
		String l10 = l9.replace("o", "0");
		String l11 = l10.replace("a", "@");
		
		return l11;
	}
	
	public static String[] allToLeet(String[] normals) {
		String[] leets = new String[normals.length];
		for (int i = 0; i < normals.length; i++) {
			leets[i] = toLeet(normals[i]);
		}
		
		return leets;
	}
}
	\end{lstlisting}

	\section*{Erklärung}
	Wir führen immer eine Ersetzung durch und speichern das Zwischenergebnis in einer neuen Variable. Sicherlich kann man das auch direkter machen, in etwa so:
	\begin{lstlisting}[style=java]
String leet = normal.replace("elite", "1337").replace("cool", "k3wl") ...;
	\end{lstlisting}
	In der Funktion \texttt{allToLeet()} erstellen wir uns zuerst ein Array der passenden Länge, was wir dann auch später zurückgeben wollen. In einer for-Schleife laufen wir dann durch das Array \texttt{normals} durch und füllen die entsprechenden Stellen in unserem \texttt{leets}-Array.
	
	
\end{document}