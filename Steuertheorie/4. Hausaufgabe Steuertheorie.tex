\documentclass{article}

\usepackage{amsmath,amssymb}
\usepackage{tikz}
\usepackage{pgfplots}
\usepackage{xcolor}
\usepackage[left=2.1cm,right=3.1cm,bottom=3cm,footskip=0.75cm,headsep=0.5cm]{geometry}
\usepackage{enumerate}
\usepackage{enumitem}
\usepackage{marvosym}
\usepackage{tabularx}
\usepackage{tikz-qtree}
\usetikzlibrary{patterns,arrows,calc,decorations.pathmorphing,backgrounds, positioning,fit,petri,decorations.fractals,trees,cd,automata,babel,shapes.geometric,arrows.meta,bending}

\usepackage[utf8]{inputenc}

\renewcommand*{\arraystretch}{1.4}

\newcolumntype{L}[1]{>{\raggedright\arraybackslash}p{#1}}
\newcolumntype{R}[1]{>{\raggedleft\arraybackslash}p{#1}}
\newcolumntype{C}[1]{>{\centering\let\newline\\\arraybackslash\hspace{0pt}}m{#1}}

\title{\textbf{Steuertheorie, Hausaufgabe 3}}
\author{\textsc{Henry Haustein}}
\date{}

\begin{document}
	\maketitle
	
	\section*{Aufgabe 1}
	\begin{enumerate}[label=(\alph*)]
		\item Die Effizienz eines Steuersystems kann erreicht werden, indem alle Güter mit dem gleichen Steuersatz besteuert werden. Das ist allerdings schwierig, da man z.B. Freizeit nicht direkt besteuern kann. Dadurch entstehen Verzerrungen.
		\item Ramsey-Regel: Ein Steuersystem ist (second-best) pareto-optimal, wenn durch eine proportionale Änderung aller Steuersätze bei allen Gütern die gleiche, proportionale Änderung der kompensierten Nachfrage ausgelöst wird.
		\item Es gilt
		\begin{align}
			Z &= L+F \notag \\
			wL &= q_1x_1 + q_2x_2 \notag \\
			L &= \frac{q_1x_1 + q_2x_2}{w} \notag \\
			Z &= \frac{q_1x_1+q_2x_2}{w} + F \notag
		\end{align}
		\item Die Lagrangefunktion lautet:
		\begin{align}
			\mathcal{L} &= \frac{q_1x_1+q_2x_2}{w} + F - \lambda\left(F^\frac{1}{4}\cdot x_1^\frac{1}{4}\cdot x_2^\frac{1}{2} - \bar{U}\right) \notag
		\end{align}
		Die Bedingungen erster Ordnung lauten
		\begin{align}
			\frac{\partial\mathcal{L}}{\partial x_1} &= \frac{q_1}{w} - \lambda \cdot F^\frac{1}{4}\cdot \frac{1}{4}\cdot x_1^{-\frac{3}{4}}\cdot x_2^\frac{1}{2} = 0 \notag \\
			\label{1.1}
			\frac{q_1}{w} &= \lambda F^\frac{1}{4}\cdot \frac{1}{4}\cdot x_1^{-\frac{3}{4}}\cdot x_2^\frac{1}{2} \tag{1.1} \\
			\frac{\partial \mathcal{L}}{\partial x_2} &= \frac{q_2}{w} - \lambda\cdot F^\frac{1}{4}\cdot x_1^\frac{1}{4}\cdot\frac{1}{2}\cdot x_2^{-\frac{1}{2}} = 0 \notag \\
			\label{1.2}
			\frac{q_2}{w} &= \lambda\cdot F^\frac{1}{4}\cdot x_1^\frac{1}{4}\cdot\frac{1}{2}\cdot x_2^{-\frac{1}{2}} \tag{1.2} \\
			\frac{\partial\mathcal{L}}{\partial F} &= 1- \lambda\cdot\frac{1}{4}\cdot F^{-\frac{3}{4}}\cdot x_1^\frac{1}{4}\cdot x_2^\frac{1}{2} = 0 \notag \\
			\label{1.3}
			1 &= \lambda\cdot\frac{1}{4}\cdot F^{-\frac{3}{4}}\cdot x_1^\frac{1}{4}\cdot x_2^\frac{1}{2} \tag{1.3} \\
			\frac{\partial\mathcal{L}}{\partial\lambda} &= F^\frac{1}{4}\cdot x_1^\frac{1}{4}\cdot x_2^\frac{1}{2} - \bar{U} = 0 \notag \\
			\label{1.4}
			\bar{U} &= F^\frac{1}{4}\cdot x_1^\frac{1}{4}\cdot x_2^\frac{1}{2} \tag{1.4}
		\end{align}
		Division von \eqref{1.1} durch \eqref{1.3} ergibt
		\begin{align}
			\frac{q_1}{w} &= \frac{\lambda F^\frac{1}{4}\cdot \frac{1}{4}\cdot x_1^{-\frac{3}{4}}\cdot x_2^\frac{1}{2}}{\lambda\cdot\frac{1}{4}\cdot F^{-\frac{3}{4}}\cdot x_1^\frac{1}{4}\cdot x_2^\frac{1}{2}} \notag \\
			&= \frac{F}{x_1} \notag \\
			\label{1.5}
			x_1 &= \frac{w}{q_1}\cdot F \tag{1.5}
		\end{align}
		Division von \eqref{1.2} durch \eqref{1.3} ergibt
		\begin{align}
			\frac{q_2}{w} &= \frac{\lambda\cdot F^\frac{1}{4}\cdot x_1^\frac{1}{4}\cdot\frac{1}{2}\cdot x_2^{-\frac{1}{2}}}{\lambda\cdot\frac{1}{4}\cdot F^{-\frac{3}{4}}\cdot x_1^\frac{1}{4}\cdot x_2^\frac{1}{2}} \notag \\
			&= \frac{2F}{x_2} \notag \\
			\label{1.6}
			x_2 &= \frac{w}{q_2}\cdot 2F \tag{1.6}
		\end{align}
		Einsetzen von \eqref{1.5} und \eqref{1.6} in \eqref{1.4}:
		\begin{align}
			\bar{U} &= F^\frac{1}{4}\left(\frac{w}{q_1}\cdot F\right)^\frac{1}{4}\cdot \left(\frac{w}{q_2}\cdot 2F\right)^\frac{1}{2} \notag \\
			&= F^\frac{1}{4}\cdot F^\frac{1}{4}\cdot F^\frac{1}{2}\cdot\left(\frac{w}{q_1}\right)^\frac{1}{4}\cdot\left(\frac{2w}{q_2}\right)^\frac{1}{2} \notag \\
			F &= \bar{U}\cdot\left(\frac{q_1\cdot q_2^2}{4\cdot w^3}\right)^\frac{1}{4} \notag \\
			x_1 &= \frac{w}{q_1}\cdot U \cdot \left(\frac{q_1\cdot q_2^2}{4\cdot w^3}\right)^\frac{1}{4} = \bar{U}\cdot\left(\frac{w\cdot q_2^2}{4\cdot q_1^3}\right)^\frac{1}{4} \notag \\
			x_2 &= \frac{2w}{q_2}\cdot \bar{U}\cdot \left(\frac{q_1\cdot q_2^2}{4\cdot w^3}\right)^\frac{1}{4} = \bar{U}\cdot\left(\frac{4\cdot w\cdot q_1}{q_2^2}\right)^\frac{1}{4} \notag
		\end{align}
		\item Es gilt
		\begin{align}
			F(\lambda q_1,\lambda q_2,\lambda w) &= \bar{U}\left(\frac{\lambda q_1\cdot\lambda^2 q_2^2}{4\cdot\lambda^3 w^3}\right)^\frac{1}{4} \notag \\
			&= F(q_1,q_2,w) \notag \\
			&= \lambda^0\cdot F(q_1,q_2,w) \notag \\
			x_1(\lambda q_1,\lambda q_2,\lambda w) &= \bar{U}\left(\frac{\lambda w\cdot \lambda^2 q_2^2}{4\cdot\lambda^3 q_1^3}\right)^\frac{1}{4} \notag \\
			&= x_1(q_1,q_2,w) \notag \\
			&= \lambda^0\cdot x_1(q_1,q_2,w) \notag \\
			x_2(\lambda q_1,\lambda q_2,\lambda w) &= \bar{U}\left(\frac{4\cdot \lambda w\cdot \lambda q_1}{\lambda^2 q_2^2}\right)^\frac{1}{4} \notag \\
			&= x_2(q_1,q_2,w) \notag \\
			&= \lambda^0 \cdot x_2(q_1,q_2,w) \notag
		\end{align}
		\item Corlett-Hague-Regel: Dasjenige Gut sollte höher besteuert werden, das eine geringere Kreuzpreiselastizität mit dem Lohnsatz aufweist.
		\item Es gilt
		\begin{align}
			\frac{\partial x_1}{\partial w}\cdot\frac{w}{x_1} &= \frac{1}{4}\left(\frac{w\cdot q_2^2}{4\cdot q_1^3}\right)^{-\frac{3}{4}}\cdot\bar{U}\cdot\frac{q_2^2}{4\cdot q_1^3}\cdot\frac{w}{\bar{U}\cdot\left(\frac{w\cdot q_2^2}{4\cdot q_1^3}\right)^\frac{1}{4}} \notag \\
			&= \frac{1}{4}\cdot\frac{4\cdot q_1^3}{w\cdot q_2^2}\cdot\frac{q_2^2}{4\cdot q_1^3}\cdot w \notag \\
			&= \frac{1}{4} \notag \\
			\frac{\partial x_2}{\partial w}\cdot\frac{w}{x_2} &= \frac{1}{4}\left(\frac{4\cdot w\cdot q_1}{q_2^2}\right)^{-\frac{3}{4}}\cdot\bar{U}\cdot\frac{4\cdot q_1}{q_2^2}\cdot\frac{w}{\bar{U}\cdot\left(\frac{4\cdot w\cdot q_1}{q_2^2}\right)^\frac{1}{4}} \notag \\
			&= \frac{1}{4}\cdot\frac{q_2^2}{4\cdot w\cdot q_1}\cdot\frac{4\cdot q_1}{q_2^2}\cdot w \notag \\
			&= \frac{1}{4} \notag
		\end{align}
		Wenn der Lohnsatz um 1 \% steigt, so steigt die Nachfrage nach Gut 1 und 2 um jeweils $\frac{1}{4}$ \%.
		\item Da beide Konsumgüter die gleiche Kreuzpreiselastizität mit dem Lohn haben, sollten diese gleich besteuert werden.
	\end{enumerate}

	\section*{Aufgabe 2}
	\begin{enumerate}[label=(\alph*)]
		\item Die Budgetbeschränkung ist $wL=q_1x_1+q_2w_2$. Der Lagrange-Ansatz liefert
		\begin{align}
			\mathcal{L} = \alpha\ln(x_1) + \beta\ln(x_2)-L-\lambda(q_1x_1+q_2x_2-wL) \notag
		\end{align}
		Die Bedingungen erster Ordnung lauten
		\begin{align}
			\frac{\partial\mathcal{L}}{\partial x_1} &= \frac{\alpha}{x_1} - \lambda q_1 = 0 \notag \\
			\label{2.1}
			\frac{\alpha}{x_1} &= \lambda q_1 \tag{2.1} \\
			\frac{\partial\mathcal{L}}{\partial x_2} &= \frac{\beta}{x_2} - \lambda q_2 = 0 \notag \\
			\label{2.2}
			\frac{\beta}{x_2} &= \lambda q_2 \tag{2.2} \\
			\frac{\partial\mathcal{L}}{\partial L} &= -1 + \lambda w = 0 \notag \\
			\label{2.3}
			\lambda w &= 1 \tag{2.3} \\
			\frac{\partial\mathcal{L}}{\partial\lambda} &= q_1x_1+q_2x_2 - wL = 0 \notag \\
			\label{2.4}
			q_1x_1+q_2x_2 &= wL \tag{2.4}
		\end{align}
		Division von \eqref{2.1} durch \eqref{2.3} ergibt
		\begin{align}
			\frac{\alpha}{x_1} &= \frac{\lambda q_1}{\lambda w} \notag \\
			x_1 &= \frac{w}{q_1}\cdot\alpha \notag
		\end{align}
		Division von \eqref{2.2} durch \eqref{2.3} ergibt
		\begin{align}
			\frac{\beta}{x_2} &= \frac{\lambda q_2}{\lambda w} \notag \\
			x_2 &= \frac{w}{q_2}\cdot\beta \notag
		\end{align}
		\item Damit die inverse Elastizitätenregel angewendet werden kann, müssen die Kreuzpreiselastizitäten gleich Null sein.
		\begin{align}
			\frac{\partial x_1}{\partial q_2} &= 0 \notag \\
			\frac{\partial x_2}{\partial q_1} &= 0 \notag
		\end{align}
		\item Die Preiselastizitäten sind
		\begin{align}
			\frac{\partial x_1}{\partial q_1}\cdot\frac{q_1}{x_1} &= -\frac{w\alpha}{q_1^2}\cdot\frac{q_1}{\frac{w\alpha}{q_1}} \notag \\
			&= -\frac{w\alpha}{q_1^2}\cdot q_1\cdot\frac{q_1}{w\alpha} \notag \\
			&= -1 \notag \\
			\frac{\partial x_2}{\partial q_2}\cdot\frac{q_2}{x_1} &= -\frac{w\beta}{q_2^2}\cdot\frac{q_2}{\frac{w\beta}{q_2}} \notag \\
			&= -\frac{w\beta}{q_2^2} \cdot q_2\cdot\frac{q_2}{w\beta} \notag \\
			&= -1 \notag
		\end{align}
		\item Der optimale Steuersatz ist $\tau_k=\frac{t_k}{q_k}=\frac{\rho}{\eta_k}=\rho$. Also gilt
		\begin{align}
			\frac{t_1}{q_1} = \rho = \frac{t_2}{q_2} \quad\Rightarrow\quad \frac{t_1}{t_2}=\frac{q_1}{q_2} \notag
		\end{align}
		\item Es gilt
		\begin{align}
			T &= x_1t_1 + x_2t_2 \notag \\
			5 &= 20\alpha\frac{t_1}{q_1} + 20\beta\underbrace{\frac{t_2}{q_2}}_{\frac{t_1}{q_1}} \notag \\
			&= 20\frac{t_1}{q_1}(\alpha + \beta) \notag \\
			\frac{1}{4} &= \frac{t_1}{q_1} \notag \\
			&= \frac{t_1}{1+t_1} \notag \\
			t_1 &= \frac{1}{3} \Rightarrow t_2 = \frac{1}{3} \notag
		\end{align}
	\end{enumerate}
	
\end{document}