\documentclass{article}

\usepackage{amsmath,amssymb}
\usepackage{tikz}
\usepackage{pgfplots}
\usepackage{xcolor}
\usepackage[left=2.1cm,right=3.1cm,bottom=3cm,footskip=0.75cm,headsep=0.5cm]{geometry}
\usepackage{enumerate}
\usepackage{enumitem}
\usepackage{marvosym}
\usepackage{tabularx}

\usepackage[utf8]{inputenc}

\renewcommand*{\arraystretch}{1.4}

\newcolumntype{L}[1]{>{\raggedright\arraybackslash}p{#1}}
\newcolumntype{R}[1]{>{\raggedleft\arraybackslash}p{#1}}
\newcolumntype{C}[1]{>{\centering\let\newline\\\arraybackslash\hspace{0pt}}m{#1}}

\title{\textbf{Steuertheorie, Hausaufgabe 1}}
\author{\textsc{Henry Haustein}}
\date{}

\begin{document}
	\maketitle
	
	\section*{Aufgabe 1}
	\begin{enumerate}[label=(\alph*)]
		\item Der Grenzsteuersatz und Durchschnittssteuersatz ist
		\begin{align}
			\frac{\partial T_1}{\partial y} &= \begin{cases}
				0 & y \le y_0 \\
				0.3 & y < y_0
			\end{cases} \notag \\
			\frac{T_1}{y} &= \begin{cases}
				0 & y \le y_0 \\
				0.3 - \frac{y_0}{y} & y > y_0
			\end{cases} \notag
		\end{align}
		Es handelt sich um einen indirekt progressiven Steuertarif mit Freibetrag.
		\begin{center}
			\begin{tikzpicture}
				\begin{axis}[
					xmin=0, xmax=1, xlabel=$x$,
					ymin=0, ymax=0.5,
					samples=400,
					axis x line=middle,
					axis y line=middle,
					domain=0:1,
					%yticklabels={,,},
					xticklabels={,,},
					xtick style={draw=none},
					%ytick style={draw=none},
					restrict y to domain=0:0.5
					]
					\addplot[mark=none,smooth,blue] {0.3*(x-0.2)};
					\draw[blue] (axis cs: 0,0) -- (axis cs: 0.2,0);
					
					\addplot[mark=none,smooth,red,domain=0.2:1] {0.3};
					\addplot[mark=none,smooth,green!80!black] {0.3-0.2/(3.3*x)};
					
				\end{axis}
			\end{tikzpicture} \\
			\textcolor{blue}{Tarifverlauf}, \textcolor{red}{Grenzsteuersatz}, \textcolor{green!80!black}{Durchschnitsssteuersatz}
		\end{center}
		\item Der Grenzsteuersatz und Durchschnittssteuersatz ist
		\begin{align}
			\frac{\partial T_1}{\partial y} &= \begin{cases}
				0 & y \le y_0 \\
				0.3 & y < y_0
			\end{cases} \notag \\
			\frac{T_1}{y} &= \begin{cases}
				0 & y \le y_0 \\
				0.3 & y > y_0
			\end{cases} \notag
		\end{align}
		Es handelt sich um einen proportionalen Steuertarif.
		\begin{center}
			\begin{tikzpicture}
				\begin{axis}[
					xmin=0, xmax=1, xlabel=$x$,
					ymin=0, ymax=0.5,
					samples=400,
					axis x line=middle,
					axis y line=middle,
					domain=0:1,
					%yticklabels={,,},
					xticklabels={,,},
					xtick style={draw=none},
					%ytick style={draw=none},
					restrict y to domain=0:0.5
					]
					\addplot[mark=none,smooth,blue,domain=0.2:1] {0.3*x};
					\draw[blue] (axis cs: 0,0) -- (axis cs: 0.2,0);
					
					\addplot[mark=none,smooth,red,domain=0.2:1] {0.3};
					\addplot[mark=none,smooth,green!80!black, domain=0.2:1] {0.3};
					
				\end{axis}
			\end{tikzpicture} \\
			\textcolor{blue}{Tarifverlauf}, \textcolor{red}{Grenzsteuersatz}, \textcolor{green!80!black}{Durchschnitsssteuersatz}
		\end{center}
		\item Der Grenzsteuersatz und Durchschnittssteuersatz ist
		\begin{align}
			\frac{\partial T_1}{\partial y} &= 0.3 \notag \\
			\frac{T_1}{y} &= 0.3-\frac{y_0}{y} \notag
		\end{align}
		Es handelt sich um einen indirekt progressiven Steuertarif.
		\begin{center}
			\begin{tikzpicture}
				\begin{axis}[
					xmin=0, xmax=1, xlabel=$x$,
					ymin=-0.2, ymax=0.5,
					samples=400,
					axis x line=middle,
					axis y line=middle,
					domain=0:1,
					%yticklabels={,,},
					xticklabels={,,},
					xtick style={draw=none},
					%ytick style={draw=none},
					restrict y to domain=-0.2:0.5
					]
					\addplot[mark=none,smooth,blue] {0.3*(x-0.2)};
					
					\addplot[mark=none,smooth,red] {0.3};
					\addplot[mark=none,smooth,green!80!black] {0.3-0.2/(3.3*x)};
					
				\end{axis}
			\end{tikzpicture} \\
			\textcolor{blue}{Tarifverlauf}, \textcolor{red}{Grenzsteuersatz}, \textcolor{green!80!black}{Durchschnitsssteuersatz}
		\end{center}
		\item Der Grenzsteuersatz und Durchschnittssteuersatz ist
		\begin{align}
			\frac{\partial T_1}{\partial y} &= 0.6y \notag \\
			\frac{T_1}{y} &= 0.3y \notag
		\end{align}
		Es handelt sich um einen direkt progressiven Steuertarif.
		\begin{center}
			\begin{tikzpicture}
				\begin{axis}[
					xmin=0, xmax=1, xlabel=$x$,
					ymin=0, ymax=0.5,
					samples=400,
					axis x line=middle,
					axis y line=middle,
					domain=0:1,
					%yticklabels={,,},
					xticklabels={,,},
					xtick style={draw=none},
					%ytick style={draw=none},
					restrict y to domain=0:0.5
					]
					\addplot[mark=none,smooth,blue] {0.3*x^2};
					
					\addplot[mark=none,smooth,red] {0.6*x};
					\addplot[mark=none,smooth,green!80!black] {0.3*x};
					
				\end{axis}
			\end{tikzpicture} \\
			\textcolor{blue}{Tarifverlauf}, \textcolor{red}{Grenzsteuersatz}, \textcolor{green!80!black}{Durchschnitsssteuersatz}
		\end{center}
	\end{enumerate}

	\section*{Aufgabe 2}
	\begin{enumerate}[label=(\alph*)]
		\item Bei einer Bruttowertsteuer gilt $p=q(1-\tau)$ und damit
		\begin{align}
			\frac{p-q}{q} = \frac{q(1-\tau)-q}{q} = (1-\tau)-1 = -\tau \notag
		\end{align}
		Bei einer Nettowertsteuer gilt $q=p(1+\theta)$ und damit
		\begin{align}
			\frac{p-q}{q} = \frac{p - p(1+\theta)}{p(1+\theta)} = \frac{1-(1+\theta)}{1+\theta} = \frac{-\theta}{1+\theta} \notag
		\end{align}
		Die Umrechnung sieht wie folgt aus
		\begin{align}
			q(1-\tau) &= \frac{q}{1+\theta} \notag \\
			1-\tau &= \frac{1}{1+\theta} \notag \\
			\Rightarrow \tau &= 1-\frac{1}{1+\theta} \notag \\
			\Rightarrow \theta &= \frac{1}{1-\tau}-1 \notag
		\end{align}
		\item Es handelt sich um eine Nettowertsteuer: Die Umsatzsteuer kommt auf den Nettopreis drauf. In 100 \EUR\, Bruttowert sind 84.03 \EUR\, Nettowert und 15.97 \EUR\, Umsatzsteuer enthalten. Die Quote beträgt also 15.97 \%.
	\end{enumerate}

	\section*{Aufgabe 3}
	Zuerst der Beweis:
	\begin{align}
		\frac{\partial T}{\partial y} = \frac{\partial (t\cdot y)}{\partial y} = \underbrace{\frac{\partial t}{\partial y}}_{0}\cdot y + t\cdot\underbrace{\frac{\partial y}{\partial y}}_{1} = t \notag
	\end{align}
	\begin{enumerate}[label=(\alph*)]
		\item Die Steueraufkommenselastizität gibt an, um wie viel Prozent sich das Steueraufkommen ändert, wenn sich die Bemessungsgrundlage um 1 Prozent ändert.
		\item Es gilt $\alpha(y)=T'(y)\cdot \frac{1}{t(y)}$. Damit gilt für einen proportionalen Steuertarif $\alpha=1$, für einen progressiven $\alpha>1$ und für einen regressiven $\alpha<1$.
		\item Es gilt
		\begin{align}
			\rho(y) &= \frac{\partial x}{\partial y} \cdot\frac{y}{x} \notag \\
			&= \frac{\partial (y-T(y))}{\partial y} \cdot\frac{y}{y-T(y)} \notag \\
			&= \left(\frac{\partial y}{\partial y} - \frac{\partial T(y)}{\partial y}\right)\cdot\frac{y}{y-T(y)} \notag \\
			&= \frac{y-T'(y)\cdot y\cdot\frac{t}{t}}{y-T(y)} \notag \\
			&= \frac{y-\alpha(y)\cdot T(y)}{y-T(y)} \notag
		\end{align}
		Wenn $\alpha>1$, dann Zähler $<$ Nenner und damit $\rho<1$. \\
		Wenn $\alpha=1$, dann Zähler $=$ Nenner und damit $\rho=1$. \\
		Wenn $\alpha<1$, dann Zähler $>$ Nenner und damit $\rho>1$.
		\item Für $T(y)=a\cdot y^\beta$ gilt
		\begin{align}
			\alpha(y) = \frac{a\cdot\beta\cdot y^{\beta-1}\cdot y}{a\cdot y^\beta} = \frac{a\cdot\beta\cdot y^\beta}{a\cdot y^\beta} = \beta \notag
		\end{align}
		Für $T(y)=y-ay^p$ ist $x=y-(y-ax^p)=ay^p$ und damit
		\begin{align}
			\rho(x) = \frac{a\cdot p\cdot y^{p-1}\cdot y}{a\cdot y^p} = \frac{a\cdot p\cdot y^p}{a\cdot y^p} = p \notag
		\end{align}
	\end{enumerate}

	\section*{Aufgabe 4}
	\begin{enumerate}[label=(\alph*)]
		\item Es gilt $T=\theta\cdot w^N\cdot N(w^B)$ und damit
		\begin{align}
			\frac{\partial T}{\partial\theta} &= w^N\cdot N(w^B) + \theta\cdot w^N\cdot\frac{\partial N(w^B)}{\partial w^B}\underbrace{\frac{\partial w^B}{\partial \theta}}_{w^N} \notag \\
			&= w^N\cdot N(w^B)\left(1+\theta\cdot \underbrace{w^N}_{\frac{w^B}{1+\theta}} \cdot \frac{\partial N(w^B)}{\partial w^B}\cdot\frac{1}{N(w^B)}\right) \notag \\
			&= w^N\cdot N(w^B)\left(1+\frac{\theta}{1+\theta}\cdot \underbrace{w^B\cdot\frac{\partial N(w^B)}{\partial w^B}\cdot\frac{1}{N(w^B)}}_{\eta}\right) \notag \\
			&= w^N\cdot N(w^B)\left(1+\frac{\theta}{1+\theta}\eta\right) \notag
		\end{align}
		Das Steueraufkommen sinkt genau dann, wenn
		\begin{align}
			1+\frac{\theta}{1+\theta}\eta &< 0 \notag \\
			1+\frac{\theta}{1+\theta}(-3) &< 0 \notag \\
			-\frac{3\theta}{1+\theta} &< -1 \notag \\
			-3\theta &< -(1+\theta) \notag \\
			-2\theta &< -1 \notag \\
			\theta &> \frac{1}{2} \notag
		\end{align}
		\item $\eta=-3$ ist schon sehr unelastisch, wenn die nachfrage noch unelastischer wird, kann der Steuersatz noch weiter angehoben werden, die Nachfrage reagiert aber kaum $\Rightarrow$ Steueraufkommen steigt.
	\end{enumerate}

	\section*{Aufgabe 5}
	\begin{enumerate}[label=(\alph*)]
		\item Der Staat maximiert $T=x(GZB(x) - GK(x))$, also
		\begin{align}
			\frac{\partial T}{\partial x} &= GZB'\cdot x + GZB - (GK' \cdot x) = 0 \notag \\
			GZB + GZB'\cdot x &= GK + GK'\cdot x \notag
		\end{align}
		Die Grenzausgabenfunktion ist dann $A'=GK + GK'\cdot x = 2+x+x=2+2x$.
		\item Die Grenzerlösfunktion ist $E'=GZB + GZB'\cdot x = 14-2x-2x = 14-4x$.
		\item Im Gleichgewicht gilt dann
		\begin{align}
			E' &= A' \notag \\
			2+2x &= 14-4x \notag \\
			12 &= 6x \notag \\
			x &= 2 \notag
		\end{align}
		Die Steuer ist dann $GZB(2) - GK(2) = 6$ und das Steueraufkommen $T=2\cdot 6=12$.
		\item Diagramm
		\begin{center}
			\begin{tikzpicture}[scale=0.9]
				\begin{axis}[
					xmin=0, xmax=8,
					ymin=0, ymax=15,
					samples=400,
					axis x line=middle,
					axis y line=middle,
					domain=0:8,
					]
					\addplot[mark=none,smooth,blue] {14-4*x};
					\addplot[mark=none,smooth,blue,dashed] {14-2*x};
					\addplot[mark=none,smooth,red] {2+2*x};
					\addplot[mark=none,smooth,red,dashed] {2+x};
					
					\draw[dotted] (axis cs: 2,0) -- (axis cs: 2,10) -- (axis cs: 0,10);
					\draw[dotted] (axis cs: 2,4) -- (axis cs: 0,4);
					
					\draw[<->] (axis cs: 0.5,4) to node[midway,right] {$t$} (axis cs: 0.5,10);
					
				\end{axis}
			\end{tikzpicture} \\
			\textcolor{blue}{Grenzausgabenfunktion/Grenzzahlungbereitschaft}, \textcolor{red}{Grenzerlösfunktion/Grenzkosten}
		\end{center}
	\end{enumerate}
	
\end{document}