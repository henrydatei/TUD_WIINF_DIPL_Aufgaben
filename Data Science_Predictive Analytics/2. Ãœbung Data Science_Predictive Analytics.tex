\documentclass{article}

\usepackage{amsmath,amssymb}
\usepackage{tikz}
\usepackage{pgfplots}
\usepackage{xcolor}
\usepackage[left=2.1cm,right=3.1cm,bottom=3cm,footskip=0.75cm,headsep=0.5cm]{geometry}
\usepackage{enumerate}
\usepackage{enumitem}
\usepackage{marvosym}
\usepackage{tabularx}
\usepackage{parskip}

\usepackage{listings}
\definecolor{lightlightgray}{rgb}{0.95,0.95,0.95}
\definecolor{lila}{rgb}{0.8,0,0.8}
\definecolor{mygray}{rgb}{0.5,0.5,0.5}
\definecolor{mygreen}{rgb}{0,0.8,0.26}
%\lstdefinestyle{java} {language=java}
\lstset{language=python,
	basicstyle=\ttfamily,
	keywordstyle=\color{lila},
	commentstyle=\color{lightgray},
	stringstyle=\color{mygreen}\ttfamily,
	backgroundcolor=\color{white},
	showstringspaces=false,
	numbers=left,
	numbersep=10pt,
	numberstyle=\color{mygray}\ttfamily,
	identifierstyle=\color{blue},
	xleftmargin=.1\textwidth, 
	%xrightmargin=.1\textwidth,
	escapechar=§,
	%literate={\t}{{\ }}1
	breaklines=true,
	postbreak=\mbox{\space},
	morekeywords={as}
}
\lstset{literate=%
	{Ö}{{\"O}}1
	{Ä}{{\"A}}1
	{Ü}{{\"U}}1
	{ß}{{\ss}}1
	{ü}{{\"u}}1
	{ä}{{\"a}}1
	{ö}{{\"o}}1
}

\usepackage[utf8]{inputenc}

\renewcommand*{\arraystretch}{1.4}

\newcolumntype{L}[1]{>{\raggedright\arraybackslash}p{#1}}
\newcolumntype{R}[1]{>{\raggedleft\arraybackslash}p{#1}}
\newcolumntype{C}[1]{>{\centering\let\newline\\\arraybackslash\hspace{0pt}}m{#1}}

\newcommand{\E}{\mathbb{E}}
\DeclareMathOperator{\rk}{rk}
\DeclareMathOperator{\Var}{Var}
\DeclareMathOperator{\Cov}{Cov}

\title{\textbf{Data Science: Predictive Analytics, Übung 2}}
\author{\textsc{Henry Haustein}}
\date{}

\begin{document}
	\maketitle

	\section*{Data Preparation}
	\begin{lstlisting}[tabsize=2]
import pandas as pd
import matplotlib.pyplot as plt
from datetime import datetime
import time
from sklearn.impute import KNNImputer
import numpy as np
from sklearn.preprocessing import LabelEncoder
from sklearn.model_selection import train_test_split

# von Übung 1
def fixCols(cols):
	liste = []
	for col in cols:
		newName = col.split("(")[0].strip().replace(" ", "_")
		liste.append(newName)
	return liste

data = pd.read_csv("dataset.csv", delimiter = ";", header = 0, index_col = 0, decimal = ",")
data.columns = fixCols(data.columns)
# print(data.dtypes)

# Übung 2

# 1
data["Date"] = pd.to_datetime(data["Date"])

# 2
data["weekday"] = data["Date"].dt.day_name()
dataDays = pd.DataFrame(data.groupby("weekday").sum()["Rented_Bike_Count"])
dataDays["daynumber"] = [time.strptime(x, "%A").tm_wday for x in list(dataDays.index.values)]
dataDays.sort_values("daynumber").iloc[:,0].plot()
plt.show()

# 3
def time2day(hour):
	if 6 <= hour and hour <= 22:
		return "day"
	else:
		return "night"

data["dayNight"] = data["Hour"].apply(time2day)
data.groupby("dayNight").sum()["Rented_Bike_Count"].plot.bar(x = "dayNight", y = "Rented_Bike_Count")
plt.show()

# 4
# wozu soll bitte die Tautemperatur notwendig sein? Also weg damit
data = data.drop(columns = ["Hotness", "Dew_point_temperature"])

# 5 + 6
data["Solar_Radiation"] = data["Solar_Radiation"].apply(lambda x: np.NaN if (x == "failure") else(float(x.replace(",", "."))))
# print(data.isna().sum())
imputer = KNNImputer(n_neighbors = 3, weights = "distance")
data[["Temperature", "Humidity", "Wind_speed", "Visibility", "Rainfall", "Snowfall", "Solar_Radiation"]] = imputer.fit_transform(data[["Temperature", "Humidity", "Wind_speed", "Visibility", "Rainfall", "Snowfall", "Solar_Radiation"]])
# print(data.isna().sum())

# 7
def fixTemp(temp):
	if temp > 100:
		return temp/10
	else:
		return temp

data["Temperature"] = data["Temperature"].apply(fixTemp)

# 8
data["Functioning_Day"] = data["Functioning_Day"].astype(bool)
data.loc[data["Seasons"] == "Herbst","Seasons"] = "Autumn"

# 9
encoder = LabelEncoder()
data["Seasons"] = encoder.fit_transform(data["Seasons"])
data["weekday"] = encoder.fit_transform(data["weekday"])
data["dayNight"] = encoder.fit_transform(data["dayNight"])
data["Holiday"] = encoder.fit_transform(data["Holiday"])
X = data[["Hour", "Temperature", "Wind_speed", "Humidity", "Visibility", "Solar_Radiation", "Rainfall", "Snowfall", "Seasons", "weekday", "dayNight", "Holiday"]]
y = data["Rented_Bike_Count"]
x_train, x_test, y_train, y_test = train_test_split(X, y, test_size = 0.3)

# print(data)
	\end{lstlisting}
	
\end{document}