\documentclass{article}

\usepackage{amsmath,amssymb}
\usepackage{tikz}
\usepackage{pgfplots}
\usepackage{xcolor}
\usepackage[left=2.1cm,right=3.1cm,bottom=3cm,footskip=0.75cm,headsep=0.5cm]{geometry}
\usepackage{enumerate}
\usepackage{enumitem}
\usepackage{marvosym}
\usepackage{tabularx}
\usepackage{parskip}
\usepackage{longtable}

\usepackage{listings}
\definecolor{lightlightgray}{rgb}{0.95,0.95,0.95}
\definecolor{lila}{rgb}{0.8,0,0.8}
\definecolor{mygray}{rgb}{0.5,0.5,0.5}
\definecolor{mygreen}{rgb}{0,0.8,0.26}
%\lstdefinestyle{java} {language=java}
\lstset{language=R,
	basicstyle=\ttfamily,
	keywordstyle=\color{lila},
	commentstyle=\color{lightgray},
	stringstyle=\color{mygreen}\ttfamily,
	backgroundcolor=\color{white},
	showstringspaces=false,
	numbers=left,
	numbersep=10pt,
	numberstyle=\color{mygray}\ttfamily,
	identifierstyle=\color{blue},
	xleftmargin=.1\textwidth, 
	%xrightmargin=.1\textwidth,
	escapechar=§,
	%literate={\t}{{\ }}1
	breaklines=true,
	postbreak=\mbox{\space}
}

\usepackage[colorlinks = true, linkcolor = blue, urlcolor  = blue, citecolor = blue, anchorcolor = blue]{hyperref}
\usepackage[utf8]{inputenc}

\renewcommand*{\arraystretch}{1.4}

\newcolumntype{L}[1]{>{\raggedright\arraybackslash}p{#1}}
\newcolumntype{R}[1]{>{\raggedleft\arraybackslash}p{#1}}
\newcolumntype{C}[1]{>{\centering\let\newline\\\arraybackslash\hspace{0pt}}m{#1}}

\newcommand{\E}{\mathbb{E}}
\DeclareMathOperator{\rk}{rk}
\DeclareMathOperator{\Var}{Var}
\DeclareMathOperator{\Cov}{Cov}

\title{\textbf{ERP-Planspiel, Short Exam}}
\author{\textsc{Henry Haustein}, \textsc{Dennis Rössel}, \textsc{Hannes Metzner}, \textsc{Felix Hauspurg}}
\date{}

\begin{document}
	\maketitle
	
	\section*{Question 1}
	\begin{enumerate}[label=(\alph*)]
		\item wrong, oat has to be at leat 30\%, blueberry has to be at least 20\%
		\item wrong, the sum of all ingredients is 0.55kg
		\item correct
		\item wrong, oat is missing, the raisins are not part of the recipe
	\end{enumerate}

	\section*{Question 2}
	\begin{enumerate}[label=(\alph*)]
		\item After step 1: As we have already the missing 30,000 units of blueberry muesli in production, there is nothing else to do.
		\item After step 1: We need to produce 30,000 more units of original muesli. \\
		After step 2: We have the ingredients for the muesli but we need 30,000 boxes and bags which we currently don't have in our inventory. We buy the 30,000 boxes and bags. \\
		After step 3: After delivery we start the production.
	\end{enumerate}
	
	\section*{Question 3}
	\textit{You should stick to your own conventions: In the American number system 30.000 units means 30 units....}
	\begin{center}
		\begin{longtable}{l|C{3cm}|C{3cm}|C{3cm}|C{3cm}}
			& \textbf{selling 30,000 units} & \textbf{invest 2M} & \textbf{reduce setup time} & \textbf{short on cash} \\
			\hline
			(a) Credit rating & positive. If we have debt the debt is reduced and so our credit rating improves. & no influence, cash is reduced but Acc. receivables increase by same amount & no influence, cash is reduced but Acc. receivables increase by same amount & credit rating is getting worse because you need a loan \\
			\hline
			(b) weekly costs & no influence. Costs stay the same. & weekly costs increase as bigger machines need more employees & weekly costs increase as faster machines need more employees, use more energy, ... & increase because we have to pay interests \\
			\hline
			(c) cash account & positive, positive cash income & cash is reduced by 2M & cash is reduced by invested amount & is on 0 because we get a loan to get back to 0 \\
			\hline
			(d) loan & positive, loan is reduced (if we are in debt) & if we have cash, then no influence, if not, then we need an additional loan & if we have cash, then no influence, if not, then we need an additional loan & increases, we need more loans \\
		\end{longtable}
	\end{center}
	
	\section*{Question 4}
	\begin{enumerate}[label=(\alph*)]
		\item calculation below:
		\begin{align}
			\text{Debt Loading} &= (1,000,000 + 500,000) + (-12,000,000 - 150,000) \notag \\
			&= -10,650,000 \Rightarrow BB \Rightarrow 7\% \notag \\
			\text{Company Value} &= \frac{12\cdot \frac{650,000 - 0}{3} + 0}{7\% + 7\%} \notag \\
			&= 18,571,428.57 \EUR \notag
		\end{align}
		\item calculation below, we pay the investment of 2M from cash:
		\begin{align}
			\text{Debt Loading} &= (800,000 + 240,000) + (-12,000,000 - 450,000) \notag \\
			&= -11,410,000 \Rightarrow BB- \Rightarrow 7.25\% \notag \\
			\text{Company Value} &= \frac{12\cdot \frac{350,000 - 2,000,000}{1} + 2,000,000}{7\% + 7,25\%} \notag \\
			&= -124,912,280.7 \EUR \notag
		\end{align}
	\end{enumerate}
	
	\section*{Question 5}
	\begin{enumerate}[label=(\alph*)]
		\item fixed costs change to 1.18 per unit, so 500g Nut Muesli will cost 2.08 and 1kg Original Muesli will cost 2.54
		\item fixed costs change to 2.06 per unit, so 500g Nut Muesli will cost 2.96 and 1kg Original Muesli will cost 3.42
		\item fixed costs change to 1.34 per unit, so 500g Nut Muesli will cost 2.24 and 1kg Original Muesli will cost 2.70
	\end{enumerate}

\end{document}