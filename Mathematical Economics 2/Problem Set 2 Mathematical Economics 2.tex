\documentclass{article}

\usepackage{amsmath,amssymb}
\usepackage{tikz}
\usepackage{pgfplots}
\usepackage{xcolor}
\usepackage[left=2.1cm,right=3.1cm,bottom=3cm,footskip=0.75cm,headsep=0.5cm]{geometry}
\usepackage{enumerate}
\usepackage{enumitem}
\usepackage{marvosym}
\usepackage{tabularx}
\usepackage{parskip}
\usepackage{multirow}

\usepackage{listings}
\definecolor{lightlightgray}{rgb}{0.95,0.95,0.95}
\definecolor{lila}{rgb}{0.8,0,0.8}
\definecolor{mygray}{rgb}{0.5,0.5,0.5}
\definecolor{mygreen}{rgb}{0,0.8,0.26}
%\lstdefinestyle{java} {language=java}
\lstset{language=R,
	basicstyle=\ttfamily,
	keywordstyle=\color{lila},
	commentstyle=\color{lightgray},
	stringstyle=\color{mygreen}\ttfamily,
	backgroundcolor=\color{white},
	showstringspaces=false,
	numbers=left,
	numbersep=10pt,
	numberstyle=\color{mygray}\ttfamily,
	identifierstyle=\color{blue},
	xleftmargin=.1\textwidth, 
	%xrightmargin=.1\textwidth,
	escapechar=§,
	%literate={\t}{{\ }}1
	breaklines=true,
	postbreak=\mbox{\space}
}

\usepackage[colorlinks = true, linkcolor = blue, urlcolor  = blue, citecolor = blue, anchorcolor = blue]{hyperref}
\usepackage[utf8]{inputenc}

\renewcommand*{\arraystretch}{1.4}

\newcolumntype{L}[1]{>{\raggedright\arraybackslash}p{#1}}
\newcolumntype{R}[1]{>{\raggedleft\arraybackslash}p{#1}}
\newcolumntype{C}[1]{>{\centering\let\newline\\\arraybackslash\hspace{0pt}}m{#1}}

\newcommand{\E}{\mathbb{E}}
\DeclareMathOperator{\rk}{rk}
\DeclareMathOperator{\Var}{Var}
\DeclareMathOperator{\Cov}{Cov}

\title{\textbf{Mathematical Economics 2, Problem Set 2}}
\author{\textsc{Henry Haustein}}
\date{}

\begin{document}
	\maketitle
	
	\section*{Task 1}
	\begin{enumerate}[label=(\alph*)]
		\item Total Action (TA): contribution to a public good
		\begin{align}
			u_i(\sigma_i, \sigma_1, ..., \sigma_n) = f\left(\sigma_i + \alpha\sum_{j\in N_i} \sigma_j\right) - c\sigma_i\notag
		\end{align}
		Best Effort (BE):
		\begin{align}
			u_i(\sigma_i, \sigma_1, ..., \sigma_n) = f\left(\max_{j\in N_i\cup \{i\}} \sigma_j\right) - c\sigma_i \notag
		\end{align}
		Average Effort (AE):
		\begin{align}
			u_i(\sigma_i, \sigma_1, ..., \sigma_n) = f\left(\sigma_i\frac{\sum_{j\in N_i} \sigma_j}{k_i}\right) - c\sigma_i \notag
		\end{align}
		Weakest Link (WL): connected banks, one goes bankrupt
		\begin{align}
			u_i(\sigma_i, \sigma_1, ..., \sigma_n) = f\left(\min_{j\in N_i\cup \{i\}} \sigma_j\right) - c\sigma_i \notag
		\end{align}
		\item TA: if the function is concave we get substitutes, if the function is convex we get complements. Property A is satisfied. \\
		BE: strategic substitutes. Property A is satisfied. \\
		AE: strategic complements. Property A is not satisfied. \\
		WL: strategic complements. Property A is not always satisfied.
		\item Table
		\begin{center}
			\begin{tabular}{c|C{5cm}|C{5cm}}
				& \textbf{All 0} & \textbf{All 1} \\
				\hline
				\textbf{TA} & $c\ge b$ & $b\ge c$ \\
				\hline
				\textbf{BE} & $c\ge b$ & never happen when $n>1$ \\
				\hline
				\textbf{AE} & always & $b\ge c$,  \\
				\hline
				\textbf{WL} & always & $b\ge c$
			\end{tabular}
		\end{center}
		\item[(Extra)] non-decreasing: If your degree is bigger then a certain $k$ then you play 1 otherwise 0. \\
		non-increasing: If your degree is bigger then a certain $k$ then you play 0 otherwise 1.
		\begin{itemize}
			\item TA: function concave: non-increasing, convex: non-decreasing
			\item BE: non-increasing
			\item AE: cannot say because it violates property A
			\item WL: non-increasing/cannot say because it violates property A
		\end{itemize}
	\end{enumerate}

	\section*{Task 2}
	\begin{enumerate}[label=(\alph*)]
		\item $u_i(\Gamma) = \sum_{j\neq i} u(d(i,j\mid \Gamma)) - k_ic$
		\item Grand star: 1 hub and $n-1$ spokes. 
		\begin{align}
			u(\text{hub}) &= (n-1)(u(1) - c) \notag \\
			u(\text{spoke}) &= (u(1) - c) + (n-2)u(2) \notag \\
			\text{total utility} &= (n-1)[u(1) - c + (n-2)u(2)] + (n-1)[u(1) - c] \notag \\
			&= (n-1)[(n-2)u(2) + 2(u(1) - c)] \notag
		\end{align}
		Empty network has a utility of 0, when is grand star better?
		\begin{align}
			(n-1)[(n-2)u(2) + 2(u(1) - c)] &> 0 \notag \\
			\underbrace{2[u(1) - c]}_{\text{negative}} + (n-2)u(2) &> 0 \notag \\
			(n-2)u(2) &> 2c - 2u(1) \notag \\
			c&< \frac{(n-2)u(2) + 2u(1)}{2} \notag
		\end{align}
		\begin{enumerate}[label=(\roman*)]
			\item efficient network is complete network
			\item efficient network is grand star
			\item efficient network is empty network
		\end{enumerate}
		\item If $c<u(1) - u(2)$ then a complete network will form. If $c\in (u(1) - u(2), u(1))$ then the hub in a grand star in receiving positive utility and the grand star is pairwise stable. If $c>u(1)$ then the hub will break all links resulting in an empty network.
	\end{enumerate}

\end{document}