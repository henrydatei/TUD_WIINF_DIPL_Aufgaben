\documentclass{article}

\usepackage{amsmath,amssymb}
\usepackage{tikz}
\usepackage{pgfplots}
\usepackage{xcolor}
\usepackage[left=2.1cm,right=3.1cm,bottom=3cm,footskip=0.75cm,headsep=0.5cm]{geometry}
\usepackage{enumerate}
\usepackage{enumitem}
\usepackage{marvosym}
\usepackage{tabularx}
\usepackage{parskip}
\usepackage{multirow}

\usepackage{listings}
\definecolor{lightlightgray}{rgb}{0.95,0.95,0.95}
\definecolor{lila}{rgb}{0.8,0,0.8}
\definecolor{mygray}{rgb}{0.5,0.5,0.5}
\definecolor{mygreen}{rgb}{0,0.8,0.26}
%\lstdefinestyle{java} {language=java}
\lstset{language=R,
	basicstyle=\ttfamily,
	keywordstyle=\color{lila},
	commentstyle=\color{lightgray},
	stringstyle=\color{mygreen}\ttfamily,
	backgroundcolor=\color{white},
	showstringspaces=false,
	numbers=left,
	numbersep=10pt,
	numberstyle=\color{mygray}\ttfamily,
	identifierstyle=\color{blue},
	xleftmargin=.1\textwidth, 
	%xrightmargin=.1\textwidth,
	escapechar=§,
	%literate={\t}{{\ }}1
	breaklines=true,
	postbreak=\mbox{\space}
}

\usepackage[colorlinks = true, linkcolor = blue, urlcolor  = blue, citecolor = blue, anchorcolor = blue]{hyperref}
\usepackage[utf8]{inputenc}

\renewcommand*{\arraystretch}{1.4}

\newcolumntype{L}[1]{>{\raggedright\arraybackslash}p{#1}}
\newcolumntype{R}[1]{>{\raggedleft\arraybackslash}p{#1}}
\newcolumntype{C}[1]{>{\centering\let\newline\\\arraybackslash\hspace{0pt}}m{#1}}

\newcommand{\E}{\mathbb{E}}
\DeclareMathOperator{\rk}{rk}
\DeclareMathOperator{\Var}{Var}
\DeclareMathOperator{\Cov}{Cov}

\title{\textbf{Mathematical Economics 2, Problem Set 3}}
\author{\textsc{Henry Haustein}}
\date{}

\begin{document}
	\maketitle
	
	\section*{Task 1}
	\begin{enumerate}[label=(\alph*)]
		\item matrix:
		\begin{align}
			\begin{pmatrix}
				2,2 & 0,2 & 0,0 \\
				2,0 & 0,0 & 0,3 \\
				0,0 & 3,0 & 4,4
			\end{pmatrix} \notag
		\end{align}
		\item Population space is a triangle
		\item pure $NE = \{e_1,e_3\}$, mixed NE:
		\begin{itemize}
			\item NE of form $(x,1-x,0)$: $u(e_1,x) = 2x = u(e_2,x)$, $u(e_3,x) = 3(1-x) \Rightarrow x\ge \frac{3}{5}$
			\item NE of form $(x,0,1-x)$: $u(e_1,x) = 2x = u(e_2,x)$, $u(e_3,x) = 4(1-x) \Rightarrow x= \frac{2}{3}$
			\item NE of form $(0,x,1-x)$: $e_3$ does better than $e_2$ $\Rightarrow$ no reason to play $e_2$
			\item NE of form $(x_1,x_2,1-x_1-x_2)$: $u(e_1,x) = 2x_1 = u(e_2,x)$, $u(e_3,x) = 3x_2 + 4(1-x_1-x_2) \Rightarrow 6x_1+x_2 = 4$
		\end{itemize}
		\item $e_3$ is a strict NE $\Rightarrow$ $e_3$ is ESS and NSS \\
		$\left(\frac{2}{3},0,\frac{1}{3}\right)$ is neither ESS nor NSS ($e_3$ can invade), same for all points $(x_1,x_2,1-x_1-x_2)$ on $6x_1+x_2=4$ \\
		$e_1$ is NSS because you can't resist $e_2$ mutants but they can't invade, same for points $(x,1-x,0)$ on the line $x\ge \frac{3}{5}$ \\
		$\left(\frac{3}{5},\frac{2}{5},0\right)$ is neither ESS nor NSS because $e_3$ can invade
		\item $u(x,x) = (x_1+x_2)(2x_1) + (1-x_1-x_2)(3x_2 + 4(1-x_1-x_2))$ \\
		$\dot{x}_1 = x_1[u(e_1,x) - u(x,x)] = x_1(1-x_1-x_2)(6x_1+x_2-4)$
		\item Sets
		\begin{itemize}
			\item Stationary: all NE, points between $e_1$ and $e_2$
			\item Lyapunov stable: NSS
			\item Asymptotically stable: $e_3$
			\item Limit states: $e_3$, points $(x,1-x,0)$ with $x\ge \frac{3}{5}$ but not $e_1$, points $(x_1,x_2,1-x_1-x_2)$ with $6x_1+x_2=4$
		\end{itemize}
	\end{enumerate}

\end{document}