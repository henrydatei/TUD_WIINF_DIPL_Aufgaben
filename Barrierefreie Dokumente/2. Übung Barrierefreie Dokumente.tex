\documentclass{article}

\usepackage{amsmath,amssymb}
\usepackage{tikz}
\usepackage{pgfplots}
\usepackage{xcolor}
\usepackage[left=2.1cm,right=3.1cm,bottom=3cm,footskip=0.75cm,headsep=0.5cm]{geometry}
\usepackage{enumerate}
\usepackage{enumitem}
\usepackage{marvosym}
\usepackage{tabularx}
\usepackage{multirow}
\usepackage[colorlinks = true, linkcolor = blue, urlcolor  = blue, citecolor = blue, anchorcolor = blue]{hyperref}
\usepackage{ulem}

\usepackage{listings}
\definecolor{lightlightgray}{rgb}{0.95,0.95,0.95}
\definecolor{lila}{rgb}{0.8,0,0.8}
\definecolor{mygray}{rgb}{0.5,0.5,0.5}
\definecolor{mygreen}{rgb}{0,0.8,0.26}
\lstdefinestyle{html} {language=html}
\lstset{language=html,
	basicstyle=\ttfamily,
	keywordstyle=\color{lila},
	commentstyle=\color{lightgray},
	stringstyle=\color{mygreen}\ttfamily,
	backgroundcolor=\color{white},
	showstringspaces=false,
	numbers=left,
	numbersep=10pt,
	numberstyle=\color{mygray}\ttfamily,
	identifierstyle=\color{blue},
	xleftmargin=.1\textwidth, 
	%xrightmargin=.1\textwidth,
	escapechar=§,
	breaklines=true,
	postbreak=\mbox{\space}
}
\lstset{literate=%
	{Ö}{{\"O}}1
	{Ä}{{\"A}}1
	{Ü}{{\"U}}1
	{ß}{{\ss}}1
	{ü}{{\"u}}1
	{ä}{{\"a}}1
	{ö}{{\"o}}1
}

\hypersetup{
	pdfauthor   = {Henry Haustein, Dennis Rössel},
	pdfkeywords = {Formular, Barrierefrei},
	pdftitle    = {PDF-Formular},
	pdflang={de-DE}
}

\usepackage[utf8]{inputenc}
\usepackage[T1]{fontenc}

\renewcommand*{\arraystretch}{1.4}

\newcolumntype{L}[1]{>{\raggedright\arraybackslash}p{#1}}
\newcolumntype{R}[1]{>{\raggedleft\arraybackslash}p{#1}}
\newcolumntype{C}[1]{>{\centering\let\newline\\\arraybackslash\hspace{0pt}}m{#1}}

\newcommand{\E}{\mathbb{E}}
\DeclareMathOperator{\rk}{rk}
\DeclareMathOperator{\Var}{Var}
\DeclareMathOperator{\Cov}{Cov}
\DeclareMathOperator{\SD}{SD}
\DeclareMathOperator{\Cor}{Cor}

\title{\textbf{Barrierefreie Dokumente, Übung 2}}
\author{\textsc{Henry Haustein}, \textsc{Dennis Rössel}}
\date{}

\begin{document}
	\maketitle
	
	\section*{Aufgabe 2.2: Barrierefreies PDF-Formular}
	\begin{Form}[encoding=html]
		\subsection*{Angaben zur Person}
		\begin{tabbing}
			xxxxxxxxxxxxx: \= \kill  % This is needed for the right tab width
			Anrede:    \>
			\ChoiceMenu[radio,name=sex]{\mbox{}}{Herr=m,Frau=f} \\
			Name:           \> \TextField[name=name,width=3cm]
			{\mbox{}} \\
			Vorname:           \> \TextField[name=vor,width=3cm]
			{\mbox{}} \\
			
			Geburtsdatum:           \>
			\ChoiceMenu[combo,name=day,width=1cm,default=01]{\mbox{}}
			{01,02,03,04,05,06,07,08,09,10,11,12,13,14,15,16,17,18,19,20,21,22,23,24,25,26,27,28,29,30,31}
			\ChoiceMenu[combo,name=month,width=1cm,default=01]{\mbox{}}
			{01,02,03,04,05,06,07,08,09,10,11,12}
			 \TextField[name=year,width=3cm]{\mbox{}}
		\end{tabbing}
		\subsection*{Anschrift}
		\begin{tabbing}
			xxxxxxxxxxxxx: \= \kill  % This is needed for the right tab width
			Straße:           \> \TextField[name=street,width=3cm]{\mbox{}} \\
			Hausnummer:           \> \TextField[name=number,width=3cm]{\mbox{}} \\
			Postleitzahl:           \>\TextField[name=plz,width=3cm]{\mbox{}} \\
			Ort:           \>\TextField[name=ort,width=3cm]{\mbox{}}
		\end{tabbing}
		\subsection*{Kontaktdaten}
		\begin{tabbing}
			xxxxxxxxxxxxx: \= \kill  % This is needed for the right tab width
			Telefonnummer:           \> \TextField[name=telefon,width=3cm]{\mbox{}} \\
			Email:           \> \TextField[name=email,width=3cm]{\mbox{}}
		\end{tabbing}
		\subsection*{Bankdaten}
		\begin{tabbing}
			xxxxxxxxxxxxx: \= \kill  % This is needed for the right tab width
			Kontoinhaber:           \> \TextField[name=street,width=3cm]{\mbox{}} \\
			IBAN:           \> \TextField[name=number,width=3cm]{\mbox{}} \\
			BIC:           \>\TextField[name=plz,width=3cm]{\mbox{}} \\
			Kreditinstitut:           \>\TextField[name=ort,width=3cm]{\mbox{}}
		\end{tabbing}
		\subsection*{Vertragsoptionen}
		\begin{tabbing}
			xxxxxxxxxxxxx: \= \kill  % This is needed for the right tab width
			Vertragsoptionen:           \> \ChoiceMenu[radio,name=vertrag]{\mbox{}}{Newsletter=n,Getränkeflatrate=g,Kursflatrate=k}
		\end{tabbing}
	\end{Form}
	
\end{document}