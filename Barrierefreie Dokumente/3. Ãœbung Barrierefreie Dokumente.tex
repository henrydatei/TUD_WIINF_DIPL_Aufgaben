\documentclass{article}

\usepackage{amsmath,amssymb}
\usepackage{tikz}
\usepackage{pgfplots}
\usepackage{xcolor}
\usepackage[left=2.1cm,right=3.1cm,bottom=3cm,footskip=0.75cm,headsep=0.5cm]{geometry}
\usepackage{enumerate}
\usepackage{enumitem}
\usepackage{marvosym}
\usepackage{tabularx}
\usepackage{multirow}
\usepackage[colorlinks = true, linkcolor = blue, urlcolor  = blue, citecolor = blue, anchorcolor = blue]{hyperref}
\usepackage{ulem}

\usepackage{listings}
\definecolor{lightlightgray}{rgb}{0.95,0.95,0.95}
\definecolor{lila}{rgb}{0.8,0,0.8}
\definecolor{mygray}{rgb}{0.5,0.5,0.5}
\definecolor{mygreen}{rgb}{0,0.8,0.26}
\lstdefinestyle{html} {language=html}
\lstset{language=html,
	basicstyle=\ttfamily,
	keywordstyle=\color{lila},
	commentstyle=\color{lightgray},
	stringstyle=\color{mygreen}\ttfamily,
	backgroundcolor=\color{white},
	showstringspaces=false,
	numbers=left,
	numbersep=10pt,
	numberstyle=\color{mygray}\ttfamily,
	identifierstyle=\color{blue},
	xleftmargin=.1\textwidth, 
	%xrightmargin=.1\textwidth,
	escapechar=§,
	breaklines=true,
	postbreak=\mbox{\space}
}
\lstset{literate=%
	{Ö}{{\"O}}1
	{Ä}{{\"A}}1
	{Ü}{{\"U}}1
	{ß}{{\ss}}1
	{ü}{{\"u}}1
	{ä}{{\"a}}1
	{ö}{{\"o}}1
}

\hypersetup{
	pdfauthor   = {Henry Haustein, Dennis Rössel},
	pdfkeywords = {Formular, Barrierefrei},
	pdftitle    = {PDF-Formular},
	pdflang={de-DE}
}

\usepackage[utf8]{inputenc}
\usepackage[T1]{fontenc}

\renewcommand*{\arraystretch}{1.4}

\newcolumntype{L}[1]{>{\raggedright\arraybackslash}p{#1}}
\newcolumntype{R}[1]{>{\raggedleft\arraybackslash}p{#1}}
\newcolumntype{C}[1]{>{\centering\let\newline\\\arraybackslash\hspace{0pt}}m{#1}}

\newcommand{\E}{\mathbb{E}}
\DeclareMathOperator{\rk}{rk}
\DeclareMathOperator{\Var}{Var}
\DeclareMathOperator{\Cov}{Cov}
\DeclareMathOperator{\SD}{SD}
\DeclareMathOperator{\Cor}{Cor}

\title{\textbf{Barrierefreie Dokumente, Übung 3}}
\author{\textsc{Henry Haustein}, \textsc{Dennis Rössel}}
\date{}

\begin{document}
	\maketitle
	
	\section*{Aufgabe 3.1: Barrier Walkthrough durchführen}
	\begin{enumerate}[label=(\alph*)]
		\item Person, die nicht richtig lesen kann (Legasthenie) und deswegen leichte Sprache benötigt. Schon viele Webseiten in leichter Sprache bedient.
		\item Aufgabe: Reisepass beantragen
		\item schwere Sprache, keine Umschaltmöglichkeit auf leichte Sprache
		\item Durchführung: Ins Suchfeld klicken und "Reisepass" eingeben. Ersten Treffer anklicken. Menüpunkt "Reisepass beantragen" anklicken, es erscheint viel komplizierter Text, keine Umstellmöglichkeit
		\item Fataler Fehler, weil es nicht möglich ist den Inhalt der Seite zu verstehen ohne externe Hilfe. Führt im Zweifelsfall zum Abbruch der Aufgabe. Beeinträchtigungslevel: hoch, Dauerhaftigkeit/Häufigkeit: gelegentlich $\to$ kritisches Problem
	\end{enumerate}
	
\end{document}