\documentclass{article}

\usepackage{amsmath,amssymb}
\usepackage{tikz}
\usepackage{pgfplots}
\usepackage{xcolor}
\usepackage[left=2.1cm,right=3.1cm,bottom=3cm,footskip=0.75cm,headsep=0.5cm]{geometry}
\usepackage{enumerate}
\usepackage{enumitem}
\usepackage{marvosym}
\usepackage{tabularx}
\usepackage{parskip}

\usepackage{listings}
\lstdefinelanguage{JavaScript}{
	keywords={typeof, new, true, false, catch, function, return, null, catch, switch, var, if, in, while, do, else, case, break, for, of, document},
	keywordstyle=\color{lila}\bfseries,
	ndkeywords={class, export, boolean, throw, implements, import, this},
	ndkeywordstyle=\color{lila}\bfseries,
	identifierstyle=\color{blue},
	sensitive=false,
	comment=[l]{//},
	morecomment=[s]{/*}{*/},
	commentstyle=\color{lightgray}\ttfamily,
	stringstyle=\color{mygreen}\ttfamily,
	morestring=[b]',
	morestring=[b]"
}
\definecolor{lightlightgray}{rgb}{0.95,0.95,0.95}
\definecolor{lila}{rgb}{0.8,0,0.8}
\definecolor{mygray}{rgb}{0.5,0.5,0.5}
\definecolor{mygreen}{rgb}{0,0.8,0.26}
\lstdefinestyle{xml} {language=xml, morekeywords={encoding, newspaper, article, title, id, published, author, xs:schema, xmlns:xs, xs:element, name, xs:sequence, xs:element, type, xs:complexType}}
\lstdefinestyle{json}{}
\lstdefinestyle{javascript}{language=javascript}
\lstdefinestyle{html}{language=html, morekeywords={main}}
\lstset{language=XML,
	basicstyle=\ttfamily,
	keywordstyle=\color{lila},
	commentstyle=\color{lightgray},
	stringstyle=\color{mygreen}\ttfamily,
	backgroundcolor=\color{white},
	showstringspaces=false,
	numbers=left,
	numbersep=10pt,
	tabsize=2,
	numberstyle=\color{mygray}\ttfamily,
	identifierstyle=\color{blue},
	xleftmargin=.1\textwidth, 
	%xrightmargin=.1\textwidth,
	escapechar=§,
	%literate={\t}{{\ }}1
	breaklines=true,
	postbreak=\mbox{\space},
	literate=%
	{Ö}{{\"O}}1
	{Ä}{{\"A}}1
	{Ü}{{\"U}}1
	{ß}{{\ss}}1
	{ü}{{\"u}}1
	{ä}{{\"a}}1
	{ö}{{\"o}}1
}

\usepackage[colorlinks = true, linkcolor = blue, urlcolor  = blue, citecolor = blue, anchorcolor = blue]{hyperref}
\usepackage[utf8]{inputenc}

\renewcommand*{\arraystretch}{1.4}

\newcolumntype{L}[1]{>{\raggedright\arraybackslash}p{#1}}
\newcolumntype{R}[1]{>{\raggedleft\arraybackslash}p{#1}}
\newcolumntype{C}[1]{>{\centering\let\newline\\\arraybackslash\hspace{0pt}}m{#1}}

\newcommand{\E}{\mathbb{E}}
\DeclareMathOperator{\rk}{rk}
\DeclareMathOperator{\Var}{Var}
\DeclareMathOperator{\Cov}{Cov}

\title{\textbf{Internet and Web Applications, Übung 5}}
\author{\textsc{Henry Haustein}}
\date{}

\begin{document}
	\maketitle
	
	\section*{Aufgabe 1: Version management}
	Vorteile:
	\begin{itemize}
		\item bei Git braucht man keine Netzwerkverbindung um Git-Operationen durchzuführen, bei SVN schon
		\item durch lokale Kopien gibt es viele unabhängige Backups des Codes
		\item Änderungen sind schneller (Merges, Branches, Commits)
		\item im Open-Source-Bereich sind Änderungen anderer einfacher: Repo forken, Änderungen rein und Pull Request stellen
	\end{itemize}
	
	\section*{Aufgabe 2: Git}
	\begin{enumerate}[label=(\alph*)]
		\item Zur Übersichtlichkeit und zur automatischen Erzeugung von z.B. Nightly Builds und Releases hat man unterschiedliche Branches:
		\begin{itemize}
			\item Master: enthält nur fertige Releases
			\item Develop: enthält neue Features
			\item Feature: enthält Features, die irgendwann oder auch gar nicht in ein Release münden sollen
			\item Hotfix: Versionsnummern, super kleine Änderungen
		\end{itemize}
		\item Die wichtigsten Ordner sind:
		\begin{itemize}
			\item hooks: This folder contains script files. Git hooks are the scripts that are executed before or after events like commit, push etc.
			\item objects: This folder represents an object database of Git.
			\item config: This is the local configuration file.
			\item refs: This folder stores information about tags and branches.
			\item HEAD: This file stores reference to the current branch. It points to the master branch by default.
			\item index: This is a binary file and stores staging information
		\end{itemize}
	\end{enumerate}

\end{document}