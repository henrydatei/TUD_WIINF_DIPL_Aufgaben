\documentclass{article}

\usepackage{amsmath,amssymb}
\usepackage{tikz}
\usepackage{pgfplots}
\usepackage{xcolor}
\usepackage[left=2.1cm,right=3.1cm,bottom=3cm,footskip=0.75cm,headsep=0.5cm]{geometry}
\usepackage{enumerate}
\usepackage{enumitem}
\usepackage{marvosym}
\usepackage{tabularx}
\usepackage{parskip}

\usepackage{listings}
\lstdefinelanguage{JavaScript}{
	keywords={typeof, new, true, false, catch, function, return, null, catch, switch, var, if, in, while, do, else, case, break, for, of, document},
	keywordstyle=\color{lila}\bfseries,
	ndkeywords={class, export, boolean, throw, implements, import, this},
	ndkeywordstyle=\color{lila}\bfseries,
	identifierstyle=\color{blue},
	sensitive=false,
	comment=[l]{//},
	morecomment=[s]{/*}{*/},
	commentstyle=\color{lightgray}\ttfamily,
	stringstyle=\color{mygreen}\ttfamily,
	morestring=[b]',
	morestring=[b]"
}
\definecolor{lightlightgray}{rgb}{0.95,0.95,0.95}
\definecolor{lila}{rgb}{0.8,0,0.8}
\definecolor{mygray}{rgb}{0.5,0.5,0.5}
\definecolor{mygreen}{rgb}{0,0.8,0.26}
\lstdefinestyle{xml} {language=xml, morekeywords={encoding, newspaper, article, title, id, published, author, xs:schema, xmlns:xs, xs:element, name, xs:sequence, xs:element, type, xs:complexType}}
\lstdefinestyle{json}{}
\lstdefinestyle{javascript}{language=javascript}
\lstdefinestyle{html}{language=html, morekeywords={main}}
\lstset{language=XML,
	basicstyle=\ttfamily,
	keywordstyle=\color{lila},
	commentstyle=\color{lightgray},
	stringstyle=\color{mygreen}\ttfamily,
	backgroundcolor=\color{white},
	showstringspaces=false,
	numbers=left,
	numbersep=10pt,
	tabsize=2,
	numberstyle=\color{mygray}\ttfamily,
	identifierstyle=\color{blue},
	xleftmargin=.1\textwidth, 
	%xrightmargin=.1\textwidth,
	escapechar=§,
	%literate={\t}{{\ }}1
	breaklines=true,
	postbreak=\mbox{\space},
	literate=%
	{Ö}{{\"O}}1
	{Ä}{{\"A}}1
	{Ü}{{\"U}}1
	{ß}{{\ss}}1
	{ü}{{\"u}}1
	{ä}{{\"a}}1
	{ö}{{\"o}}1
}

\usepackage[colorlinks = true, linkcolor = blue, urlcolor  = blue, citecolor = blue, anchorcolor = blue]{hyperref}
\usepackage[utf8]{inputenc}

\renewcommand*{\arraystretch}{1.4}

\newcolumntype{L}[1]{>{\raggedright\arraybackslash}p{#1}}
\newcolumntype{R}[1]{>{\raggedleft\arraybackslash}p{#1}}
\newcolumntype{C}[1]{>{\centering\let\newline\\\arraybackslash\hspace{0pt}}m{#1}}

\newcommand{\E}{\mathbb{E}}
\DeclareMathOperator{\rk}{rk}
\DeclareMathOperator{\Var}{Var}
\DeclareMathOperator{\Cov}{Cov}

\title{\textbf{Internet and Web Applications, Übung 9}}
\author{\textsc{Henry Haustein}}
\date{}

\begin{document}
	\maketitle
	
	\section*{Aufgabe 1: Evaluating XMPP}
	Following are the benefits or advantages of XMPP protocol:
	\begin{itemize}
		\item Extensible: It can be customized to individual user requirements.
		\item Messaging: Short messages are used for fast communication between user and server.
		\item Presence: It is reactive to presence of user and his/her status.
		\item Protocol: It is an open platform which is constantly evolving.
		\item Secured: It uses TLS and SASL to provide secured end to end connection.
	\end{itemize}
	Following are the drawbacks or disadvantages of XMPP protocol:
	\begin{itemize}
		\item It does not have QoS mechanism as used by MQTT protocol.
		\item Streaming XML has overhead due to text based communication compare to binary based communication.
		\item XML content transports asynchronously.
		\item Server may overload with presence and instant messaging.
	\end{itemize}
	(\url{https://www.rfwireless-world.com/Terminology/Advantages-and-Disadvantages-of-XMPP-protocol.html})
	
	XMPP is based on a decentralized client-server architecture. In this architecture, clients don’t communicate directly with each other; instead, there’s a decentralized server acting as the intermediary between them. XMPP network shares a similar architectural design with email servers - there’s no central master server, and anyone can run their own XMPP server. (\url{https://ably.com/periodic-table-of-realtime/xmpp})
	
	\section*{Aufgabe 2: Serverless Messaging}
	Mein Client startet einen Daemon, der DNS-based Service Discovery und Multicast DNS unterstützt. Daemon veröffentlicht unter der Multicast DNS Adresse 224.0.0.251 (FF02::FB für IPv6) unter anderem IP und Port des XMPP Clients (kann noch mit mehr Informationen in einem DNS TXT Record angereichert werden). Andere Chatteilnehmer senden Multicast DNS Query und finden mich.
	
	\section*{Aufgabe 3: Jingle}
	In XMPP wird ein Channel aufgebaut (außerhalb von XMPP) in dem Daten mittels RTP gesendet werden. (\url{https://en.wikipedia.org/wiki/Jingle_(protocol)})
\end{document}