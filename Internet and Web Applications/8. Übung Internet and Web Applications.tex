\documentclass{article}

\usepackage{amsmath,amssymb}
\usepackage{tikz}
\usepackage{pgfplots}
\usepackage{xcolor}
\usepackage[left=2.1cm,right=3.1cm,bottom=3cm,footskip=0.75cm,headsep=0.5cm]{geometry}
\usepackage{enumerate}
\usepackage{enumitem}
\usepackage{marvosym}
\usepackage{tabularx}
\usepackage{parskip}

\usepackage{listings}
\lstdefinelanguage{JavaScript}{
	keywords={typeof, new, true, false, catch, function, return, null, catch, switch, var, if, in, while, do, else, case, break, for, of, document},
	keywordstyle=\color{lila}\bfseries,
	ndkeywords={class, export, boolean, throw, implements, import, this},
	ndkeywordstyle=\color{lila}\bfseries,
	identifierstyle=\color{blue},
	sensitive=false,
	comment=[l]{//},
	morecomment=[s]{/*}{*/},
	commentstyle=\color{lightgray}\ttfamily,
	stringstyle=\color{mygreen}\ttfamily,
	morestring=[b]',
	morestring=[b]"
}
\definecolor{lightlightgray}{rgb}{0.95,0.95,0.95}
\definecolor{lila}{rgb}{0.8,0,0.8}
\definecolor{mygray}{rgb}{0.5,0.5,0.5}
\definecolor{mygreen}{rgb}{0,0.8,0.26}
\lstdefinestyle{xml} {language=xml, morekeywords={encoding, newspaper, article, title, id, published, author, xs:schema, xmlns:xs, xs:element, name, xs:sequence, xs:element, type, xs:complexType}}
\lstdefinestyle{json}{}
\lstdefinestyle{javascript}{language=javascript}
\lstdefinestyle{html}{language=html, morekeywords={main}}
\lstset{language=XML,
	basicstyle=\ttfamily,
	keywordstyle=\color{lila},
	commentstyle=\color{lightgray},
	stringstyle=\color{mygreen}\ttfamily,
	backgroundcolor=\color{white},
	showstringspaces=false,
	numbers=left,
	numbersep=10pt,
	tabsize=2,
	numberstyle=\color{mygray}\ttfamily,
	identifierstyle=\color{blue},
	xleftmargin=.1\textwidth, 
	%xrightmargin=.1\textwidth,
	escapechar=§,
	%literate={\t}{{\ }}1
	breaklines=true,
	postbreak=\mbox{\space},
	literate=%
	{Ö}{{\"O}}1
	{Ä}{{\"A}}1
	{Ü}{{\"U}}1
	{ß}{{\ss}}1
	{ü}{{\"u}}1
	{ä}{{\"a}}1
	{ö}{{\"o}}1
}

\usepackage[colorlinks = true, linkcolor = blue, urlcolor  = blue, citecolor = blue, anchorcolor = blue]{hyperref}
\usepackage[utf8]{inputenc}

\renewcommand*{\arraystretch}{1.4}

\newcolumntype{L}[1]{>{\raggedright\arraybackslash}p{#1}}
\newcolumntype{R}[1]{>{\raggedleft\arraybackslash}p{#1}}
\newcolumntype{C}[1]{>{\centering\let\newline\\\arraybackslash\hspace{0pt}}m{#1}}

\newcommand{\E}{\mathbb{E}}
\DeclareMathOperator{\rk}{rk}
\DeclareMathOperator{\Var}{Var}
\DeclareMathOperator{\Cov}{Cov}

\title{\textbf{Internet and Web Applications, Übung 8}}
\author{\textsc{Henry Haustein}}
\date{}

\begin{document}
	\maketitle
	
	\section*{Aufgabe 1: Multicast Communication on the Internet}
	Weaknesses:
	\begin{itemize}
		\item only for live events, not for video on demand
		\item client software must be deployed, managed and updated
		\item slower start times/delays duw to push architecture
		\item network switches and wireless access points must be multicast enabled
		\item network locations not multicast enables will not benefit
	\end{itemize}
	\url{https://www.rampecdn.com/enterprise/what-is-multicast/}
	
	\section*{Aufgabe 2: Streaming on Video-on-Demand Portals}
	Netflix nutzt MPEG-DASH und CDNs um seinen Content zu verteilen \\
	\url{https://ieeexplore.ieee.org/document/6195531}
	
	\section*{Aufgabe 3: Digital Rights Management (DRM)}
	Google Widevine ist eine proprietäre Technologie zur digitalen Rechteverwaltung (DRM) von Google, welche von den Webbrowsern Google Chrome und Mozilla Firefox, einigen Derivaten dieser Browser sowie Google Android MediaDRM, Android TV und anderen Geräten der Unterhaltungselektronik verwendet wird. Sie unterstützt verschiedene Schemata zur Verschlüsselung und Hardware-Sicherung, um den Zugriff auf verteilte Videodaten durch Endverbraucher nach den Regeln von Content-Eigentümern zu beschränken. In der Hauptsache bietet Widevine ein Content Decryption Module (CDM) als einen Client für Google Chrome und andere Browser und Geräte an. Von Inhalteanbietern kann Widevine kostenlos verwendet werden, da der Hersteller keine Lizenzgebühren für die Integration erhebt, auch nicht für die Integration in Geräten. \\
	Die drei Sicherheitsniveaus von Widevine sind:
	\begin{itemize}
		\item L1 - keine Beschränkung auf die Auflösung oder HDR; höchste Stufe des Schutzes. Sowohl Kryptografie als auch Medienverarbeitung finden in einer vertrauenswürdigen Ausführungsumgebung (TEE) statt.
		\item L2 - (typischerweise) 540p Auflösungsbegrenzung. Nur die kryptografischen Operationen werden in einer TEE ausgeführt, die Medienverarbeitung nicht.
		\item L3 - (typischerweise) 480p Auflösungsbegrenzung. Nur softwarebasiertes DRM ohne Hardwarebeschränkung.
	\end{itemize}
	Widevine DRM wird mit Chromium-basierten proprietären Webbrowsern und auf Android benutzt. Es verwendet dynamisches adaptives Streaming über HTTP und HTTP Live Streaming. Widevine gehört zu den DRM-Systemen, die in einem Browser von den Encrypted Media Extensions und den Media Source Extensions Gebrauch machen. Über dreißig Chipsets, sechs größere Desktop- und mobile Betriebssysteme, sowie Google-Produkte wie Chromecast und Android TV verwenden Widevine. \\
	Konzerne wie Amazon Prime Video, BBC, Hulu, Netflix, Spotify und Disney+ verwenden Widevine DRM, um die Verteilung ihrer Inhalte zu verwalten. \\
	Ebenfalls wird es von Mozilla Firefox seit seiner Version 47 im Jahr 2016 verwendet, und unter Microsoft Windows standardmäßig aktiviert. Unter Linux-Systemen wird dem Benutzer die Aktivierung von Widevine im Browser angeboten; dort kann er sie auch abstellen und deinstallieren. Vor der Einführung von Widevine verwendete Mozilla das Produkt Primetime DRM von Adobe für einige Versionen. Vor der Markteinführung der dafür notwendigen Hardwarevoraussetzungen in jeden multimediafähigen Computer wurden vor allem softwarebasierte DRM-Lösungen auf der Basis von proprietären Browser-Plug-ins wie Microsoft Silverlight oder Adobe Flash verwendet. \\
	\url{https://de.wikipedia.org/wiki/Widevine}
	
	Die Stufe L3 ist öffentlich gebrochen, es gibt ein Writeup dafür (\url{https://github.com/tomer8007/widevine-l3-decryptor/wiki/Reversing-the-old-Widevine-Content-Decryption-Module}). Die anderen Stufen sind auch ent\-schlüsselbar, allerdings sind die Schlüssel nicht öffentlich verfügbar. Raubkopiergruppen müssen aber in Besitz dieser Schlüssel sein, anders ist es nicht erklärbar, dass wenige Minuten nach der Veröffentlichung einer Serie bereits Kopien existieren.
\end{document}