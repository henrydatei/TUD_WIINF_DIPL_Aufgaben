\documentclass{article}

\usepackage{amsmath,amssymb}
\usepackage{tikz}
\usepackage{pgfplots}
\usepackage{xcolor}
\usepackage[left=2.1cm,right=3.1cm,bottom=3cm,footskip=0.75cm,headsep=0.5cm]{geometry}
\usepackage{enumerate}
\usepackage{enumitem}
\usepackage{marvosym}
\usepackage{tabularx}
\usepackage{parskip}

\usepackage{listings}
\lstdefinelanguage{JavaScript}{
	keywords={typeof, new, true, false, catch, function, return, null, catch, switch, var, if, in, while, do, else, case, break, for, of, document},
	keywordstyle=\color{lila}\bfseries,
	ndkeywords={class, export, boolean, throw, implements, import, this},
	ndkeywordstyle=\color{lila}\bfseries,
	identifierstyle=\color{blue},
	sensitive=false,
	comment=[l]{//},
	morecomment=[s]{/*}{*/},
	commentstyle=\color{lightgray}\ttfamily,
	stringstyle=\color{mygreen}\ttfamily,
	morestring=[b]',
	morestring=[b]"
}
\definecolor{lightlightgray}{rgb}{0.95,0.95,0.95}
\definecolor{lila}{rgb}{0.8,0,0.8}
\definecolor{mygray}{rgb}{0.5,0.5,0.5}
\definecolor{mygreen}{rgb}{0,0.8,0.26}
\lstdefinestyle{xml} {language=xml, morekeywords={encoding, newspaper, article, title, id, published, author, xs:schema, xmlns:xs, xs:element, name, xs:sequence, xs:element, type, xs:complexType}}
\lstdefinestyle{json}{}
\lstdefinestyle{javascript}{language=javascript}
\lstdefinestyle{html}{language=html, morekeywords={main}}
\lstset{language=XML,
	basicstyle=\ttfamily,
	keywordstyle=\color{lila},
	commentstyle=\color{lightgray},
	stringstyle=\color{mygreen}\ttfamily,
	backgroundcolor=\color{white},
	showstringspaces=false,
	numbers=left,
	numbersep=10pt,
	tabsize=2,
	numberstyle=\color{mygray}\ttfamily,
	identifierstyle=\color{blue},
	xleftmargin=.1\textwidth, 
	%xrightmargin=.1\textwidth,
	escapechar=§,
	%literate={\t}{{\ }}1
	breaklines=true,
	postbreak=\mbox{\space},
	literate=%
	{Ö}{{\"O}}1
	{Ä}{{\"A}}1
	{Ü}{{\"U}}1
	{ß}{{\ss}}1
	{ü}{{\"u}}1
	{ä}{{\"a}}1
	{ö}{{\"o}}1
}

\usepackage[colorlinks = true, linkcolor = blue, urlcolor  = blue, citecolor = blue, anchorcolor = blue]{hyperref}
\usepackage[utf8]{inputenc}

\renewcommand*{\arraystretch}{1.4}

\newcolumntype{L}[1]{>{\raggedright\arraybackslash}p{#1}}
\newcolumntype{R}[1]{>{\raggedleft\arraybackslash}p{#1}}
\newcolumntype{C}[1]{>{\centering\let\newline\\\arraybackslash\hspace{0pt}}m{#1}}

\newcommand{\E}{\mathbb{E}}
\DeclareMathOperator{\rk}{rk}
\DeclareMathOperator{\Var}{Var}
\DeclareMathOperator{\Cov}{Cov}

\title{\textbf{Internet and Web Applications, Übung 3}}
\author{\textsc{Henry Haustein}}
\date{}

\begin{document}
	\maketitle
	
	\section*{Aufgabe 1: Interaction models}
	Es gab die folgenden 3 Modelle:
	\begin{itemize}
		\item Synchronous model (\textit{Click, wait, and refresh user interaction paradigm}): synchronous (user activities are blocked), high degree of dependence between HTTP Request and HTTP Response
		\item Non blocking model (AJAX): asynchronous (user activities are not blocked), high degree of dependence between HTTP Request and HTTP Response
		\item Asynchronous model (Comet via long- polling, WebSockets): asynchronous, minimum degree of dependence between HTTP Request and HTTP Response
	\end{itemize}
	
	\section*{Aufgabe 2: JSON}
	\begin{enumerate}[label=(\alph*)]
		\item Vorteile: viel weniger Overhead, JavaScript-Support; Nachteil: Daten sind nicht deklarativ
		\item JSON-Schema:
		\begin{lstlisting}[style=json]
{
	"definitions": {},
	"$schema": "https://json-schema.org/draft/2020-12/schema",
	"$id": "https://example.com/newspaper.schema.json",
	"title": "Root",
	"type": "object",
	"required": [
	"articles"
	],
	"properties": {
		"articles": {
			"$id": "#root/articles",
			"title": "Articles",
			"type": "object",
			"required": [
				"article"
			],
			"properties": {
				"article": {
					"$id": "#root/articles/article",
					"title": "Article",
					"type": "array",
					"default": [],
					"items": {
						"$id": "#root/articles/article/items",
						"title": "Items",
						"type": "object",
						"required": [
							"title",
							"author",
							"published",
							"_id"
						],
						"properties": {
							"title": {
								"$id": "#root/articles/article/items/title",
								"title": "Title",
								"type": "string",
								"default": "",
								"examples": [
									"First man on moon"
								],
								"pattern": "^.*$"
							},
							"author": {
								"$id": "#root/articles/article/items/author",
								"title": "Author",
								"type": "string",
								"default": "",
								"examples": [
									"Ted Markson"
								],
								"pattern": "^.*$"
							},
							"published": {
								"$id": "#root/articles/article/items/published",
								"title": "Published",
								"type": "object",
								"required": [
									"day",
									"month",
									"year"
								],
								"properties": {
									"day": {
										"$id": "#root/articles/article/items/published/day",
										"title": "Day",
										"type": "integer",
										"examples": [
											21
										],
										"default": 0
									},
									"month": {
										"$id": "#root/articles/article/items/published/month",
										"title": "Month",
										"type": "integer",
										"examples": [
											7
										],
										"default": 0
									},
									"year": {
										"$id": "#root/articles/article/items/published/year",
										"title": "Year",
										"type": "integer",
										"examples": [
											1969
										],
										"default": 0
									}
								}
							},
							"_id": {
								"$id": "#root/articles/article/items/_id",
								"title": "_id",
								"type": "string",
								"default": "",
								"examples": [
									"5"
								],
								"pattern": "^.*$"
							}
						}
					}
				}
			}
		}
	}
}
		\end{lstlisting}
	\end{enumerate}
	
	\section*{Aufgabe 3: Web Sockets}
	HTML-Seite
	\begin{lstlisting}[style=html]
<!doctype html>
	<head>
		<script type="text/javascript">
			function WS() {
				var ws = new WebSocket("ws://echo.websocket.events")
				ws.onopen = function() {ws.send("Hallo")}
				ws.onmessage = function(evt) {alert(evt.data)}
			}
		</script>
	</head>
	<body>
		<a href="javascript:WS()">Run example</a>
	</body>
</html>
	\end{lstlisting}

\end{document}