\documentclass{article}

\usepackage{amsmath,amssymb}
\usepackage{tikz}
\usepackage{pgfplots}
\usepackage{xcolor}
\usepackage[left=2.1cm,right=3.1cm,bottom=3cm,footskip=0.75cm,headsep=0.5cm]{geometry}
\usepackage{enumerate}
\usepackage{enumitem}
\usepackage{marvosym}
\usepackage{tabularx}
\usepackage{parskip}
\usepackage{multirow}

\usepackage{listings}
\definecolor{lightlightgray}{rgb}{0.95,0.95,0.95}
\definecolor{lila}{rgb}{0.8,0,0.8}
\definecolor{mygray}{rgb}{0.5,0.5,0.5}
\definecolor{mygreen}{rgb}{0,0.8,0.26}
%\lstdefinestyle{java} {language=java}
\lstset{language=R,
	basicstyle=\ttfamily,
	keywordstyle=\color{lila},
	commentstyle=\color{lightgray},
	stringstyle=\color{mygreen}\ttfamily,
	backgroundcolor=\color{white},
	showstringspaces=false,
	numbers=left,
	numbersep=10pt,
	numberstyle=\color{mygray}\ttfamily,
	identifierstyle=\color{blue},
	xleftmargin=.1\textwidth, 
	%xrightmargin=.1\textwidth,
	escapechar=§,
	%literate={\t}{{\ }}1
	breaklines=true,
	postbreak=\mbox{\space}
}

\usepackage[colorlinks = true, linkcolor = blue, urlcolor  = blue, citecolor = blue, anchorcolor = blue]{hyperref}
\usepackage[utf8]{inputenc}

\renewcommand*{\arraystretch}{1.4}

\newcolumntype{L}[1]{>{\raggedright\arraybackslash}p{#1}}
\newcolumntype{R}[1]{>{\raggedleft\arraybackslash}p{#1}}
\newcolumntype{C}[1]{>{\centering\let\newline\\\arraybackslash\hspace{0pt}}m{#1}}

\newcommand{\E}{\mathbb{E}}
\DeclareMathOperator{\rk}{rk}
\DeclareMathOperator{\Var}{Var}
\DeclareMathOperator{\Cov}{Cov}

\title{\textbf{Mathematical Economics 1A, Problem Set 1}}
\author{\textsc{Henry Haustein}}
\date{}

\begin{document}
	\maketitle
	
	\section*{Task 1: Ye olde game of Chicken}
	\begin{enumerate}[label=(\alph*)]
		\item The normal form of this game is
		\begin{center}
			\begin{tabular}{C{0.5cm}c|c|c}
				& & \multicolumn{2}{c}{\textbf{Kate}} \\
				& & swerve & not swerve \\
				\hline
				\multirow{2}{0.5cm}{\rotatebox[origin=c]{90}{\textbf{Jane}}} & swerve & (0,0) & (-1,2) \\
				\cline{3-4}
				& not swerve & (2,-1) & (-5,-5)
			\end{tabular}
		\end{center}
		\item Jane's strategy is $p\cdot\text{not swerve} + (1-p)\cdot\text{swerve}$. Then
		\begin{itemize}
			\item $u_K(p\cdot\text{not swerve} + (1-p)\cdot\text{swerve}, \text{not swerve}) = p(-5) + (1-p)2 = 2-7p$
			\item $u_K(p\cdot\text{not swerve} + (1-p)\cdot\text{swerve}, \text{swerve}) = p(-1) + (1-p)0 = -p$
		\end{itemize}
		\begin{center}
			\begin{tikzpicture}
				\begin{axis}[
					xmin=0, xmax=1, xlabel={$p$},
					ymin=-5, ymax=2, ylabel={$u_K$},
					samples=400,
					axis x line=middle,
					axis y line=middle,
					domain=0:1,
					]
					\addplot[blue] {2-7*x};
					\addplot[red] {-x};
				\end{axis}
			\end{tikzpicture} \\
			\textcolor{blue}{$2-7p$}, \textcolor{red}{$-p$}
		\end{center}
		\item Not swerve is the best reponse if
		\begin{align}
			u_K(p\cdot\text{not swerve} + (1-p)\cdot\text{swerve}, \text{not swerve}) &\ge u_K(p\cdot\text{not swerve} + (1-p)\cdot\text{swerve}, \text{swerve}) \notag \\
			2-7p &\ge -p \notag \\
			p &\le \frac{1}{3} \notag
		\end{align}
		This leads to
		\begin{align}
			BR_K(p) &= \begin{cases}
				\text{not swerve} & p<\frac{1}{3} \\
				\text{choose random} & p=\frac{1}{3} \\
				\text{swerve} & p>\frac{1}{3}
			\end{cases} \notag
		\end{align}
		\begin{center}
			\begin{tikzpicture}
				\begin{axis}[
					xmin=0, xmax=1, xlabel={$p$},
					ymin=0, ymax=1, ylabel={$q$},
					samples=400,
					axis x line=middle,
					axis y line=middle,
					domain=0:1,
					]
					\draw[blue,thick] (axis cs: 0,1) -- (axis cs: 0.33,1) -- (axis cs: 0.33,0) -- (axis cs: 1,0);
					\draw[red,thick] (axis cs: 0,1) -- (axis cs: 0,0.33) -- (axis cs: 1,0.33) -- (axis cs: 1,0);
				\end{axis}
			\end{tikzpicture} \\
			\textcolor{blue}{Kate}, \textcolor{red}{Jane}
		\end{center}
		\item You can see that (swerve, not swerve) and (not swerve, swerve) are pure strategy Nash Equilibria. If we assign probabilities to each strategy we can find a NE in mixed strategies.
		\begin{center}
			\begin{tabular}{C{0.5cm}c|c|c|r}
				& & \multicolumn{2}{c}{\textbf{Kate}} & \\
				& & swerve & not swerve & \\
				\hline
				\multirow{2}{0.5cm}{\rotatebox[origin=c]{90}{\textbf{Jane}}} & swerve & (0,0) & (-1,2) & $p$ \\
				\cline{3-5}
				& not swerve & (2,-1) & (-5,-5) & $1-p$ \\
				\hline
				& & $q$ & $1-q$
			\end{tabular}
		\end{center}
		Then the outcome for each player is
		\begin{align}
			u_K(p,q) &= 2p(1-q) - q(1-p) - 5(1-p)(1-q) = 7p - 6pq + 4q - 5 \notag \\
			u_J(p,q) &= -p(1-q) + 2q(1-p) - 5(1-p)(1-q) = 7q - 6pq + 4p - 5 \notag
		\end{align}
		Maximizing
		\begin{align}
			\frac{\partial u_K}{\partial q} &= -6p+4 \overset{!}{=} 0 \notag \\
			p &= \frac{2}{3} \notag \\
			\frac{\partial u_J}{\partial p} &= -6q + 4 \overset{!}{=} 0 \notag \\
			q &= \frac{2}{3} \notag
		\end{align}
		So the third NE is $\left(\frac{2}{3}\text{swerve} + \frac{1}{3}\text{not swerve}, \frac{2}{3}\text{swerve} + \frac{1}{3}\text{not swerve}\right)$.
		\item With the underline-approach you find that (not swerve, swerve) is definitely a NE. If Kate played not swerve and Jane has do decide what she plays, the question is $x-5 > -1$ (we want her to choose not swerve so that (swerve, not swerve) doesn't become a NE). The equation above leads to $x>4$ and then the only NE is (not swerve, swerve).
	\end{enumerate}

	\section*{Task 2: Auctions!}
	\begin{enumerate}[label=(\alph*)]
		\item Die utility function for player 1 with bid $b_i$ is
		\begin{align}
			u_1(x_1,x_2) = \begin{cases}
				500-x_1 & x_1>x_2 \\
				\frac{1}{2}(500-x_1) & x_1=x_2 \\
				0 & x_1 < x_2
			\end{cases} \notag
		\end{align}
		\item The best response for player 1 is
		\begin{align}
			BR_1(x_2) = \begin{cases}
				[0,x_2) & x_2 > 500 \\
				[0,500] & x_2 = 500 \\
				\emptyset & x_2 < 500
			\end{cases} \notag
		\end{align}
		\item
	\end{enumerate}
	
	\section*{Task 3: Cournot Price Competition}
	\begin{enumerate}[label=(\alph*)]
		\item The utility functions are
		\begin{align}
			u_1(q_1,q_2) &= q_1 \cdot (1 - q_1 - q_2) \notag \\
			u_2(q_1,q_2) &= q_2 \cdot (1 - q_1 - q_2) \notag
		\end{align}
		\item Maximizing
		\begin{align}
			\frac{\partial u_1}{\partial q_1} &= 1 - 2q_1 - q_2 \overset{!}{=} 0 \notag \\
			q_1 &= \frac{1-q_2}{2} \label{eq1} \\
			\frac{\partial u_2}{\partial q_2} &= 1 - 2q_2 - q_1 \overset{!}{=} 0 \notag \\
			q_2 &= \frac{1-q_1}{2} \label{eq2}
		\end{align}
		\item Insert \eqref{eq2} in \eqref{eq1}:
		\begin{align}
			q_1 &= \frac{1 - \frac{1 - q_1}{2}}{2} \notag \\
			&= \frac{1}{2} - \frac{1-q_1}{4} \notag \\
			4q_1 &= 2 - (1 - q_1) \notag \\
			3q_1 &= 1 \notag \\
			q_1 &= \frac{1}{3} \notag \\
			\Rightarrow q_2 &= \frac{1}{3} \notag
		\end{align}
		\item 
	\end{enumerate}

\end{document}