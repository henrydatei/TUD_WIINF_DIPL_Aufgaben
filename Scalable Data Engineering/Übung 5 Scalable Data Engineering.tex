\documentclass{article}

\usepackage{amsmath,amssymb}
\usepackage{tikz}
\usepackage{pgfplots}
\usepackage{xcolor}
\usepackage[left=2.1cm,right=3.1cm,bottom=3cm,footskip=0.75cm,headsep=0.5cm]{geometry}
\usepackage{enumerate}
\usepackage{enumitem}
\usepackage{marvosym}
\usepackage{tabularx}
\usepackage{parskip}

\usepackage{listings}
\definecolor{lightlightgray}{rgb}{0.95,0.95,0.95}
\definecolor{lila}{rgb}{0.8,0,0.8}
\definecolor{mygray}{rgb}{0.5,0.5,0.5}
\definecolor{mygreen}{rgb}{0,0.8,0.26}
\lstdefinestyle{R} {language=R}
\lstset{language=SQL,
	basicstyle=\ttfamily,
	keywordstyle=\color{lila},
	commentstyle=\color{lightgray},
	stringstyle=\color{mygreen}\ttfamily,
	backgroundcolor=\color{white},
	showstringspaces=false,
	numbers=left,
	numbersep=10pt,
	numberstyle=\color{mygray}\ttfamily,
	identifierstyle=\color{blue},
	xleftmargin=.1\textwidth, 
	%xrightmargin=.1\textwidth,
	escapechar=§,
	%literate={\t}{{\ }}1
	breaklines=true,
	postbreak=\mbox{\space},
	morekeywords={with, data, refresh, materialized, explain, rank, over, partition, uuid}
}

\usepackage[colorlinks = true, linkcolor = blue, urlcolor  = blue, citecolor = blue, anchorcolor = blue]{hyperref}
\usepackage[utf8]{inputenc}

\renewcommand*{\arraystretch}{1.4}

\newcolumntype{L}[1]{>{\raggedright\arraybackslash}p{#1}}
\newcolumntype{R}[1]{>{\raggedleft\arraybackslash}p{#1}}
\newcolumntype{C}[1]{>{\centering\let\newline\\\arraybackslash\hspace{0pt}}m{#1}}

\newcommand{\E}{\mathbb{E}}
\DeclareMathOperator{\rk}{rk}
\DeclareMathOperator{\Var}{Var}
\DeclareMathOperator{\Cov}{Cov}

\title{\textbf{Scalable Data Engineering, Exercise 5}}
\author{\textsc{Henry Haustein}}
\date{}

\begin{document}
	\maketitle
	
	\section*{Task 1}
	\begin{enumerate}[label=(\alph*)]
		\item True, technically there is also a base part, but often this is included into the trend.
		\item False, the season mask is fixed in length.
		\item False, this does piece wise aggregate approximation. SAX represents segments with an alphabet.
		\item False, with very large time series this takes forever.
		\item False, for some forecasting methods you need the season length too.
		\item False, e.g. no one can predict the value of a share in the future.
		\item True.
	\end{enumerate}

	\section*{Task 2}
	\begin{enumerate}[label=(\alph*)]
		\item Visualization
		\begin{center}
			\begin{tikzpicture}
				\begin{axis}[
					xmin=1, xmax=15, xlabel=$x$,
					ymin=-5, ymax=20, ylabel=$y$,
					samples=400,
					axis x line=middle,
					axis y line=middle,
					domain=1:15,
					]
					\addplot[mark=x,blue] coordinates {
						(1,-1)
						(2,2)
						(3,7)
						(4,4)
						(5,3)
						(6,7)
						(7,12)
						(8,9)
						(9,8)
						(10,9)
						(11,12)
						(13,17)
						(14,14)
						(15,13)
					};
					
				\end{axis}
			\end{tikzpicture}
		\end{center}
		The local maxima are at 3, 8 and 13. This suggests a season length of 5. Running the following R code
		\begin{lstlisting}[style=R]
series = c(-1, 2, 7, 4, 3, 4, 7, 12, 9, 8, 9, 12, 17, 14, 13)
trend = filter(series, rep_len(1,5)/5)
detrend = series - trend
mat = matrix(detrend, ncol = 5, byrow = TRUE)
figure = apply(mat, 2, mean, na.rm = TRUE)
season = figure - mean(figure)
		\end{lstlisting}
		We get as \texttt{detrend}: \texttt{NA, NA, 4, -8.881784e-16, -2, -2, 0, 4, 0, -2, -2, -1.776357e-15, 4, NA, NA}. And as season mask: \texttt{-2, 0, 4, 4.440892e-16, -2}.
		\item I don't see any problem. Obviously one season has a length of 4, so running
		\begin{lstlisting}[style=R]
series = c(-1, 2, 5, 4, 3, 6, 9, 8, 7, 10, 13, 12, 11, 14, 17, 16)
decompose(ts(series, frequency = 4))
		\end{lstlisting}
		doesn't give any errors. All residuals are just 0.
	\end{enumerate}
	

	\section*{Task 3}
	Let's plot the data first
	\begin{center}
		\begin{tikzpicture}
			\begin{axis}[
				xmin=1, xmax=10, xlabel=$x$,
				ymin=-5, ymax=20, ylabel=$y$,
				samples=400,
				axis x line=middle,
				axis y line=middle,
				domain=1:10,
				]
				\addplot[mark=x,blue] coordinates {
					(1,4)
					(2,7)
					(3,12)
					(4,9)
					(5,8)
					(6,9)
					(7,12)
					(8,17)
					(9,14)
					(10,13)
				};
				
			\end{axis}
		\end{tikzpicture}
	\end{center}
	We see $s=5$. Then the predicted values are
	\begin{center}
		\begin{tabular}{r|rrr|rrr}
			\textbf{$x_t$} & \textbf{AR(1)} & \textbf{AR(2)} & \textbf{EGRV} & \textbf{$\varepsilon_{AR(1)}^2$} & \textbf{$\varepsilon_{AR(2)}^2$} & \textbf{$\varepsilon_{EGRV}^2$} \\
			\hline
			4 & 4 & 4 & 4 & 0 & 0 & 0 \\
			\hline
			7 & 2 & 7 & 7 & 25 & 0 & 0 \\
			\hline
			12 & 1 & 3 & 12 & 121 & 81 & 0 \\
			\hline
			9 & 0,5 & -0,3 & 9 & 72,25 & 86,49 & 0 \\
			\hline
			8 & 0,25 & -1,08 & 8 & 60,0625 & 82,4464 & 0 \\
			\hline
			9 & 0,125 & -0,558 & 9 & 78,765625 & 91,355364 & 0 \\
			\hline
			12 & 0,0625 & -0,0108 & 12 & 142,50390625 & 144,25931664 & 0 \\
			\hline
			17 & 0,03125 & 0,16092 & 17 & 287,9384765625 & 283,5546152464 & 0 \\
			\hline
			14 & 0,015625 & 0,099792 & 14 & 195,562744140625 & 193,215782443264 & 0 \\
			\hline
			13 & 0,0078125 & 0,0115992 & 13 & 168,796936035156 & 168,698555341441 & 0 
		\end{tabular}
	\end{center}
	The AIC is
	\begin{itemize}
		\item for AR(1): 2305,760376
		\item for AR(2): 2264,040067
		\item for EGRV: 2
	\end{itemize}
	$\Rightarrow$ EGRV is for this time series the best forecast.
\end{document}