\documentclass{article}

\usepackage{amsmath,amssymb}
\usepackage{tikz}
\usepackage{pgfplots}
\usepackage{xcolor}
\usepackage[left=2.1cm,right=3.1cm,bottom=3cm,footskip=0.75cm,headsep=0.5cm]{geometry}
\usepackage{enumerate}
\usepackage{enumitem}
\usepackage{marvosym}
\usepackage{tabularx}
\usepackage{parskip}

\usepackage{listings}
\definecolor{lightlightgray}{rgb}{0.95,0.95,0.95}
\definecolor{lila}{rgb}{0.8,0,0.8}
\definecolor{mygray}{rgb}{0.5,0.5,0.5}
\definecolor{mygreen}{rgb}{0,0.8,0.26}
%\lstdefinestyle{java} {language=java}
\lstset{language=SQL,
	basicstyle=\ttfamily,
	keywordstyle=\color{lila},
	commentstyle=\color{lightgray},
	stringstyle=\color{mygreen}\ttfamily,
	backgroundcolor=\color{white},
	showstringspaces=false,
	numbers=left,
	numbersep=10pt,
	numberstyle=\color{mygray}\ttfamily,
	identifierstyle=\color{blue},
	xleftmargin=.1\textwidth, 
	%xrightmargin=.1\textwidth,
	escapechar=§,
	%literate={\t}{{\ }}1
	breaklines=true,
	postbreak=\mbox{\space},
	morekeywords={with, data, refresh, materialized, explain, rank, over, partition}
}

\usepackage[colorlinks = true, linkcolor = blue, urlcolor  = blue, citecolor = blue, anchorcolor = blue]{hyperref}
\usepackage[utf8]{inputenc}

\renewcommand*{\arraystretch}{1.4}

\newcolumntype{L}[1]{>{\raggedright\arraybackslash}p{#1}}
\newcolumntype{R}[1]{>{\raggedleft\arraybackslash}p{#1}}
\newcolumntype{C}[1]{>{\centering\let\newline\\\arraybackslash\hspace{0pt}}m{#1}}

\newcommand{\E}{\mathbb{E}}
\DeclareMathOperator{\rk}{rk}
\DeclareMathOperator{\Var}{Var}
\DeclareMathOperator{\Cov}{Cov}

\title{\textbf{Scalable Data Engineering, Exercise 3}}
\author{\textsc{Henry Haustein}}
\date{}

\begin{document}
	\maketitle
	
	\section*{Task 1}
	\begin{enumerate}[label=(\alph*)]
		\item True
		\item False, maintenance is different
		\item False, it is an NP-hard problem
		\item False, it depends on the use case
		\item True
	\end{enumerate}

	\section*{Task 2}
	\begin{enumerate}[label=(\alph*)]
		\item SQL:
		\begin{lstlisting}
EXPLAIN(SELECT * FROM ORDERS)
		\end{lstlisting}
		\item SQL:
		\begin{lstlisting}
SELECT * FROM ORDERS WHERE O_CLERK = "Clerk#000000322"
		\end{lstlisting}
		took 600ms.
		\item SQL:
		\begin{lstlisting}
CREATE INDEX ocleak_idx ON ODERS(O_CLERK)
		\end{lstlisting}
		\item Query from (b) then takes around 300ms
	\end{enumerate}

	\section*{Task 3}
	\begin{enumerate}[label=(\alph*)]
		\item SQL:
		\begin{lstlisting}[tabsize=2]
CREATE MATERIALIZED VIEW custperreg AS
	SELECT REGION.R_NAME, COUNT(*)
	FROM CUSTOMER, NATION, REGION
	WHERE 
		CUSTOMER.C_NATIONKEY = NATION.N_NATIONKEY AND 
		NATION.N_REGIONKEY = REGION.R_REGIONKEY
	GROUP BY REGION.R_NAME
WITH DATA
		\end{lstlisting}
		\item SQL:
		\begin{lstlisting}
INSERT INTO REGION(R_REGIONKEY, R_NAME, R_COMMENT) VALUES (5, "AUSTRALIA", "down under")

REFRESH MATERIALIZED VIEW custperreg
		\end{lstlisting}
		\item In the materialized view Australia is missing. That's why there are no customers for this region and since we are doing inner joins for the materialized view there are no joining partners for Australia. To fix this:
		\begin{lstlisting}[tabsize=2]
DROP MATERIALIZED VIEW custperreg

CREATE MATERIALIZED VIEW custperreg AS
	SELECT REGION.R_NAME, COUNT(CUSTOMER.C_CUSTKEY)
	FROM 
		CUSTOMER RIGHT OUTER JOIN (
			NATION RIGHT OUTER JOIN REGION 
			ON(NATION.N_REGIONKEY = REGION.R_REGIONKEY)
		) 
		ON(CUSTOMER.C_NATIONKEY = NATION.N_NATIONKEY)
	GROUP BY REGION.R_NAME
WITH DATA
		\end{lstlisting}
	\end{enumerate}

	\section*{Task 4}
	\begin{enumerate}[label=(\alph*)]
		\item We create a materialized view from the subquery:
		\begin{lstlisting}[tabsize=2]
CREATE MATERIALIZED VIEW linepart AS
	SELECT 
		PART.P_NAME,
		SUM(LINEITEM.L_QUANTITY) AS qty
	FROM PART, LINEITEM
	WHERE PART.P_NAME = LINEITEM.L_PARTKEY
	GROUP BY PART.P_NAME
WITH DATA

SELECT PART.P_NAME,  qty,  RANK() OVER(ORDER BY qty DESC)
FROM linepart
		\end{lstlisting}
		Creating the materialized view takes some time but then the actual query is done in about a second (from 10 seconds before).
		\item SQL:
		\begin{lstlisting}[tabsize=2]
CREATE MATERIALIZED VIEW linepart2 AS
	SELECT 
		PART.P_NAME,
		SUM(LINEITEM.L_QUANTITY) AS qty,
		LINEITEM.L_SUPPKEY
	FROM PART, LINEITEM
	WHERE PART.P_NAME = LINEITEM.L_PARTKEY
	GROUP BY PART.P_NAME, LINEITEM.L_SUPPKEY
WITH DATA
		\end{lstlisting}
		The query from (a) becomes then
		\begin{lstlisting}[tabsize=2]
SELECT PART.P_NAME,  SUM(qty),  RANK() OVER(ORDER BY SUM(qty) DESC)
FROM linepart2
		\end{lstlisting}
		with roughly 4 seconds run time. The other query becomes
		\begin{lstlisting}[tabsize=2]
SELECT
	PART.P_NAME,
	NATION.N_NAME,
	SUM(qty),
	RANK() OVER(
		PARTITION BY PART.P_NAME 
		ORDER BY SUM(qty) DESC
	)
FROM linepart2, NATION, SUPPLIER
WHERE
	LINEITEM.L_SUPPKEY = SUPPLIER.S_SUPPKEY AND
	SUPPLIER.S_NATIONKEY = NATION.N_NATIONKEY
GROUP BY PART.P_NAME, NATION.N_NAME
		\end{lstlisting}
		with now 21 seconds runtime (was 60 seconds before).
	\end{enumerate}

\end{document}