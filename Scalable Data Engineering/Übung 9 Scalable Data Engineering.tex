\documentclass{article}

\usepackage{amsmath,amssymb}
\usepackage{tikz}
\usepackage{pgfplots}
\usepackage{xcolor}
\usepackage[left=2.1cm,right=3.1cm,bottom=3cm,footskip=0.75cm,headsep=0.5cm]{geometry}
\usepackage{enumerate}
\usepackage{enumitem}
\usepackage{marvosym}
\usepackage{tabularx}
\usepackage{parskip}
\usepackage{longtable}

\usepackage{listings}
\definecolor{lightlightgray}{rgb}{0.95,0.95,0.95}
\definecolor{lila}{rgb}{0.8,0,0.8}
\definecolor{mygray}{rgb}{0.5,0.5,0.5}
\definecolor{mygreen}{rgb}{0,0.8,0.26}
\lstdefinestyle{R} {language=R}
\lstdefinestyle{Python} {language=Python}
\lstset{language=SQL,
	basicstyle=\ttfamily,
	keywordstyle=\color{lila},
	commentstyle=\color{lightgray},
	stringstyle=\color{mygreen}\ttfamily,
	backgroundcolor=\color{white},
	showstringspaces=false,
	numbers=left,
	numbersep=10pt,
	numberstyle=\color{mygray}\ttfamily,
	identifierstyle=\color{blue},
	xleftmargin=.1\textwidth, 
	%xrightmargin=.1\textwidth,
	escapechar=§,
	%literate={\t}{{\ }}1
	breaklines=true,
	postbreak=\mbox{\space},
	morekeywords={with, data, refresh, materialized, explain, rank, over, partition, uuid, extension, replace, function, returns, language}
}

\usepackage[colorlinks = true, linkcolor = blue, urlcolor  = blue, citecolor = blue, anchorcolor = blue]{hyperref}
\usepackage[utf8]{inputenc}

\renewcommand*{\arraystretch}{1.4}

\newcolumntype{L}[1]{>{\raggedright\arraybackslash}p{#1}}
\newcolumntype{R}[1]{>{\raggedleft\arraybackslash}p{#1}}
\newcolumntype{C}[1]{>{\centering\let\newline\\\arraybackslash\hspace{0pt}}m{#1}}

\newcommand{\E}{\mathbb{E}}
\DeclareMathOperator{\rk}{rk}
\DeclareMathOperator{\Var}{Var}
\DeclareMathOperator{\Cov}{Cov}

\title{\textbf{Scalable Data Engineering, Exercise 9}}
\author{\textsc{Henry Haustein}}
\date{}

\begin{document}
	\maketitle
	
	\section*{Task 1}
	\begin{enumerate}[label=(\alph*)]
		\item Python
		\begin{lstlisting}[style=Python]
print(torch.sum(x, 0))
print(torch.sum(x, 1))
print(torch.sum(x, 2))
		\end{lstlisting}
		\item Python
		\begin{lstlisting}[style=Python]
ones = torch.ones([1,2,4,5])
random = torch.randn(10,4)
random + ones.reshape([10, 4])
		\end{lstlisting}
	\end{enumerate}

	\section*{Task 2}
	Right answers are:
	\begin{itemize}
		\item Matrix Multiplication Backwards
		\item Backward Step of the linear layer
	\end{itemize}

	\section*{Task 3}
	Right answer is [0.27, 0.73]
	
	\section*{Task 4}
	\begin{enumerate}[label=(\alph*)]
		\item Python
		\begin{lstlisting}[style=Python, tabsize=2]
class NGramLanguageModeler(nn.Module):

	def __init__(self, vocab_size, embedding_dim, context_size, hidden_size):
		super(NGramLanguageModeler, self).__init__()
		self.embeddings = nn.Embedding(vocab_size, embedding_dim)
		self.linear1 = nn.Linear(context_size * embedding_dim, hidden_size)
		self.linear2 = nn.Linear(hidden_size, vocab_size)

	def forward(self, inputs):
		embeds = self.embeddings(inputs).view((1, -1))
		out = F.relu(self.linear1(embeds))
		out = self.linear2(out)
		log_probs = F.log_softmax(out, dim=1)
		return log_probs
		\end{lstlisting}
		\item Python, only the loop
		\begin{lstlisting}[style=Python, tabsize=2]
for epoch in tqdm.tqdm(range(EPOCHS),total=EPOCHS):
	total_loss = 0
	for context, target in trigrams:
		# Step 1. Prepare the inputs to be passed to the model (i.e, turn the words into integer indices and wrap them in tensors)
		context_idxs = torch.tensor([word_to_ix[w] for w in context], dtype=torch.long)
		
		# Step 2. Recall that torch *accumulates* gradients. Before passing in a new instance, you need to zero out the gradients from the old instance
		model.zero_grad()
		
		# Step 3. Run the forward pass, getting log probabilities over next words
		log_probs = model(context_idxs)
		
		# Step 4. Compute your loss function. (Again, Torch wants the target word wrapped in a tensor)
		loss = loss_function(log_probs, torch.tensor([word_to_ix[target]], dtype=torch.long))
		
		# Step 5. Do the backward pass and update the gradient
		loss.backward()
		optimizer.step()
		
		# Get the Python number from a 1-element Tensor by calling tensor.item()
		total_loss += loss.item()
	
	#print("\t", total_loss)
	losses.append(total_loss)
		\end{lstlisting}
		\item Python, only the second function
		\begin{lstlisting}[style=Python, tabsize=2]
def most_similar(word_to_test, word_to_ix):
	test_embedding = get_word_embedding_for_word(word_to_test, word_to_ix)
	
	# get embeddings for all other possible words like aaa bbb ccc
	cos = torch.nn.CosineSimilarity()
	results = {}
	for c in string.ascii_lowercase:
		c_embedding = get_word_embedding_for_word(c+c+c, word_to_ix)
		cosine_similarity = cos(test_embedding, c_embedding)
		results[c+c+c] = cosine_similarity.item()
		
	sorted_results =  dict(sorted(results.items(), key=lambda item: -item[1]))
	return sorted_results
		\end{lstlisting}
	\end{enumerate}

	\section*{Task 5}
	\begin{lstlisting}[style=Python, tabsize=2]
# Let's load the model
model = gensim.models.KeyedVectors.load_word2vec_format("word2vec_embeddings.bin", binary=True)

# Print the length fo the whole vocabulary 
print(len(model.wv.vocab))

# Print the embedding of the  word  good
print(model["good"])

# Get the 10 most similar words of "good"
print(model.most_similar("good", topn = 10))

# Check whether our embeddings are good at the analogy task, what is the result to "?-man" when looking at "queen-king"?
print(model.most_similar(positive = ["woman", "king"], negative = ["man"], topn = 1))
	\end{lstlisting}
	
\end{document}