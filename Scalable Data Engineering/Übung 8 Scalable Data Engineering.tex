\documentclass{article}

\usepackage{amsmath,amssymb}
\usepackage{tikz}
\usepackage{pgfplots}
\usepackage{xcolor}
\usepackage[left=2.1cm,right=3.1cm,bottom=3cm,footskip=0.75cm,headsep=0.5cm]{geometry}
\usepackage{enumerate}
\usepackage{enumitem}
\usepackage{marvosym}
\usepackage{tabularx}
\usepackage{parskip}
\usepackage{longtable}

\usepackage{listings}
\definecolor{lightlightgray}{rgb}{0.95,0.95,0.95}
\definecolor{lila}{rgb}{0.8,0,0.8}
\definecolor{mygray}{rgb}{0.5,0.5,0.5}
\definecolor{mygreen}{rgb}{0,0.8,0.26}
\lstdefinestyle{R} {language=R}
\lstset{language=SQL,
	basicstyle=\ttfamily,
	keywordstyle=\color{lila},
	commentstyle=\color{lightgray},
	stringstyle=\color{mygreen}\ttfamily,
	backgroundcolor=\color{white},
	showstringspaces=false,
	numbers=left,
	numbersep=10pt,
	numberstyle=\color{mygray}\ttfamily,
	identifierstyle=\color{blue},
	xleftmargin=.1\textwidth, 
	%xrightmargin=.1\textwidth,
	escapechar=§,
	%literate={\t}{{\ }}1
	breaklines=true,
	postbreak=\mbox{\space},
	morekeywords={with, data, refresh, materialized, explain, rank, over, partition, uuid, extension, replace, function, returns, language}
}

\usepackage[colorlinks = true, linkcolor = blue, urlcolor  = blue, citecolor = blue, anchorcolor = blue]{hyperref}
\usepackage[utf8]{inputenc}

\renewcommand*{\arraystretch}{1.4}

\newcolumntype{L}[1]{>{\raggedright\arraybackslash}p{#1}}
\newcolumntype{R}[1]{>{\raggedleft\arraybackslash}p{#1}}
\newcolumntype{C}[1]{>{\centering\let\newline\\\arraybackslash\hspace{0pt}}m{#1}}

\newcommand{\E}{\mathbb{E}}
\DeclareMathOperator{\rk}{rk}
\DeclareMathOperator{\Var}{Var}
\DeclareMathOperator{\Cov}{Cov}

\title{\textbf{Scalable Data Engineering, Exercise 8}}
\author{\textsc{Henry Haustein}}
\date{}

\begin{document}
	\maketitle
	
	\section*{Task 1}
	\begin{enumerate}[label=(\alph*)]
		\item False. Supervised Learning needs labelled data.
		\item False. The output is $f(Xw)$.
		\item True.
		\item False. For non-linearity you need at least one non-linear activation function.
		\item True.
		\item False. Three derivatives.
		\item False. Exploding gradients lead to instability because of huge changes in weights.
		\item False. Problem are dead neurons.
		\item True.
	\end{enumerate}

	\section*{Task 2}
	Using an Excel Spreadsheet we get
	\begin{center}
		\begin{longtable}{ccc|cccc|cc}
			$x_1$ & $x_2$ & $t$ & $w_{11}^{old}$ & $w_{21}^{old}$ & $z$ & $a$ & $w_{11}^{new}$ & $w_{21}^{new}$ \\
			\hline
			1 & 0 & 1 & -0,2 & 0,4 & -0,2 & 0 & 0,1 & 0,4
\\
			0 & 1 & 1 & 0,1 & 0,4 & 0,4 & 0 & 0,1 & 0,7
\\
			0 & 0 & 0 & 0,1 & 0,7 & 0 & 0 & 0,1 & 0,7
\\
			1 & 1 & 1 & 0,1 & 0,7 & 0,8 & 0 & 0,4 & 1
\\
			\hline
			1 & 0 & 1 & 0,4 & 1 & 0,4 & 0 & 0,7 & 1
\\
		 	0 & 1 & 1 & 0,7 & 1 & 1 & 1 & 0,7 & 1
\\
			0 & 0 & 0 & 0,7 & 1 & 0 & 0 & 0,7 & 1
\\
			1 & 1 & 1 & 0,7 & 1 & 1,7 & 1 & 0,7 & 1
\\
			\hline
			1 & 0 & 1 & 0,7 & 1 & 0,7 & 0 & 1 & 1
\\
			0 & 1 & 1 & 1 & 1 & 1 & 1 & 1 & 1
\\
			0 & 0 & 0 & 1 & 1 & 0 & 0 & 1 & 1
\\
			1 & 1 & 1 & 1 & 1 & 2 & 1 & 1 & 1
\\
			\hline
			1 & 0 & 1 & 1 & 1 & 1 & 1 & 1 & 1
\\
			0 & 1 & 1 & 1 & 1 & 1 & 1 & 1 & 1
\\
			0 & 0 & 0 & 1 & 1 & 0 & 0 & 1 & 1
\\
			1 & 1 & 1 & 1 & 1 & 2 & 1 & 1 & 1
		\end{longtable}
	\end{center}
	The weights converge to 1.

	\section*{Task 3}
	Since we know the true values $t_{a_1}$, $t_{a_2}$ and $t_{a_3}$ we can train all three perceptrons individually. For the first perceptron we see that it should only fire if and only if $x_1$ fires. This leads to $w_{11} = 1$ and $w_{21} = -1$. For the second perceptron it is the opposite: It should only fire if and only if $x_2$ fires. Therefore $w_{21} = -1$ and $w_{22} = 1$.
	
	The third perceptron models the XOR function which is not linear separable. Therefore we won't find weights $w_{13}$ and $w_{23}$ that will produce correct results.
	
\end{document}