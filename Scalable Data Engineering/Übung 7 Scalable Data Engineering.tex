\documentclass{article}

\usepackage{amsmath,amssymb}
\usepackage{tikz}
\usepackage{pgfplots}
\usepackage{xcolor}
\usepackage[left=2.1cm,right=3.1cm,bottom=3cm,footskip=0.75cm,headsep=0.5cm]{geometry}
\usepackage{enumerate}
\usepackage{enumitem}
\usepackage{marvosym}
\usepackage{tabularx}
\usepackage{parskip}

\usepackage{listings}
\definecolor{lightlightgray}{rgb}{0.95,0.95,0.95}
\definecolor{lila}{rgb}{0.8,0,0.8}
\definecolor{mygray}{rgb}{0.5,0.5,0.5}
\definecolor{mygreen}{rgb}{0,0.8,0.26}
\lstdefinestyle{R} {language=R}
\lstset{language=SQL,
	basicstyle=\ttfamily,
	keywordstyle=\color{lila},
	commentstyle=\color{lightgray},
	stringstyle=\color{mygreen}\ttfamily,
	backgroundcolor=\color{white},
	showstringspaces=false,
	numbers=left,
	numbersep=10pt,
	numberstyle=\color{mygray}\ttfamily,
	identifierstyle=\color{blue},
	xleftmargin=.1\textwidth, 
	%xrightmargin=.1\textwidth,
	escapechar=§,
	%literate={\t}{{\ }}1
	breaklines=true,
	postbreak=\mbox{\space},
	morekeywords={with, data, refresh, materialized, explain, rank, over, partition, uuid, extension, replace, function, returns, language}
}

\usepackage[colorlinks = true, linkcolor = blue, urlcolor  = blue, citecolor = blue, anchorcolor = blue]{hyperref}
\usepackage[utf8]{inputenc}

\renewcommand*{\arraystretch}{1.4}

\newcolumntype{L}[1]{>{\raggedright\arraybackslash}p{#1}}
\newcolumntype{R}[1]{>{\raggedleft\arraybackslash}p{#1}}
\newcolumntype{C}[1]{>{\centering\let\newline\\\arraybackslash\hspace{0pt}}m{#1}}

\newcommand{\E}{\mathbb{E}}
\DeclareMathOperator{\rk}{rk}
\DeclareMathOperator{\Var}{Var}
\DeclareMathOperator{\Cov}{Cov}

\title{\textbf{Scalable Data Engineering, Exercise 7}}
\author{\textsc{Henry Haustein}}
\date{}

\begin{document}
	\maketitle
	
	\textit{I don't know what happened to exercise 6, there aren't any tasks.}
	
	\section*{Task 1}
	\begin{enumerate}[label=(\alph*)]
		\item False. Most obviously SQL is not Turing complete.
		\item False. Functions do calculations and have a return value, procedures don't have a return value and can do data manipulations.
		\item False. An aggregate function calls the step function for every processed tuple and a final function on the end.
		\item False. Regression can do this but it's not limited to that.
		\item False. Overfitting is a real issue.
		\item False. We don't necessarily want good performance on the training set, we want it on the test set. Overfitting does not give that.
		\item False. In general this is not a good idea.
	\end{enumerate}

	\section*{Task 2}
	With R this is pretty easy
	\begin{lstlisting}[style=R]
x = c(1, 2.5, 3, 4.25, 5)
y = c(4.5, 4, 2.5, 2, 1)
summary(lm(y ~ x))
	\end{lstlisting}
	Gives
	\begin{table}[ht]
	\centering
	\begin{tabular}{rrrrr}
	  \hline
	 & Estimate & Std. Error & t value & Pr($>$$|$t$|$) \\ 
	  \hline
	(Intercept) & 5.5928 & 0.5247 & 10.66 & 0.0018 \\ 
	  x & -0.8866 & 0.1523 & -5.82 & 0.0101 \\ 
	   \hline
	\end{tabular}
\end{table}

	\section*{Task 3}
	In SQL:
	\begin{lstlisting}[tabsize=2]
CREATE TABLE reg (size numeric, price numeric);

INSERT INTO reg (size, price) VALUES (1000, 275000), (2500, 370000), (800, 175000), (1900, 225000), (3000, 500000);

CREATE EXTENSION plpython3u;

CREATE OR REPLACE FUNCTION ols() RETURNS numeric[] AS
$$
	from sklearn.linear_model import LinearRegression
	
	X = []
	y = []
	
	rv = plpy.execute("SELECT * FROM reg")
	for row in rv:
		X.append([row["size"]])
		y.append(row["price"])
	
	model = LinearRegression().fit(X, y)
	return [*model.coef_, model.intercept]
$$
LANGUAGE plpython3u;

SELECT ols();
	\end{lstlisting}
\end{document}