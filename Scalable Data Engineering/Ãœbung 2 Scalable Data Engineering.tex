\documentclass{article}

\usepackage{amsmath,amssymb}
\usepackage{tikz}
\usepackage{pgfplots}
\usepackage{xcolor}
\usepackage[left=2.1cm,right=3.1cm,bottom=3cm,footskip=0.75cm,headsep=0.5cm]{geometry}
\usepackage{enumerate}
\usepackage{enumitem}
\usepackage{marvosym}
\usepackage{tabularx}
\usepackage{parskip}

\usepackage{listings}
\definecolor{lightlightgray}{rgb}{0.95,0.95,0.95}
\definecolor{lila}{rgb}{0.8,0,0.8}
\definecolor{mygray}{rgb}{0.5,0.5,0.5}
\definecolor{mygreen}{rgb}{0,0.8,0.26}
%\lstdefinestyle{java} {language=java}
\lstset{language=SQL,
	basicstyle=\ttfamily,
	keywordstyle=\color{lila},
	commentstyle=\color{lightgray},
	stringstyle=\color{mygreen}\ttfamily,
	backgroundcolor=\color{white},
	showstringspaces=false,
	numbers=left,
	numbersep=10pt,
	numberstyle=\color{mygray}\ttfamily,
	identifierstyle=\color{blue},
	xleftmargin=.1\textwidth, 
	%xrightmargin=.1\textwidth,
	escapechar=§,
	%literate={\t}{{\ }}1
	breaklines=true,
	postbreak=\mbox{\space}
}

\usepackage[colorlinks = true, linkcolor = blue, urlcolor  = blue, citecolor = blue, anchorcolor = blue]{hyperref}
\usepackage[utf8]{inputenc}

\renewcommand*{\arraystretch}{1.4}

\newcolumntype{L}[1]{>{\raggedright\arraybackslash}p{#1}}
\newcolumntype{R}[1]{>{\raggedleft\arraybackslash}p{#1}}
\newcolumntype{C}[1]{>{\centering\let\newline\\\arraybackslash\hspace{0pt}}m{#1}}

\newcommand{\E}{\mathbb{E}}
\DeclareMathOperator{\rk}{rk}
\DeclareMathOperator{\Var}{Var}
\DeclareMathOperator{\Cov}{Cov}

\title{\textbf{Scalable Data Engineering, Exercise 2}}
\author{\textsc{Henry Haustein}}
\date{}

\begin{document}
	\maketitle
	
	\section*{Task 1}
	\begin{enumerate}[label=(\alph*)]
		\item False, for some reporting functions we need \texttt{OVER}
		\item True
		\item False, this does \texttt{CUBE}
		\item True
		\item True
		\item False, SQL integer division returns an integer
		\item False, \texttt{ORDER BY} in an \texttt{OVER} clause has no effect on the sorting of the result. And an \texttt{ORDER BY} outside an \texttt{OVER} only sorts the result.
		\item False, MDX operates on data cubes. This can sometimes work on relational databases but it's not always the case.
	\end{enumerate}

	\section*{Task 2}
	\begin{enumerate}[label=(\alph*)]
		\item SQL:
		\begin{lstlisting}[tabsize=2]
SELECT ORDERS.O_ORDERKEY, AVG(ORDERS.O_TOTALPRICE)
FROM ORDERS
GROUP BY ORDERS.O_ORDERKEY
		\end{lstlisting}
		\item SQL:
		\begin{lstlisting}[tabsize=2]
SELECT REGION.R_NAME, NATION.N_NAME,  SUM(ORDERS.O_TOTALPRICE)
FROM REGION, ORDERS, CUSTOMER, NATION
WHERE 
	ORDERS.O_CUSTKEY = CUSTOMER.C_CUSTKEY AND 
	CUSTOMER.C_NATIONKEY = NATION.N_NATIONKEY AND 
	NATION.N_REGIONKEY = REGION.R_REGIONKEY 
GROUP BY ROLLUP(REGION.R_NAME, NATION.N_NAME)
		\end{lstlisting}
		\item SQL:
		\begin{lstlisting}[tabsize=2]
SELECT
	year, quarter, sales, sales/SUM(sales)
FROM (
	SELECT 
		EXTRACT(year FROM ORDERS.O_ORDERDATE) AS year, 
		EXTRACT(quarter FROM ORDERS.O_ORDERDATE) AS quarter,
		COUNT(*) AS sales
	FROM ORDERS
	GROUP BY year, quarter
) AS x
		\end{lstlisting}
		\item SQL:
		\begin{lstlisting}[tabsize=2]
SELECT
	PART.P_NAME, 
	qty, 
	RANK() OVER(ORDER BY qty DESC)
FROM (
	SELECT 
		PART.P_NAME,
		SUM(LINEITEM) AS qty
	FROM PART, LINEITEM
	WHERE PART.P_NAME = LINEITEM.L_PARTKEY
	GROUP BY PART.P_NAME
) AS x
		\end{lstlisting}
		\item SQL:
		\begin{lstlisting}[tabsize=2]
SELECT
	PART.P_NAME,
	NATION.N_NAME,
	SUM(LINEITEM.L_QUANTITY),
	RANK() OVER(
		PARTITION BY PART.P_NAME 
		ORDER BY SUM(LINEITEM.L_QUANTITY) DESC
		)
FROM PART, NATION, LINEITEM, SUPPLIER
WHERE
	PART.P_PARTKEY = LINEITEM.L_PARTKEY AND
	LINEITEM.L_SUPPKEY = SUPPLIER.S_SUPPKEY AND
	SUPPLIER.S_NATIONKEY = NATION.N_NATIONKEY
GROUP BY PART.P_NAME, NATION.N_NAME
		\end{lstlisting}
	\end{enumerate}

\end{document}