\documentclass{article}

\usepackage{amsmath,amssymb}
\usepackage{tikz}
\usetikzlibrary{er,positioning}
\usepackage{pgfplots}
\usepackage{xcolor}
\usepackage[left=2.1cm,right=3.1cm,bottom=3cm,footskip=0.75cm,headsep=0.5cm]{geometry}
\usepackage{enumerate}
\usepackage{enumitem}
\usepackage{marvosym}
\usepackage{tabularx}
\usepackage{parskip}
\usepackage{longtable}

\usepackage{listings}
\definecolor{lightlightgray}{rgb}{0.95,0.95,0.95}
\definecolor{lila}{rgb}{0.8,0,0.8}
\definecolor{mygray}{rgb}{0.5,0.5,0.5}
\definecolor{mygreen}{rgb}{0,0.8,0.26}
\lstdefinestyle{sql} {language=sql}
\lstdefinestyle{python} {language=python, morekeywords={with, as}}
\lstset{basicstyle=\ttfamily,
	keywordstyle=\color{lila},
	commentstyle=\color{lightgray},
	stringstyle=\color{mygreen}\ttfamily,
	backgroundcolor=\color{white},
	showstringspaces=false,
	numbers=left,
	numbersep=10pt,
	numberstyle=\color{mygray}\ttfamily,
	identifierstyle=\color{blue},
	xleftmargin=.1\textwidth, 
	%xrightmargin=.1\textwidth,
	escapechar=§,
	breaklines=true,
	postbreak=\mbox{\space}
}

\usepackage[utf8]{inputenc}

\renewcommand*{\arraystretch}{1.4}

\newcolumntype{L}[1]{>{\raggedright\arraybackslash}p{#1}}
\newcolumntype{R}[1]{>{\raggedleft\arraybackslash}p{#1}}
\newcolumntype{C}[1]{>{\centering\let\newline\\\arraybackslash\hspace{0pt}}m{#1}}

\newcommand{\E}{\mathbb{E}}
\DeclareMathOperator{\rk}{rk}
\DeclareMathOperator{\Var}{Var}
\DeclareMathOperator{\Cov}{Cov}

\def\ojoin{\setbox0=\hbox{$\bowtie$}%
	\rule[-.02ex]{.25em}{.4pt}\llap{\rule[\ht0]{.25em}{.4pt}}}
\def\leftouterjoin{\mathbin{\ojoin\mkern-5.8mu\bowtie}}
\def\rightouterjoin{\mathbin{\bowtie\mkern-5.8mu\ojoin}}
\def\fullouterjoin{\mathbin{\ojoin\mkern-5.8mu\bowtie\mkern-5.8mu\ojoin}}

\title{\textbf{Datenbanken, Hands-on 4}}
\author{\textsc{Henry Haustein}}
\date{}

\begin{document}
	\maketitle

	\section*{Aufgabe 1}
	Wir definieren dazu am Anfang des Quelltextes eine Variable \texttt{neuaufbau}, deren Wert wir auf 0 setzen. Weiterhin müssen wir den Wert dieser Variable immer dann um 1 erhöhen, wenn die Funktion \texttt{rehash\_tables()} aufgerufen wird:
	\begin{lstlisting}[style=python,tabsize=2]
def rehash_tables(self) -> None:
	global neuaufbau
	neuaufbau = neuaufbau + 1
	# create new hash functions
	...
	\end{lstlisting}
	Das Keyword \texttt{global} ist hier nötig, weil die Variable außerhalb der Funktion definiert wurde und damit normalerweise nicht veränderbar innerhalb der Funktion ist. Mit \texttt{global} kann man das ändern. Nach durchlaufen des Skripts hat \texttt{neuaufbau} dann den Wert von 841, das heißt die Hashtabellen wurden 841 mal neu aufgebaut.

	\section*{Aufgabe 2}
	Hierfür legen wir uns ein Dictionary an, welches alle Sprachen und die Anzahl wie oft diese vertrieben worden sind, enthält:
	\begin{lstlisting}[style=python,tabsize=2]
counter = {sprache: 0 for sprache in languages_list}
	\end{lstlisting}
	Wir ändern dann einen Teil der \texttt{insert()}-Funktion:
	\begin{lstlisting}[style=python,tabsize=2]
tmp: Node = self.hash_table_1[self.hash_func_1(key)]
self.hash_table_1[self.hash_func_1(key)] = Node(key, value)
key = tmp.key
value = tmp.value
# sprache (value) wurde vertrieben -> counter erhoehen
counter[value] = counter[value] + 1

if self.hash_table_2[self.hash_func_2(key)] == None:
self.hash_table_2[self.hash_func_2(key)] = Node(key, value)
return True

# same as before for the other hash_table
tmp: Node = self.hash_table_2[self.hash_func_2(key)]
self.hash_table_2[self.hash_func_2(key)] = Node(key, value)
key = tmp.key
value = tmp.value
# sprache (value) wurde vertrieben -> counter erhoehen
counter[value] = counter[value] + 1
	\end{lstlisting}
	Ganz zum Schluss lassen wir dieses Dictionary noch sortieren mittels
	\begin{lstlisting}[style=python,tabsize=2]
print(sorted(counter.items(), key=lambda x: x[1], reverse=True))
	\end{lstlisting}
	und erhalten
	\begin{lstlisting}[style=python]
[('C', 6812), ('Basic', 6377), ('C#', 5710), ('Ada', 5155), ('C++', 4998), ('D', 4291), ('Kotlin', 4093), ('Perl', 3709), ('PHP', 3609), ('Prolog', 3540), ('MATLAB', 3439), ('Javascript', 3357), ('Eiffel', 3241), ('Pascal', 3193), ('Erlang', 3112), ('Fortran', 3109), ('Lisp', 2874), ('Go', 2808), ('Smalltalk', 2594), ('Ruby', 2355), ('Scala', 2109), ('Haskell', 2055), ('Python', 2030), ('SQL', 1866), ('F#', 1732), ('Java', 1710), ('Swift', 1260), ('TypeScript', 228)]
	\end{lstlisting}
	Offensichtlich wurde C mit 6812 mal am öftesten aus dem Nest vertrieben.
	
	\section*{Aufgabe 3}
	Die Löschfunktion ist
	\begin{lstlisting}[style=python, tabsize=2]
def delete(self, value):
	# check hash table 1
	for n in self.hash_table_1:
		try:
			if n.value == value:
				self.hash_table_1[self.hash_func_1(n.key)] = None
				break
		except Exception as e:
			pass

	# check hash table 2
	for n in self.hash_table_2:
		try:
			if n.value == value:
				self.hash_table_2[self.hash_func_2(n.key)] = None
				break
		except Exception as e:
			pass
	\end{lstlisting}
	Und dann noch
	\begin{lstlisting}[style=python,tabsize=2]
language_cuckoo.delete("Erlang")
language_cuckoo.delete("Python")
language_cuckoo.delete("TypeScript")
language_cuckoo.insert(8, "Erlang")
language_cuckoo.insert(27, "TypeScript")
language_cuckoo.insert(21, "Python")
language_cuckoo.print_hash_tables()
	\end{lstlisting}
	ergibt die folgenden Hashtables
	\begin{center}
		\begin{tabular}{c|c|c}
			\textbf{Position} & \textbf{Hashtable 1} & \textbf{Hashtable 2} \\
			\hline
			0 & (0, Ada) & (1, Basic) \\
			1 & & (13, Javascript) \\
			2 & (14, Kotlin) & \\
			3 & (24, Smalltalk) & \\
			4 & (2, C) & (12, Java) \\
			5 & (25, SQL) & \\
			6 & (16, MATLAB) & (10, Go)  \\
			7 & (26, Swift) & (22, Ruby) \\
			8 & (4, C++) & (8, Erlang) \\
			9 & (27, TypeScript) & (21, Python) \\
			10 & (18, Perl) & (7, F\#) \\
			11 & (9, Fortran) & (19, PHP) \\
			12 & (6, Eiffel) & (5, D) \\
			13 & (23, Scala) & (17, Pascal) \\
			14 & (20, Prolog) & (3, C\#) \\
			15 & (11, Haskell) & (15, Lisp)
		\end{tabular}
	\end{center}
	An der letzten Position der zweiten Hashtabelle steht also Lisp.
	
	\section*{Aufgabe 4}
	Der Programmcode in (d) verläuft fälschlicherweise richtig. Problem ist hier, dass wir in der ersten Hashtabelle an der Position \texttt{hash2} nachschauen, obwohl die richtige Position hier \texttt{hash1} wäre. Selbiges für die 2. Hashtabelle.
\end{document}