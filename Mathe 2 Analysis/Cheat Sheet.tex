\documentclass[10pt,landscape,a4paper]{article}
\usepackage{multicol}
\usepackage{calc}
\usepackage{ifthen}
\usepackage[landscape,top=1cm,bottom=1cm,right=1cm,left=1cm,noheadfoot,bindingoffset=0pt,marginparwidth=0pt,marginparsep=0pt]{geometry}
\usepackage{amsmath,amsfonts,amssymb,mathtools}
\usepackage{graphicx}
\usepackage{fontenc}
%\usepackage{lua-visual-debug}
\usepackage{enumitem}
\usepackage[utf8]{inputenc}

% Turn off header and footer
\pagestyle{empty}

% Redefine section commands to use less space
\makeatletter
\renewcommand{\section}{\@startsection{section}{1}{0mm}%
                                {5pt}{1pt}%x
                                {\normalfont\small\bfseries}}
\renewcommand{\subsection}{\@startsection{subsection}{2}{0mm}%
                                {5pt}%
                                {1pt}%
                                {\normalfont\small\underline}}
\makeatother

\setlength{\parindent}{0pt}
\setlength{\parskip}{0pt plus 0.5ex}

\setlist{
	noitemsep,
	topsep=-\parskip,
	leftmargin=2em
}
\setcounter{secnumdepth}{0}


% -----------------------------------------------------------------------

\begin{document}

\raggedright
\footnotesize
\begin{multicols*}{4}


% multicol parameters
% These lengths are set only within the two main columns
\setlength{\columnseprule}{0.1pt}
\setlength{\premulticols}{1pt}
\setlength{\postmulticols}{1pt}
\setlength{\multicolsep}{1pt}
\setlength{\columnsep}{1pt}

\begin{center}
	\normalsize{\textbf{Komplexe Zahlen}}
\end{center}
\textbf{Darstellung}
\begin{itemize}
	\item Algebraische Form: $z=x+yi$
	\item Trigonometrische Form: $z=r\cdot (\cos(\varphi) + i\sin(\varphi))$
	\item Exponentialform: $z=r\cdot e^{i\varphi}$
	\item[$\Rightarrow$] $\varphi$ heißt \textit{Argument} oder \textit{Phase}
	\item[$\Rightarrow$] $r$ heißt \textit{Betrag} oder \textit{Modul}
\end{itemize}
\textbf{Umwandlung}
\begin{align}
	r &= \vert z\vert=\sqrt{x^2+y^2} \notag \\
	\varphi &= \arctan\left(\frac{y}{x}\right) + \begin{cases}
		0 &z \text{ im 1. od. 4. Quad.} \\
		-\pi &z \text{ im 2. Quad.} \\
		\pi &z \text{ im 3. Quad.}
	\end{cases} \notag \\
	x &= r\cdot \cos(\varphi) \notag \\
	y &= r\cdot \sin(\varphi) \notag
\end{align}
\textbf{Rechenregeln}
\begin{itemize}
	\item $(a+bi)+(c+di) = (a+c) + (b+d)i$
	\item $(a+bi)-(c+di) = (a+c) - (b+d)i$
	\item $(a+bi)\cdot (c+di) = (ac-bd) + (ad+bc)i$
	\item $\overline{(a+bi)} = a-bi$
	\item $\vert a+bi\vert = \sqrt{a^2+b^2}$
	\item $\frac{a+bi}{c+di}=\frac{(a+bi)\cdot\overline{(c+di)}}{\vert c+di\vert^2}$
	\item Komplexe Lösungen quadratischer Gleichungen:
	\begin{align}
		0 &= x^2 + px + q \notag \\
		x_{1/2} &= -\frac{p}{2} \pm i\cdot\sqrt{q-\frac{p^2}{4}} \notag
	\end{align}
\end{itemize}

\begin{center}
	\normalsize{\textbf{Folgen}}
\end{center}
\textbf{Eigenschaften}
\begin{itemize}
	\item geometrische Folge: $\frac{a_{n+1}}{a_n}=q$, nächstes Folgenglied ist immer um bestimmten Faktor $q$ größer
	\item arithmetische Folge: $a_{n+1}-a_n=d$, nächstes Folgenglied ist immer um bestimmten Wert $d$ größer.
\end{itemize}
\textbf{Konvergenz}
\begin{itemize}
	\item Grenzwert: $\lim_{n\to\infty} a_n = x$
	\item Nullfolge: $x=0$
	\item Folge beschränkt und monoton fallend/steigend $\Rightarrow$ konvergent
\end{itemize}
\begin{align}
	\lim_{n\to\infty}\left(1+\frac{a}{b\cdot k}\right)^k = e^{\frac{a}{b}} \notag
\end{align}

\begin{center}
	\normalsize{\textbf{Reihen}}
\end{center}
\textbf{Begriffe}
\begin{itemize}
	\item Reihe: $s=\sum_{k=0}^{\infty} a_k$
	\item $n$-te Partialsumme: $s_n=\sum_{k=0}^{n} a_k$
	\item Grenzwert: $\lim_{n\to\infty} s_n$
\end{itemize}
\begin{align}
	\sum_{k=0}^\infty a_0\cdot q^k = \frac{a_0}{1-q} \notag
\end{align}
\textbf{Konvergenzkriterien}
\begin{itemize}
	\item Trivialkriterium: $\sum_{k=0}^{\infty} a_k$ ist konvergent, wenn $a_k$ eine Nullfolge ist
	\item Leibnitz-Kriterium (für alternierende Folgen): $\sum_{k=0}^{\infty} (-1)^k\cdot a_k$ ist konvergent, wenn $a_k$ eine monoton fallende Nullfolge ist
	\item Majorantenkriterium: $\sum_{k=0}^{\infty} a_k$ ist konvergent, wenn $\sum_{k=0}^{\infty} b_k$ konvergent ist und $0\le a_k\le b_k$ gilt. $b_k=\frac{1}{k^2}$ funktioniert meistens gut als Majorante.
	\item Minorantenkriterium: $\sum_{k=0}^{\infty} a_k$ ist divergent, wenn $\sum_{k=0}^{\infty} b_k$ divergent ist und $0\le b_k\le a_k$ gilt. $b_k=\frac{1}{k}$ funktioniert meistens gut als Minorante.
	\item Quotientenkriterium: $\sum_{k=0}^{\infty} a_k$ ist konvergent, wenn
	\begin{align}
		\lim_{k\to\infty}\left\vert\frac{a_{k+1}}{a_k}\right\vert<1 \notag
	\end{align}
	\item Wurzelkriterium: $\sum_{k=0}^{\infty} a_k$ ist konvergent, wenn
	\begin{align}
		\lim_{k\to\infty}\sqrt[k]{\vert a_k\vert} <1 \notag
	\end{align}
\end{itemize}

\begin{center}
	\normalsize{\textbf{Funktionen einer Variablen}}
\end{center}
\textbf{Umkehrfunktion berechnen:} Gleichung $y=f(x)$ nach $x$ umstellen und anschließend $x$ und $y$ vertauschen

\textbf{Gebrochen rationale Funktionen $y=\frac{P_m(x)}{Q_n(x)}$}
\begin{itemize}
	\item Nullstelle: $P_m(x_0)=0$ und $Q_n(x_0)\neq 0$
	\item Polstelle: $P_m(x_0)\neq 0$ und $Q_n(x_0)=0$
	\item Lücke: $P_m(x_0)=0$ und $Q_n(x_0)=0$
\end{itemize}

\begin{center}
	\normalsize{\textbf{Differentialrechnung für Funktionen einer Variablen}}
\end{center}
Funktion $f$ ist in $x_0$ differenzierbar, wenn $\lim_{h\to 0} \frac{f(x_0+h)-f(x_0)}{h}$ existiert.

\textbf{Differentiationsregeln}
\begin{itemize}
	\item $(cf)'(x)=c\cdot f'(x)$
	\item $(f\pm g)'(x) = f'(x) \pm g'(x)$
	\item $(f\cdot g)'(x) = f'(x)g(x) + f(x)g'(x)$
	\item $\left(\frac{f}{g}\right)'(x)=\frac{f'(x)g(x) - f(x)g'(x)}{(g(x))^2}$
	\item $(f\circ g)'(x) = f'(g(x)) \cdot g'(x)$
	\item $(f^{-1})'(x)=\frac{1}{f'(f^{-1}(x))}$
	\item Logarithmische Differentiation: $f'(x)=f(x)\cdot (\ln(x))'$
\end{itemize}

\begin{center}
	\begin{tabular}{c|c}
		$f(x)$ & $f'(x)$ \\
		\hline
		$x^n$ & $nx^{n-1}$ \\
		$a^x$ & $a^x\cdot\ln (x)$ \\
		$e^x$ & $e^x$ \\
		$\log_a(x)$ & $\frac{1}{x\cdot\ln(a)}$ \\
		$\sin(x)$ & $\cos(x)$ \\
		$\cos(x)$ & $-\sin(x)$ \\
		$\tan(x)$ & $\frac{1}{cos^2(x)}$ \\
		$\cot(x)$ & $-\frac{1}{\sin^2(x)}$
	\end{tabular}
\end{center}

\textbf{Kurvendiskussion:} Nullstellen, Polstellen, Minima, Maxima, Monotonie, Wendepunkte, Konkavität, Konvexität, Verhalten im Unendlichen
\begin{itemize}
	\item $f'(x) >0$: monoton steigend
	\item $f''(x) >0$: konvex
	\item $f'(x_E)=0$ und $f''(x_E)\neq 0$: Extremstelle
	\item $f''(x_W)=0$ und $f'''(x_W)\neq 0$: Wendepunkt
\end{itemize}

\textbf{Änderungsrate und Elastizität}
\begin{itemize}
	\item Änderungsrate: $\varrho_f(x)=\frac{f'(x)}{f(x)}$
	\item Elastizität: $\varepsilon_f(x)=x\cdot\frac{f'(x)}{f(x)}=x\cdot\varrho_f(x)$
	\item $\varepsilon_{cf}(x)=\varepsilon_f(x)$
	\item $\varepsilon_{f+g}(x)=\frac{f(x)\varepsilon_f(x) + g(x)\varepsilon_g(x)}{f(x)+g(x)}$
	\item $\varepsilon_{fg}(x)=\varepsilon_f(x) + \varepsilon_g(x)$
	\item $\varepsilon_{\frac{f}{g}}(x)=\varepsilon_f(x)-\varepsilon_g(x)$
	\item $\varepsilon_{f\circ g}(x)=\varepsilon_f(g(x))\cdot \varepsilon_g(x)$
	\item Amoroso-Robinson-Gleichung
	\begin{align}
		f'(x) = \varepsilon_f(x) \cdot\frac{f(x)}{x} = \frac{f(x)}{x}\left(1+\varepsilon_{\frac{f(x)}{x}}(x)\right) \notag
	\end{align}
\end{itemize}

\begin{center}
	\normalsize{\textbf{Integralrechnung}}
\end{center}
$F(x)$ ist Stammfunktion von $f(x)$, wenn $F'(x)=f(x)$ gilt.

\textbf{Rechenregeln}
\begin{itemize}
	\item $\int \alpha f(x) + \beta g(x) \mathrm{d}x = \alpha\int f(x)\mathrm{d}x + \beta\int g(x)\mathrm{d}x$
	\item Partielle Integration: $f(x)\cdot g(x)=\int f'(x)g(x)\mathrm{d}x + \int f(x)g'(x)\mathrm{d}x$
	\item Substitution: $\int f(ax+b)\mathrm{d}x = \frac{1}{a}F(ax+b)+C$
	\item $\int_a^b f(x)\mathrm{d}x = F(b)-F(a)$
	\item $\int_a^b f(x)\mathrm{d}x = -\int_b^a f(x) \mathrm{d}x$
	\item $\int_a^a f(x)\mathrm{d}x = 0$
	\item $\int_a^c f(x)\mathrm{d}x = \int_a^b f(x)\mathrm{d}x + \int_b^c f(x)\mathrm{d}x$
\end{itemize}

\begin{center}
	\begin{tabular}{c|c}
		$f(x)$ & $\int f(x)\mathrm{d}x$ \\
		\hline
		$x^n$ & $\frac{x^{n+1}}{n+1}+C$ \\
		$\frac{1}{x}$ & $\ln(\vert x\vert) + C$ \\
		$\frac{1}{x^m}$ & $-\frac{1}{(m-1)x^{m-1}}+C$ \\
		$x^\alpha$ & $\frac{x^{\alpha+1}}{\alpha+1}+C$ \\
		$a^x$ & $\frac{a^x}{\ln(a)}+C$ \\
		$e^{\alpha x+\beta}$ & $\frac{e^{\alpha x+\beta}}{\alpha}+C$ \\
		$\ln(x)$ & $x\cdot\ln(x)-x+C$ \\
		$\sin(x)$ & $-\cos(x)+C$ \\
		$\cos(x)$ & $\sin(x)+C$
	\end{tabular}
\end{center}

\begin{center}
	\normalsize{\textbf{Differentialrechnung bezüglich mehrerer Variablen}}
\end{center}

\textbf{Höhenlinie} der Höhe $C$: $f:D_f\to\mathbb{R}^2$ $\Rightarrow$ $f(x_1,x_2)=C$

\textbf{Homogenität}: $f(\lambda x_1,...,\lambda x_n) = \lambda^\alpha f(x_1,...,x_n)$
\begin{itemize}
	\item $\alpha=1$: linear-homogen
	\item $\alpha>1$: superlinear-homogen
	\item $\alpha<1$: sublinear-homogen
\end{itemize}

\textbf{Partielle Änderungsrate und Elastizität}
\begin{itemize}
	\item Partielle Änderungsrate: $\varrho_f^{(x_k)}=\frac{f_{x_k}(x)}{f(x)}$
	\item Partielle Elastizität: $\varepsilon_f^{(x_k)}=x_k\cdot\varrho_f^{(x_k)}$
	\item Elastizitätsmatrix
	\begin{align}
		\varepsilon(x)=\begin{pmatrix}
		\varepsilon_{f_1}^{(x_1)}(x) & \dots & \varepsilon_{f_1}^{(x_n)}(x) \\
		\vdots & & \vdots \\
		\varepsilon_{f_l}^{(x_1)}(x) & \dots & \varepsilon_{f_l}^{(x_n)}(x)
		\end{pmatrix} \notag
	\end{align}
\end{itemize}

\textbf{Extremwertaufgaben} - Bedingungen
\begin{itemize}
	\item $f_{x_1}=0$ und $f_{x_2}=0$
	\item $\text{det}\begin{pmatrix}
		f_{x_1,x_1} & f_{x_1,x_2} \\ f_{x_2,x_1} & f_{x_2,x_2}
	\end{pmatrix}>0$
	\item $f_{x_1,x_1}<0$ (Maximalstelle) oder $f_{x_1,x_1}>0$ (Minimalstelle)
\end{itemize}

\textbf{Regression} - Methode der kleinsten Quadrate $\Rightarrow$ $f(x)=\hat{a}x+\hat{b}$
\begin{align}
	\hat{a} &= \frac{n \sum_{i=1}^{n} x_{i} y_{i}-\sum_{i=1}^{n} x_{i} \sum_{i=1}^{n} y_{i}}{n \sum_{i=1}^{n} x_{i}^{2}-\left(\sum_{i=1}^{n} x_{i}\right)^{2}} \notag \\
	\hat{b} &= \frac{\sum_{i=1}^{n} x_{i}^{2} \sum_{i=1}^{n} y_{i}-\sum_{i=1}^{n} x_{i} y_{i} \sum_{i=1}^{n} x_{i}}{n \sum_{i=1}^{n} x_{i}^{2}-\left(\sum_{i=1}^{n} x_{i}\right)^{2}} \notag
\end{align}

\textbf{Extremwertaufgaben mit Nebenbedingungen} - Variablensubstitution $\Rightarrow$ einfach ineinander einsetzen und Ableitung 0 setzen

\textbf{Extremwertaufgaben mit Nebenbedingungen} - Lagrange-Faktoren
\begin{itemize}
	\item Funktion $f(x_1,...,x_n)$ und Nebenbedingungen $g_i(x_1,...,x_n)=0$
	\item[$\Rightarrow$] Aufstellen der Lagrange-Funktion
	\begin{align}
		L(x_1,...,x_n,\lambda_1,...,\lambda_m) = f(...) + \sum_{i=1}^m (\lambda_i\cdot g_i(...)) \notag
	\end{align}
	\item[$\Rightarrow$] Partielle Ableitung von $L$ nach jeder Variable und Nullsetzen
	\item[$\Rightarrow$] Gleichungssystem lösen
\end{itemize}

\begin{center}
	\normalsize{\textbf{Lineare Differentialgleichungen erster Ordnung}}
\end{center}
Differentialgleichung $f'(x) + a(x)\cdot f(x)=b(x)$ hat allgemeine Lösung
\begin{align}
	f^\ast(x) = e^{-A(x)}\cdot\int b(x)e^{A(x)}\mathrm{d}x \notag
\end{align}

\end{multicols*}
\end{document}