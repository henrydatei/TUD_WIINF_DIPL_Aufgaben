\documentclass{article}

\usepackage{amsmath,amssymb}
\usepackage{tikz}
\usepackage{pgfplots}
\usepackage{xcolor}
\usepackage[left=2.1cm,right=3.1cm,bottom=3cm,footskip=0.75cm,headsep=0.5cm]{geometry}
\usepackage{enumerate}
\usepackage{enumitem}
\usepackage{marvosym}
\usepackage{tabularx}
\usepackage{parskip}
\usepackage{longtable}

\usepackage{listings}
\definecolor{lightlightgray}{rgb}{0.95,0.95,0.95}
\definecolor{lila}{rgb}{0.8,0,0.8}
\definecolor{mygray}{rgb}{0.5,0.5,0.5}
\definecolor{mygreen}{rgb}{0,0.8,0.26}
%\lstdefinestyle{java} {language=java}
\lstset{language=R,
	basicstyle=\ttfamily,
	keywordstyle=\color{lila},
	commentstyle=\color{lightgray},
	stringstyle=\color{mygreen}\ttfamily,
	backgroundcolor=\color{white},
	showstringspaces=false,
	numbers=left,
	numbersep=10pt,
	numberstyle=\color{mygray}\ttfamily,
	identifierstyle=\color{blue},
	xleftmargin=.1\textwidth, 
	%xrightmargin=.1\textwidth,
	escapechar=§,
	%literate={\t}{{\ }}1
	breaklines=true,
	postbreak=\mbox{\space}
}

\usepackage[colorlinks = true, linkcolor = blue, urlcolor  = blue, citecolor = blue, anchorcolor = blue]{hyperref}
\usepackage[utf8]{inputenc}

\renewcommand*{\arraystretch}{1.4}

\newcolumntype{L}[1]{>{\raggedright\arraybackslash}p{#1}}
\newcolumntype{R}[1]{>{\raggedleft\arraybackslash}p{#1}}
\newcolumntype{C}[1]{>{\centering\let\newline\\\arraybackslash\hspace{0pt}}m{#1}}

\newcommand{\E}{\mathbb{E}}
\DeclareMathOperator{\rk}{rk}
\DeclareMathOperator{\Var}{Var}
\DeclareMathOperator{\Cov}{Cov}

\title{\textbf{Kryptografie und -analyse, Übung 7}}
\author{\textsc{Henry Haustein}}
\date{}

\begin{document}
	\maketitle

	\section*{AES}
	\begin{enumerate}[label=(\alph*)]
		\item 10 Rundenschlüssel + 1 Schlüssel am Anfang \\
		11 Runden $\cdot$ 4 Blöcke = 44 Blöcke \\
		$w_4$: Rcon[$j=i/N_k$] = Rcon[1] = [$x^{j-1}$, 00, 00, 00] = [01, 00, 00, 00] \\
		$w_8$: Rcon[$j=8/4$] = Rcon[2] = [$x^{j-1}$, 00, 00, 00] = [02, 00, 00, 00] \\
		$w_{40}$: Rcon[$j=40/4$] = Rcon[10] = [$x^{j-1}$, 00, 00, 00] = [36, 00, 00, 00]. Muss noch modulo $x^8+x^4+x^3+x+1$ gerechnet werden:
		\begin{align}
			x^9 \div (x^8+x^4+x^3+x+1) = x \quad R: x^5+x^4+x^2+x = 00110110_2 = 36_{16} \notag
		\end{align}
		\item $k_0$
		\begin{center}
			\begin{tabular}{c|c|c|c}
				$w_0$ & $w_1$ & $w_2$ & $w_3$ \\
				\hline
				2b & 28 & ab & 09 \\
				7e & ae & f7 & cf  \\
				15 & d2 & 15 & 4f \\
				16 & a6 & 88 & 3c
			\end{tabular}
		\end{center}
		$k_1 = w_4w_5w_6w_7$ mit
		\begin{itemize}
			\item $w_4 = w_0 \oplus (\text{Rcon}[1] \oplus \text{SubWord}(\text{Rot}(w_3)))$
			\begin{itemize}
				\item Rot($w_3$) = cf4f3c09
				\item SubWord(cf4f3c09) = 8a84eb01
				\item Rcon[1] $\oplus$ 8a84eb01 = 01000000 $\oplus$ 8a84eb01 = 8b84eb01
				\item $w_0 \oplus \text{8b84eb01} = \text{2b7e1516} \oplus \text{8b84eb01} = \text{a0fafe17}$
			\end{itemize}
			\item $w_5 = w_4 \oplus w_1$ = a0fafe17 $\oplus$ 28aed2a6 = 8b542cb1
			\item $w_6 = w_5 \oplus w_2$ = 8b542cb1 $\oplus$ abf71588 = 23a33939
			\item $w_7 = w_6 \oplus w_3$ = 23a33939 $\oplus$ 09cf4f3c  = 2a6c7605
		\end{itemize}
		\item Runde 0: $m\oplus k_0$
		\begin{center}
			\begin{tabular}{c|c|c|c}
				32 & 08 & f1 & e0 \\
				43 & 5a & 31 & 37  \\
				f6 & 30 & 58 & 07 \\
				68 & 8d & a2 & 34
			\end{tabular} $\oplus$
			\begin{tabular}{c|c|c|c}
				2b & 28 & ab & 09 \\
				7e & ae & f7 & cf  \\
				15 & d2 & 15 & 4f \\
				16 & a6 & 88 & 3c
			\end{tabular} = 
			\begin{tabular}{c|c|c|c}
				19 &  &  &  \\
				 & f4 &  &   \\
				 &  & 4d &  \\
				 &  &  & 08
			\end{tabular}
		\end{center}
		Runde 1: Ergebnis SubBytes, Ergebnis ShiftRow, Ergebnis Mixcolumn, $\oplus k_1$
		\begin{center}
			\begin{tabular}{c|c|c|c}
				d4 & & & \\
				& bf & & \\
				& & e3 & \\
				& & & 30 \\
			\end{tabular},
			\quad
			\begin{tabular}{c|c|c|c}
				d4 & & & \\
				bf & & & \\
				e3 & & & \\
				30 & & & \\
			\end{tabular},
			\quad
			\begin{tabular}{c|c|c|c}
				ba & & & \\
				& & & \\
				& & & \\
				& & & \\
			\end{tabular}
			$\oplus$
			\begin{tabular}{c|c|c|c}
				a0 & 8b & 23 & 2a \\
				fa & 54 & a3 & 6c \\
				fe & 2c & 39 & 76 \\
				17 & b1 & 39 & 05 \\
			\end{tabular} = 
			\begin{tabular}{c|c|c|c}
				1a & & & \\
				& & & \\
				& & & \\
				& & & \\
			\end{tabular}
		\end{center}
		Ergebnis von MixColumn: $d_i = a(x) \otimes c_i \mod x^4+1$, mit $a(x) = 03x^3 + 01x^2 + 01x + 02$ das heißt
		\begin{align}
			\begin{pmatrix}
				d_{0i} \\ d_{1i} \\ d_{2i} \\ d_{3i}
			\end{pmatrix} &= \begin{pmatrix}
				02 & 03 & 01 & 01 \\
				01 & 02 & 03 & 01 \\
				01 & 01 & 02 & 03 \\
				03 & 01 & 01 & 02
			\end{pmatrix}
			\begin{pmatrix}
				c_{0i} \\ c_{1i} \\ c_{2i} \\ c_{3i}
			\end{pmatrix} \notag \\
			d_{0,0} &= 02 \cdot d4 \oplus 03 \cdot bf \oplus e3 \oplus 30 \mod x^8+x^4+x^3+x+1 \notag \\
			&= 00000010_2 \cdot 11010100 \dots \mod x^8+x^4+x^3+x+1 \notag \\
			&= x \cdot (x^7+x^6+x^4+x^2) \dots \mod x^8+x^4+x^3+x+1 \notag \\
			&= x^8 + x^7 + x^5 + x^3 \dots \mod x^8+x^4+x^3+x+1 \notag \\
			&= x^7 + x^5 + x^4 + x + 1 \dots \mod x^8+x^4+x^3+x+1 \notag \\
			&= 10110011_2 \dots \mod x^8+x^4+x^3+x+1 \notag \\
			&= b3 \dots \mod x^8+x^4+x^3+x+1 \notag \\
			&= ba \notag
		\end{align}
		\item letzte Runde: Shift$^{-1}$, Sub$^{-1}$ vertauschen \\
		vorletzte Runden: $\oplus k_{r-1}$, MC$^{-1}$ vertauschen ($k_{r-1} \to k'_{r-1}$) und Shift$^{-1}$, Sub$^{-1}$ vertauschen, ... \\
		$\Rightarrow$ neue Reihenfolge: Sub$^{-1}$, Shift$^{-1}$, MC$^{-1}$, $\oplus k'_{r-1}$, Sub$^{-1}$, Shift$^{-1}$, ... $\Rightarrow$ selbe Reihenfolge wie bei der Verschlüsselung
	\end{enumerate}

\end{document}