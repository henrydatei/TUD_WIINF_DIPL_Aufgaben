\documentclass{article}

\usepackage{amsmath,amssymb}
\usepackage{tikz}
\usepackage{pgfplots}
\usepackage{xcolor}
\usepackage[left=2.1cm,right=3.1cm,bottom=3cm,footskip=0.75cm,headsep=0.5cm]{geometry}
\usepackage{enumerate}
\usepackage{enumitem}
\usepackage{marvosym}
\usepackage{tabularx}
\usepackage{parskip}

\usepackage{listings}
\definecolor{lightlightgray}{rgb}{0.95,0.95,0.95}
\definecolor{lila}{rgb}{0.8,0,0.8}
\definecolor{mygray}{rgb}{0.5,0.5,0.5}
\definecolor{mygreen}{rgb}{0,0.8,0.26}
%\lstdefinestyle{java} {language=java}
\lstset{language=R,
	basicstyle=\ttfamily,
	keywordstyle=\color{lila},
	commentstyle=\color{lightgray},
	stringstyle=\color{mygreen}\ttfamily,
	backgroundcolor=\color{white},
	showstringspaces=false,
	numbers=left,
	numbersep=10pt,
	numberstyle=\color{mygray}\ttfamily,
	identifierstyle=\color{blue},
	xleftmargin=.1\textwidth, 
	%xrightmargin=.1\textwidth,
	escapechar=§,
	%literate={\t}{{\ }}1
	breaklines=true,
	postbreak=\mbox{\space}
}

\usepackage[colorlinks = true, linkcolor = blue, urlcolor  = blue, citecolor = blue, anchorcolor = blue]{hyperref}
\usepackage[utf8]{inputenc}

\renewcommand*{\arraystretch}{1.4}

\newcolumntype{L}[1]{>{\raggedright\arraybackslash}p{#1}}
\newcolumntype{R}[1]{>{\raggedleft\arraybackslash}p{#1}}
\newcolumntype{C}[1]{>{\centering\let\newline\\\arraybackslash\hspace{0pt}}m{#1}}

\newcommand{\E}{\mathbb{E}}
\DeclareMathOperator{\rk}{rk}
\DeclareMathOperator{\Var}{Var}
\DeclareMathOperator{\Cov}{Cov}

\title{\textbf{Kryptografie und -analyse, Übung 4}}
\author{\textsc{Henry Haustein}}
\date{}

\begin{document}
	\maketitle
	
	\section*{Aufgabe 1: DES}
	\begin{enumerate}[label=(\alph*)]
		\item Ergebnis der Expansionsabbildung: 110110$\vert$100110 \\
		$\oplus$ Schlüssel: 000010$\vert$111100 \\
		1. Block in $S1_0$: 0100 \\
		2. Block in $S2_3$: 0010
		\item Wenn die Register C und D immer gleich sind, so ist der Rundenschlüssel immer gleich. Damit sind Ver- und Entschlüsselung identisch.
		\item Es gilt
		\begin{center}
			\begin{tabular}{l|cccccccc}
				$m$ & $m_1$ & $m_2$ & $m_3$ & $m_4$ & $m_5$ & $m_6$ & $m_7$ & $m_8$ \\
				$P$ & & & & & $m_6$ & $m_7$ & $m_4$ & $m_5$ \\
				$\oplus k$ & & & & & $k_0$ & $k_1$ & $k_2$ & $k_3$ \\
				$\oplus L_0$ & & & & & $m_1$ & $m_2$ & $m_3$ & $m_4$ \\
				\hline
				$c$ & $c_1$ & $c_2$ & $c_3$ & $c_4$ & $c_5$ & $c_6$ & $c_7$ & $c_8$
			\end{tabular}
		\end{center}
		$\Rightarrow k_0 = m_0 \oplus m_6 \oplus c_4 = 1$ \\
		$\Rightarrow k_1 = m_1 \oplus m_7 \oplus c_5 = 0$ \\
		$\Rightarrow k_2 = m_2 \oplus m_4 \oplus c_6 = 0$ \\
		$\Rightarrow k_3 = m_3 \oplus m_5 \oplus c_7 = 1$
		\item Meet-in-the-middle-Angriff, zweifache Verschlüsselung (Aufwand $2^{56}\cdot 2^{56} = 2^{112}$) und einfache Entschlüsselung (Aufwand $2^{56}$)
	\end{enumerate}

	\section*{Differentielle Kryptoanalyse}
	\begin{enumerate}[label=(\alph*)]
		\item $S1_I^\ast = S1_I \oplus S1_I' = 110110 \oplus 011011 = 101101$ \\
		$S1_O = S1_2(110110) = 0111$, $S1_O^\ast = S1_3(101101) = 0001$ \\
		$S1_O' = 0111 \oplus 0001 = 0110$
		\item $S1_E' = 010001 \oplus 010010 = 000011$. Von den 64 möglichen Inputpaaren brauchen wir diejenigen, die Inputdifferenz von $3_{16}$ und Outputdifferenz $9_{16}$ haben. Dazu schauen wir in der Verteilungstabelle in der Spalte 9 nach Einsen. Es gibt 4 Inputpaare: (4,7), (7,4), (31,32), (32,31). \\
		$S1_K = S1_I \oplus S1_E$
		\begin{center}
			\begin{tabular}{c|c}
				$S1_I$, $S1_I^\ast$ & Schlüsselkandidaten \\
				\hline
				4, 7 & 15, 16 \\
				31, 32 & 20, 23
			\end{tabular}
		\end{center}
		$\Rightarrow$ gesuchter Schlüssel ist 23 \\
		$\Rightarrow$ Differenz zwischen den Schlüsselkandidaten ist die Inputdifferenz der Eingaben. Mit immer derselben Differenz ist es nicht möglich einen eindeutigen Schlüssel zu erhalten.
	\end{enumerate}

\end{document}