\documentclass{article}

\usepackage{amsmath,amssymb}
\usepackage{tikz}
\usepackage{pgfplots}
\usepackage{xcolor}
\usepackage[left=2.1cm,right=3.1cm,bottom=3cm,footskip=0.75cm,headsep=0.5cm]{geometry}
\usepackage{enumerate}
\usepackage{enumitem}
\usepackage{marvosym}
\usepackage{tabularx}
\usepackage{parskip}

\usepackage{listings}
\definecolor{lightlightgray}{rgb}{0.95,0.95,0.95}
\definecolor{lila}{rgb}{0.8,0,0.8}
\definecolor{mygray}{rgb}{0.5,0.5,0.5}
\definecolor{mygreen}{rgb}{0,0.8,0.26}
%\lstdefinestyle{java} {language=java}
\lstset{language=R,
	basicstyle=\ttfamily,
	keywordstyle=\color{lila},
	commentstyle=\color{lightgray},
	stringstyle=\color{mygreen}\ttfamily,
	backgroundcolor=\color{white},
	showstringspaces=false,
	numbers=left,
	numbersep=10pt,
	numberstyle=\color{mygray}\ttfamily,
	identifierstyle=\color{blue},
	xleftmargin=.1\textwidth, 
	%xrightmargin=.1\textwidth,
	escapechar=§,
	%literate={\t}{{\ }}1
	breaklines=true,
	postbreak=\mbox{\space}
}

\usepackage[colorlinks = true, linkcolor = blue, urlcolor  = blue, citecolor = blue, anchorcolor = blue]{hyperref}
\usepackage[utf8]{inputenc}

\renewcommand*{\arraystretch}{1.4}

\newcolumntype{L}[1]{>{\raggedright\arraybackslash}p{#1}}
\newcolumntype{R}[1]{>{\raggedleft\arraybackslash}p{#1}}
\newcolumntype{C}[1]{>{\centering\let\newline\\\arraybackslash\hspace{0pt}}m{#1}}

\newcommand{\E}{\mathbb{E}}
\DeclareMathOperator{\rk}{rk}
\DeclareMathOperator{\Var}{Var}
\DeclareMathOperator{\Cov}{Cov}

\title{\textbf{Kryptografie und -analyse, Zusammenfassung Vorlesung 9}}
\author{\textsc{Henry Haustein}}
\date{}

\begin{document}
	\maketitle
	
	\section*{Wie kann ein Generator einer Gruppe $G$ gefunden werden?}
	Primfaktorzerlegung $\vert G\vert = p_1\cdot p_2\cdot ...$, Wahl eines zufälligen Elementes $a\in G$, für jedes $p_i$:
	\begin{align}
		a^{\frac{\vert G\Vert}{p_i}} \overset{?}{\equiv} 1 \mod \vert G\vert \notag
	\end{align}
	Wenn $\equiv 1$, dann $a$ kein Generator.
	
	\section*{Was versteht man unter dem Problem des Diskreten Logarithmus?}
	Bestimme $x$:
	\begin{align}
		x = \log_g(y) \mod p \notag
	\end{align}
	
	\section*{Wie funktioniert der BSGS-Algorithmus?}
	$m = \lceil \sqrt{\vert G\vert}\rceil$, Ansatz $x=qm + r$
	\begin{itemize}
		\item Babystep-Liste: $B=\{(i,y(g^i)^{-1})\mod p\} \to r$
		\item Giantstep-Liste: $G=\{(j,(g^m)^j)\mod p\} \to q$
	\end{itemize}
	Berechnung der Elemente von $G$, bis $(g^m)^j$ als zweite Komponente eines Elements in $B$ gefunden wurde.
	
	\section*{Wie funktioniert der DH-Schlüsselaustausch?}
	Öffentlich: $p,g$, A wählt $x_A$, B wählt $x_B$, berechnet $y_A = g^{x_A}\mod p$, Austausch $y_A$, $y_B$, $\Rightarrow k = y_B^{x_A}\mod p$
	
	\section*{Worauf beruht die Sicherheit (DH-Problem)?}
	Bestimme $x_A$ ($x_B$ analog):
	\begin{align}
		x_A = \log_g(y_A) \mod p \notag
	\end{align}
	
	\section*{Ist der DH-Schlüsselaustausch sicher gegen passive bzw. aktive Angriffe?}
	sicher gegen passive Angriffe, nicht sicher gegen aktive Angriffe (Man-in-the-middle)
	
	\section*{Wie werden bei ElGamal die Schlüssel bestimmt?}
	Jeder Teilnehmer:
	\begin{itemize}
		\item wählt Primzahl $p$ und Generator $p\in\mathbb{Z}_p^\ast$
		\item wählt zufällige Zahl $k_d$ mit $0\le k_d\le p-2$
		\item berechnet $k_e = g^{k_d}\mod p$
		\item[$\Rightarrow$] öffentlicher Schlüssel: $(p,g,k_e)$
		\item[$\Rightarrow$] privater Schlüssel: $k_d$
	\end{itemize}
	
	\section*{Wie funktioniert ElGamal als Konzelationssystem?}
	Verschlüsselung (B $\to$ A):
	\begin{itemize}
		\item B benötigt öffentlichen Schlüssel von A: $(p,g,k_e)$
		\item wählt Zufallszahl $r$ mit $0\le r\le p-2$
		\item berechnet: $c = (c_1,c_2)$ mit
		\begin{align}
			c_1 &= g^r\mod p \notag \\
			c_2 &= mk_e^r\mod p \notag
		\end{align}
	\end{itemize}
	Entschlüsselung:
	\begin{align}
		m = (c_1^{k_d})^{-1}c_2\mod p \notag
	\end{align}
	
	\section*{Worauf beruht die Sicherheit des ElGamal-Kryptosystems (DH-Problem, wie lautet es hier konkret)?}
	Bestimme $k_d$:
	\begin{align}
		k_d = \log_g(k_e)\mod p \notag
	\end{align}
	
	\section*{Was ist bei der sicheren Verwendung von ElGamal als Konzelationssystem zu beachten? (Warum darf die Zufallszahl nur einmal vewendet werden? Welcher aktive Angriff ist möglich? Wie ist er zu verhindern?)}
	Wenn die Zufallszahl mehrfach verwendet wird, so kann, falls eine Nachricht $m_1$ bekannt ist, eine andere Nachricht $m_2$ (bei selben $r$) berechnet werden.
	
	Gewählter Klartext-Schlüsseltext-Angriff ist auch möglich $\Rightarrow$ Einfügen von Redundanz: $m = m, h(m)$
	
\end{document}