\documentclass{article}

\usepackage{amsmath,amssymb}
\usepackage{tikz}
\usepackage{pgfplots}
\usepackage{xcolor}
\usepackage[left=2.1cm,right=3.1cm,bottom=3cm,footskip=0.75cm,headsep=0.5cm]{geometry}
\usepackage{enumerate}
\usepackage{enumitem}
\usepackage{marvosym}
\usepackage{tabularx}
\usepackage{parskip}

\usepackage{listings}
\definecolor{lightlightgray}{rgb}{0.95,0.95,0.95}
\definecolor{lila}{rgb}{0.8,0,0.8}
\definecolor{mygray}{rgb}{0.5,0.5,0.5}
\definecolor{mygreen}{rgb}{0,0.8,0.26}
%\lstdefinestyle{java} {language=java}
\lstset{language=R,
	basicstyle=\ttfamily,
	keywordstyle=\color{lila},
	commentstyle=\color{lightgray},
	stringstyle=\color{mygreen}\ttfamily,
	backgroundcolor=\color{white},
	showstringspaces=false,
	numbers=left,
	numbersep=10pt,
	numberstyle=\color{mygray}\ttfamily,
	identifierstyle=\color{blue},
	xleftmargin=.1\textwidth, 
	%xrightmargin=.1\textwidth,
	escapechar=§,
	%literate={\t}{{\ }}1
	breaklines=true,
	postbreak=\mbox{\space}
}

\usepackage[colorlinks = true, linkcolor = blue, urlcolor  = blue, citecolor = blue, anchorcolor = blue]{hyperref}
\usepackage[utf8]{inputenc}

\renewcommand*{\arraystretch}{1.4}

\newcolumntype{L}[1]{>{\raggedright\arraybackslash}p{#1}}
\newcolumntype{R}[1]{>{\raggedleft\arraybackslash}p{#1}}
\newcolumntype{C}[1]{>{\centering\let\newline\\\arraybackslash\hspace{0pt}}m{#1}}

\newcommand{\E}{\mathbb{E}}
\DeclareMathOperator{\rk}{rk}
\DeclareMathOperator{\Var}{Var}
\DeclareMathOperator{\Cov}{Cov}

\title{\textbf{Kryptografie und -analyse, Zusammenfassung Vorlesung 3}}
\author{\textsc{Henry Haustein}}
\date{}

\begin{document}
	\maketitle
	
	\section*{Wie funktionieren Transpositionen, MM-Substitutionen und PM-Substitutionen?}
	Transposition = Vertauschen der Zeichen des Klartextes \\
	MM-Substitutionen (monoalphabetisch, monographisch): ein Buchstabe des Klartextes wird mit einem Buchstaben ersetzt. Die Buchstaben zu denen ersetzt wird kommen aus einem Alphabet $\Rightarrow$ eineindeutige Zuordnung
	PM-Substitutionen (polyalphabetisch, monographisch): wie MM-Substitutionen, nur dass die Buchstaben zu denen ersetzt wird, aus mehreren Alphabeten kommen $\Rightarrow$ eindeutige Zuordnung

	\section*{Was sind Ansätze zur Analyse dieser Verfahren?}
	Da die statistischen Eigenschaften der Klartexte erhalten bleiben (zumindest bei MM-Substitutionen und Transpositionen), versucht man über diese, wieder an die Klartexte heranzukommen.
	
	\section*{Wie wird bei der Analyse von PM-Substitutionen, in denen der Schlüssel periodisch wiederholt wird, vorgegangen?}
	Zuerst muss die Schlüssellänge bestimmt werden, z.B. mit dem Kasiski-Test oder Friedman-Test. Danach kann man den Schlüsseltext in Blöcke unterteilen, die mit dem selben Schlüssel verschlüsselt worden sind. Innerhalb dieser Blöcke findet nur eine MM-Substitution statt, diese kann man knacken.
	
	\section*{Wie werden statistische Charakteristika von Klartexten in natürlichen Sprachen durch die Verschlüsselung mit klassischen Verfahren beeinflusst?}
	Die Häufigkeiten der Buchstaben, Digramme und Trigramme bleiben erhalten bei MM-Substitutionen.
\end{document}