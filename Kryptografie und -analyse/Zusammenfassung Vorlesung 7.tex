\documentclass{article}

\usepackage{amsmath,amssymb}
\usepackage{tikz}
\usepackage{pgfplots}
\usepackage{xcolor}
\usepackage[left=2.1cm,right=3.1cm,bottom=3cm,footskip=0.75cm,headsep=0.5cm]{geometry}
\usepackage{enumerate}
\usepackage{enumitem}
\usepackage{marvosym}
\usepackage{tabularx}
\usepackage{parskip}

\usepackage{listings}
\definecolor{lightlightgray}{rgb}{0.95,0.95,0.95}
\definecolor{lila}{rgb}{0.8,0,0.8}
\definecolor{mygray}{rgb}{0.5,0.5,0.5}
\definecolor{mygreen}{rgb}{0,0.8,0.26}
%\lstdefinestyle{java} {language=java}
\lstset{language=R,
	basicstyle=\ttfamily,
	keywordstyle=\color{lila},
	commentstyle=\color{lightgray},
	stringstyle=\color{mygreen}\ttfamily,
	backgroundcolor=\color{white},
	showstringspaces=false,
	numbers=left,
	numbersep=10pt,
	numberstyle=\color{mygray}\ttfamily,
	identifierstyle=\color{blue},
	xleftmargin=.1\textwidth, 
	%xrightmargin=.1\textwidth,
	escapechar=§,
	%literate={\t}{{\ }}1
	breaklines=true,
	postbreak=\mbox{\space}
}

\usepackage[colorlinks = true, linkcolor = blue, urlcolor  = blue, citecolor = blue, anchorcolor = blue]{hyperref}
\usepackage[utf8]{inputenc}

\renewcommand*{\arraystretch}{1.4}

\newcolumntype{L}[1]{>{\raggedright\arraybackslash}p{#1}}
\newcolumntype{R}[1]{>{\raggedleft\arraybackslash}p{#1}}
\newcolumntype{C}[1]{>{\centering\let\newline\\\arraybackslash\hspace{0pt}}m{#1}}

\newcommand{\E}{\mathbb{E}}
\DeclareMathOperator{\rk}{rk}
\DeclareMathOperator{\Var}{Var}
\DeclareMathOperator{\Cov}{Cov}

\title{\textbf{Kryptografie und -analyse, Zusammenfassung Vorlesung 7}}
\author{\textsc{Henry Haustein}}
\date{}

\begin{document}
	\maketitle
	
	\section*{Auf welchem Prinzip beruht der Algorithmus AES?}
	Substitutions-Permutations-Netzwerk mit wahlweise 10, 12 oder 14 Runden

	\section*{Wie funktioniert AES prinzipiell (Struktur, Iterationsrunden, Anzahl der Runden …)?}
	Verschlüsselung von Klartextblöcken der Länge 128 Bit (vorgeschlagene Längen von 192 und 256 Bits nicht standardisiert), Schlüssellänge wahlweise 128, 192 oder 256 Bits, Rundenanzahl $r$ hängt von Schlüssel- und Klartextlänge ab:
	\begin{itemize}
		\item Schlüssellänge 128 Bit: 10 Runden
		\item Schlüssellänge 192 Bit: 12 Runden
		\item Schlüssellänge 256 Bit: 14 Runden
	\end{itemize}

	Runde 0: $\oplus k_0$ \\
	Struktur der ersten $r-1$ Runden: SubBytes, ShiftRow, MixColumn, $\oplus k_i$ \\
	Struktur der $r$-ten Runde: SubBytes, ShiftRow, $\oplus k_r$
	
	\section*{Wie funktioniert die Entschlüsselung beim AES?}
	Runde $r$: $\oplus k_r$, ShiftRow$^{-1}$, SubBytes$^{-1}$ \\
	Runde $1,...,r-1$: $\oplus k_i$, MixColumn$^{-1}$, ShiftRow$^{-1}$, SubBytes$^{-1}$ \\
	Runde 0: $\oplus k_0$
	
	\section*{Wie kann die Entschlüsselung in äquivalenter Reihenfolge wie die Verschlüsselung durchgeführt werden?}
	Es gilt:
	\begin{itemize}
		\item SubBytes(ShiftRow($s_i$))=ShiftRow(SubBytes($s_i$))
		\item SubBytes$^{-1}$(ShiftRow$^{-1}$($s_i$)) = ShiftRow$^{-1}$(SubBytes$^{-1}$($s_i$))
		\item MixColumn($s_i \oplus k_i$) = MixColumn($s_i$) $\oplus$ MixColumn($k_i$)
		\item MixColumn$^{-1}$($s_i \oplus k_i$) = MixColumn$^{-1}$($s_i$) $\oplus$ MixColumn$^{-1}$($k_i$)
	\end{itemize}
	Reihenfolge der Abarbeitung wie bei Verschlüsselung, $k_i'$ = MixColumn$^{-1}$($k_i$)
	
	Runde 0: $\oplus k_r$ \\
	Runde $1,...,r-1$: SubBytes$^{-1}$, ShiftRow$^{-1}$, MixColumn$^{-1}$, $\oplus k_i'$ \\
	Runde $r$: SubBytes$^{-1}$, ShiftRow$^{-1}$, $\oplus k_0$
	
	\section*{Was versteht man unter synchroner bzw. selbstsynchronisierender Chiffre?}
	Synchrone Stromchiffre: Verschlüsselung eines Zeichens ist abhängig von der Position bzw. von vorhergehenden Klartext- oder Schlüsselzeichen
	
	Selbstsynchronisierende Stromchiffre: Verschlüsselung ist nur von begrenzter Anzahl vorhergehender Zeichen abhängig
	
	\section*{Wie funktionieren die Betriebsarten ECB und CBC, welche Eigenschaften haben sie?}
	ECB (Electronic Code Book)
	\begin{itemize}
		\item Selbstsynchronisierend (Abhängigkeit von 0 Blöcken)
		\item Länge der verarbeiteten Einheiten: entsprechend Blockgröße der Blockchiffre (AES: $l = 128$ Bit)
		\item Keine Abhängigkeiten zwischen den Blöcken
	\end{itemize}

	CBC (Cipher Block Chaining)
	\begin{itemize}
		\item Selbstsynchronisierend (Abhängigkeit von 1 Block)
		\item Länge der verarbeiteten Einheiten: entsprechend Blockgröße der Blockchiffre (AES: $l = 128$ Bit)
		\item Abhängigkeiten zwischen den Blöcken: gleiche Klartextblöcke liefern unterschiedliche Schlüsseltextblöcke
		\item Initialisierungsvektor IV muss nicht geheim sein, darf aber nicht vorhersagbar sein
	\end{itemize}
	
	\section*{Welchen Nachteil hat ECB bzgl. Sicherheit?}
	gleiche Klartextblöcke liefern gleiche Schlüsseltextblöcke $\Rightarrow$ ggf. Kodebuchanalysen möglich
	
	\section*{Wie wirken sich Fehler bzw. Manipulationen während der Übertragung bei ECB und CBC aus?}
	EBC: Synchronisationsfehler: keine Fehlerfortpflanzung $\Rightarrow$ gezieltes Einfügen und Entfernen von Blöcken möglich
	
	CBC: Fehler während der Übertragung
	\begin{itemize}
		\item additive Fehler: Fehlerfortpflanzung in den Folgeblock
		\item Synchronisationsfehler: 2 Blöcke betroffen
	\end{itemize}
	$\Rightarrow$ Verfahren eignet sich zur Authentikation: Manipulationen, Einfügen und Entfernen von Blöcken erkennbar
	
\end{document}