\documentclass{article}

\usepackage{amsmath,amssymb}
\usepackage{tikz}
\usepackage{pgfplots}
\usepackage{xcolor}
\usepackage[left=2.1cm,right=3.1cm,bottom=3cm,footskip=0.75cm,headsep=0.5cm]{geometry}
\usepackage{enumerate}
\usepackage{enumitem}
\usepackage{marvosym}
\usepackage{tabularx}
\usepackage{parskip}

\usepackage{listings}
\definecolor{lightlightgray}{rgb}{0.95,0.95,0.95}
\definecolor{lila}{rgb}{0.8,0,0.8}
\definecolor{mygray}{rgb}{0.5,0.5,0.5}
\definecolor{mygreen}{rgb}{0,0.8,0.26}
%\lstdefinestyle{java} {language=java}
\lstset{language=R,
	basicstyle=\ttfamily,
	keywordstyle=\color{lila},
	commentstyle=\color{lightgray},
	stringstyle=\color{mygreen}\ttfamily,
	backgroundcolor=\color{white},
	showstringspaces=false,
	numbers=left,
	numbersep=10pt,
	numberstyle=\color{mygray}\ttfamily,
	identifierstyle=\color{blue},
	xleftmargin=.1\textwidth, 
	%xrightmargin=.1\textwidth,
	escapechar=§,
	%literate={\t}{{\ }}1
	breaklines=true,
	postbreak=\mbox{\space}
}

\usepackage[colorlinks = true, linkcolor = blue, urlcolor  = blue, citecolor = blue, anchorcolor = blue]{hyperref}
\usepackage[utf8]{inputenc}

\renewcommand*{\arraystretch}{1.4}

\newcolumntype{L}[1]{>{\raggedright\arraybackslash}p{#1}}
\newcolumntype{R}[1]{>{\raggedleft\arraybackslash}p{#1}}
\newcolumntype{C}[1]{>{\centering\let\newline\\\arraybackslash\hspace{0pt}}m{#1}}

\newcommand{\E}{\mathbb{E}}
\DeclareMathOperator{\rk}{rk}
\DeclareMathOperator{\Var}{Var}
\DeclareMathOperator{\Cov}{Cov}

\title{\textbf{Kryptografie und -analyse, Zusammenfassung Vorlesung 10}}
\author{\textsc{Henry Haustein}}
\date{}

\begin{document}
	\maketitle
	
	\section*{Wie funktioniert ElGamal als Signatursystem?}
	Schlüsselgenerierung identisch
	
	A führt folgende Schritte aus:
	\begin{itemize}
		\item wählt Zufallszahl $r\in\mathbb{Z}_p^\ast$
		\item berechnet $r^{-1}$ mit $rr^{-1} \equiv 1 \mod (p-1)$
		\item berechnet $s=(s_1,s_2)$ mit
		\begin{align}
			s_1 &= g^r\mod p \notag \\
			s_2 &= r^{-1}(h(m) - k_ss_1) \mod (p-1) \notag
		\end{align}
	\end{itemize}

	Test der Signatur:
	\begin{itemize}
		\item testet ob $1\le s_1\le p-1$
		\item berechnet $v_1 = k_t^{s_1}s_1^{s_2} \mod p$
		\item berechnet $h(m)$ und $v_2 = g^{h(m)}\mod p$
		\item akzeptiert Signatur, wenn $v_1 \equiv v_2$
	\end{itemize}
	
	\section*{Was ist bei der sicheren Verwendung von ElGamal als Signatursystem zu beachten? Warum darf die Zufallszahl nur einmal verwendet werden? Warum wird eine Hashfunktion auf die Nachricht angewendet?}
	selbiges wie bei ElGamal als Konzelationssystem
	
	Verwendung einer Hash-Funktion
	
	\section*{Wie werden die Parameter bei RSA gewählt?}
	Wahl von $p$ und $q$ als große Primzahlen, die nicht dicht beieinander liegen, aber auch nicht zu weit auseinander
	
	\section*{Wie wird ver- und entschlüsselt bzw. signiert und getestet?}
	Verschlüsselung: $c = m^{k_e} \mod n$ \\
	Entschlüsselung: $m = c^{k_d} \mod n$
	
	Signieren: $s = m^{k_s}\mod n$ \\
	Testen: $m = s^{k_t} \mod n$?
	
	\section*{Nachweis der Entschlüsselung?}
	Zu zeigen, dass
	\begin{align}
		m &= \left(m^{k_e}\right)^{k_d} \mod n \notag \\
		m &= m^{k_e\cdot k_d} \mod n \notag
	\end{align}
	Wir wissen $k_e\cdot k_d \equiv 1 \mod \Phi(n)$, also $k_e\cdot k_d = l(p-1)(q-1) + 1$. Damit
	\begin{align}
		m^{k_e\cdot k_d} &= m^{l(p-1)(q-1) + 1} \mod n \notag \\
		&= \left(m^{p-1}\right)^{l(q-1)} \cdot m \mod n \notag
	\end{align}
	Da $p$ eine Primzahl ist, gilt nach dem kleinen Satz von Fermat ($a^{p-1}\equiv 1 \mod p$) und weil $p\mid n$:
	\begin{align}
		1^{l(q-1)}\cdot m = m \mod p \notag
	\end{align}
	
	\section*{Was ist bei der Parameterwahl von RSA zu beachten?}
	Wahl von $p$ und $q$ als große Primzahlen, die nicht dicht beieinander liegen, aber auch nicht zu weit auseinander
	
\end{document}