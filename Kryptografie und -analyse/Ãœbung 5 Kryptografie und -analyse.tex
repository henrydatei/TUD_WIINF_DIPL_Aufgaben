\documentclass{article}

\usepackage{amsmath,amssymb}
\usepackage{tikz}
\usepackage{pgfplots}
\usepackage{xcolor}
\usepackage[left=2.1cm,right=3.1cm,bottom=3cm,footskip=0.75cm,headsep=0.5cm]{geometry}
\usepackage{enumerate}
\usepackage{enumitem}
\usepackage{marvosym}
\usepackage{tabularx}
\usepackage{parskip}

\usepackage{listings}
\definecolor{lightlightgray}{rgb}{0.95,0.95,0.95}
\definecolor{lila}{rgb}{0.8,0,0.8}
\definecolor{mygray}{rgb}{0.5,0.5,0.5}
\definecolor{mygreen}{rgb}{0,0.8,0.26}
%\lstdefinestyle{java} {language=java}
\lstset{language=R,
	basicstyle=\ttfamily,
	keywordstyle=\color{lila},
	commentstyle=\color{lightgray},
	stringstyle=\color{mygreen}\ttfamily,
	backgroundcolor=\color{white},
	showstringspaces=false,
	numbers=left,
	numbersep=10pt,
	numberstyle=\color{mygray}\ttfamily,
	identifierstyle=\color{blue},
	xleftmargin=.1\textwidth, 
	%xrightmargin=.1\textwidth,
	escapechar=§,
	%literate={\t}{{\ }}1
	breaklines=true,
	postbreak=\mbox{\space}
}

\usepackage[colorlinks = true, linkcolor = blue, urlcolor  = blue, citecolor = blue, anchorcolor = blue]{hyperref}
\usepackage[utf8]{inputenc}

\renewcommand*{\arraystretch}{1.4}

\newcolumntype{L}[1]{>{\raggedright\arraybackslash}p{#1}}
\newcolumntype{R}[1]{>{\raggedleft\arraybackslash}p{#1}}
\newcolumntype{C}[1]{>{\centering\let\newline\\\arraybackslash\hspace{0pt}}m{#1}}

\newcommand{\E}{\mathbb{E}}
\DeclareMathOperator{\rk}{rk}
\DeclareMathOperator{\Var}{Var}
\DeclareMathOperator{\Cov}{Cov}

\title{\textbf{Kryptografie und -analyse, Übung 5}}
\author{\textsc{Henry Haustein}}
\date{}

\begin{document}
	\maketitle

	\section*{Differentielle Kryptoanalyse}
	\begin{enumerate}[label=(\alph*)]
		\item $S1_I^\ast = S1_I \oplus S1_I' = 110110 \oplus 011011 = 101101$ \\
		$S1_O = S1_2(110110) = 0111$, $S1_O^\ast = S1_3(101101) = 0001$ \\
		$S1_O' = 0111 \oplus 0001 = 0110$
		\item $S1_E' = 010001 \oplus 010010 = 000011$. Von den 64 möglichen Inputpaaren brauchen wir diejenigen, die Inputdifferenz von $3_{16}$ und Outputdifferenz $9_{16}$ haben. Dazu schauen wir in der Verteilungstabelle in der Spalte 9 nach Einsen. Es gibt 4 Inputpaare: (4,7), (7,4), (31,32), (32,31). \\
		$S1_K = S1_I \oplus S1_E$
		\begin{center}
			\begin{tabular}{c|c}
				$S1_I$, $S1_I^\ast$ & Schlüsselkandidaten \\
				\hline
				4, 7 & 15, 16 \\
				31, 32 & 20, 23
			\end{tabular}
		\end{center}
		$\Rightarrow$ gesuchter Schlüssel ist 23 \\
		$\Rightarrow$ Differenz zwischen den Schlüsselkandidaten ist die Inputdifferenz der Eingaben. Mit immer derselben Differenz ist es nicht möglich einen eindeutigen Schlüssel zu erhalten.
		\item Da die Wahrscheinlichkeit der Charakteristik auch von den anderen S-Boxen abhängt, müssen die Wahrscheinlichkeiten der anderen S-Boxen möglichst groß sein (exakt 1). Da die S-Boxen deterministisch arbeiten, müssen die anderen S-Boxen eine Inputdifferenz von 0 verarbeiten. Die äußeren Bits sind für die Wahl der S-Box zuständig, deswegen dürfen nur die mittleren Bits von $S2_I' \neq 0$ ($\Rightarrow$ nur S2 aktiv). Damit ergeben sich folgende möglichen Inputdifferenzen:
		\begin{itemize}
			\item 000100 = 04 $\Rightarrow$ Outputdifferenz 7 $\left(\frac{14}{64}\right)$
			\item 001000 = 08 $\Rightarrow$ Outputdifferenz A $\left(\frac{16}{64}\right)$
			\item 001100 = 0C $\Rightarrow$ Outputdifferenz 5 $\left(\frac{14}{64}\right)$
		\end{itemize}
		größte Wahrscheinlichkeit für 08 $\to$ A mit $p=\frac{16}{64} = \frac{1}{4}$. 001000 sah vor der Expansion so aus: 0100. Alle anderen 4er-Blöcke sind 0. Damit $x' = 04000000_{16}$. Die Wahrscheinlichkeit für diese 1-Runden-Charakteristik ist damit:
		\begin{align}
			\underbrace{1}_{S1}\cdot \underbrace{\frac{1}{4}}_{S2} \cdot \underbrace{1 \cdots 1}_{S3-S8} = \frac{1}{4} \notag
		\end{align}
		$S2'_O = A$ $\Rightarrow S1_O' \, S2_O' \, S3_O' \, ... = 0000 \, 1010 \, 0000 ...$ nach der Permutation werden die Einsen auf Position 5 und 7 auf die Positionen 13 und 2 permutiert. Alle anderen Bits sind 0. Damit ist $y'=40080000$.
		\item Die Wahrscheinlichkeit für diese Charakteristik ist $p_1^\Omega \cdot p_2^\Omega$. Die Charakteristik ist so konstruiert, dass $p_1^\Omega = 1$ ist. Mit $L_{\Omega m} = 19600000$ sind die Inputdifferenzen für die S-Boxen (die Charakteristik ist so definiert, dass $S_O' = 0$ ist):
		\begin{itemize}
			\item Umwandlung in Binärdarstellung: 0001 1001 0110 0000 0000 0000 0000 0000
			\item Expansion: 000011 110010 101100 000000 000000 000000 000000 000000
		\end{itemize}
		Inputdifferenzen mit Wahrscheinlichkeiten
		\begin{itemize}
			\item $S1_I' = 3 \Rightarrow S1_O' = 0 \left(p = \frac{14}{64}\right)$
			\item $S2_I' = 32 \Rightarrow S2_O' = 0 \left(p = \frac{8}{64}\right)$
			\item $S3_I' = 2C \Rightarrow S3_O' = 0 \left(p = \frac{10}{64}\right)$
			\item $S4_I' = 0 \Rightarrow S4_O' = 0 \left(p = 1\right)$
			\item $S5_I' = 0 \Rightarrow S5_O' = 0 \left(p = 1\right)$
			\item $S6_I' = 0 \Rightarrow S6_O' = 0 \left(p = 1\right)$
			\item $S7_I' = 0 \Rightarrow S7_O' = 0 \left(p = 1\right)$
			\item $S8_I' = 0 \Rightarrow S8_O' = 0 \left(p = 1\right)$
		\end{itemize}
		$\Rightarrow p_2^\Omega =  \frac{14}{64}\cdot \frac{8}{64}\cdot \frac{10}{64} \cdot 1\cdots 1 = \frac{35}{8192}$ und somit $p^\Omega = p_1^\Omega \cdot p_2^\Omega = \frac{35}{8192}$.
	\end{enumerate}

	\section*{Lineare Kryptoanalyse}
	\begin{enumerate}[label=(\alph*)]
		\item für $n=1$:
		\begin{align}
			\mathbb{P}(X_1 = 0) &= \frac{1}{2} + 2^{1-1}\left(p_1 - \frac{1}{2}\right) \notag \\
			&= p_1 \notag
		\end{align}
		für $n=2$:
		\begin{align}
			\mathbb{P}(X_1\oplus X_2 = 0) &= \frac{1}{2} + 2^{2-1}\left(p_1 - \frac{1}{2}\right)\left(p_2 - \frac{1}{2}\right) \notag \\
			&= \frac{1}{2} + 2\left(p_1p_2  - \frac{1}{2}p_1 - \frac{1}{2}p_2 + \frac{1}{4}\right) \notag \\
			&= \frac{1}{2} + 2p_1p_2 - p_1 - p_2 + \frac{1}{2} \notag \\
			&= 2p_1p_2 - p_1 - p_2 + 1 \notag
		\end{align}
		vgl. aus Vorlesung $\mathbb{P}(X_1\oplus X_2=0) = p_1p_2 + (1-p_1)(1-p_2)$
	\end{enumerate}

\end{document}