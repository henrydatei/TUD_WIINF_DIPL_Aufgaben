\documentclass{article}

\usepackage{amsmath,amssymb}
\usepackage{tikz}
\usepackage{pgfplots}
\usepackage{xcolor}
\usepackage[left=2.1cm,right=3.1cm,bottom=3cm,footskip=0.75cm,headsep=0.5cm]{geometry}
\usepackage{enumerate}
\usepackage{enumitem}
\usepackage{marvosym}
\usepackage{tabularx}
\usepackage{parskip}
\usepackage{longtable}

\usepackage{listings}
\definecolor{lightlightgray}{rgb}{0.95,0.95,0.95}
\definecolor{lila}{rgb}{0.8,0,0.8}
\definecolor{mygray}{rgb}{0.5,0.5,0.5}
\definecolor{mygreen}{rgb}{0,0.8,0.26}
%\lstdefinestyle{java} {language=java}
\lstset{language=R,
	basicstyle=\ttfamily,
	keywordstyle=\color{lila},
	commentstyle=\color{lightgray},
	stringstyle=\color{mygreen}\ttfamily,
	backgroundcolor=\color{white},
	showstringspaces=false,
	numbers=left,
	numbersep=10pt,
	numberstyle=\color{mygray}\ttfamily,
	identifierstyle=\color{blue},
	xleftmargin=.1\textwidth, 
	%xrightmargin=.1\textwidth,
	escapechar=§,
	%literate={\t}{{\ }}1
	breaklines=true,
	postbreak=\mbox{\space}
}

\usepackage[colorlinks = true, linkcolor = blue, urlcolor  = blue, citecolor = blue, anchorcolor = blue]{hyperref}
\usepackage[utf8]{inputenc}

\renewcommand*{\arraystretch}{1.4}

\newcolumntype{L}[1]{>{\raggedright\arraybackslash}p{#1}}
\newcolumntype{R}[1]{>{\raggedleft\arraybackslash}p{#1}}
\newcolumntype{C}[1]{>{\centering\let\newline\\\arraybackslash\hspace{0pt}}m{#1}}

\newcommand{\E}{\mathbb{E}}
\DeclareMathOperator{\rk}{rk}
\DeclareMathOperator{\Var}{Var}
\DeclareMathOperator{\Cov}{Cov}
\DeclareMathOperator{\dec}{dec}
\DeclareMathOperator{\enc}{enc}
\DeclareMathOperator{\ggT}{ggT}
\DeclareMathOperator{\ord}{ord}

\title{\textbf{Kryptografie und -analyse, Übung 8}}
\author{\textsc{Henry Haustein}}
\date{}

\begin{document}
	\maketitle

	\section*{Betriebsarten}
	\begin{enumerate}[label=(\alph*)]
		\item Die Blöcke $c_1$ bis $c_{i-1}$ können ohne Probleme entschlüsselt werden. Der Block $c_i$ kann nicht entschlüsselt werden, er wurde ja gelöscht. Für den Block $c_{i+1}$ muss folgendes berechnet werden: $m_{i+1} = \dec(k, c_{i+1}) \oplus c_i$, was nicht geht. Ab Block $c_{i+2}$ kann wieder alles entschlüsselt werden, $c_{i+2} = \dec(k, c_{i+2}) \oplus c_{i+1}$.
		\item Ja, man kann unterschiedliche IVs auf Sender- und Empfängerseite verwenden. Auf Senderseite wird verschlüsselt:
		\begin{itemize}
			\item $c_1 = \enc(k, m_1 \oplus IV_S)$
			\item $c_2 = \enc(k, m_2 \oplus c_1)$
		\end{itemize}
		Auf Empfängerseite wird entschlüsselt:
		\begin{itemize}
			\item $m_1 = \dec(k, c_1) \oplus IV_E$ $\Rightarrow$ klappt nicht
			\item $m_2 = \dec(k, c_2) \oplus c_1$ $\Rightarrow$ funktioniert
		\end{itemize}
		Bei CFB ist die Beeinflussung länger, nämlich $\lceil\frac{l}{r}\rceil$, bei OFB geht das gar nicht, weil nur der IV immer wieder verschlüsselt wird. Ist der IV anders, so werden eine völlig andere Pseudo-Schlüssel generiert mit denen die Nachricht $\oplus$ wird.
		\item $m$ = 128 Bit, Blocklänge 64 Bit, $r$ = 8 Bit. Bei CBC wird die Verschlüsselungfunktion zwei mal aufgerufen, weil es 2 Blöcke gibt. Bei CFB kommt es auf $r$ an, hier wird die Verschlüsselungfunktion $\frac{128}{8} = 16$ mal ausgeführt.
		\item Es gilt:
		\begin{center}
			\begin{tabular}{L{2cm}|L{4cm}|L{4cm}|L{4cm}}
				& \textbf{Direktzugriff} & \textbf{Parallelisierbarkeit} & \textbf{Vorausberechnung} \\
				\hline
				\textbf{ECB} & ja & ja & nein \\
				\hline
				\textbf{CBC} & enc: nein, dec: ja & enc: nein, dec: ja & nein \\
				\hline
				\textbf{CFB} & ähnlich CBC & ähnlich CBC & nein (nur 1 Block) \\
				\hline
				\textbf{OFB} & wenn Schlüsselblöcke nicht gespeichert werden: nein & wenn Schlüsselblöcke nicht gespeichert werden: nein & ja \\
				\hline
				\textbf{CTR} & ja & ja & ja \\
			\end{tabular}
		\end{center}
		\item Direktzugriff: ob eine Abhängigkeit von vorherigen Cipherblöcken/Klartextblöcken vorliegt. \\
		Parallelisierbarkeit: wenn Direktzugriff vorliegt \\
		Vorausberechnung: ob Verschlüsselung auf Klartextblöcke oder Schlüsselblöcke angewendet wird
	\end{enumerate}

	\section*{Grundlagen}
	\begin{enumerate}[label=(\alph*)]
		\item $\mathbb{Z}_{77}^\ast = \{a\in\mathbb{Z}_{77}\mid \ggT(a,77) = 1\}$. Offensichtlich $\ggT(20,77) = 1$ und $\ggT(14,77) = 7$ und $20^{-1} = 27$ mit WolframAlpha (\texttt{20\^\,-1 mod 77})
		\item Satz von Lagrange: Wenn $H$ Untergruppe von $G$, dann $\ord(H)\mid \ord(G)$, damit haben die Untergruppen von $\mathbb{Z}_{13}^\ast$ die Ordnungen 1, 2, 3, 4, 6 und 12 (die Ordnung von $\mathbb{Z}_{13}^\ast$ ist $\Phi(13) = 12$).
		\item Primfaktorzerlegung von Gruppenordnung: $12 = 2^2\cdot 3$. Für $a_1=5$:
		\begin{itemize}
			\item $b= a_1^{\frac{n}{p_1}} = 5^{\frac{12}{2}} = 5^6 \equiv 12 \mod 13$
			\item $b= a_1^{\frac{n}{p_2}} = 5^{\frac{12}{3}} = 5^4 \equiv 1 \mod 13$
		\end{itemize}
		Für $a_2 = 6$:
		\begin{itemize}
			\item $b= a_2^{\frac{n}{p_1}} = 6^{\frac{12}{2}} = 6^6 \equiv 12 \mod 13$
			\item $b= a_2^{\frac{n}{p_2}} = 6^{\frac{12}{3}} = 6^4 \equiv 9 \mod 13$
		\end{itemize}
		$\Rightarrow$ $a_1=5$ ist kein Generator, $a_2=6$ ist ein Generator.
		\item 
	\end{enumerate}

\end{document}