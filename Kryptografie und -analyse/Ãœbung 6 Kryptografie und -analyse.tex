\documentclass{article}

\usepackage{amsmath,amssymb}
\usepackage{tikz}
\usepackage{pgfplots}
\usepackage{xcolor}
\usepackage[left=2.1cm,right=3.1cm,bottom=3cm,footskip=0.75cm,headsep=0.5cm]{geometry}
\usepackage{enumerate}
\usepackage{enumitem}
\usepackage{marvosym}
\usepackage{tabularx}
\usepackage{parskip}
\usepackage{longtable}

\usepackage{listings}
\definecolor{lightlightgray}{rgb}{0.95,0.95,0.95}
\definecolor{lila}{rgb}{0.8,0,0.8}
\definecolor{mygray}{rgb}{0.5,0.5,0.5}
\definecolor{mygreen}{rgb}{0,0.8,0.26}
%\lstdefinestyle{java} {language=java}
\lstset{language=R,
	basicstyle=\ttfamily,
	keywordstyle=\color{lila},
	commentstyle=\color{lightgray},
	stringstyle=\color{mygreen}\ttfamily,
	backgroundcolor=\color{white},
	showstringspaces=false,
	numbers=left,
	numbersep=10pt,
	numberstyle=\color{mygray}\ttfamily,
	identifierstyle=\color{blue},
	xleftmargin=.1\textwidth, 
	%xrightmargin=.1\textwidth,
	escapechar=§,
	%literate={\t}{{\ }}1
	breaklines=true,
	postbreak=\mbox{\space}
}

\usepackage[colorlinks = true, linkcolor = blue, urlcolor  = blue, citecolor = blue, anchorcolor = blue]{hyperref}
\usepackage[utf8]{inputenc}

\renewcommand*{\arraystretch}{1.4}

\newcolumntype{L}[1]{>{\raggedright\arraybackslash}p{#1}}
\newcolumntype{R}[1]{>{\raggedleft\arraybackslash}p{#1}}
\newcolumntype{C}[1]{>{\centering\let\newline\\\arraybackslash\hspace{0pt}}m{#1}}

\newcommand{\E}{\mathbb{E}}
\DeclareMathOperator{\rk}{rk}
\DeclareMathOperator{\Var}{Var}
\DeclareMathOperator{\Cov}{Cov}

\title{\textbf{Kryptografie und -analyse, Übung 6}}
\author{\textsc{Henry Haustein}}
\date{}

\begin{document}
	\maketitle

	\section*{Lineare Kryptoanalyse}
	\begin{enumerate}[label=(\alph*)]
		\item für $n=1$:
		\begin{align}
			\mathbb{P}(X_1 = 0) &= \frac{1}{2} + 2^{1-1}\left(p_1 - \frac{1}{2}\right) \notag \\
			&= p_1 \notag
		\end{align}
		für $n=2$:
		\begin{align}
			\mathbb{P}(X_1\oplus X_2 = 0) &= \frac{1}{2} + 2^{2-1}\left(p_1 - \frac{1}{2}\right)\left(p_2 - \frac{1}{2}\right) \notag \\
			&= \frac{1}{2} + 2\left(p_1p_2  - \frac{1}{2}p_1 - \frac{1}{2}p_2 + \frac{1}{4}\right) \notag \\
			&= \frac{1}{2} + 2p_1p_2 - p_1 - p_2 + \frac{1}{2} \notag \\
			&= 2p_1p_2 - p_1 - p_2 + 1 \notag
		\end{align}
		vgl. aus Vorlesung $\mathbb{P}(X_1\oplus X_2=0) = p_1p_2 + (1-p_1)(1-p_2)$
		\item für $n=1$:
		\begin{align}
			\mathbb{P}(X_1 = 0) &= \frac{1}{2} + 2^{1-1}\cdot\varepsilon_1 \notag \\
			&= \frac{1}{2} + \varepsilon_1 \notag
		\end{align}
		für $n=2$:
		\begin{align}
			\mathbb{P}(X_1\oplus X_2 = 0) &= \frac{1}{2} + 2^{2-1}\cdot\varepsilon_1\cdot\varepsilon_2 \notag
		\end{align}
		\item Tabelle
		\begin{center}
			\begin{longtable}{c|c|c|c|c|c}
				$x$ & $f(x)$ & $x^{[1,4]}$ & $f(x)^{[2,3]}$ & $x^{[1]} \oplus x^{[4]}$ & $f(x)^{[2]} \oplus f(x)^{[3]}$ \\
				\hline
				0 = 0000 & e = 1110 & 00 & 11 & 0 & 0 \\
				1 = 0001 & 4 = 0100 & 01 & 10 & 1 & 1 \\
				2 = 0010 & d = 1101 & 00 & 10 & 0 & 1 \\
				3 = 0011 & 1 = 0001 & 01 & 00 & 1 & 0 \\
				4 = 0100 & 2 = 0010 & 00 & 01 & 0 & 1 \\
				5 = 0101 & f = 1111 & 01 & 11 & 1 & 0 \\
				6 = 0110 & b = 1011 & 00 & 01 & 0 & 1 \\
				7 = 0111 & 8 = 1000 & 01 & 00 & 1 & 0 \\
				8 = 1000 & 3 = 0011 & 10 & 01 & 1 & 1 \\
				9 = 1001 & a = 1010 & 11 & 01 & 0 & 1 \\
				a = 1010 & 6 = 0110 & 10 & 11 & 1 & 0 \\
				b = 1011 & c = 1100 & 11 & 10 & 0 & 1 \\
				c = 1100 & 5 = 0101 & 10 & 10 & 1 & 1 \\
				d = 1101 & 9 = 1001 & 11 & 00 & 0 & 0 \\
				e = 1110 & 0 = 0000 & 10 & 00 & 1 & 0 \\
				f = 1111 & 7 = 0111 & 11 & 11 & 0 & 0
			\end{longtable}
		\end{center}
		$\Rightarrow$ 6 Übereinstimmungen $\Rightarrow \frac{6}{16} = 0.375$. Das ist schlechter als Raten, es bietet sich hier an, eine affine lineare Approximation zu nutzen, indem man $f(x)^{[2,3]} = x^{[1,4]} \oplus 1$ nutzt. Die Güte ist dann $\frac{10}{16}$.
		\item Es gilt:
		\begin{itemize}
			\item $m_r = x_1$
			\item $m_l = y_1 \oplus x_2$
			\item $c_r = x_5$
			\item $c_l = y_5 \oplus x_4$
			\item $x_3 = m_r \oplus y_2 = c_r \oplus y_4$
		\end{itemize}
		Einsetzen in Approximationsgleichung für Runde 2:
		\begin{align}
			k_2^{[26]} &= x_2^{[17]} \oplus y_2^{[3,8,14,25]} \oplus 1 \notag \\
			&= m_l^{[17]} \oplus y_1^{[17]} \oplus x_3^{[3,8,14,25]} \oplus m_r^{[3,8,14,25]} \oplus 1 \notag
		\end{align}
		Approximationsgleichung für Runde 4:
		\begin{align}
			k_4^{[26]} &= x_4^{[17]} \oplus y_4^{[3,8,14,25]} \oplus 1 \notag \\
			&= c_l^{[17]} \oplus y_5^{[17]} \oplus x_3^{[3,8,14,25]} \oplus c_r^{[3,8,14,25]} \oplus 1 \notag
		\end{align}
		Approximationsgleichung für Runde 1:
		\begin{align}
			k_1^{[2,3,5,6]} &= x_1^{[1,2,4,5]} \oplus y_1^{[17]} \oplus 1 \notag \\
			&= m_r^{[1,2,4,5]} \oplus y_1^{[17]} \oplus 1 \notag
		\end{align}
		Approximationsgleichung für Runde 5:
		\begin{align}
			k_5^{[2,3,5,6]} &= x_5^{[1,2,4,5]} \oplus y_5^{[17]} \oplus 1 \notag \\
			&= c_r^{[1,2,4,5]} \oplus y_5^{[17]} \oplus 1 \notag
		\end{align}
		Addition der 4 Gleichungen liefert (Aufhebung von z.B. $y_1^{[17]} \oplus y_1^{[17]} = 0$)
		\begin{align}
			k_1^{[2,3,5,6]} \oplus k_2^{[26]} \oplus k_4^{[26]} \oplus k_5^{[2,3,5,6]} &= m_r^{[1,2,4,5]} \oplus m_l^{[17]} \oplus m_r^{[3,8,14,25]} \oplus c_l^{[17]} \oplus c_r^{[3,8,14,25]} \oplus c_r^{[1,2,4,5]} \notag \\
			&= m_l^{[17]} \oplus m_r^{[1,2,3,4,5,8,14,15]} \oplus c_l^{[17]} \oplus c_r^{[1,2,3,4,5,8,14,15]} \notag \\
			&= m^{[17,33,34,35,36,37,40,46,57]} \oplus c^{[17,33,34,35,36,37,40,46,57]} \notag
		\end{align}
		Güte der Approximation mit Pilling-up-Lemma mit $n=4$: $\frac{1}{2} + 2^3\left(\frac{52}{64} - \frac{32}{64}\right)^2\cdot\left(\frac{42}{64} - \frac{32}{64}\right)^2 = 0.519$ 
	\end{enumerate}

\end{document}