\documentclass{article}

\usepackage{amsmath,amssymb}
\usepackage{tikz}
\usepackage{pgfplots}
\usepackage{xcolor}
\usepackage[left=2.1cm,right=3.1cm,bottom=3cm,footskip=0.75cm,headsep=0.5cm]{geometry}
\usepackage{enumerate}
\usepackage{enumitem}
\usepackage{marvosym}
\usepackage{tabularx}
\usepackage{parskip}
\usepackage{longtable}

\usepackage{listings}
\definecolor{lightlightgray}{rgb}{0.95,0.95,0.95}
\definecolor{lila}{rgb}{0.8,0,0.8}
\definecolor{mygray}{rgb}{0.5,0.5,0.5}
\definecolor{mygreen}{rgb}{0,0.8,0.26}
%\lstdefinestyle{java} {language=java}
\lstset{language=R,
	basicstyle=\ttfamily,
	keywordstyle=\color{lila},
	commentstyle=\color{lightgray},
	stringstyle=\color{mygreen}\ttfamily,
	backgroundcolor=\color{white},
	showstringspaces=false,
	numbers=left,
	numbersep=10pt,
	numberstyle=\color{mygray}\ttfamily,
	identifierstyle=\color{blue},
	xleftmargin=.1\textwidth, 
	%xrightmargin=.1\textwidth,
	escapechar=§,
	%literate={\t}{{\ }}1
	breaklines=true,
	postbreak=\mbox{\space}
}

\usepackage[colorlinks = true, linkcolor = blue, urlcolor  = blue, citecolor = blue, anchorcolor = blue]{hyperref}
\usepackage[utf8]{inputenc}

\renewcommand*{\arraystretch}{1.4}

\newcolumntype{L}[1]{>{\raggedright\arraybackslash}p{#1}}
\newcolumntype{R}[1]{>{\raggedleft\arraybackslash}p{#1}}
\newcolumntype{C}[1]{>{\centering\let\newline\\\arraybackslash\hspace{0pt}}m{#1}}

\newcommand{\E}{\mathbb{E}}
\DeclareMathOperator{\rk}{rk}
\DeclareMathOperator{\Var}{Var}
\DeclareMathOperator{\Cov}{Cov}
\DeclareMathOperator{\dec}{dec}
\DeclareMathOperator{\enc}{enc}
\DeclareMathOperator{\ggT}{ggT}
\DeclareMathOperator{\ord}{ord}

\title{\textbf{Kryptografie und -analyse, Übung 10}}
\author{\textsc{Henry Haustein}}
\date{}

\begin{document}
	\maketitle

	\section*{RSA}
	\begin{enumerate}[label=(\alph*)]
		\item $k_s = k_t^{-1}\mod\Phi(n) \Rightarrow k_s = 3^{-1}\mod \Phi(3\cdot 17) \equiv 3^{-1}\mod 32 \Rightarrow k_s = 11$ \\
		$s = 7^{11} \mod 51 = 31$ \\
		$3 \overset{?}{=} 12^3\mod 51 \Rightarrow 45 = 12^3\mod 51 \Rightarrow$ Signatur nicht gültig
		\begin{align}
			m &= \left(m^{k_s}\right)^{k_t} \mod n \notag \\
			m &= m^{k_s\cdot k_t} \mod n \notag
		\end{align}
		Wir wissen $k_s\cdot k_t \equiv 1 \mod \Phi(n)$, also $k_s\cdot k_t = l(p-1)(q-1) + 1$. Damit
		\begin{align}
			m^{k_s\cdot k_t} &= m^{l(p-1)(q-1) + 1} \mod n \notag \\
			&= \left(m^{p-1}\right)^{l(q-1)} \cdot m \mod n \notag
		\end{align}
		Da $p$ eine Primzahl ist, gilt nach dem kleinen Satz von Fermat ($a^{p-1}\equiv 1 \mod p$) und weil $p\mid n$:
		\begin{align}
			1^{l(q-1)}\cdot m = m \mod p \notag
		\end{align}
		\item $k_{dp} = k_e^{-1}\mod (p-1) = 3^{-1} \mod 2 \equiv 1$ \\
		$k_{dq} = k_e^{-1}\mod (q-1) = 3^{-1} \mod 10 \equiv 7$ \\
		$u$, $v$ mit $up+vq=1 \Rightarrow EEA(3,11) \Rightarrow u=4$, $v=-1$ \\
		$\Rightarrow y_p = c^{k_{dp}} \mod p = 19^1 \mod 3 \equiv 1$ \\
		$\Rightarrow y_q = c^{k_{dq}} \mod q = 19^7 \mod 11 \equiv 2$ \\
		$CRA(y_p, y_q) = upy_q + vqy_p = 3\cdot 4\cdot 2 + (-1)\cdot 11\cdot 1 = 13 = m$ \\
		Probe: $13^3\mod 33 \equiv 19$
		\item Angreifer beobachtet $c$, will $m$ ermitteln, bekommt $n, k_e$, wählt $r\in \mathbb{Z}_n^\ast$, berechnet $r^{-1}\mod n$, berechnet $c' = c\cdot r^{k_e}\mod n$ und lässt sich das vom Empfänger entschlüsseln $\Rightarrow m' = (c')^{k_d}\mod n$. Angreifer berechnet
		\begin{align}
			m'\cdot r^{-1}\mod n &= (c')^{k_d}\cdot r^{-1} \mod n \notag \\
			&= (m^{k_e}\cdot r^{k_e})^{k_d} \cdot r^{-1} \mod n \notag \\
			&= m\cdot r\cdot r^{-1} \mod n \notag \\
			&= m \notag
		\end{align}
		\item $r^{-1}\mod 33 \equiv 5$ \\
		$c' = 5\cdot 20^3 \mod 33 \equiv 4$ \\
		$m' = 4^7\mod 33 \equiv 16$ \\
		$m = 16\cdot 5 = 14$
		\item Wer faktorisieren kann, zerlegt $n = p\cdot q$ und berechnet dann $k_d = k_e^{-1}\mod \Phi(n)$.	
	\end{enumerate}

	\section*{Kryptosysteme auf Basis elliptischer Kurven}
	\begin{enumerate}[label=(\alph*)]
		\item Diskriminante $\neq 0$
		\begin{align}
			D &= 4a^3 + 27b^2 \mod p \notag \\
			&= 4\cdot 4^3 + 27\cdot 7^2\mod 11 \notag \\
			&= 6 \mod 11 \notag
		\end{align}
		Für $x = 0$:
		\begin{align}
			z &= 0^3 + 4\cdot 0 + 7 \mod 11 \equiv 7 \notag
		\end{align}
		Ist $7\in QR_{11}$? $\Rightarrow z^{\frac{p-1}{2}}\mod p\overset{?}{\equiv} 1$. $7^5\mod 11\equiv -1$ $\Rightarrow$ es gibt keinen Punkt mit $x=0$ auf dieser Kurve. \\
		Für $x=5$:
		\begin{align}
			z &= 5^3 + 4\cdot 5 + 7 \mod 11 \equiv 9 \notag
		\end{align}
		Ist $9\in QR_{11}$? $\Rightarrow z^{\frac{p-1}{2}}\mod p\overset{?}{\equiv} 1$. $9^5\mod 11\equiv 1$ $\Rightarrow$ es gibt einen Punkt mit $x=0$ auf dieser Kurve $P(5, 3)$ (und auch $Q(5,8)$). \\
		Punkt-Kompression: $(5,3) \to (5,3\mod 2) = (5,1)$ und $(5,8)\to(5,8\mod 2) = (5,0)$ \\
		Punkt-Dekompression:
		\begin{align}
			z &= 6^3 + 4\cdot 6 + 7 \mod 11 \equiv 5 \notag \\
			y &= \sqrt{5} \mod 11 \equiv 4 \notag
		\end{align}
		Da $4\not\equiv 1 \mod 2$, ist $(6,1)\to (6,-4) = (6,7)$.
		\item $827 = 1100111011_2$ $\Rightarrow$ 10 Punktverdopplungen, 7 Punktadditionen \\
		nichtadjazenter Form: $[1,0,-1,0,1,0,0,0,-1,0,-1]_2 \Rightarrow$ 11 Punktverdopplungen, 5 Punktadditionen/-subtraktionen (online Tool zur Umwandlung Dezimal $\to$ NAF: \url{https://codegolf.stackexchange.com/questions/235319/convert-to-a-non-adjacent-form})
		\item $Q_A = 2\cdot (1,1)\mod 11$
		\begin{align}
			s &= \frac{3x^2 + a}{2y} = \frac{3\cdot 1^2 + 4}{2\cdot 1} = \frac{7}{2} \notag \\
			x_Q &= s^2 - 2x = \frac{49}{4} - 2 = \frac{41}{4} \notag \\
			y_Q &= -y + s(x - x_Q) = -1 + \frac{7}{2}\left(1 - \frac{41}{4}\right) = -\frac{267}{8} \notag
		\end{align}
		$Q_B = 3\cdot (1,1) \mod 11 = (1,1) + \left(\frac{41}{4},-\frac{267}{8}\right)\mod 11$
		\begin{align}
			s &= \frac{\Delta y}{\Delta x} = \frac{-\frac{267}{8} - 1}{\frac{41}{4} - 1} = -\frac{275}{74} \notag \\
			x_Q &= \left(-\frac{275}{74}\right)^2 - \frac{41}{4} - 1 = \frac{3505}{1369} \notag \\
			y_Q &= -1 - \frac{275}{74}\left(1 - \frac{3505}{1369}\right) = \frac{243047}{50653} \notag
		\end{align}
		gemeinsamer Schlüssel $k_{AB} = 2\cdot \left(\frac{3505}{1369},\frac{243047}{50653}\right)$
		\begin{align}
			s &= \frac{3x^2 + a}{2y} = \frac{3\cdot \left(\frac{3505}{1369}\right)^2 + 4}{2\cdot \frac{243047}{50653}} = \frac{44351719}{17985478} \notag \\
			x_k &= s^2 - 2x = \left(\frac{44351719}{17985478}\right)^2 - 2\cdot \frac{3505}{1369} = \frac{310700466634601}{323477418888484} \notag \\
			y_k &= -y + s(x - x_k) = -\frac{243047}{50653} + \frac{44351719}{17985478}\left(\frac{3505}{1369} - \frac{310700466634601}{323477418888484}\right) = -\frac{4964433566291413215147}{5817896000915613435352} \notag
		\end{align}
	\end{enumerate}

\end{document}