\documentclass{article}

\usepackage{amsmath,amssymb}
\usepackage{tikz}
\usepackage{pgfplots}
\usepackage{xcolor}
\usepackage[left=2.1cm,right=3.1cm,bottom=3cm,footskip=0.75cm,headsep=0.5cm]{geometry}
\usepackage{enumerate}
\usepackage{enumitem}
\usepackage{marvosym}
\usepackage{tabularx}
\usepackage{parskip}

\usepackage{listings}
\definecolor{lightlightgray}{rgb}{0.95,0.95,0.95}
\definecolor{lila}{rgb}{0.8,0,0.8}
\definecolor{mygray}{rgb}{0.5,0.5,0.5}
\definecolor{mygreen}{rgb}{0,0.8,0.26}
%\lstdefinestyle{java} {language=java}
\lstset{language=R,
	basicstyle=\ttfamily,
	keywordstyle=\color{lila},
	commentstyle=\color{lightgray},
	stringstyle=\color{mygreen}\ttfamily,
	backgroundcolor=\color{white},
	showstringspaces=false,
	numbers=left,
	numbersep=10pt,
	numberstyle=\color{mygray}\ttfamily,
	identifierstyle=\color{blue},
	xleftmargin=.1\textwidth, 
	%xrightmargin=.1\textwidth,
	escapechar=§,
	%literate={\t}{{\ }}1
	breaklines=true,
	postbreak=\mbox{\space}
}

\usepackage[colorlinks = true, linkcolor = blue, urlcolor  = blue, citecolor = blue, anchorcolor = blue]{hyperref}
\usepackage[utf8]{inputenc}

\renewcommand*{\arraystretch}{1.4}

\newcolumntype{L}[1]{>{\raggedright\arraybackslash}p{#1}}
\newcolumntype{R}[1]{>{\raggedleft\arraybackslash}p{#1}}
\newcolumntype{C}[1]{>{\centering\let\newline\\\arraybackslash\hspace{0pt}}m{#1}}

\newcommand{\E}{\mathbb{E}}
\DeclareMathOperator{\rk}{rk}
\DeclareMathOperator{\Var}{Var}
\DeclareMathOperator{\Cov}{Cov}

\title{\textbf{Kryptografie und -analyse, Zusammenfassung Vorlesung 6}}
\author{\textsc{Henry Haustein}}
\date{}

\begin{document}
	\maketitle
	
	\section*{Was ist eine iterative Charakteristik?}
	Ein Sonderfall sind sogenannte iterative Charakteristiken, mit $\Omega_1 = \Omega_2$, welche immer wieder an sich selbst angehängt werden können. Die vertauschten Hälften der Klartextdifferenz sind also gleich der Geheimtextdifferenz derselben Charakteristik. Diese lassen sich also leicht zu beliebig großen $n$-Runden-Charakteristiken zusammenhängen. Während bei nicht-iterativen Charakteristiken die Wahrscheinlichkeit mit größerem $n$, bedingt durch den Avalanche-Effekt, immer schneller abnimmt, bleiben die Wahrscheinlichkeiten der Teilcharakteristiken aus denen iterative Charakteristiken zusammengesetzt sind gleich. Iterative Charakteristiken werden deshalb bei einem Angriff bevorzugt eingesetzt.

	\section*{Wie wirkt sich die Anzahl aktiver S-Boxen auf die differentielle Kryptoanalyse aus?}
	Es wird aufwendiger, je mehr S-Boxen aktiv sind.
	
	\section*{Was ist das Ziel der linearen Kryptoanalyse?}
	Klartext-Schlüsseltext-Angriff
	\begin{itemize}
		\item Ziel: Approximation der Chiffrierfunktion durch eine lineare Abbildung
		\item Suche nach Approximationsgleichungen mit möglichst hoher Güte
		\item Untersuchung genügend vieler Klartext-Schlüsseltext-Paare liefert Schlüsselbits
	\end{itemize}
	Lineare Abhängigkeit einzelner Ausgabebits einer S-Box $Si_{O}[i]$? gesucht: Funktionen $\phi: \mathbb{F}_2^6 \to \mathbb{F}_2$ mit
	\begin{align}
		Si_O[i] = \phi(Si) = \bigoplus_{k=1}^{6} l_k\cdot Si_I[k] \notag
	\end{align}
	
	\section*{Wie werden lineare Approximationen für die Substitutionsboxen ermittelt?}
	Systematische Suche
	
	\section*{Wie kann mit Hilfe einer solchen Approximationsgleichung eine Runde analysiert werden?}
	Mit Auswahlvektor $u=(010000)$ und $v=(1111)$ (Güte $\frac{12}{64}$ $\Rightarrow$ affine Approximation, Güte $\frac{52}{64}$) ergibt sich für S5:
	\begin{align}
		u^T\cdot S5_I &= v^T\cdot S5_O \oplus 1 \notag \\
		u^T\cdot (m\oplus k) &= v^T\cdot c \oplus 1 \notag \\
		(010000)^T\cdot m \oplus (010000)^T \cdot k &= (1111)^T \cdot c \oplus 1 \notag \\
		m^{[2]} \oplus k^{[2]} &= c^{[1,2,3,4]} \oplus 1 \notag
	\end{align}
	Umstellen nach $k^{[2]}$:
	\begin{align}
		k^{[2]} = m^{[2]} \oplus c^{[1,2,3,4]} \oplus 1 \notag
	\end{align}
	Analyse von genügend Klartext-Schlüsseltext-Paaren liefert $k^{[2]}$.
	
	\section*{Wie ist das allgemeine Vorgehen bei der linearen Kryptoanalyse (einfacher Algorithmus)?}
	Vorbereitung:
	\begin{itemize}
		\item Auswahlvektoren $u$, $v$, $w$ bestimmen mit:
		\begin{align}
			w^k &= u^T\cdot m \oplus v^T\cdot c \quad\text{oder} \notag \\
			w^k &= u^T\cdot m \oplus v^T\cdot c \oplus 1 \notag
		\end{align}
		\item Güte der Approximation $p_A > 0.5$
	\end{itemize}
	1. Schritt
	\begin{itemize}
		\item Untersuchung von $N$ Klartext-Schlüsseltext-Paaren
		\item $Z$: Anzahl von Paaren, für die die rechte Seite der entsprechenden Gleichung 0 ist
	\end{itemize}
	2. Schritt: $Z > \frac{N}{2}: w^T\cdot k = 0$ oder $Z < \frac{N}{2}: w^T\cdot k = 1$
	
	\section*{Wie erfolgt die Analyse des DES mit 3 Runden?}
	Zwei verschiedene Approximationsgleichungen für erste und dritte Runde
	\begin{itemize}
		\item 1. Runde: $k_1^{[26]} = x_1^{[17]} \oplus y_1^{[3,8,14,25]} \oplus 1$
		\item 3. Runde: $k_3^{[26]} = x_3^{[17]} \oplus y_3^{[3,8,14,25]} \oplus 1$
	\end{itemize}
	Ersetzen von $y_1 = L_m \oplus x_2$ und $y_3 = L_c \oplus x_2$ $\Rightarrow$ Addieren der Gleichungen entfernt $x_2$ (für Näheres siehe Übung)
	
\end{document}