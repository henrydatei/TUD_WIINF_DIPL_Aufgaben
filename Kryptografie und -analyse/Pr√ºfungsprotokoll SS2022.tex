\documentclass{article}

\usepackage{amsmath,amssymb}
\usepackage{tikz}
\usepackage{pgfplots}
\usepackage{xcolor}
\usepackage[left=2.1cm,right=3.1cm,bottom=3cm,footskip=0.75cm,headsep=0.5cm]{geometry}
\usepackage{enumerate}
\usepackage{enumitem}
\usepackage{marvosym}
\usepackage{tabularx}
\usepackage{parskip}

\usepackage{listings}
\definecolor{lightlightgray}{rgb}{0.95,0.95,0.95}
\definecolor{lila}{rgb}{0.8,0,0.8}
\definecolor{mygray}{rgb}{0.5,0.5,0.5}
\definecolor{mygreen}{rgb}{0,0.8,0.26}
%\lstdefinestyle{java} {language=java}
\lstset{language=R,
	basicstyle=\ttfamily,
	keywordstyle=\color{lila},
	commentstyle=\color{lightgray},
	stringstyle=\color{mygreen}\ttfamily,
	backgroundcolor=\color{white},
	showstringspaces=false,
	numbers=left,
	numbersep=10pt,
	numberstyle=\color{mygray}\ttfamily,
	identifierstyle=\color{blue},
	xleftmargin=.1\textwidth, 
	%xrightmargin=.1\textwidth,
	escapechar=§,
	%literate={\t}{{\ }}1
	breaklines=true,
	postbreak=\mbox{\space}
}

\usepackage[colorlinks = true, linkcolor = blue, urlcolor  = blue, citecolor = blue, anchorcolor = blue]{hyperref}
\usepackage[utf8]{inputenc}

\renewcommand*{\arraystretch}{1.4}

\newcolumntype{L}[1]{>{\raggedright\arraybackslash}p{#1}}
\newcolumntype{R}[1]{>{\raggedleft\arraybackslash}p{#1}}
\newcolumntype{C}[1]{>{\centering\let\newline\\\arraybackslash\hspace{0pt}}m{#1}}

\newcommand{\E}{\mathbb{E}}
\DeclareMathOperator{\rk}{rk}
\DeclareMathOperator{\Var}{Var}
\DeclareMathOperator{\Cov}{Cov}

\title{\textbf{Kryptografie und -analyse, Prüfungsprotokoll SS 2022}}
\date{}

\begin{document}
	\maketitle
	Prüferin: Dr.-Ing. Elke Franz
	
	\section*{Wir haben die Schlüsselverteilung in symmetrisch und asymmetrisch unterteilt. Was bedeutet das und was sind die Vorteile?}
	
	\section*{Was ist informationstheoretisch perfekte Sicherheit? Geben Sie ein Beispielverfahren an! Wird das Verfahren praktisch angewendet?}
	
	\section*{Schutzziele der Kryptographie, Warum benutzen wir nicht das informationstheoretisch perfekt sichere Verfahren in Bezug auf die Schutzziele?}
	
	\section*{Was sind Betriebsarten? Was kann man damit erreichen?}
	
	\section*{Was ist Electronic Code Book und was ist daran das Problem?}
	
	\section*{Wie funktioniert Cipher Block Chaining?}
	
	\section*{Wie funktioniert RSA? Schlüsselgenerierung (Bedingungen, $\Phi(n)$, ...), Ver- und Entschlüsselung?}
	
	\section*{Ist das gerade eben skizzierte RSA-System sicher?}
	
	\section*{Was besagt das Prinzip von Kerkhoff?}
	
\end{document}