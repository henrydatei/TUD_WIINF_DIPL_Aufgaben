\documentclass{article}

\usepackage{amsmath,amssymb}
\usepackage{tikz}
\usepackage{pgfplots}
\usepackage{xcolor}
\usepackage[left=2.1cm,right=3.1cm,bottom=3cm,footskip=0.75cm,headsep=0.5cm]{geometry}
\usepackage{enumerate}
\usepackage{enumitem}
\usepackage{marvosym}
\usepackage{tabularx}
\usepackage{parskip}
\usepackage{longtable}

\usepackage{listings}
\definecolor{lightlightgray}{rgb}{0.95,0.95,0.95}
\definecolor{lila}{rgb}{0.8,0,0.8}
\definecolor{mygray}{rgb}{0.5,0.5,0.5}
\definecolor{mygreen}{rgb}{0,0.8,0.26}
%\lstdefinestyle{java} {language=java}
\lstset{language=R,
	basicstyle=\ttfamily,
	keywordstyle=\color{lila},
	commentstyle=\color{lightgray},
	stringstyle=\color{mygreen}\ttfamily,
	backgroundcolor=\color{white},
	showstringspaces=false,
	numbers=left,
	numbersep=10pt,
	numberstyle=\color{mygray}\ttfamily,
	identifierstyle=\color{blue},
	xleftmargin=.1\textwidth, 
	%xrightmargin=.1\textwidth,
	escapechar=§,
	%literate={\t}{{\ }}1
	breaklines=true,
	postbreak=\mbox{\space}
}

\usepackage[colorlinks = true, linkcolor = blue, urlcolor  = blue, citecolor = blue, anchorcolor = blue]{hyperref}
\usepackage[utf8]{inputenc}

\renewcommand*{\arraystretch}{1.4}

\newcolumntype{L}[1]{>{\raggedright\arraybackslash}p{#1}}
\newcolumntype{R}[1]{>{\raggedleft\arraybackslash}p{#1}}
\newcolumntype{C}[1]{>{\centering\let\newline\\\arraybackslash\hspace{0pt}}m{#1}}

\newcommand{\E}{\mathbb{E}}
\DeclareMathOperator{\rk}{rk}
\DeclareMathOperator{\Var}{Var}
\DeclareMathOperator{\Cov}{Cov}
\DeclareMathOperator{\dec}{dec}
\DeclareMathOperator{\enc}{enc}
\DeclareMathOperator{\ggT}{ggT}
\DeclareMathOperator{\ord}{ord}

\title{\textbf{Kryptografie und -analyse, Übung 9}}
\author{\textsc{Henry Haustein}}
\date{}

\begin{document}
	\maketitle

	\section*{Grundlagen}
	\begin{enumerate}[label=(\alph*)]
		\item $\mathbb{Z}_{77}^\ast = \{a\in\mathbb{Z}_{77}\mid \ggT(a,77) = 1\}$. Offensichtlich $\ggT(20,77) = 1$ und $\ggT(14,77) = 7$ und $20^{-1} = 27$ mit WolframAlpha (\texttt{20\^\,-1 mod 77})
		\item Satz von Lagrange: Wenn $H$ Untergruppe von $G$, dann $\ord(H)\mid \ord(G)$, damit haben die Untergruppen von $\mathbb{Z}_{13}^\ast$ die Ordnungen 1, 2, 3, 4, 6 und 12 (die Ordnung von $\mathbb{Z}_{13}^\ast$ ist $\Phi(13) = 12$).
		\item Primfaktorzerlegung von Gruppenordnung: $12 = 2^2\cdot 3$. Für $a_1=5$:
		\begin{itemize}
			\item $b= a_1^{\frac{n}{p_1}} = 5^{\frac{12}{2}} = 5^6 \equiv 12 \mod 13$
			\item $b= a_1^{\frac{n}{p_2}} = 5^{\frac{12}{3}} = 5^4 \equiv 1 \mod 13$
		\end{itemize}
		Für $a_2 = 6$:
		\begin{itemize}
			\item $b= a_2^{\frac{n}{p_1}} = 6^{\frac{12}{2}} = 6^6 \equiv 12 \mod 13$
			\item $b= a_2^{\frac{n}{p_2}} = 6^{\frac{12}{3}} = 6^4 \equiv 9 \mod 13$
		\end{itemize}
		$\Rightarrow$ $a_1=5$ ist kein Generator, $a_2=6$ ist ein Generator.
		\item Binärdarstellung des Exponenten: 22 = 10110$_2$. $z_0 = 1$, $z_1 = 5$, $z_2 = 3$, $z_3 = 1$, $z_4 = 5$, $z_5 = 3$
	\end{enumerate}

	\section*{Diskreter Logarithmus/ElGamal}
	\begin{enumerate}[label=(\alph*)]
		\item $x = \log_5(9)\mod 11$ und $x = qm + r$ \\
		$m = \lceil\sqrt{\vert G\vert}\rceil = \lceil\sqrt{\Phi(11)}\rceil = \lceil\sqrt{10}\rceil = 4$ \\ Babystep-Liste: $B = \left\lbrace\left(i,y(g^i)^{-1}\mod p\right), 0\le i <m\right\rbrace$ mit $y=9 \equiv -2$, $g=5$ und $p=11$. $B=\{(0, 9), (1,4), (2,3), (3,5)\}$ \\
		Giantstep-Liste: $G=\{(j, (g^m)^j \mod p), 0\le j<m\} = \{(0,1), (1,9), (2,4), (3,3)\}$ \\
		erstes zusammengehörendes Element: (0,9) und (1,9) $\Rightarrow x = 1\cdot 4 + 0 = 4$ \\
		Probe: $5^4 \equiv 9 \mod 11$
		\item $k_e = g^{k_d} = 5^3 \equiv 6 \mod 17$ \\
		$c_1 = g^r = 5^2 \equiv 8 \mod 17$ \\
		$c_2 = m\cdot k_e^r = 4\cdot 6^2 \equiv 8 \mod 17$ \\
		$m = (c_1^{k_d})^{-1}\cdot c_2 = (6^3)^{-1}\cdot 16 \equiv 12^{-1}\cdot 16 \equiv 10\cdot 16 \equiv 7 \mod 17$ \\
		$k_d = \log_g(k_e) \mod p$
		\item $m_2 = m_1\cdot c_{1,2}^{-1} \cdot c_{2,2} = 4 \cdot 7^{-1}\cdot 2 = 4\cdot 8\cdot 2 \equiv 9 \mod 11$
		\item $k_t = g^{k_s} = 2^3 \equiv 8\mod 11$ \\
		Berechnung $r^{-1}$ mit $rr^{-1} \equiv 1\mod (p-1) \Rightarrow r^{-1} = 3$
		$s_1 = g^r = 2^7 \equiv 7 \mod 11$ \\
		$s_2 = r^{-1}(m - k_ss_1) = 3\cdot (4 - 3\cdot 7) = -51 \equiv 9 \mod 10$ \\
		$v_1 = k_t^{s_1}\cdot s_1^{s_2} = 8^7\cdot 7^9 \equiv 2\cdot 8 \equiv 5 \mod 11$ \\
		$v_2 = g^m = 2^4 \equiv 5 \mod 11 \Rightarrow v_1 = v_2$
	\end{enumerate}

\end{document}