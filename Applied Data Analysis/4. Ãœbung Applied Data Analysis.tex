\documentclass{article}

\usepackage{amsmath,amssymb}
\usepackage{tikz}
\usepackage{pgfplots}
\usepackage{xcolor}
\usepackage[left=2.1cm,right=3.1cm,bottom=3cm,footskip=0.75cm,headsep=0.5cm]{geometry}
\usepackage{enumerate}
\usepackage{enumitem}
\usepackage{marvosym}
\usepackage{tabularx}
\usepackage{parskip}

\usepackage{listings}
\definecolor{lightlightgray}{rgb}{0.95,0.95,0.95}
\definecolor{lila}{rgb}{0.8,0,0.8}
\definecolor{mygray}{rgb}{0.5,0.5,0.5}
\definecolor{mygreen}{rgb}{0,0.8,0.26}
%\lstdefinestyle{java} {language=java}
\lstset{language=R,
	basicstyle=\ttfamily,
	keywordstyle=\color{lila},
	commentstyle=\color{lightgray},
	stringstyle=\color{mygreen}\ttfamily,
	backgroundcolor=\color{white},
	showstringspaces=false,
	numbers=left,
	numbersep=10pt,
	numberstyle=\color{mygray}\ttfamily,
	identifierstyle=\color{blue},
	xleftmargin=.1\textwidth, 
	%xrightmargin=.1\textwidth,
	escapechar=§,
	%literate={\t}{{\ }}1
	breaklines=true,
	postbreak=\mbox{\space}
}

\usepackage[utf8]{inputenc}

\renewcommand*{\arraystretch}{1.4}

\newcolumntype{L}[1]{>{\raggedright\arraybackslash}p{#1}}
\newcolumntype{R}[1]{>{\raggedleft\arraybackslash}p{#1}}
\newcolumntype{C}[1]{>{\centering\let\newline\\\arraybackslash\hspace{0pt}}m{#1}}

\newcommand{\E}{\mathbb{E}}
\DeclareMathOperator{\rk}{rk}
\DeclareMathOperator{\Var}{Var}
\DeclareMathOperator{\Cov}{Cov}

\title{\textbf{Applied Data Analysis, Übung 4}}
\author{\textsc{Henry Haustein}}
\date{}

\begin{document}
	\maketitle
	
	\section*{Vorbereitung}
	\begin{lstlisting}
install.packages("dplyr")
library(dplyr)
install.packages("readxl")
library(readxl)
data = read_excel("data.xlsx", na = "NA")

data$gender = factor(data$gender)
levels(data$gender) = c("male", "female", "diverse")
data$employment = factor(data$employment)
levels(data$employment) = c("student", "employed", "unemployed")
data$education = factor(data$education)
levels(data$education) = c("no degree", "secondary", "intermediate", "high school", "academic")
data$play_frequency = factor(data$play_frequency)
levels(data$play_frequency) = c("never", "every few months", "every few weeks", "1-2 days a week", "3-5 days a week", "daily")
data$treatment = factor(data$treatment)
levels(data$treatment) = c("control", "lootbox in task reward", "lootbox picture", "badge")
data$age = sapply(data$age, function(year) {2016-year})
data$rt6 = as.numeric(data$rt6)
data$rt7 = as.numeric(data$rt7)
data$rt8 = as.numeric(data$rt8)
data$rt9 = as.numeric(data$rt9)
data$rt10 = as.numeric(data$rt10)
data$rt11 = as.numeric(data$rt11)
data$rt12 = as.numeric(data$rt12)
data$rt13 = as.numeric(data$rt13)
data$rt14 = as.numeric(data$rt14)
subControl = subset(data, treatment == "control")
	\end{lstlisting}
	
	\section*{Task 1}
	\begin{lstlisting}
install.packages("ggplot2")
library(ggplot2)
ggplot(data, aes(x = age, y = `total time`)) 
+ geom_point() 
+ geom_smooth(method = "lm")
lm_tt_age = lm(`total time` ~ age, data = data)
summary(lm_tt_age)
ggplot(subControl, aes(x = age, y = `total time`)) 
+ geom_point() 
+ geom_smooth(method = "lm")
lm_tt_age_control = lm(`total time` ~ age, data = subControl)
summary(lm_tt_age_control)

ggplot(data, aes(x = tasks_completed, y = `total time`)) 
+ geom_point() 
+ geom_smooth(method = "lm")
lm_tt_tc = lm(`total time` ~ tasks_completed, data = data)
summary(lm_tt_tc)
ggplot(subControl, aes(x = tasks_completed, y = `total time`)) 
+ geom_point() 
+ geom_smooth(method = "lm")
lm_tt_tc_control = lm(`total time` ~ tasks_completed, data = subControl)
summary(lm_tt_tc_control)

ggplot(data, aes(x = rt0, y = `total time`)) 
+ geom_point() 
+ geom_smooth(method = "lm")
lm_tt_rt0 = lm(`total time` ~ rt0, data = data)
summary(lm_tt_rt0)
ggplot(subControl, aes(x = rt0, y = `total time`)) 
+ geom_point() 
+ geom_smooth(method = "lm")
lm_tt_rt0_control = lm(`total time` ~ rt0, data = subControl)
summary(lm_tt_rt0_control)
	\end{lstlisting}
	Das führt zu folgenden Modellen:
\begin{center}
\begin{tabular}{l c c}
\hline
 & Full Model & Control Group \\
\hline
(Intercept) & $230665.09^{***}$ & $298612.42$   \\
            & $(53287.10)$      & $(156127.98)$ \\
age         & $3635.06$         & $-1457.79$    \\
            & $(2032.53)$       & $(5666.45)$   \\
\hline
R$^2$       & $0.01$            & $0.00$        \\
Adj. R$^2$  & $0.01$            & $-0.01$       \\
Num. obs.   & $437$             & $99$          \\
\hline
\multicolumn{3}{l}{\scriptsize{$^{***}p<0.001$; $^{**}p<0.01$; $^{*}p<0.05$}}
\end{tabular}
\end{center}
\begin{center}
\begin{tabular}{l c c}
\hline
 & Full Model & Control Group \\
\hline
(Intercept)      & $170842.92^{***}$ & $280409.31^{*}$ \\
                 & $(44899.68)$      & $(128200.11)$   \\
tasks\_completed & $14859.12^{***}$  & $-2518.17$      \\
                 & $(4095.99)$       & $(15008.84)$    \\
\hline
R$^2$            & $0.03$            & $0.00$          \\
Adj. R$^2$       & $0.03$            & $-0.01$         \\
Num. obs.        & $437$             & $99$            \\
\hline
\multicolumn{3}{l}{\scriptsize{$^{***}p<0.001$; $^{**}p<0.01$; $^{*}p<0.05$}}
\end{tabular}
\end{center}
\begin{center}
\begin{tabular}{l c c}
\hline
 & Full Model & Control Group \\
\hline
(Intercept) & $157695.17^{***}$ & $119595.89^{***}$ \\
            & $(5821.01)$       & $(5304.52)$       \\
rt0         & $1.01^{***}$      & $0.99^{***}$      \\
            & $(0.01)$          & $(0.01)$          \\
\hline
R$^2$       & $0.91$            & $0.99$            \\
Adj. R$^2$  & $0.91$            & $0.99$            \\
Num. obs.   & $437$             & $99$              \\
\hline
\multicolumn{3}{l}{\scriptsize{$^{***}p<0.001$; $^{**}p<0.01$; $^{*}p<0.05$}}
\end{tabular}
\end{center}
Man sieht, dass offensichtlich nur die Variable \texttt{rt0} die Variable \texttt{total time} gut erklären kann.

	\section*{Task 2}
	\begin{lstlisting}
lm_tt_age_rt0 = lm(`total time` ~ age + rt0, data = data)
summary(lm_tt_age_rt0)

lm_tt_age_rt0_interaction = lm(`total time` ~ age + rt0 + age:rt0, data = data)
summary(lm_tt_age_rt0_interaction)

lm_tt_treatment_rt0 = lm(`total time` ~ treatment + treatment:rt0 + 0, data = data)
summary(lm_tt_treatment_rt0)
	\end{lstlisting}
	Das führt zu folgenden Modellen:
\begin{center}
\begin{tabular}{l c c}
\hline
 & without interaction terms &  with interaction terms \\
\hline
(Intercept) & $107465.81^{***}$ & $115139.46^{***}$ \\
            & $(15600.65)$      & $(20861.24)$      \\
age         & $2048.80^{***}$   & $1700.27^{*}$     \\
            & $(591.55)$        & $(863.35)$        \\
rt0         & $1.01^{***}$      & $0.98^{***}$      \\
            & $(0.01)$          & $(0.06)$          \\
age:rt0     &                   & $0.00$            \\
            &                   & $(0.00)$          \\
\hline
R$^2$       & $0.92$            & $0.92$            \\
Adj. R$^2$  & $0.92$            & $0.92$            \\
Num. obs.   & $437$             & $437$             \\
\hline
\multicolumn{3}{l}{\scriptsize{$^{***}p<0.001$; $^{**}p<0.01$; $^{*}p<0.05$}}
\end{tabular}
\end{center}
\begin{center}
\begin{tabular}{l c}
\hline
 & Model 1 \\
\hline
treatmentcontrol                    & $119595.89^{***}$ \\
                                    & $(10859.57)$      \\
treatmentlootbox in task reward     & $224714.96^{***}$ \\
                                    & $(12861.18)$      \\
treatmentlootbox picture            & $135344.63^{***}$ \\
                                    & $(18221.65)$      \\
treatmentbadge                      & $153816.42^{***}$ \\
                                    & $(10165.92)$      \\
treatmentcontrol:rt0                & $0.99^{***}$      \\
                                    & $(0.02)$          \\
treatmentlootbox in task reward:rt0 & $1.03^{***}$      \\
                                    & $(0.04)$          \\
treatmentlootbox picture:rt0        & $1.06^{***}$      \\
                                    & $(0.13)$          \\
treatmentbadge:rt0                  & $1.01^{***}$      \\
                                    & $(0.02)$          \\
\hline
R$^2$                               & $0.96$            \\
Adj. R$^2$                          & $0.96$            \\
Num. obs.                           & $437$             \\
\hline
\multicolumn{2}{l}{\scriptsize{$^{***}p<0.001$; $^{**}p<0.01$; $^{*}p<0.05$}}
\end{tabular}
\end{center}
Sobald wir Interaktionsterme erlauben, ist das Alter der Probanden gar nicht mehr so wichtig. Die Behandlung scheint aber einen großen Einfluss auf die \texttt{total time} zu haben.
	
\end{document}