\documentclass{article}

\usepackage{amsmath,amssymb}
\usepackage{tikz}
\usepackage{pgfplots}
\usepackage{xcolor}
\usepackage[left=2.1cm,right=3.1cm,bottom=3cm,footskip=0.75cm,headsep=0.5cm]{geometry}
\usepackage{enumerate}
\usepackage{enumitem}
\usepackage{marvosym}
\usepackage{tabularx}
\usepackage{parskip}

\usepackage{listings}
\definecolor{lightlightgray}{rgb}{0.95,0.95,0.95}
\definecolor{lila}{rgb}{0.8,0,0.8}
\definecolor{mygray}{rgb}{0.5,0.5,0.5}
\definecolor{mygreen}{rgb}{0,0.8,0.26}
%\lstdefinestyle{java} {language=java}
\lstset{language=R,
	basicstyle=\ttfamily,
	keywordstyle=\color{lila},
	commentstyle=\color{lightgray},
	stringstyle=\color{mygreen}\ttfamily,
	backgroundcolor=\color{white},
	showstringspaces=false,
	numbers=left,
	numbersep=10pt,
	numberstyle=\color{mygray}\ttfamily,
	identifierstyle=\color{blue},
	xleftmargin=.1\textwidth, 
	%xrightmargin=.1\textwidth,
	escapechar=§,
	%literate={\t}{{\ }}1
	breaklines=true,
	postbreak=\mbox{\space}
}

\usepackage[utf8]{inputenc}

\renewcommand*{\arraystretch}{1.4}

\newcolumntype{L}[1]{>{\raggedright\arraybackslash}p{#1}}
\newcolumntype{R}[1]{>{\raggedleft\arraybackslash}p{#1}}
\newcolumntype{C}[1]{>{\centering\let\newline\\\arraybackslash\hspace{0pt}}m{#1}}

\newcommand{\E}{\mathbb{E}}
\DeclareMathOperator{\rk}{rk}
\DeclareMathOperator{\Var}{Var}
\DeclareMathOperator{\Cov}{Cov}

\title{\textbf{Applied Data Analysis, Übung 1}}
\author{\textsc{Henry Haustein}}
\date{}

\begin{document}
	\maketitle
	
	\section*{Task 1}
	\begin{lstlisting}
install.packages("readxl")
library(readxl)
data = read_excel("data.xlsx", na = "NA")
	\end{lstlisting}

	\section*{Task 2}
	\begin{lstlisting}
str(data)
summary(data)
head(data, 1)["age"]
tail(data, 1)["age"]
dim(data)[1]
	\end{lstlisting}
	Wir stellen fest, dass die numerische Kodierung bei einigen Variablen nicht gut umgesetzt ist: Manchmal starten die Level bei 0, mal bei 1 und nicht immer sind alle Zahlen genutzt. Zudem ist in der Spalte \textit{Alter} nicht das Alter verzeichnet, sondern das Geburtsdatum.
	
	Die erste Person wurde 1967 geboren, die letzte 1996. Insgesamt gibt es 437 Beobachtungen in der Tabelle.
	
	\section*{Task 3}
	\begin{lstlisting}
data$gender = factor(data$gender)
levels(data$gender) = c("male", "female", "diverse")
data$employment = factor(data$employment)
levels(data$employment) = c("student", "employed", "unemployed")
data$education = factor(data$education)
levels(data$education) = c("no degree", "secondary", "intermediate", "high school", "academic")
data$play_frequency = factor(data$play_frequency)
levels(data$play_frequency) = c("never", "every few months", "every few weeks", "1-2 days a week", "3-5 days a week", "daily")
data$treatment = factor(data$treatment)
levels(data$treatment) = c("control", "lootbox in task reward", "lootbox picture", "badge")
data$age = sapply(data$age, function(year) {2016-year})
data$rt6 = as.numeric(data$rt6)
data$rt7 = as.numeric(data$rt7)
data$rt8 = as.numeric(data$rt8)
data$rt9 = as.numeric(data$rt9)
data$rt10 = as.numeric(data$rt10)
data$rt11 = as.numeric(data$rt11)
data$rt12 = as.numeric(data$rt12)
data$rt13 = as.numeric(data$rt13)
data$rt14 = as.numeric(data$rt14)
	\end{lstlisting}
	Statt
	\begin{lstlisting}
data$rt6 = as.numeric(data$rt6)
	\end{lstlisting}
	könnte/sollte man
	\begin{lstlisting}
data$rt6 = sapply(data$rt6, as.numeric)
	\end{lstlisting}
	nutzen, aber das erzeugt benannte Listen in den Spalten, statt den Typ auf \texttt{num} zu ändern. Deswegen habe ich das angepasst.
	
	\section*{Task 4}
	\begin{lstlisting}
subsetControl = subset(data, treatment == "control")
subsetLootPic = subset(data, treatment == "lootbox picture")
summary(data$tasks_completed)
summary(subsetControl$tasks_completed)
summary(subsetLootPic$tasks_completed)
	\end{lstlisting}
	Wir sehen, dass der Mittelwert über alle Daten bei 10.06 liegt, in der Kontrollgruppe bei 7.788 und in der Lootbox-Picture-Gruppe bei 8.807.
\end{document}