\documentclass{article}

\usepackage{amsmath,amssymb}
\usepackage{tikz}
\usepackage{pgfplots}
\usepackage{xcolor}
\usepackage[left=2.1cm,right=3.1cm,bottom=3cm,footskip=0.75cm,headsep=0.5cm]{geometry}
\usepackage{enumerate}
\usepackage{enumitem}
\usepackage{marvosym}
\usepackage{tabularx}
\usepackage{parskip}

\usepackage{listings}
\definecolor{lightlightgray}{rgb}{0.95,0.95,0.95}
\definecolor{lila}{rgb}{0.8,0,0.8}
\definecolor{mygray}{rgb}{0.5,0.5,0.5}
\definecolor{mygreen}{rgb}{0,0.8,0.26}
%\lstdefinestyle{java} {language=java}
\lstset{language=R,
	basicstyle=\ttfamily,
	keywordstyle=\color{lila},
	commentstyle=\color{lightgray},
	stringstyle=\color{mygreen}\ttfamily,
	backgroundcolor=\color{white},
	showstringspaces=false,
	numbers=left,
	numbersep=10pt,
	numberstyle=\color{mygray}\ttfamily,
	identifierstyle=\color{blue},
	xleftmargin=.1\textwidth, 
	%xrightmargin=.1\textwidth,
	escapechar=§,
	%literate={\t}{{\ }}1
	breaklines=true,
	postbreak=\mbox{\space}
}

\usepackage[utf8]{inputenc}

\renewcommand*{\arraystretch}{1.4}

\newcolumntype{L}[1]{>{\raggedright\arraybackslash}p{#1}}
\newcolumntype{R}[1]{>{\raggedleft\arraybackslash}p{#1}}
\newcolumntype{C}[1]{>{\centering\let\newline\\\arraybackslash\hspace{0pt}}m{#1}}

\newcommand{\E}{\mathbb{E}}
\DeclareMathOperator{\rk}{rk}
\DeclareMathOperator{\Var}{Var}
\DeclareMathOperator{\Cov}{Cov}

\title{\textbf{Applied Data Analysis, Übung 2}}
\author{\textsc{Henry Haustein}}
\date{}

\begin{document}
	\maketitle
	
	\section*{Task 1}
	\begin{lstlisting}
install.packages("dplyr")
library(dplyr)

# Bekannt aus Uebung 1
install.packages("readxl")
library(readxl)
data = read_excel("data.xlsx", na = "NA")
data$gender = factor(data$gender)
levels(data$gender) = c("male", "female", "diverse")
data$employment = factor(data$employment)
levels(data$employment) = c("student", "employed", "unemployed")
data$education = factor(data$education)
levels(data$education) = c("no degree", "secondary", "intermediate", "high school", "academic")
data$play_frequency = factor(data$play_frequency)
levels(data$play_frequency) = c("never", "every few months", "every few weeks", "1-2 days a week", "3-5 days a week", "daily")
data$treatment = factor(data$treatment)
levels(data$treatment) = c("control", "lootbox in task reward", "lootbox picture", "badge")
data$age = sapply(data$age, function(year) {2016-year})
data$rt6 = as.numeric(data$rt6)
data$rt7 = as.numeric(data$rt7)
data$rt8 = as.numeric(data$rt8)
data$rt9 = as.numeric(data$rt9)
data$rt10 = as.numeric(data$rt10)
data$rt11 = as.numeric(data$rt11)
data$rt12 = as.numeric(data$rt12)
data$rt13 = as.numeric(data$rt13)
data$rt14 = as.numeric(data$rt14)

data = data %>% filter(age <= 80) %>% filter(age >= 18)
data = data %>% filter(`total time` <= 2000000)
data = data %>% mutate(old = age > 25)
summary(data)

data %>% group_by(treatment) %>% summarise(completeTasks_mean = mean(tasks_completed), completeTasks_median = median(tasks_completed), completeTasks_var = sd(tasks_completed))

subsetControl = data %>% filter(treatment == "control")
subsetLootPic = data %>% filter(treatment == "lootbox picture")
subsetLootInTask = data %>% filter(treatment == "lootbox in task reward")
subsetBadge = data %>% filter(treatment == "badge")
subsetYoung = data %>% filter(old == FALSE)
subsetOld = data %>% filter(old == TRUE)
	\end{lstlisting}
	121 Personen sind alt und 268 sind jung. 
	
	Am meiste Aufgaben wurden in der Gruppe mit der Lootbox im task reward erfüllt, und die Gruppe mit der Badge hatte die größte Standardabweichung.

	\section*{Task 2}
	\begin{lstlisting}
install.packages("ggplot2")
library(ggplot2)

ggplot(data, aes(x = tasks_completed)) + geom_histogram(binwidth = 1)
ggplot(data, aes(x = tasks_completed, fill = treatment)) + geom_histogram(binwidth = 1)
ggplot(data, aes(x = tasks_completed)) + geom_histogram(binwidth = 1) + facet_wrap(~ treatment)

ggplot(data, aes(x = treatment, y = tasks_completed)) + geom_boxplot()
ggplot(data, aes(y = tasks_completed)) + geom_boxplot() + facet_wrap(~ treatment)

ggplot(data, aes(x = `total time`)) + geom_density()
ggplot(data, aes(x = `total time`, color = treatment)) + geom_density()
ggplot(data, aes(x = `total time`)) + geom_density() + facet_wrap(~ treatment)
	\end{lstlisting}
	
	\section*{Task 3}
	\begin{lstlisting}
t.test(data$`total time`, mu = 320000, alternative = "two.sided", conf.level = 0.95)
t.test(subsetYoung$`total time`, mu = 320000, alternative = "two.sided", conf.level = 0.95)
t.test(subsetOld$`total time`, mu = 320000, alternative = "two.sided", conf.level = 0.95)

t.test(subsetYoung$tasks_completed, mu = mean(subsetOld$tasks_completed), alternative = "two.sided", conf.level = 0.95)

pairwise.t.test(data$tasks_completed, data$treatment, p.adjust.method = "none", pool.sd = FALSE)
	\end{lstlisting}
	Bei allen Personen und den jungen Personen ist der p-value unter 5\%, das heißt wir können $H_0$ nicht ablehnen. Bei den alten Personen ist der p-value bei 50\%, wir lehnen $H_0$ ab.
	
	Bei allen andern Tests sind die p-values unter 5\%, auch da lehnen wir $H_0$ ab. Die Mittelwerte sind also verschieden voneinander.
	
\end{document}