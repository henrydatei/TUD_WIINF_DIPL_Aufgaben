\documentclass{article}

\usepackage{amsmath,amssymb}
\usepackage{tikz}
\usepackage{pgfplots}
\usepackage{xcolor}
\usepackage[left=2.1cm,right=3.1cm,bottom=3cm,footskip=0.75cm,headsep=0.5cm]{geometry}
\usepackage{enumerate}
\usepackage{enumitem}
\usepackage{marvosym}
\usepackage{tabularx}
\usepackage{hyperref}
\usepackage{longtable}
\usepackage{parskip}

\usepackage{listings}
\definecolor{lightlightgray}{rgb}{0.95,0.95,0.95}
\definecolor{lila}{rgb}{0.8,0,0.8}
\definecolor{mygray}{rgb}{0.5,0.5,0.5}
\definecolor{mygreen}{rgb}{0,0.8,0.26}
\lstdefinestyle{java} {language=java}
\lstset{language=java,
	basicstyle=\ttfamily,
	keywordstyle=\color{lila},
	commentstyle=\color{lightgray},
	stringstyle=\color{mygreen}\ttfamily,
	backgroundcolor=\color{white},
	showstringspaces=false,
	numbers=left,
	numbersep=10pt,
	numberstyle=\color{mygray}\ttfamily,
	identifierstyle=\color{blue},
	xleftmargin=.1\textwidth, 
	%xrightmargin=.1\textwidth,
	escapechar=§,
}

\usepackage[utf8]{inputenc}

\renewcommand*{\arraystretch}{1.4}

\newcolumntype{L}[1]{>{\raggedright\arraybackslash}p{#1}}
\newcolumntype{R}[1]{>{\raggedleft\arraybackslash}p{#1}}
\newcolumntype{C}[1]{>{\centering\let\newline\\\arraybackslash\hspace{0pt}}m{#1}}

\newcommand{\E}{\mathbb{E}}
\DeclareMathOperator{\rk}{rk}
\DeclareMathOperator{\Var}{Var}
\DeclareMathOperator{\Cov}{Cov}

\title{\textbf{AMCS-Fragen Rechnernetze}}
\author{\textsc{Henry Haustein}}
\date{}

\begin{document}
	\maketitle
	
	\section{Einführung}
	
	Welche der folgenden Protokolle sind verbindungsorientiert? 
	\begin{itemize}
		\item UDP (User Datagram Protocol)
		\item \textbf{TCP (Transmission Control Protocol)}
		\item \textbf{ISDN-Telefonnetz}
		\item IP (Internet Protocol)
	\end{itemize}

	Ordnen Sie die Schichten im ISO/OSI-Referenzmodell zu!
	\begin{enumerate}
		\item Bitübertragung
		\item Sicherung
		\item Vermittlung
		\item Transport
		\item Sitzung
		\item Darstellung
		\item Anwendung
	\end{enumerate}
	
	Welche Aufgabe werden von der Transportschicht übernommen?
	\begin{itemize}
		\item \textbf{Flußsteuerung zwischen Endsystemen}
		\item Komprimierung
		\item Wegewahl/Routing
		\item \textbf{Sichere Ende-zu-Ende-Kommunikation}
		\item \textbf{Multiplexing}
		\item Elektrische Kopplung
	\end{itemize}

	Ordnen Sie die Protokollaufrufe für einen Verbindungsaufbau nach ihrer Reihenfolge! 
	\begin{enumerate}
		\item ConReq
		\item ConInd
		\item ConRsp
		\item ConCnf
		\item DatReq
		\item DatInd
	\end{enumerate}

	Ein Dienst im Sinne des ISO/OSI-Referenzmodells ist ... 
	\begin{itemize}
		\item ... die horizontale Kommunikation zwischen zwei Prozessen auf zwei Hosts derselben Ebene. (virtuelle/gedachte Direktkommunikation)
		\item \textbf{... eine vertikale Kommunikation zwischen zwei Schichten auf einem Host.}
		\item  ... ist die Regel, die das Format und die Bedeutung der von den Partnereinheiten innerhalb einer Schicht ausgetauschten Rahmen, Pakete oder Nachrichten festlegt.  
	\end{itemize}

	Welche Netzstrukturen gehören zu den Broadcast-Topologien? 
	\begin{itemize}
		\item Stern
		\item \textbf{Ring}
		\item Baum
		\item \textbf{Bus}
		\item \textbf{Satellit}
	\end{itemize}

	Sie möchten in Ihrer Wohnung 2 Rechner und 2 mobile Geräte miteinander koppeln. Zu welcher Netzklasse würde dieses Netz zugeordnet?
	\begin{itemize}
		\item WAN
		\item PAN
		\item MAN
		\item \textbf{LAN}
		\item NFC
	\end{itemize}

	Ordnen Sie die verschiedenen Netztechnologien der passenden Netzklasse zu! 
	\begin{itemize}
		\item Ethernet - LAN
		\item Bluetooth - PAN
		\item DSL - MAN
		\item 5G - WAN
	\end{itemize}

	 Welcher Schicht werden Ethernet-Switches im Referenzmodell zugeordnet?
	 \begin{itemize}
	 	\item Schicht 1
	 	\item \textbf{Schicht 2}
	 	\item Schicht 3
	 	\item Schicht 4
	 \end{itemize}
 
 	Ordnen Sie die Schichten in das Internet-Modell ein! 
 	\begin{enumerate}
 		\item Sicherungsschicht
 		\item Internetschicht
 		\item Transportschicht
 		\item Answendungsschicht
 	\end{enumerate}
 
 	Welcher Schicht im Internetmodell sind die Protokolle TCP und UDP zugeordnet?
 	\begin{itemize}
 		\item Netzzugangsschicht
 		\item Internetschicht
 		\item \textbf{Transportschicht}
 		\item Anwendungsschicht
 	\end{itemize}
 
 	Wie erfolgt die Vermittlung mit dem Internet Protokoll (IP)?
 	\begin{itemize}
 		\item nachrichten-orientiert
 		\item verbindungs-orientiert 
 		\item \textbf{paket-orientiert}
 	\end{itemize}
	
	\section{Bitübertragungschicht}
	
	Wie nennt man die Einheit, die angibt, wieviele Signalschritte pro Sekunde kodiert werden?
	\begin{itemize}
		\item Bitrate
		\item Bandbreite
		\item \textbf{Baudrate}
		\item Schrittrate
	\end{itemize}

	Laut dem Nyquist-Theorem ... 
	\begin{itemize}
		\item ... steigt die Bitrate $b$ exponentiell mit der Abtastfrequenz $f_a$.
		\item  ... wird die Bitrate umso kleiner, je größer die Anzahl der Signalschritte ist. 
		\item \textbf{... ist die Bitrate $b$ immer kleiner als die doppelte Bandbreite, wenn wir binär (also mit 2 Signalstufen) abtasten.} 
	\end{itemize}

	Welche Aussagen treffen auf die Manchester-Kodierung zu? 
	\begin{itemize}
		\item ... erfordert, dass Sender und Empfänger gleich getaktet sind.
		\item \textbf{... wird durch Anwenden der XOR Operation auf das Taktsignal und den Datenstrom (NRZ-Signal) erzeugt.}
		\item \textbf{... ist ein selbsttaktender Code.}
		\item ... erzeugt keinen Overhead.
		\item \textbf{... "1" werden durch keine Pegeländerung kodiert.}
	\end{itemize}

	Bitte sortieren Sie die Übertragungsmedien entsprechend der verfügbaren Bandbreite. Beginnen Sie mit der geringsten Bandbreite. 
	\begin{enumerate}
		\item Twisted Pair
		\item Koaxialkabel
		\item Zellularfunk (GSM)
		\item Satellitenkommunikation
		\item Lichtwellenleiter
	\end{enumerate}

	In welcher Topologie sind die verschiedenen Ebenen der strukturierten Verkabelung jeweils verbunden? 
	\begin{itemize}
		\item Ring
		\item Binärbaum
		\item \textbf{Stern}
		\item Bus
	\end{itemize}

	Was bedeutet der Begriff \textit{Switched Medium}?
	\begin{itemize}
		\item Die Bandbreite des Übertragungskanals wird auf alle Teilnehmer gleichmäßig aufgeteilt. 
		\item Die Übertragung über das Medium wird je nach Bedarf an- und ausgeschaltet. 
		\item \textbf{Allen Teilnehmern wird abwechselnd (bei Bedarf) die gesamte Bandbreite des Übertragungsmediums exklusiv freigegeben. }
	\end{itemize}

	Welche Multiplexing-Verfahren werden bei GSM verwendet? 
	\begin{itemize}
		\item Codemultiplex
		\item \textbf{Frequenzmultiplex}
		\item \textbf{Zeitmultiplex}
		\item \textbf{Raummultiplex}
		\item Orthogonales Frequenzmultiplex
	\end{itemize}

	Bei welchen Übertragungstechnologien wird Orthogonales Frequenzmultiplex verwendet?
	\begin{itemize}
		\item GSM
		\item \textbf{5G}
		\item ISDN
		\item \textbf{Kabelnetze}
		\item \textbf{WiFi}
		\item Lichtwellenleiter
	\end{itemize}

	Welche Eigenschaften treffen auf digitale Signale zu? 
	\begin{itemize}
		\item \textbf{Wertdiskret}
		\item Zeitkontinuierlich
		\item \textbf{Zeitdiskret}
		\item Wertkontinuierlich
	\end{itemize}

	Wie groß muss die Abtastrate eines analogen Signals laut Shannon-Theorem mindestens sein?
	\begin{itemize}
		\item kleiner als die doppelte Grenzfrequenz des analogen Signals 
		\item mindestens halb so groß wie die Grenzfrequenz des analogen Signals
		\item größer als die dreifache Grenzfrequenz des analogen Signals
		\item \textbf{größer als die doppelte Grenzfrequenz des analogen Signals }
	\end{itemize}

	Welche Schritte sind notwendig, um ein digitales Signal über einen analogen Kanal zu übertragen? Ordnen Sie die Schritte nach ihrer Reihenfolge.
	\begin{enumerate}
		\item Digitale Modulation
		\item Analoge Modulation
		\item Empfang des modulierten Signals
		\item Analoge Demodulation
		\item Digitale Demodulation
		\item Synchronisation auf Sender
	\end{enumerate}

	Orden Sie die Modulationsarten richtig zu! 
	\begin{itemize}
		\item Phasenmodulation: Bei der binären "0" wird ein Signal mit einer Phase von 0$^\circ$ ausgesendet, bei "1" ein Signal mit der Phase 180$^\circ$
		\item Frequenzmodulation: Bei der binären "0" wird eine niedrige Frequenz $f_0$ ausgesendet, bei "1" die höhere Frequenz $f_1$
		\item Amplitudenmodulation: Bei der binären "0" wird eine niedrige Amplitude übertragen, bei "1" eine hohe Amplitude. 
	\end{itemize}
	
	\section{Netztechnologien I}
	
	Wie wird bei Slotted ALOHA eine Verbesserung der Ausnutzung des Kanaldurchsatzes erreicht? 
	\begin{itemize}
		\item durch die Nutzung von Signaturen
		\item \textbf{durch die Einführung von Zeitslots}
		\item durch die Reservierung von Kanälen
		\item durch das gleichzeitige Versenden mehrerer Pakete 
	\end{itemize}

	Ordnen Sie die unterschiedlichen CSMA-und ALOHA-Verfahren nach Ihrer Kollisionswahrscheinlichkeit! Beginnen Sie mit dem Verfahren mit der geringsten Wahrscheinlichkeit! 
	\begin{enumerate}
		\item Nonpersistant CSMA
		\item 0,5-persistant CSMA 
		\item 1-persistant CSMA
		\item slotted ALOHA
		\item ALOHA
	\end{enumerate}

	Welche Aussagen treffen auf das CSMA/CD-Verfahren zu? 
	\begin{itemize}
		\item Ein zentraler Knoten überwacht das Medium und regelt den Zugriff.
		\item \textbf{Viele Teilnehmer teilen sich ein Übertragungsmedium (Shared Medium)}
		\item Kollisionen werden nicht erkannt.
		\item Jeder Teilnehmer kann auf einem exklusiven Kanal mit der Zentrale kommunizieren.
		\item \textbf{Alle Teilnehmer überwachen das Medium.}
		\item \textbf{Kollisionen werden erkannt und ein JAM-Signal gesendet.}
		\item Wenn eine Kollision erkannt wurde, senden alle Teilnehmer direkt noch einmal.
	\end{itemize}

	Welches Verfahren wurde beim klassischen Ethernet verwendet? 
	\begin{itemize}
		\item Slotted ALOHA
		\item Nonpersistant CSMA
		\item 0,5-persistant CSMA
		\item \textbf{1-persistant CSMA}
		\item Pure ALOHA
	\end{itemize}

	Warum sind Kolisionsvermeidungsverfahren im heutigen Ethernet nicht mehr dringend notwendig?
	\begin{itemize}
		\item \textbf{Weil durch die Sterntopologie jeder Kanal nur durch einen Rechner exklusiv genutzt wird.}
		\item \textbf{Durch die Verwendung von Switches und Duplexleitungen treten so gut wie keine Kollisionen mehr auf.}
		\item Weil eine vollvermaschte Topologie verwendet wird und somit Punkt-zu-Punkt-Verbindungen entstehen.
		\item Weil die Segmentlänge kleiner als 100 m gewählt wird.
	\end{itemize}

	Mit wie vielen Bit wird die Zieladresse im Ethernet-Frame kodiert? 
	\begin{itemize}
		\item 6 Bit
		\item 32 Bit
		\item \textbf{48 Bit}
		\item 64 Bit
		\item 128 Bit
	\end{itemize}

	Auf welcher Schicht im OSI/ISO-Referenzmodell werden Switches eingeordnet?
	\begin{itemize}
		\item Schicht 1
		\item \textbf{Schicht 2}
		\item Schicht 3
		\item Schicht 4
		\item Schicht 5
		\item Schicht 6
		\item Schicht 7
	\end{itemize}

	Welche Funktionen werden durch einen Switch umgesetzt?
	\begin{itemize}
		\item \textbf{Anpassung der Rahmenlängen zwischen verschiedenen Ethernet-Standards}
		\item \textbf{Selektive Datenweiterleitung}
		\item Ermitteln der besten Route durch das Internet
		\item Kolissionserkennung
		\item \textbf{Pufferung der Daten bei Kopplung unterschiedl. Ethernet-Standards}
	\end{itemize}

	Welche Aussagen treffen auf Cut-Through-Switches zu? 
	\begin{itemize}
		\item Sie prüfen erst die Prüfsumme bevor das Datenpaket weitergesendet wird.
		\item Sie können zwischen verschiedenen Ethernet-Standards Formatanpassungen vornehmen.
		\item \textbf{Die Daten werden ohne Zwischenspeicherung/Pufferung weitergeleitet.}
		\item \textbf{Es können nur Kanäle mit gleicher Datenrate gekoppelt werden.}
		\item Die Datenpakete werden komplett im Switch zwischen gepuffert und dann weitergeleitet.
	\end{itemize}

	Wie viele Kabelverbindungen zwischen 2 Switches sind notwendig, wenn man Datenpakete in 3 verschiedenen virtuellen LANs mit taggedVLAN verteilen möchte? 
	\begin{itemize}
		\item 3 Leitungen, weil jedes VLAN über einen getrennten Port verbunden werden muss.
		\item 2 Leitungen für den Vollduplexbetrieb.
		\item \textbf{1 Leitung, da die Verteilung auf die VLANs per VLAN-ID/Tag im Switch vorgenommen wird.} 
	\end{itemize}

	Ordnen Sie die verschiedenen Funktechnologien nach der Reichweite der Netzwerkverbindung. Beginnen Sie mit der kürzesten Reichweite! 
	\begin{enumerate}
		\item NFC
		\item Bluetooth
		\item ZigBee
		\item WiFi
		\item WiMax
		\item LTE
	\end{enumerate}

	Welches MultipleAccess-Verfahren wird beim WLAN-Standard IEEE 802.11 verwendet? 
	\begin{itemize}
		\item CDMA
		\item CSMA/CD
		\item \textbf{CSMA/CA}
		\item Slotted ALOHA
		\item Pure ALOHA
	\end{itemize}

	Ein WLAN Interface Controller z.B. in einem Smartphone arbeitet ... 
	\begin{itemize}
		\item ... nur im Infrastruktur-Mode.
		\item ... nur im Adhoc-Mode.
		\item \textbf{... sowohl im Infrastruktur als auch im Adhoc-Mode. }
		\item ... WDS mode.
	\end{itemize}

	In welcher Topologie sind die ZigBee-Router (ZR) mit den Endgeräten (ZED) verbunden? 
	\begin{itemize}
		\item Ring
		\item Bus
		\item \textbf{Stern}
		\item vollvermascht, P2P
	\end{itemize}

	Bei der drahtlosen Bezahlung mit einem Smartphone erfolgt die Übertragung der Daten mit ...
	\begin{itemize}
		\item ... einem passiven RFID-Tag 
		\item \textbf{... einem aktiven NFC-Transmitter}
		\item ... mit einer adhoc WLAN-Verbindung 
		\item ... mit Bluetooth 
	\end{itemize}
	
	\section{Netztechnologien II}
	
	Welche Art von Multiplexing wird bei WiMax genutzt?
	\begin{itemize}
		\item \textbf{time division multiplexing}
		\item space division multiplexing
		\item \textbf{orthogonal frequency division multiplexing}
		\item channel division multiplexing
	\end{itemize}

	Was bedeutet MIMO im Zusammenhang mit WiMax? 
	\begin{itemize}
		\item Nutzung einer Sende- und mehrerer Empfangsantennen pro Gerät
		\item \textbf{Nutzung mehrerer Sende- und Empfangsantennen pro Gerät}
		\item Nutzung mehrerer Sende- und einer Empfangsantenne pro Gerät 
		\item Nutzung einer Antenne pro Gerät, die sowohl Senden als auch Empfangen kann. 
	\end{itemize}

	Für welche Größe von Netzen wird WiMax verwendet? 
	\begin{itemize}
		\item PAN
		\item LAN
		\item WAN
		\item \textbf{MAN}
		\item NFC
	\end{itemize}

	Welcher IEEE Standard gehört zu Resilient Packet Rings?
	\begin{itemize}
		\item 802.3
		\item 802.5
		\item 802.11
		\item 802.16
		\item \textbf{802.17}
	\end{itemize}

	Über welche Topology sind Resilient Packet Rings verbunden? 
	\begin{itemize}
		\item Dual Kupferkabel 
		\item \textbf{Dual Glasfaser }
		\item Wireless
		\item Laser
	\end{itemize}

	Wie schnell kann im RPR ein Fehler entdeckt werden? (z.B. wenn eine Leitung beschädigt wird)
	\begin{itemize}
		\item 10 s
		\item 100 ms
		\item \textbf{50 ms}
		\item 50 $\mu$s
	\end{itemize}

	In welcher Topologie sind die Stationen und Switches beim WAN/MAN-Ethernet verbunden?
	\begin{itemize}
		\item Ring
		\item \textbf{Duplex P2P}
		\item Bus
		\item half-duplex P2P
	\end{itemize}

	Wofür werden im Carrier Ethernet sogenannte Ethernet Virtual Connections erstellt?
	\begin{itemize}
		\item exklusive Reservierung einer Verbindung zwischen 2 Switches
		\item \textbf{Zusicherung einer Dienstgüte für eine konkrete Verbindung über mehrere Switches.}
		\item Verschlüsselung der Kommunikation zwischen 2 Stationen im Ethernet
		\item Nutzung anderer Netzwerktechnologien innerhalb des Ethernet 
	\end{itemize}
	
	Wie werden bei MPLS die Label vergeben? 
	\begin{itemize}
		\item Jede Verbindung zwischen zwei Punkten hat ein eindeutiges Label. Ein Pfad, der eine bestimmte Verbindung verwendet, nimmt auf dieser ihr Label an.
		\item Jeder Pfad hat ein eindeutiges Label. Jede Punkt-zu-Punkt-Verbindung nimmt für Pakete dieses Pfades das gegebene Label an. 
		\item \textbf{Für jede Ende-zu-Ende-Kommunikation wird zwischen zwei Vermittlungsstellen ein für all ihre Ports eindeutiges Label vergeben. }
	\end{itemize}

	Was bedeuten Forward Equivalence Classes?
	\begin{itemize}
		\item bestimmte Pakete werden bevorzugt behandelt. 
		\item \textbf{gleiche Behandlung aller Pakete eines Datenstroms }
		\item die verschiedenen Ethernet-Standards sind vorwärts kompatibel.
		\item Die Datenpakete des Vorgängers werden mit höherer Priorität weitergesendet.
	\end{itemize}

	Wie groß ist die Basisdatenrate beim SSONET (STS-1)? 
	\begin{itemize}
		\item 51,84 kBit/s 
		\item 100 Mbit/s
		\item \textbf{51,84 MBit/s}
		\item 10 Mbit/s
		\item 50 GBit/s
	\end{itemize}

	Sortieren Sie die verschiedenen Multiplexing-Verfahren für SSONET nach der zu erreichenden Datenrate. Beginnen Sie mit dem Verfahren, dass die höchste Datenrate ermöglicht?
	\begin{enumerate}
		\item DWDM - Dense Wavelength Division Multiplexing
		\item CWDM - Coarse Wavelength Division Multiplexing 
		\item WDM - Wavelength Division Multiplexing
	\end{enumerate}
	
	\section{Sicherungsschicht}
	
	Was gehört zu den Aufgaben der Sicherungsschicht? 
	\begin{itemize}
		\item Routing der Pakete
		\item \textbf{Bildung von Übertragungsrahmen}
		\item \textbf{Fehlerbehandlung, Kontrolle von Prüfsummen}
		\item Ende-zu-Ende-Sicherung von Nachrichten
		\item \textbf{Flusskontrolle zur Überlastvermeidung}
		\item \textbf{Interface für die Vermittlungsschicht}
		\item Modulation der Daten auf physikalischen Übertragungskanal
		\item Schnittstelle für die Anwendungsschicht
	\end{itemize}

	Wie groß ist die Anzahl korrigierbarer Fehler, wenn die Hamming-Distanz $d=5$ ist?
	\begin{itemize}
		\item 1
		\item \textbf{2}
		\item 3
		\item 4
		\item 5
	\end{itemize}

	Welche der folgenden Codes können Fehler nur erkennen jedoch nicht korrigieren? 
	\begin{itemize}
		\item \textbf{CRC-32}
		\item \textbf{eindimensionale Paritätsbits}
		\item Hamming-Code
		\item Reed-Solomon-Code
	\end{itemize}

	Wie lange muss der Sender mindestens warten, bis er den nächsten Rahmen senden kann? 
	\begin{itemize}
		\item Halbe Round-Trip-Time
		\item Doppelte Round-Trip-Zeit 
		\item \textbf{Einfache Round-Trip-Zeit }
		\item Ablauf des Timeouts 
	\end{itemize}

	Bitte ordnen Sie die jeweiligen Funktionen/Eigenschaften des ARQ den Übertragungsproblemen zu, die sie lösen. 
	\begin{itemize}
		\item verspätetes Acknowledge: Sequenznummer des ACK, ermöglicht Zuordnung zum Frame
		\item verlorene Daten: automatische Sendewiederholung nach Timeout 
		\item verlorene Quittung: Sequenznummer in den Daten zur Duplikaterkennung 
	\end{itemize}

	Ordnen Sie die entsprechenden Dienstprimitive den Zustandsübergängen zu!
	\begin{itemize}
		\item DatReq, DatInd: Verbindung aufgebaut - Verbindung aufgebaut
		\item DisReq, DisInd: Verbindung aufgebaut - Ruhezustand
		\item ConReq, ConInd: Ruhezustand - Verbindung im Aufbau 
		\item ConRsp, ConCnf: Verbindung im Aufbau - Verbindung aufgebaut 
	\end{itemize}

	HDLC ist ein .... Protokoll. 
	\begin{itemize}
		\item byte-orientiertes
		\item verbindungs-orientiertes
		\item \textbf{bit-orientiertes}
		\item paket-orientiertes 
		\item service-orientiertes
	\end{itemize}

	Bitte ordnen Sie die Bestandteile eines Rahmens des HDLC-Protokolls in die richtige Reihenfolge!
	\begin{enumerate}
		\item Opening Flag 
		\item Adresse 
		\item Steuerung
		\item Nutzdaten
		\item Prüfsumme
		\item Closing Flag
	\end{enumerate}

	In welchen der folgenden Netze kommt das Point-to-Point-Protocol zum Einsatz?
	\begin{itemize}
		\item \textbf{Kabelmodem}
		\item LTE
		\item \textbf{xDSL}
		\item \textbf{GPRS/UMTS}
		\item ATM
		\item NFC
	\end{itemize}
	
	\section{Vermittlungsschicht}
	
	Welche Aufgaben werden durch die Vermittlungsschicht erfüllt?
	\begin{itemize}
		\item Bildung von Übertragungsrahmen
		\item \textbf{Routing der Pakete}
		\item \textbf{Einheitliche Adressierung}
		\item Ende-zu-Ende-Sicherung von Nachrichten
		\item Modulation der Daten auf physikalischen Übertragungskanal
		\item \textbf{Anpassung von Formaten und Adressen}
	\end{itemize}

	Ordnen Sie die Protokolle der Art der Wegewahl zu! 
	\begin{itemize}
		\item Global: OSPF (Open Shortest Path First) 
		\item Lokal: HotPotato
		\item Verteilt: RIP (Routing Information Protocol) 
	\end{itemize}

	 Welcher Algorithmus wird in OSPF verwendet, um den kürzesten Pfad zu bestimmen? 
	 \begin{itemize}
	 	\item Bellman-Ford Algorithmus
	 	\item Distance Vector Routing 
	 	\item Routing Information Protocol 
	 	\item \textbf{Dijkstra’s Algorithmus}
	 \end{itemize}
 
 	Wie heißen die Pakete, die bei der Überlastung eines Kanals an den Sender zurückgesendet werden?
 	\begin{itemize}
 		\item Joke-Pakete
 		\item \textbf{Choke-Pakete }
 		\item Choice-Pakete
 		\item Overload-Pakete
 		\item Stop-Pakete
 	\end{itemize}
 
 	Wie werden bei ECN (Explicit Congestion Notification) die Pakete markiert, um dem Sender eine Überlastsituation anzuzeigen?
 	\begin{itemize}
 		\item Byte im Paket-Header
 		\item Flag am Ende des Paketes
 		\item \textbf{Bit im Paket-Header}
 		\item Paket mit leerem Payload
 	\end{itemize}
	
	Was passiert beim LeakyBucket-Verfahren? 
	\begin{itemize}
		\item Daten werden gepuffert, um sie dann später im Bulk auszuliefern.
		\item \textbf{Daten mit unterschiedlicher Datenrate werden gepuffert, um sie dann an einen Empfänger mit konstanter Datenrate ausliefern zu können.}
		\item Daten werden nur solange gepuffert, bis der Puffer (Eimer) überläuft, dann werden sie verworfen.
		\item Multimediadaten werden synchronisiert, um den Jitter zu verringern.
	\end{itemize}

	Wie groß ist der IPv4-Adressraum? 
	\begin{itemize}
		\item 16 Bit
		\item \textbf{32 Bit}
		\item 64 Bit
		\item 128 Bit
		\item 256 Bit
	\end{itemize}

	Wie groß ist der IPv6-Adressraum? 
	\begin{itemize}
		\item 16 Bit
		\item 32 Bit
		\item 64 Bit
		\item \textbf{128 Bit}
		\item 256 Bit
	\end{itemize}

	Welche der folgenden IPv4-Adressen gehören zu privaten Netzwerken? 
	\begin{itemize}
		\item \textbf{10.0.0.2}
		\item 141.76.40.1
		\item \textbf{192.168.54.3}
		\item \textbf{192.168.0.7}
		\item 214.15.23.1
		\item 188.1.144.217
	\end{itemize}

	Welche Eigenschaften treffen auf das OSPF-Protokoll zu? 
	\begin{itemize}
		\item statisches Routing-Verfahren
		\item \textbf{unterstützt Intra-Domain-Routing}
		\item \textbf{Berechnung des kürzesten Pfades mit Dijkstra-Algorithmus}
		\item \textbf{dynamisches Routing-Verfahren}
		\item unterstützt Inter-Domain-Routing
		\item \textbf{ist ein Link State Protocol}
		\item verwendet Distance Vector Routing
	\end{itemize}

	Wieviele Adressen (gerundet) könnte man theoretisch mit den 128 bit langen Adressen in IPv6 vergeben?
	\begin{itemize}
		\item 4,3 Milliarden 
		\item 300 Milliarden 
		\item 640 Trilliarden
		\item \textbf{340 Sextillionen }
		\item 600 Billiarden
	\end{itemize}

	Ordnen Sie die Adressarten den passenden Eigenschaften zu! 
	\begin{itemize}
		\item Unicast: Adressen für ein einzelnes Interface 
		\item Broadcast: Existieren nicht und wird mit Multicast-Adressen realisiert. 
		\item Anycast: Adressen für mehrere Interfaces, wobei nur eines davon das Paket empfängt. 
		\item Multicast: Adressen für mehrere Interfaces, die alle das selbe Paket empfangen. 
	\end{itemize}

	Mit IPv6 besteht die Möglichkeit, dass ein Interface sich selbst eine link-lokale Adresse zuweist und auf dieser Basis auch eine globale Unicast-Adresse erzeugen und sich selbst zuweisen kann. Wie lautet die link-lokale IPv6-Adresse zur MAC-Adresse 24:48:3F:8E:F1:26?
	\begin{itemize}
		\item fe80::2448:3FFE:FF8E:F126
		\item fe80:2448:3FFF::FE8E:F126 
		\item fe80::2648:3FFE:FF8E:F126
		\item \textbf{fe80::2648:3FFF:FE8E:F126 }
		\item fe80::2448:3FFF:FE8E:F126
		\item fe80:2648:3FFF::FE8E:F126 
	\end{itemize} 

	In IPSec wird für die Verschlüsselung folgendes Verfahren verwendet? 
	\begin{itemize}
		\item RSA
		\item Caesar-Chiffre 
		\item \textbf{AES}
		\item One-Time-Pad 
		\item DES
		\item Twofish
	\end{itemize}

	Bringen Sie die DHCP-Protokollnachrichten in die richtige Reihenfolge! 
	\begin{enumerate}
		\item DHCP DISCOVER 
		\item DHCP OFFER 
		\item DHCP REQUEST 
		\item DHCP ACK 
	\end{enumerate}
	
	\section{Transportschicht}
	
	Welche Aufgaben werden von der Transportschicht übernommen? 
	\begin{itemize}
		\item \textbf{Ende-zu-Ende-Fehlerbehandlung}
		\item Sicherung der Punkt-zu-Punkt-Verbindung
		\item Wegewahl durch das Netz
		\item \textbf{Aushandlung von Dienstgüte-Eigenschaften}
		\item \textbf{Ende-zu-Ende-Flusskontrolle}
		\item Modulation der Daten auf ein physikalisches Übertragungsmedium
		\item \textbf{Überlaststeuerung/Congestion Control}
	\end{itemize}

	Was versteht man unter einem Port?
	\begin{itemize}
		\item \textbf{Transport Service Access Point}
		\item Network Service Access Point
		\item IP-Schnittstelle
		\item \textbf{standardisiert für verschiedene Anwendungsdienste}
		\item bildet einen Host im Netzwerk ab
		\item \textbf{bildet einen Prozess auf einem Host ab}
	\end{itemize}

	Ordnen Sie die Ports den einzelnen Diensten zu (entsprechend IANA -Standard). 
	\begin{itemize}
		\item 80: HTTP
		\item 443: HTTPS
		\item 53: Domain Name System (DNS) 
		\item 23: Telnet
		\item 25: Simple Mail Transfer Protocol (SMTP) 
		\item 110: POP-3
		\item 143: IMAP
	\end{itemize}

	Wie groß sollen die Laufnummern beim Schiebefensterprotokoll mindestens sein, damit Fehler in der Übertragung und Duplikate erkannt werden können?
	\begin{itemize}
		\item 0 bis $F-1$
		\item \textbf{0 bis $F$}
		\item 0 bis $F+1$
		\item 0 bis $2\cdot F$
	\end{itemize}

	Was passiert bei einem Schiebefensterprotokoll der Fenstergröße 4, wenn der 2. Frame verlorengeht? Zur Fehlerbehandlung wird Go-back-n verwendet.
	\begin{itemize}
		\item Da keine Bestätigung für Frame 2 eintrifft, wiederholt der Sender sofort den Frame 2, bevor er Frame 3 und 4 sendet.
		\item \textbf{Der 3. und 4. Frame werden beim Empfänger verworfen, der Sender wiederholt nach Timeout die Frames 2, 3 und 4, der Empfänger bestätigt jeweils die Ankunft} 
		\item Der 3. und 4. Frame werden beim Empfänger gepuffert, der Sender wiederholt nach Timeout den Frame 2, der Empfänger bestätigt in einer Sammelquittung, dass 2, 3 und 4 angekommen sind. 
		\item Der Sender verschickt alle Rahmen gemeinsam und wartet auf die Bestätigung für alle Pakete, nach Timeout wiederholt er alle Pakete noch einmal.
	\end{itemize}

	Welche der folgenden Eigenschaften sind TCP zuzuordnen? 
	\begin{itemize}
		\item verbindungsloses Protokoll
		\item \textbf{verbindungsorientiertes Protokoll}
		\item geeignet für Echtzeit-Multimedia-Anwendungen
		\item \textbf{Reihenfolgegarantie}
		\item \textbf{Fenster-basierte Flusskontrolle}
		\item Datagram-basierte Kommunikation
		\item \textbf{Congestion Control}
	\end{itemize}

	Wie groß ist der UDP-Header, wenn das UDP-Segment 64kByte groß ist? 
	\begin{itemize}
		\item 8 bit
		\item 2 Byte
		\item \textbf{8 Byte}
		\item 16 Byte
		\item 24 Byte
		\item 32 Byte
	\end{itemize}

	UDP ist ein sogenannter "Best-Effort"-Dienst. Das bedeutet, dass ... 
	\begin{itemize}
		\item Segmente im Fehlerfall wiederholt werden.
		\item \textbf{Segmente verloren gehen können.}
		\item \textbf{Segmente in der falschen Reihenfolge ankommen können.}
		\item dass die richtige Reihenfolge der Segmente garantiert wird.
		\item dass alle Segmente einer Verbindung gemeinsam als Strom behandelt werden.
		\item \textbf{dass jedes Segment unabhängig von anderen behandelt wird.}
		\item \textbf{dass keine Staukontrolle stattfindet.}
	\end{itemize}

	Was geschieht beim TCP Slow Start, wenn keine Stausituation auftritt? 
	\begin{itemize}
		\item Die Fenstergröße ist zu Beginn sehr klein, sie wird bis zur max. Fenstergröße exponentiell erhöht. 
		\item Die anfänglich sehr kleine Fenstergröße wird bis zu einem Schwellwert linear erhöht und dann bis zur maximalen Fenstergröße exponentiell.
		\item \textbf{Die Fenstergröße wird bis zum Schwellwert exponentiell erhöht, dann bis zur maximalen Fenstergröße linear. }
		\item Die Fenstergröße beginnt mit einem maximalen Wert und wird dann bei Problemen linear verringert. 
		\item Die Fenstergröße wird bis zur maximalen Fenstergröße exponetiell erhöht und steigt danach weiter linear an, bis es zu Stausituationen kommt. 
	\end{itemize}

	 Was geschieht beim TCP Slow Start, wenn eine Stausituation auftritt?
	 \begin{itemize}
	 	\item \textbf{Sendefenster wird nach Timeout auf 1 gesetzt}
	 	\item Sendefenster wird halbiert
	 	\item Schwellwert wird verdoppelt
	 	\item \textbf{Schwellwert wird halbiert}
	 	\item \textbf{Timeout wird verdoppelt}
	 	\item Timeout wird halbiert
	 	\item Timeout wird nicht erhöht
	 \end{itemize}
 
 	Welche Eigenschaften beschreiben das Verhalten bei Fast Recovery?
 	\begin{itemize}
 		\item \textbf{Wiederholung der Übertragung nach Erhalt dreier gleicher ACKs}
 		\item Timeout wird verdoppelt
 		\item Sendefenster wird nach Timeout auf 1 gesetzt
 		\item \textbf{Sendefenster wird halbiert }
 		\item Schwellwert wird halbiert
 		\item \textbf{keine Erhöhung des Timeouts}
 	\end{itemize}
 
 	Ordnen Sie die Filtermöglichkeiten den entsprechenden Firewall-Funktionen zu!
 	\begin{itemize}
 		\item Protokolle via TCP-Ports: Circuit Relay, Schicht 4 
 		\item anwendungsbezogene Authentisierung: Application Gateway, Schicht 7 
 		\item IP-Quell-/Zieladressen: Paketfilter, Schicht 3 
 		\item Domain Names (Quelle/Ziel): DNS, Schicht 7 
 	\end{itemize}
	
	\section{Netzwerkperformance}
	
	Welche Performance-Probleme können auf Schicht 4 durch Nutzung von TCP entstehen?
	\begin{itemize}
		\item Anpassung der Übertragungsgeschwindigkeit zwischen unterschiedlichen Ethernet-Varianten
		\item \textbf{starke Drosselung der Geschwindigkeit durch Slow Start bei Paketverlusten}
		\item Ineffiziente Wegewahlverfahren
		\item \textbf{Wiederholte Übertragung verloren gegangener Pakete verzögert Übertragung}
		\item \textbf{unnötige Verzögerung bei Paketverlusten, weil Timeout zu groß gewählt}
		\item Fragmentierung und Reassemblierung von Paketen
	\end{itemize}

	Wie kann Jitter ausgeglichen werden?
	\begin{itemize}
		\item Drosselung der Sendegeschwindigkeit, dann gehen weniger Pakete verloren
		\item Vergrößerung der Fenstergröße bei TCP
		\item \textbf{Durch TCP-Sequenznummern kann richtige Reihenfolge der Pakete wiederhergestellt werden.}
		\item \textbf{Pufferung auf Empfängerseite, dann verzögerte Wiedergabe mit gleichmäßiger Geschwindigkeit}
		\item Pufferung auf Senderseite, um Pakete synchron zu verschicken
	\end{itemize}

	Wie groß ist (rechnerisch betrachtet) die minimale Framelänge $F$ im Gigabit Ethernet?
	\begin{itemize}
		\item 128 Bit
		\item 200 Bit
		\item 250 Bit
		\item \textbf{2000 Bit}
	\end{itemize}

	Um welches Vielfaches erhöht sich die Framegröße von Jumbo-Frames gegenüber klassischen Ethernet-Frames? 
	\begin{itemize}
		\item das Doppelte 
		\item das Dreifache 
		\item das Vierfache 
		\item \textbf{das 6-fache }
		\item das 10-fache 
	\end{itemize}

	Durch welche Einflussfaktoren kann die TCP-Datenrate sinken? 
	\begin{itemize}
		\item durch die Verringerung der Round Trip Time
		\item \textbf{durch die Vergrößerung der Round Trip Time}
		\item durch die Verringerung der Entfernung zwischen Senden und Empfänger
		\item \textbf{durch die Vergrößerung der Entfernung zwischen Sender und Empfänger}
		\item durch die Verringerung der Paketverlustrate
		\item \textbf{durch die Erhöhung der Paketverlustrate}
		\item \textbf{durch die Verringerung der maximalen Segmentgröße}
		\item durch die Erhöhung der maximalen Segmentgröße
	\end{itemize}

	Welcher Fairness-Algorithmus ermöglicht eine nahezu gleichmäßige Übertragungszeit im gesamten Netz? 
	\begin{itemize}
		\item FirstCome - FirstServe
		\item Round-Robin Fair-Queueing
		\item Round-Robin Weighted-Queueing 
		\item \textbf{Scheduling anhand von Paketzeitstempeln}
		\item Prioritätsbasiertes Scheduling
	\end{itemize}

	Welche Fairness-Algorithmen könnn Sie verwenden, wenn Sie Pakten einer Videokonferenzübertragung den Vorrang bei der Übertragung einräumen wollen? 
	\begin{itemize}
		\item Scheduling anhand von Paketzeitstempeln
		\item \textbf{Prioritätsscheduling}
		\item Round-Robin Fair-Queuing
		\item \textbf{Weighted Round-Robin}
		\item First In - First Out
	\end{itemize}

	Bei welcher Auslastung der Kanalkapazität sollte man die Übertragung bereits drosseln? 
	\begin{itemize}
		\item bei 50 \% 
		\item \textbf{bei 70 - 80 \% }
		\item bei 90 \%
		\item erst bei voller Auslastung von 100 \%
	\end{itemize}

	Ordnen Sie den TCP Varianten die richtigen Verfahren zu! 
	\begin{itemize}
		\item TCP klassisch: Paketverlust führt zu "Neustart" (Slow Start) der Senderate, dann exponentielle Erhöhung bis Schwellwert 
		\item TCP-Reno - Fast Recovery: bei Empfang von 3 Dup-Acks wird Congestion Window halbiert 
		\item Fast-Retransmit: Sender erkennt doppelte Acks und sendet fehlendes Paket vor Ablauf des Timers für Paketverlust
		\item FAST TCP: Paketanzahl in der Warteschlange wird gemessen, viele Pakete in der Warteschlange, Senderate wird proaktiv reduziert 
	\end{itemize}

	Welche Anpassungen am Host-System können zu Leistungssteigerungen führen?
	\begin{itemize}
		\item \textbf{Übertragung von Großen Jumbo-Frames}
		\item Verkürzung der Timeouts beim TCP.
		\item \textbf{Zusammenfassen der Schichten (Integrated Layer Processing)}
		\item \textbf{Netzmanager in Betriebssystemkern verlegen}
		\item \textbf{Prüfsummenberechnung während des Kopiervorgangs}
		\item Verkürzung der Warteschlangen
	\end{itemize}
	
	Auf wieviele Byte wird der TCP/IP-Header bei Robust Header Compression reduziert?
	\begin{itemize}
		\item 2 Byte
		\item \textbf{3 Byte}
		\item 5 Byte
		\item 16 Byte
		\item 32 Byte
	\end{itemize}
	
	\section{Internetdienste}
	
	Was versteht man unter einem "autoritativem Datensatz" in DNS?
	\begin{itemize}
		\item ist ein Ressourcendatensatz der im Cache zwischengespeichert wird 
		\item \textbf{stammt von der Stelle, die den konkreten Domain-Namen verwaltet }
		\item wird von der root-Domain verwaltet 
		\item ist ein Datensatz, der vom primären Name-Server einer übergeordneten Zone stammt. 
	\end{itemize}

	Eine DNS-Anfrage nach dem Host www.inf.tu-dresden.de wird von einem Client innerhalb der Domäne computer.com abgesendet. Ordnen Sie, in welcher Reihenfolge welche DNS-Server angefragt werden, wenn das iterative Vorgehen (referral) genutzt wird und alle DNS-Caches leer sind.
	\begin{enumerate}
		\item dns.computer.com 
		\item Root-Nameserver 
		\item .de Nameserver 
		\item tu-dresden.de 
		\item inf.tu-dresden.de 
	\end{enumerate}

	Wie viele Domänen umfasst eine DNS-Zone? 
	\begin{itemize}
		\item Genau einen vollständig qualifiziert Domänenname (FQDN), bspw. www.rn.inf.tu-dresden.de
		\item Genau eine Subdomäne, bspw. rn.inf.tu-dresden.de. Weitere Subdomänen dieser Subdomäne gehören zur selben Zone. 
		\item \textbf{Die Zone ist unabhängig von der Domäne und kann daher verschiedene/beliebig viele Domänen umfassen, die eine Autorität (Organisation/Administration/Verwaltung) zugehören. Subdomänen einer Domäne können zur selben Zone gehören, müssen es aber nicht.}
		\item Alle Domänen unterhalb der Top-Level-Domäne, bspw. *.de 
	\end{itemize}

	Welche Protokolle werden für den Abruf der Emails vom Server durch einen Client verwendet? 
	\begin{itemize}
		\item SMTP
		\item \textbf{IMAP}
		\item MIME
		\item HTTP
		\item \textbf{POP3}
	\end{itemize}

	Ordnen Sie die Standardports den jeweiligen Protokollen zu! 
	\begin{itemize}
		\item 80: HTTP
		\item 443: HTTPS
		\item 53: DNS
		\item 23: Telnet
		\item 25: SMTP
		\item 110: POP
		\item 143: IMAP
	\end{itemize}

	Sie wollen eine Email an rektor@tu-dresden.de senden. Woher weiß der SMTP-Server, an welchen Server er die Nachricht senden soll?
	\begin{itemize}
		\item Er fragt per Broadcast alle Hosts der Domäne an, welcher Server SMTP-Nachrichten empfangen kann 
		\item Der SMTP-Server führt eine Liste ihm bekannter Adressen und Server. 
		\item \textbf{Er sucht im DNS nach einem Resource Record vom Typ MX. }
		\item Er fragt beim DHCP-Server der Domäne an, der ihm die richtige IP-Adresse zurück gibt. 
	\end{itemize}

	Welche Aufgabe(n) übernimmt Base64? 
	\begin{itemize}
		\item Komprimierung von Binärdaten zur schnelleren Übertragung. 
		\item \textbf{Kodierung von Binärdaten in überwiegend Buchstaben und Ziffern. }
		\item Verschlüsselung von E-Mail-Nachrichten. 
		\item Verschlüsselung von Anhängen.
	\end{itemize}

	Warum adressiert man Webseiten nicht direkt über eine IP-Adresse?
	\begin{itemize}
		\item \textbf{Die Inhalte können so auch auf mehrere Web-Server verteilt werden und ein Load Balancing durchgeführt werden.}
		\item IP-Adressen sind nicht eindeutig, die können auch auf mehrere Hosts zeigen.
		\item Heutige Browser können IPv6-Adressen nicht verarbeiten.
		\item \textbf{Weil man sich die Namen einfacher merken kann als Zahlenkombinationen.}
		\item \textbf{IP-Adressen von Webservern können sich ändern. Dann müsste man auch alle Hyperlinks ständig anpassen.}
	\end{itemize}

	Welches sind korrekte URLs für Dokumente im WWW? 
	\begin{itemize}
		\item \textbf{ftp://IP-Adresse des Hostrechners/Pfad/Dateiname.html}
		\item DNS-Name des Hostrechners:Pfad:Protokoll
		\item \textbf{https://DNS-Name des Hostrechners/Pfad/}
		\item \textbf{http://IP-Adresse des Hostrechners/index.html}
		\item http@DNS-Name der Domäne/Pfad/Dateiname.xml
		\item \textbf{http://IP-Adresse des Hostrechners:Portnummer/Pfad}
	\end{itemize}

	Welches Transportprotokoll wird von HTTP verwendet? 
	\begin{itemize}
		\item UDP (User Datagram Protocol) 
		\item \textbf{TCP (Transmission Control Protocol) }
		\item TCP oder UDP (beides möglich) 
		\item RTP
	\end{itemize}

	Was bedeutet die Angabe "Connection: Keep-Alive" im HTTP-Header? 
	\begin{itemize}
		\item Der HTTP-Request erfolgt asynchron.
		\item \textbf{Die aufgebaute TCP-Verbindung wird nicht nach jedem Request-Response wieder abgebaut, sondern längerfristig aufrechterhalten.} 
		\item Die Webseite wird immer zuerst aus dem Cache geladen. 
		\item Kurze HTTP-Anfrage, ob der Server noch erreichbar ist. Ohne Inhaltsübertragung. 
	\end{itemize}

	Was passiert beim Caching von Webseiten? 
	\begin{itemize}
		\item Besonders oft nachgefragte Webseiten werden im Cache zwischengespeichert, damit sie dann beim ersten Aufruf schneller ausgeliefert werden können.
		\item \textbf{Webseiten, die einmal abgerufen wurden, werden im Cache gespeichert und bei erneuter Anfrage sofort aus dem Cache ausgeliefert. }
		\item Webseiten werden redundant in mehreren Webproxies gespeichert, damit sie bei vielen Anfragen performant verteilt werden können.
		\item Für bestimmte Webseiten werden Anwendungsdaten im Browser-Speicher abgelegt, z.B. Sessiondaten. 
	\end{itemize}
	
	Nach welchem Kriterium wird in CDNs entschieden, an welchen Knoten die Anfrage eines Clients weitergeleitet wird? 
	\begin{itemize}
		\item nach der Auslastung der Knoten
		\item \textbf{nach der örtlichen Nähe zum Anfragenden }
		\item nach der Länge der Warteschlage mit offenen Anfragen 
		\item nach der IP-Range aus der die Anfrage kommt 
		\item Die Anfragen werden zufällig an einen Knoten verteilt, was insgesamt zu einer relativ gleichmäßigen Belastung aller Server führt. 
	\end{itemize}

	SNMP Agenten sind ...
	\begin{itemize}
		\item zentrale Management-Geräte 
		\item NMS (Network Management Stations)
		\item \textbf{Netzwerkgeräte wie Router, Switches oder Hubs }
		\item PDUs
	\end{itemize}

	Wann wird bei SNMP eine TRAP-Nachricht erzeugt?
	\begin{itemize}
		\item Wenn der Manager dem Agent signalisieren will, dass ein Neustart erfolgen soll. 
		\item Als Antwort auf einen GetBulkRequest. 
		\item \textbf{Wenn der Agent einen Alarm oder ein außergewöhnliches Ereignis an den Manager melden will, z.B. bei Pufferüberlauf.} 
		\item Wenn der Agent alle Einstellungen verwerfen und auf Werkseinstellungen zurücksetzen soll. 
		\item Wenn der Manager signalisiert, dass die Puffergröße verändert werden soll. 
	\end{itemize}
	
	\section{Multimediakommunikation}
	
	Welche der folgenden Anwendungen ordnen Sie nicht dem Bereich der Multimediakommunikation zu? 
	\begin{itemize}
		\item IP-TV (Fernsehen über Internet)
		\item Video on Demand (Filme nach Bedarf anschauen), Streaming-Dienste
		\item \textbf{Videodownloads (Filme laden zum später schauen)}
		\item Audio-Podcasts
		\item \textbf{MP3s herunterladen zum später hören}
		\item \textbf{Download von Bilddateien von einer Webseite}
		\item YouTube/Vimeo
	\end{itemize}

	Welche Anforderungen ergeben sich aus der Multimediakommunikation an die Protokolle der Transportschicht? 
	\begin{itemize}
		\item \textbf{Echtzeitfähigkeit}
		\item Paketwiederholung
		\item \textbf{Reihenfolgegarantie für Pakete}
		\item paketorientierte Vermittlung
		\item Synchronisation von Audio und Video
	\end{itemize}

	Was glauben Sie ist eine günstige Reihenfolge, Frames in JPEG auszuliefern? 
	\begin{itemize}
		\item IPBB...BBP 
		\item \textbf{IBBP...BBP}
		\item PIBB...BB
		\item PBB...BBI
	\end{itemize}

	Was ist die wichtigste Aufgabe von Codecs?
	\begin{itemize}
		\item Codecs übernehmen die Aufgabe eines Digital/Analog-Wandlers.
		\item Codecs stellen die Paketgrößen automatisch auf sich ändernde Kommunikationsparameter ein.
		\item \textbf{Codecs erlauben eine Anpassung des Datenstroms an den Kommunikationskanal, z.B. durch Komprimierung.}
		\item Codecs haben auf die Effizienz der Multimediakommunikation keinerlei Auswirkung, sie wandeln die Multimediadaten nur in digitale Daten um.
	\end{itemize}

	Welche Eigenschaften treffen auf Quellkodierungen zu? 
	\begin{itemize}
		\item \textbf{medienabhängig}
		\item medienunabhängig
		\item verlustfrei
		\item \textbf{verlustbehaftet}
		\item \textbf{kontextbehaftet}
		\item kontextfrei
	\end{itemize}

	Auf welchen Mechanismus greift Multimediakommunikation auf der Transportschicht (OSI-4) zurück? 
	\begin{itemize}
		\item Pures UDP 
		\item \textbf{Erweiterung von UDP}
		\item Pures TCP
		\item Erweiterung von TCP
	\end{itemize}

	Warum nutzt man RTP für die Übertragung von Medienströmen? 
	\begin{itemize}
		\item Medienströme sollten so schnell wie möglich durch das Netz zugestellt werden. RTP findet die schnellste Route. 
		\item Medienströme müssen im Netzwerk als echtzeitkritisch markiert werden. Da TCP/IP dafür keinen Header hat muss eine neue Protokollebene eingeführt werden. 
		\item \textbf{Medienströme sollten mittels UDP durch das Netz zugestellt werden. RTP ermöglicht dem Empfänger, die korrekte Paketreihenfolge und -abspielposition wieder herzustellen. }
	\end{itemize}

	Warum ist beim Empfänger ein Puffer notwendig, bevor die Multimediadaten abgespielt werden? 
	\begin{itemize}
		\item \textbf{Pakete können in falscher Reihenfolge ankommen und müssen vorm Abspielen sortiert werden}
		\item Pakete können verlorengehen und müssen wiederholt werden
		\item \textbf{um Jitter auszugleichen}
		\item \textbf{um Audio und Videopakete zu synchronisieren}
		\item um das Abspielen der Pakete zu verzögern
	\end{itemize}

	Was zeichnet echtes Medien-Streaming aus?
	\begin{itemize}
		\item Getrennte Server für Adressierung und Streaming, Verwendung eines eigenen Protokolls zur Steuerung, sowie RTP-Medienströme.
		\item Getrennte Server für Mediensteuerung und Medienauslieferung, Verwendung von HTTP zur Steuerung, sowie RTP-Medienströme.
		\item \textbf{Getrennte Server für Medienbeschreibung und Medien, Verwendung eines eigenen Protokolls zur Steuerung, sowie RTP-Medienströme.}
		\item Jeder Server trennt seine Streams durch HTTP-basierte Mediensteuerung und RTP-basierte Medienauslieferung.
	\end{itemize}

	Welche Streaming-Technologie haben Sie beim Schauen der Vorlesungsvideos verwendet? (Schauen Sie doch mal in die Entwicklerkonsole des Browsers oder nutzen Wireshark.)
	\begin{itemize}
		\item "echtes" Streaming 
		\item \textbf{HTTP-Streaming}
		\item MPEG Dash Streaming
	\end{itemize}

	Was sind die wichtigsten Funktionen von SIP?
	\begin{itemize}
		\item Kodierung und Komprimierung des Audiostroms
		\item \textbf{Lokalisierung von Teilnehmern}
		\item \textbf{Anzeige der Verfügbarkeit (Presence) von Teilnehmern }
		\item Zerteilung des Audiostroms in Pakete
		\item Analog-Digital-Wandlung des analogen Audiosignals
		\item \textbf{Aushandlung von Parametern für die Kommunikation}
		\item \textbf{Aufbau und Management einer Session, z.B. Rufweiterleitung und Abbau von Sessions} 
		\item Zeitstempel für Jitter-Vermeidung und Synchronisation
	\end{itemize}

	Ordnen Sie die SIP-Nachrichten in die richtige Reihenfolge!
	\begin{enumerate}
		\item INVITE
		\item TRYING
		\item RINGING
		\item OK
		\item ACK
		\item RTP-Session
		\item BYE
	\end{enumerate}

	Was ist die Aufgabe eines SIP-Registrars?
	\begin{itemize}
		\item Er arbeitet als Vermittler (Routing) der auf der einen Seite Anfragen entgegennimmt, um dann über seine eigene Adresse eine Verbindung zu einer anderen Seite herzustellen. 
		\item \textbf{Er registriert ein oder mehrere IP-Adressen (Geräte) zu einer bestimmten SIP-URI. }
		\item Er erzeugt Weiterleitungen, um eingehende Anträge in einer alternativen Gruppe von URIs kontaktieren zu können 
		\item Er verbindet ein SIP-Netz mit anderen Netzen, wie beispielsweise dem öffentlichen Fernsprechnetz.
		\item Er ist die Schnittstelle zum Benutzer, die Inhalte darstellt und Befehle entgegennimmt, z.B. SIP-Telefon. 
	\end{itemize}
	
	\section{Mobile Computing}
	
	Welche Aussagen treffen bei der Datenübertragung über Funksignale zu?
	\begin{itemize}
		\item \textbf{Die Durchdringung von Objekten wird mit zunehmender Frequenz schlechter.}
		\item Je niedriger die Frequenz, umso höher wird die mögliche Datenrate. (Nyquist)
		\item Um mit niedrigen Frequenzen zu funken, benötigt man mehr Energie.
		\item \textbf{Je höher die Frequenz, desto höher ist die theoretisch mögliche Datenrate}
		\item  \textbf{Bei sehr hohen Frequenzen ist eine Sichtverbindung (Line of sight) notwendig.}
		\item Je höher die Frequenz, um so besser können die Signale Körper durchdringen.
	\end{itemize}

	Wie groß sollte der Mindestabstand zwischen 2 Zellen mit der gleichen Frequenz sein, wenn 12 verschiedene Zelltypen mit einem Empfangsradius von 500 Hz zur Verfügung stehen?
	\begin{itemize}
		\item 100
		\item 300
		\item \textbf{3000}
		\item 5000
	\end{itemize}

	Was versteht man unter einem "Handover" in Mobilfunknetzen? 
	\begin{itemize}
		\item Den Wechsel zwischen Funkzellen unterschiedlicher Mobilfunkanbieter. 
		\item Den Wechsel zwischen verschiedenen Funktechnologien zur Datenübertragung, z.B. zwischen LTE und 5G. 
		\item \textbf{Den Wechsel zwischen Funkzellen bei der Bewegung innerhalb eines Mobilfunknetzes.}
		\item Den Wechsel in andere Mobilfunknetze z.B. bei Grenzübertritt in ein anderes Land. 
	\end{itemize}

	Ordnen Sie den Ablauf bei Anfrage einer IP-Adresse über DHCP! 
	\begin{enumerate}
		\item DHCP DISCOVER 
		\item DHCP OFFER 
		\item DHCP REQUEST 
		\item DHCP ACK 
	\end{enumerate}

	Wann versucht der Client erstmalig sein DHCP-Lease zu erneuern? 
	\begin{itemize}
		\item 30 \% der Lease-Time 
		\item \textbf{50 \% der Lease-Time }
		\item 70 \% der Lease-Time
		\item 87,5 \% der Lease-Time 
	\end{itemize}

	Welche Art der Adressierung wird bei Mobile IPv6 standard-mäßig verwendet? 
	\begin{itemize}
		\item Foreign-Agent Care-of-Address
		\item \textbf{Co-located Care-of-Address }
		\item Home-Agent Care-of-Address
		\item Permanent Address in allen Subnetzen
	\end{itemize}

	Welche Aussagen treffen auf TCP Fast Recovery zu? 
	\begin{itemize}
		\item \textbf{Wiederholung der Übertragung nach Erhalt dreier gleicher ACKs}
		\item Timeout wird verdoppelt
		\item Sendefenster wird nach Timeout auf 1 gesetzt
		\item \textbf{Sendefenster wird halbiert }
		\item Schwellwert wird halbiert
		\item \textbf{keine Erhöhung des Timeouts}
	\end{itemize}

	Welche Erweiterungen wurden für Mobile TCP eingeführt?
	\begin{itemize}
		\item \textbf{Explicit Loss Notification}
		\item Explicit Success Notification
		\item \textbf{Explicit Congestion Notification}
		\item \textbf{Selective Acknowledgements}
		\item \textbf{größeres initiales Sendefenster}
		\item linearer Anstieg der Senderate nach Slow Start
		\item \textbf{Fast Retransmit}
		\item Slow Retransmit
	\end{itemize}

	Ordnen Sie die Hauptkonzepte des Responsive Webdesign den richtigen Funktionen zu! 
	\begin{itemize}
		\item Flexible images: Erzwingt die Skalierung des Bildes zur Breite des Parent-Elements unter Wahrung der Seitenproportionen. 
		\item CSS Media Queries: Der Gerätetyp sowie die spezifischen Eigenschaften eines Gerätes werden abgefragt. 
		\item HTML5 Picture Element: Bereitstellung mehrerer Qualitätsstufen eines Bildes, Auswahl entsprechend Auflösung des Gerätes. 
		\item Fluid grids: Die Größe der Seite wird durch Prozentwerte definiert und passt sich so in Relation zur Größe des Browserfensters an. 
	\end{itemize}
	
	Welches sind die wichtigsten Prinzipien, die Sie bei der Entwicklung von Webseiten nach dem Mobile First Prinzip beachten müssen? 
	\begin{itemize}
		\item \textbf{Weniger ist mehr - Optimiert für kleinere Bildschirmgrößen}
		\item Viele Bilder statt Text
		\item \textbf{Touch-Eingabe und andere Gesten unterstützen}
		\item Entwicklung unterschiedlicher Webseiten für mobile und stationäre Geräte
		\item \textbf{Energieverbrauch minimieren}
		\item \textbf{Datensparsamkeit wegen geringerer Bandbreiten}
	\end{itemize}
	
\end{document}