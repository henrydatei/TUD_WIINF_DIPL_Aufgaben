\documentclass{article}

\usepackage{amsmath,amssymb}
\usepackage{tikz}
\usepackage{pgfplots}
\usepackage{xcolor}
\usepackage[left=2.1cm,right=3.1cm,bottom=3cm,footskip=0.75cm,headsep=0.5cm]{geometry}
\usepackage{enumerate}
\usepackage{enumitem}
\usepackage{marvosym}
\usepackage{tabularx}
\usepackage{hyperref}
\usepackage{longtable}

\usepackage{listings}
\definecolor{lightlightgray}{rgb}{0.95,0.95,0.95}
\definecolor{lila}{rgb}{0.8,0,0.8}
\definecolor{mygray}{rgb}{0.5,0.5,0.5}
\definecolor{mygreen}{rgb}{0,0.8,0.26}
\lstdefinestyle{java} {language=java}
\lstset{language=java,
	basicstyle=\ttfamily,
	keywordstyle=\color{lila},
	commentstyle=\color{lightgray},
	stringstyle=\color{mygreen}\ttfamily,
	backgroundcolor=\color{white},
	showstringspaces=false,
	numbers=left,
	numbersep=10pt,
	numberstyle=\color{mygray}\ttfamily,
	identifierstyle=\color{blue},
	xleftmargin=.1\textwidth, 
	%xrightmargin=.1\textwidth,
	escapechar=§,
}

\usepackage[utf8]{inputenc}

\renewcommand*{\arraystretch}{1.4}

\newcolumntype{L}[1]{>{\raggedright\arraybackslash}p{#1}}
\newcolumntype{R}[1]{>{\raggedleft\arraybackslash}p{#1}}
\newcolumntype{C}[1]{>{\centering\let\newline\\\arraybackslash\hspace{0pt}}m{#1}}

\newcommand{\E}{\mathbb{E}}
\DeclareMathOperator{\rk}{rk}
\DeclareMathOperator{\Var}{Var}
\DeclareMathOperator{\Cov}{Cov}

\title{\textbf{Rechnernetze, Übung 9}}
\author{\textsc{Henry Haustein}}
\date{}

\begin{document}
	\maketitle
	
	\section*{Aufgabe 1}
	\begin{enumerate}[label=(\alph*)]
		\item Interpretation:
		\begin{itemize}
			\item \textit{www}: hostname
			\item \textit{inf}: subdomain
			\item \textit{tu-dresden}: domain
			\item \textit{de}: toplevel-domain
		\end{itemize}
		\item Client sendet Namen, zu dem er die IP wissen will. DNS-Server organisiert die IP, indem er selbst in seiner Tabelle nachschaut, oder, wenn es keinen passenden Eintrag gibt, fragt er einen übergeordneten DNS-Server. Hat der DNS-Server die IP beschafft, sendet er sie an den Client zurück.
		\item \texttt{dig www.example.com A} liefert
		\begin{lstlisting}[tabsize=2]
; <<>> DiG 9.10.6 <<>> www.example.com A
;; global options: +cmd
;; Got answer:
;; ->>HEADER<<- opcode: QUERY, status: NOERROR, id: 44774
;; flags: qr rd ra ad; QUERY: 1, ANSWER: 1, AUTHORITY: 0, ADDITIONAL: 1

;; OPT PSEUDOSECTION:
; EDNS: version: 0, flags:; udp: 4096
;; QUESTION SECTION:
;www.example.com.		IN	A

;; ANSWER SECTION:
www.example.com.	2197	IN	A	93.184.216.34

;; Query time: 62 msec
;; SERVER: 208.67.222.222#53(208.67.222.222)
;; WHEN: Thu Jun 24 13:23:31 CEST 2021
;; MSG SIZE  rcvd: 60
		\end{lstlisting}
		und damit die IP-Adresse 93.184.216.34. 
		
		\texttt{dig tu-dresden.de MX} liefert
		\begin{lstlisting}[tabsize=2]
; <<>> DiG 9.10.6 <<>> tu-dresden.de MX
;; global options: +cmd
;; Got answer:
;; ->>HEADER<<- opcode: QUERY, status: NOERROR, id: 26730
;; flags: qr rd ra; QUERY: 1, ANSWER: 2, AUTHORITY: 0, ADDITIONAL: 1

;; OPT PSEUDOSECTION:
; EDNS: version: 0, flags:; udp: 4096
;; QUESTION SECTION:
;tu-dresden.de.			IN	MX

;; ANSWER SECTION:
tu-dresden.de.		2932	IN	MX	20 mailin5.zih.tu-dresden.de.
tu-dresden.de.		2932	IN	MX	20 mailin6.zih.tu-dresden.de.

;; Query time: 62 msec
;; SERVER: 208.67.222.222#53(208.67.222.222)
;; WHEN: Thu Jun 24 13:24:39 CEST 2021
;; MSG SIZE  rcvd: 94
		\end{lstlisting}
		und damit die Adressen mailin5.zih.tu-dresden.de und mailin6.zih.tu-dresden.de. 
		
		\texttt{dig tu-dresden.de} liefert
		\begin{lstlisting}[tabsize=2]
; <<>> DiG 9.10.6 <<>> tu-dresden.de
;; global options: +cmd
;; Got answer:
;; ->>HEADER<<- opcode: QUERY, status: NOERROR, id: 41062
;; flags: qr rd ra; QUERY: 1, ANSWER: 1, AUTHORITY: 0, ADDITIONAL: 1

;; OPT PSEUDOSECTION:
; EDNS: version: 0, flags:; udp: 4096
;; QUESTION SECTION:
;tu-dresden.de.			IN	A

;; ANSWER SECTION:
tu-dresden.de.		600	IN	A	141.76.39.140

;; Query time: 62 msec
;; SERVER: 208.67.222.222#53(208.67.222.222)
;; WHEN: Thu Jun 24 13:26:17 CEST 2021
;; MSG SIZE  rcvd: 58
		\end{lstlisting}
		und \texttt{dig www.tu-dresden.de} liefert
		\begin{lstlisting}[tabsize=2]
; <<>> DiG 9.10.6 <<>> www.tu-dresden.de
;; global options: +cmd
;; Got answer:
;; ->>HEADER<<- opcode: QUERY, status: NOERROR, id: 61835
;; flags: qr rd ra; QUERY: 1, ANSWER: 2, AUTHORITY: 0, ADDITIONAL: 1

;; OPT PSEUDOSECTION:
; EDNS: version: 0, flags:; udp: 4096
;; QUESTION SECTION:
;www.tu-dresden.de.		IN	A

;; ANSWER SECTION:
www.tu-dresden.de.	4450	IN	CNAME	tucms.wcms.tu-dresden.de.
tucms.wcms.tu-dresden.de. 600	IN	A	141.76.39.140

;; Query time: 62 msec
;; SERVER: 208.67.222.222#53(208.67.222.222)
;; WHEN: Thu Jun 24 13:27:19 CEST 2021
;; MSG SIZE  rcvd: 87
		\end{lstlisting}
		Ein CNAME Resource Record (CNAME RR) ist im Domain Name System (DNS) dazu vorgesehen, einer Domäne einen weiteren Namen zuzuordnen. Die Abkürzung "CNAME“" steht für \textit{canonical name} (engl. \textit{canonical} = anerkannt) und bezeichnet daher den primären, quasi echten Namen. (\url{https://de.wikipedia.org/wiki/CNAME_Resource_Record}). 
		
		\texttt{dig \_ldap.\_tcp.google.com SRV} liefert
		\begin{lstlisting}[tabsize=2]
; <<>> DiG 9.10.6 <<>> _ldap._tcp.google.com SRV
;; global options: +cmd
;; Got answer:
;; ->>HEADER<<- opcode: QUERY, status: NOERROR, id: 53873
;; flags: qr rd ra; QUERY: 1, ANSWER: 1, AUTHORITY: 0, ADDITIONAL: 1

;; OPT PSEUDOSECTION:
; EDNS: version: 0, flags:; udp: 4096
;; QUESTION SECTION:
;_ldap._tcp.google.com.		IN	SRV

;; ANSWER SECTION:
_ldap._tcp.google.com.	86400	IN	SRV	5 0 389 ldap.google.com.

;; Query time: 63 msec
;; SERVER: 208.67.222.222#53(208.67.222.222)
;; WHEN: Thu Jun 24 13:31:41 CEST 2021
;; MSG SIZE  rcvd: 85
		\end{lstlisting}
		\item 1 Tag = 86400 Sekunden. Damit ist die Tabelle
		\begin{center}
			\begin{tabular}{c|c|c|c|c}
				\textbf{NAME} & \textbf{TTL} & \textbf{CLASS} & \textbf{TYPE} & \textbf{VALUE} \\
				\hline
				jupiter & 86400 & IN & A & 117.186.1.1 \\
				jupiter & 86400 & IN & AAAA & 2001:db8:85a3:8d3::1 \\
				saturn & 86400 & IN & A & 117.186.1.2 \\
				saturn & 86400 & IN & AAAA & 2001:db8:85a3:8d3::2 \\
				rn-edu.de. & 86400 & IN & NS & sonne \\
				sonne & 86400 & IN & A & 117.186.1.3 \\
				sonne & 86400 & IN & AAAA & 2001:db8:85a3:8d3::3 \\
				\_sip.\_tcp.rn-edu.de. & 86400 & IN & SRV & 0 0 5060 mond.rn-edu.de. \\
				\_sip.\_udp.rn-edu.de. & 86400 & IN & SRV & 0 0 5060 mond.rn-edu.de. \\
				mond & 86400 & IN & A & 117.186.1.4 \\
				mond & 86400 & IN & AAAA & 2001:db8:85a3:8d3::4 \\
			\end{tabular}
		\end{center}
	\end{enumerate}

	\section*{Aufgabe 2}
	Bei Base64 wird das ASCII-Alphabet, in dem jedes Zeichen aus 8 Bit besteht (insgesamt $2^8=256$ mögliche Zeichen), in ein Alphabet transformiert, in welchem die Zeichen nur 6 Bit benötigen (insgesamt $2^6=64$ mögliche Zeichen). Dazu nimmt man 3 Zeichen des ASCII-Alphabetes (24 Bit) und wandelt diese in 4 Zeichen des Base64-Alphabets um (24 Bit). Falls solche Gruppen aus 3 Zeichen nicht voll werden, füllt man diese einfach auf 24 Bit auf. Bei 20.000 Byte Binärdatei gibt es 6.666 volle Gruppen und 1 Gruppe, die mit einem Zeichen aufgefüllt werden muss, also insgesamt 6.667 Gruppen. Jede Gruppe wird durch 4 Zeichen des Base64-Alphabets repräsentiert, sodass wir auf insgesamt 26.668 Byte kommen. Nach je 76 Byte wird ein CR-LF gesendet, das sind insgesamt 351 á 2 Byte. Insgesamt werden nun also 27.370 Byte gesendet.
	
	Moderne Webanwendungen nutzen Base64-Codierung zur Übertragung von Binärdaten zwischen Client und Browser. Base64-Codierung kann dabei clientseitig im Browser (JavaScript) durchgeführt werden.

	\section*{Aufgabe 3}
	\begin{enumerate}[label=(\alph*)]
		\item DNS-Request stellen, Daten vom Webserver holen und darstellen
		\item URL: gaia.cs.umass.edu/cs453/index.html \\
		HTTP-Version: 1.1 \\
		persistente Verbindung: ja, wird angefragt \\
		IP ist nicht Bestandteil von HTTP
		\item Statuscode 200 $\to$ Dokument wurde gefunden \\
		Datum der Antwort: Tue, 07 Mar 2006 12:39:45GMT \\
		letzte Modifikation: Sat, 10 Dec 2005 18:27:46GMT \\
		Größe in Bytes: 3874 \\
		ersten 5  Bytes: <!doc \\
		persistente Verbindung wurde aufgebaut
		\item Protokoll: HTTP, Dienst: Programm, welches Funktionen über Schnittstellen bereitstellt
	\end{enumerate}

	\section*{Aufgabe 4}
	Tabelle
	\begin{center}
		\begin{longtable}{c|c|c|c|c}
			\textbf{Absender} & \textbf{Ziel} & \textbf{Protokoll} & \textbf{Methode/Antwort} & \textbf{Inhalt} \\
			\hline
			Client & DNS-Server & DNS & query & URL vom Webserver (www.heise.de) \\
			DNS-Server & Client & DNS & query response & IP vom Webserver \\
			Client & Webserver & TCP & SYN & \\
			Webserver & Client & TCP & SYN, ACk & \\
			Client & Webserver & TCP & ACK & \\
			Client & Webserver & HTTP & GET /tools & Ziel-URL \\
			Webserver & Client & TCP & ACK & \\
			Webserver & Client & HTTP & 200 OK & Dokumentdaten \\
			Client & Webserver & TCP & ACK & \\
			Webserver & Client & TCP & ACK & \\
			Client & Webserver & TCP & TCP Keep-alive & \\
			Webserver & Client & TCP & TCP Keep-alive & \\
			\dots & \dots & \dots & \dots & \\
			Webserver & Client & TCP & FW & \\
			Client & Webserver & TCP & ACK
		\end{longtable}
	\end{center}

	\section*{Aufgabe 5}
	\begin{enumerate}[label=(\alph*)]
		\item Schichten:
		\begin{itemize}
			\item Sicherungsschicht: Quelle und Ziel sind der vorangegangene und nachfolgende Knoten, Protokoll z.B. HDLC oder PPP
			\item Vermittlungsschicht: Quelle und Ziel sind der vorangegangene und nachfolgende Knoten, Protokoll z.B. IP
			\item Transportschicht: Quelle: 141.76.42.42, Ziel: 141.30.111.111, Protokoll: z.B. TCP
		\end{itemize}
		Auf dem Rückweg vertauschen sich Quelle und Ziel.
		\item ?
	\end{enumerate}

	\section*{Aufgabe 6}
	\begin{enumerate}[label=(\alph*)]
		\item Manager $\leftarrow$ GET $\leftarrow$ Agent \\
		Manager $\rightarrow$ RESPONSE $\rightarrow$ Agent
		\item Manager $\leftarrow$ GETBULK $\leftarrow$ Agent \\
		Manager $\rightarrow$ RESPONSE $\rightarrow$ Agent
		\item Manager $\rightarrow$ TRAP $\rightarrow$ Agent
		\item Manager $\leftarrow$ SET $\leftarrow$ Agent \\
		Manager $\rightarrow$ RESPONSE $\rightarrow$ Agent
	\end{enumerate}
	
\end{document}