\documentclass{article}

\usepackage{amsmath,amssymb}
\usepackage{tikz}
\usepackage{pgfplots}
\usepackage{xcolor}
\usepackage[left=2.1cm,right=3.1cm,bottom=3cm,footskip=0.75cm,headsep=0.5cm]{geometry}
\usepackage{enumerate}
\usepackage{enumitem}
\usepackage{marvosym}
\usepackage{tabularx}
\usepackage[amsmath,thmmarks,standard]{ntheorem}
\usepackage{mathtools}

\usepackage[utf8]{inputenc}

\renewcommand*{\arraystretch}{1.4}
\newcommand{\E}{\mathbb{E}}

\newcolumntype{L}[1]{>{\raggedright\arraybackslash}p{#1}}
\newcolumntype{R}[1]{>{\raggedleft\arraybackslash}p{#1}}
\newcolumntype{C}[1]{>{\centering\let\newline\\\arraybackslash\hspace{0pt}}m{#1}}

\DeclareMathOperator{\tr}{tr}
\DeclareMathOperator{\Var}{Var}
\DeclareMathOperator{\Cov}{Cov}
\renewcommand{\E}{\mathbb{E}}

\newtheorem{thm}{Theorem}
\newtheorem{lem}{Lemma}

\title{\textbf{Einführung in die Produktion, Tutorium 7}}
\author{\textsc{Henry Haustein}}
\date{}

\begin{document}
	\maketitle
	
	\section*{Aufgabe 15}
	\begin{enumerate}[label=(\alph*)]
		\item offene Fertigung: sofortige Weitergabe zur nächsten Stufe \\
		geschlossene Fertigung: Weitergabe erfolgt erst, wenn gesamtes Los produziert wurde
		\item Es handelt sich um ein Staulager, da $x_p>x_v$. Da der Lagerbestand auf Null sinkt, muss eine sofortige Weitergabe der Produkte erfolgen, also handelt es sich hier um ein offenes Staulager.
		\item Es gilt
		\begin{itemize}
			\item $t_p=b$
			\item $t_v=a$
			\item $t_f=t_v-t_p=c$
			\item $x=e$
			\item $L_{max}=d$
		\end{itemize}
		\item Lagerhaltungskosten je Los:
		\begin{align}
			K_{L,Los} &= \frac{L_{max}}{2} \cdot t_v \cdot c_L \notag \\
			L_{max} &= t_p(x_p-x_v) \notag \\
			&= \frac{x}{x_p}(x_p-x_v) \notag \\
			&= x\left(1-\frac{x_v}{x_p}\right) \notag \\
			t_v &= \frac{x}{x_v} \notag \\
			K_{L,Los} &= \frac{x}{2}\left(1-\frac{x_v}{x_p}\right) \cdot \frac{x}{x_v}\cdot c_L \notag \\
			&= \frac{x^2}{2}\left(\frac{1}{x_v}-\frac{1}{x_p}\right)\cdot c_L \notag
		\end{align}
		\item Multiplikation mit der Losauflagehäufigkeit $n=\frac{B}{x}$ ergibt:
		\begin{align}
			K_L &= \frac{x}{2}\left(\frac{1}{x_v}-\frac{1}{x_p}\right)\cdot c_L\cdot B \notag
		\end{align}
		Gesamtkosten sind also
		\begin{align}
			K(x) &= K_L + K_R \to\min \notag \\
			&= \frac{x}{2}\left(\frac{1}{x_v}-\frac{1}{x_p}\right)\cdot c_L\cdot B + k_R\frac{B}{x} \to\min \notag
		\end{align}
		\item Ableiten und Nullsetzen
		\begin{align}
			\frac{\partial K}{\partial x} = \frac{1}{2}\left(\frac{1}{x_v}-\frac{1}{x_p}\right)\cdot c_L\cdot B - k_R\frac{B}{x^2} &= 0 \notag \\
			\frac{1}{2}\left(\frac{1}{x_v}-\frac{1}{x_p}\right)\cdot c_L\cdot B &= k_R\frac{B}{x^2} \notag \\
			x_{opt} &= \sqrt{\frac{2k_R}{\left(\frac{1}{x_v}-\frac{1}{x_p}\right)\cdot c_L}} \notag
		\end{align}
	\end{enumerate}
	
\end{document}