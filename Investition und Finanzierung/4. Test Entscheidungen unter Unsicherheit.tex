\documentclass{article}

\usepackage{amsmath,amssymb}
\usepackage{tikz}
\usepackage{pgfplots}
\usepackage{xcolor}
\usepackage[left=2.1cm,right=3.1cm,bottom=3cm,footskip=0.75cm,headsep=0.5cm]{geometry}
\usepackage{enumerate}
\usepackage{enumitem}
\usepackage{marvosym}
\usepackage{tabularx}
\usepackage{multirow}
\usepackage[colorlinks = true, linkcolor = blue, urlcolor  = blue, citecolor = blue, anchorcolor = blue]{hyperref}
\usepackage{parskip}

\usepackage{listings}
\definecolor{lightlightgray}{rgb}{0.95,0.95,0.95}
\definecolor{lila}{rgb}{0.8,0,0.8}
\definecolor{mygray}{rgb}{0.5,0.5,0.5}
\definecolor{mygreen}{rgb}{0,0.8,0.26}
\lstdefinestyle{java} {language=java}
\lstset{language=java,
	basicstyle=\ttfamily,
	keywordstyle=\color{lila},
	commentstyle=\color{lightgray},
	stringstyle=\color{mygreen}\ttfamily,
	backgroundcolor=\color{white},
	showstringspaces=false,
	numbers=left,
	numbersep=10pt,
	numberstyle=\color{mygray}\ttfamily,
	identifierstyle=\color{blue},
	xleftmargin=.1\textwidth, 
	%xrightmargin=.1\textwidth,
	escapechar=§,
}

\usepackage[utf8]{inputenc}

\renewcommand*{\arraystretch}{1.4}

\newcolumntype{L}[1]{>{\raggedright\arraybackslash}p{#1}}
\newcolumntype{R}[1]{>{\raggedleft\arraybackslash}p{#1}}
\newcolumntype{C}[1]{>{\centering\let\newline\\\arraybackslash\hspace{0pt}}m{#1}}

\newcommand{\E}{\mathbb{E}}
\DeclareMathOperator{\rk}{rk}
\DeclareMathOperator{\Var}{Var}
\DeclareMathOperator{\Cov}{Cov}
\DeclareMathOperator{\SD}{SD}
\DeclareMathOperator{\Cor}{Cor}
\DeclareMathOperator{\RBF}{RBF}

\title{\textbf{Investition und Finanzierung, Test Entscheidungen unter Unsicherheit}}
\author{\textsc{Henry Haustein}}
\date{}

\begin{document}
	\maketitle
	
	\section*{Risiko}
	Rechnet man die Kapitalwerte für alle Projekte und Szenarien durch ergibt sich folgende Übersicht:
	\begin{center}
		\begin{tabular}{l|r|r|r}
			& Normal & Produktionskosten steigen & Verkaufserlöse steigen \\
			& 60\% & 20\% & 20\% \\
			\hline
			$P_1$ & 628,11 & -1496,76 & 4003,83 \\
			\hline
			$P_2$ & 2248,63 & 74,04 & 6185,11 \\
			\hline
			$P_3$ & -2123,16 & -6484,26 & 3781,55
		\end{tabular}
	\end{center}
	Die Erwartungswerte sind damit
	\begin{align}
		\E(BW(P_1)) &= 0.6\cdot 628,11 + 0,2\cdot -1496,76 + 0,2\cdot 4003,83 = 878,28 \notag \\
		\E(BW(P_2)) &= 2601,01 \notag \\
		\E(BW(P_3)) &= -1814,44 \notag
	\end{align}
	Die Standardabweichungen sind dann
	\begin{align}
		\SD(BW(P_1)) &= \sqrt{0,6(628,11-878,28)^2 + 0,2(-1496,76 - 878,28)^2 + 0,2(4003,83 - 878,28)^2} = 1766,22 \notag \\
		\SD(BW(P_2)) &= 1980,09 \notag \\
		\SD(BW(P_3)) &= 3268,28 \notag
	\end{align}
	Es sollte also das Projekt 1 mit einer Standardabweichung von 1766,22 durchgeführt werden.
	
	\section*{Nutzenfunktion}
	Aus den Barwerten ergibt sich folgende Nutzentabelle:
	\begin{center}
		\begin{tabular}{l|r|r|r}
			Umweltzustand & 1 (10\%) & 2 (30\%) & 3 (60\%) \\
			\hline
			Nutzen $P_1$ & 1150 & 2100 & 2976 \\
			\hline
			Nutzen $P_2$ & -2256 & -516 & 2716 \\
			\hline
			Nutzen $P_3$ & -2574 & 1150 & 1744
		\end{tabular} \\
		\textit{Hinweis wenn man das selber mit einer Tabellenkalkulation berechnen möchte: Leider hält sich keine Tabellenkalkulation (Google Spreadsheets, MS Excel, LibreOffice Calc, Apple Numbers) konsequent an Operatorprioritäten. So sollte folgende Formel zu -36 ausgewertet werden:
		\begin{align}
			-(0,004\cdot 1500)^2 = -(6)^2 = -36 \notag
		\end{align}
		Aber stattdessen wird folgendes berechnet:
		\begin{align}
			-(0,004\cdot 1500)^2 = -(6)^2 \overset{?}{=} ... = 36 \notag
		\end{align}
		Der Taschenrechner rechnet das Ergebnis übrigens richtig aus.}
	\end{center}
	Berechnet man die Erwartungswerte der Nutzen, so ergibt sich:
	\begin{align}
		\E(N(P_1)) &= 0,1\cdot 1150 + 0,3\cdot 2100 + 0,6\cdot 2976 = 2530,6 \notag \\
		\E(N(P_2)) &= 1249,2 \notag \\
		\E(N(P_3)) &= 1134 \notag
	\end{align}
	
	\section*{Kalkulationszins}
	Die Marktrendite ist
	\begin{align}
		r_M = 0,2\cdot 4\% + 0,1\cdot 8\% + 0,7\cdot 9\% = 7,9\% \notag
	\end{align}
	Damit ist der Zins unter Unsicherheit für das Unternehmen DEF
	\begin{align}
		r = 3\% + 1,6\cdot (7,9\% - 3\%) = 10,84\% \notag
	\end{align}

	\section*{Kritischer Wert}
	Die Erlöse und Kosten in den einzelnen Perioden sind
	\begin{center}
		\begin{tabular}{l|r|r|r}
			Periode & 1 & 2 & 3 \\
			\hline
			Stückerlöse & $20x$ & $37x$ & $30x$ \\
			\hline
			Stückkosten & $18x$ & $33x$ & $25x$ \\
			\hline
			fixe Erlöse & 5900 & 8700 & 7500 \\
			\hline
			fixe Kosten & 7600 & 13200 & 11200 \\
			\hline
			Periodenüberschuss & $-1700+2x$ & $-4500+4x$ & $-3700+5x$
		\end{tabular}
	\end{center}
	Der Barwert ist dann
	\begin{align}
		BW &= -17000 + \frac{-1700+2x}{1,05} + \frac{-4500+4x}{1,05^2} + \frac{-3700+5x}{1,05^3} \overset{!}{\ge} 0 \notag \\
		x &\ge 2629 \notag
	\end{align}
	
	\section*{Outputänderung}
	Analoge Rechnung wie oben ergibt für eine Produktionsmenge von 0 Stück einen Barwert von -20110,08, bei einer Produktionsmenge von 1 Stück ist der Barwert -20095,83, also eine Differenz von 14,24.
	
\end{document}