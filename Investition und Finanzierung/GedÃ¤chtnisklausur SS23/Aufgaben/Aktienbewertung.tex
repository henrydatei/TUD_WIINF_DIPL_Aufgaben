\question[5] Eine Aktie kostet heute 30 EUR und gibt einmal pro Jahr eine Dividende von 1,50. Der Zins betrage $r_E = 8\%$.
\begin{parts}
    \part Wie hoch ist die Dividendenrendite?
    \part Wie hoch ist der Preis der Aktie nach dem Dividend-Discount-Modell, wenn wir von einem Dividendenwachstum von 2\% ausgehen? Warum ist in manchen Fällen das Dividend-Discount-Modell nicht geeignet?
    \part In der Aktienbewertung nutzt man häufig auch Multiplikatoren (Kurs-Buchwert-Verhältnis, etc.). Welchen Vorteil bieten diese?
\end{parts}
\begin{solution}
    \begin{parts}
        \part[1] $r = \frac{1,50}{30} = 0,05$
        \part[3] Die Dividende wird im zweiten Jahr $1,50\cdot 1,02 = 1,53$ betragen. Dann sollte der Preis heute sein:
        \begin{align}
        	P_0 &= \frac{1,50}{1,08} + \frac{\frac{1,53}{0,08 - 0,02}}{0,08} \notag \\
        	&= 25 \notag
        \end{align}
    	Das Dividend-Discount-Modell berücksichtigt keine Aktienrückkäufe.
        \part[1] Basieren auf tatsächlichen Marktpreisen (und nicht auf Prognosen)
    \end{parts}
\end{solution}