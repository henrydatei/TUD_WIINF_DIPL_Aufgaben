\question[16] Betrachten Sie die folgenden 2 Nullkupon-Anleihen mit einem Nennwert von 100 EUR:
\begin{center}
	\begin{tabular}{c|c}
		\textbf{Laufzeit} & \textbf{Preis} \\
		\hline
		1 & 97,54 EUR \\
		\hline
		3 & 89,62 EUR
	\end{tabular}
\end{center}
\begin{parts}
    \part Ermitteln Sie die Effektivverzinsung der beiden Anleihen.
    \part Welchen Preis hat eine 3-jährige Anleihe mit einem Kuponzins von 4\% und einem Nennwert von 1000 EUR? Nutzen Sie das Duplizierungsportfolio und ermitteln Sie den effektiven Zins einer zweijährigen Nullkupon-Anleihe mittels linearer Interpolation.
    \part Wie heißt das finanzwirtschaftliche Prinzip, welches Sie in (b) benutzt haben?
    \part Welchen Preis hat eine Kuponanleihe mit 4\% Kupon, einer Effektivverzinsung von 3,61\% p.a., einem Nennwert von 1000 EUR und einer Laufzeit von 1,5 Jahren, die ihren Kupon halbjährlich auszahlt?
    \part Was ist der Unterschied zwischen Clean und Dirty Price? Und welchen sollte man benutzen, um Preisunterschiede zu untersuchen?
\end{parts}
\begin{solution}
    \begin{parts}
        \part[2] $r_1 = \frac{100}{97,54} - 1 = 2,52\%$ \\
        $r_3 = \sqrt[3]{\frac{100}{89,62}} - 1 = 3,72\%$
        \part[6] Lineare Interpolation für $r_2 = \frac{r_3-r_1}{2} = \frac{2,52\% + 3,72\%}{2} = 3,12\%$. Der Preis einer solchen Anleihe ist dann $\frac{100}{1,0312^2} = 94,04$ EUR. Duplizierungsportfolio:
        \begin{center}
        	\begin{tabular}{c|c|ccc|c}
        		\textbf{Laufzeit} & \textbf{Preis} & $CF_1$ & $CF_2$ & $CF_3$ & \textbf{Stückzahl} \\
        		\hline
        		1 & 97,54 & 40 & & & 0,4 \\
        		2 & 94,04 & & 40 & & 0,4 \\
        		3 & 89,62 & & & 1040 & 10,4 \\
        		\hline
        		Kuponanleihe & 1008,68 & 40 & 40 & 1040 & 1
        	\end{tabular}
        \end{center}
        \part[1] Law of One Price / Gesetz des einheitlichen Preises
        \part[4] Der Effektivzins für ein halbes Jahr ist $r_{halb} = \sqrt{1,0361}-1 = 0,0179$. Der Preis ist dann
        \begin{align}
        	P_0 &= \frac{20}{1,0179} + \frac{20}{1,0179^2} + \frac{1020}{1,0179^3} \notag \\
        	&= 1006,08 \notag
        \end{align}
        \part[3] Dirty Price enthält aufgelaufene Stückzinsen, welche vorhersagbarem Muster folgen, Clean Price enthält keine aufgelaufenen Stückzinsen wodurch unvorhergesehene Änderungen der Preise durch eine Änderung der Effektivverzinsung betrachtet werden
    \end{parts}
\end{solution}