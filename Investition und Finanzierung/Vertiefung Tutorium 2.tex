\documentclass{article}

\usepackage{amsmath,amssymb}
\usepackage{tikz}
\usepackage{pgfplots}
\usepackage{xcolor}
\usepackage[left=2.1cm,right=3.1cm,bottom=3cm,footskip=0.75cm,headsep=0.5cm]{geometry}
\usepackage{enumerate}
\usepackage{enumitem}
\usepackage{marvosym}
\usepackage{tabularx}
\usepackage{multirow}
\usepackage[colorlinks = true, linkcolor = blue, urlcolor  = blue, citecolor = blue, anchorcolor = blue]{hyperref}
\usepackage{parskip}

\usepackage{listings}
\definecolor{lightlightgray}{rgb}{0.95,0.95,0.95}
\definecolor{lila}{rgb}{0.8,0,0.8}
\definecolor{mygray}{rgb}{0.5,0.5,0.5}
\definecolor{mygreen}{rgb}{0,0.8,0.26}
\lstdefinestyle{java} {language=java}
\lstset{language=java,
	basicstyle=\ttfamily,
	keywordstyle=\color{lila},
	commentstyle=\color{lightgray},
	stringstyle=\color{mygreen}\ttfamily,
	backgroundcolor=\color{white},
	showstringspaces=false,
	numbers=left,
	numbersep=10pt,
	numberstyle=\color{mygray}\ttfamily,
	identifierstyle=\color{blue},
	xleftmargin=.1\textwidth, 
	%xrightmargin=.1\textwidth,
	escapechar=§,
}

\usepackage[utf8]{inputenc}

\renewcommand*{\arraystretch}{1.4}

\newcolumntype{L}[1]{>{\raggedright\arraybackslash}p{#1}}
\newcolumntype{R}[1]{>{\raggedleft\arraybackslash}p{#1}}
\newcolumntype{C}[1]{>{\centering\let\newline\\\arraybackslash\hspace{0pt}}m{#1}}

\newcommand{\E}{\mathbb{E}}
\DeclareMathOperator{\rk}{rk}
\DeclareMathOperator{\Var}{Var}
\DeclareMathOperator{\Cov}{Cov}
\DeclareMathOperator{\SD}{SD}
\DeclareMathOperator{\Cor}{Cor}
\DeclareMathOperator{\RBF}{RBF}

\title{\textbf{Investition und Finanzierung, Vertiefung Tutorium 2}}
\author{\textsc{Henry Haustein}}
\date{}

\begin{document}
	\maketitle
	
	\section*{Rentenbarwert}
	Der Rentenbarwertfaktor ist
	\begin{align}
		\RBF = \frac{q^n - 1}{q^n\cdot (q-1)} \notag
	\end{align}
	Damit ergibt sich für die ersten 6 Jahre:
	\begin{align}
		BW_1 &= \frac{1.06^6-1}{1.06^6\cdot 0.06}\cdot 50000 \notag \\
		&= 245866.22 \notag
	\end{align}
	Der Barwert der Rente für die zweiten 6 Jahre ist (im Jahr 6!). Dieser muss dann noch 6 Jahre abgezinst werden:
	\begin{align}
		BW_2 &= \frac{\frac{1.04^6-1}{1.04^6\cdot 0.04}\cdot 50000}{1.06^6} \notag \\
		&= 184774.98 \notag
	\end{align}
	In Summe ergibt sich 430641.20.

	\section*{Restrentenansprüche}
	Dazu müssen wir die zukünftigen Rentenansprüche auf das Jahr 2020 abzinsen. Silvio bekommt noch 3 Renten á 50000 mit 6\% und 6 Renten á 50000 mit 4\%, die dann aber noch 3 mal abgezinst werden müssen:
	\begin{align}
		BW_1 &= \frac{1.06^3-1}{1.06^3\cdot 0.06}\cdot 50000 \notag \\
		&= 133650.60 \notag \\
		BW_2 &= \frac{\frac{1.04^6-1}{1.04^6\cdot 0.04}\cdot 50000}{1.06^3} \notag \\
		&= 220069.96 \notag
	\end{align}
	In Summe 353720.56.

	\section*{Annuitätendarlehen}
	Berechnen wir zuerst den Barwert zu Anfang des Jahres 4 (bzw. Ende des Jahres 3), den wir dann noch 3 mal abzinsen müssen:
	\begin{align}
		BW &= \frac{1.1^{12} - 1}{1.1^{12}\cdot 0.1}\cdot 10000 \notag \\
		&= 68136.92 \notag
	\end{align}
	3 mal abzinsen liefert das Ergebnis von $\frac{68136.92}{1.1^3} = 51192.28$.

	\section*{Zeitungsabo}
	\begin{enumerate}[label=(\alph*)]
		\item Wir brauchen zuerst den Monatszins $i$, für den gilt:
		\begin{align}
			1.05 &= (1+i)^{12} \notag \\
			i &= 4.074123784\cdot 10^{-3} \notag
		\end{align}
		Damit ist er Barwert:
		\begin{align}
			BW_{monatlich} &= \frac{(1+i)^{36}-1}{(1+i)^{36}\cdot i}\cdot 30 \notag \\
			&= 1002.64 \notag
		\end{align}
		\item Hier können wir den Barwert klassisch berechnen (Achtung: Zahlung am Jahresanfang!):
		\begin{align}
			BW_{jaehrlich} &= 340 + \frac{340}{1.05} + \frac{340}{1.05^2} \notag \\
			&= 972.20 \notag
		\end{align}
		\item Wir lösen das wieder mit dem Monatszins $i$:
		\begin{align}
			BW_{einmalig} &= \frac{1020}{(1+i)^{18}} \notag \\
			&= 948.02 \notag
		\end{align}
	\end{enumerate}
	Offensichtlich ist der Barwert des 3-Jahres-Abos am geringsten, also sollten wir das kaufen.
	
	\section*{Sparkonto}
	Am Ende der Ansparphase sind auf dem Konto:
	\begin{align}
		K &= 10000\cdot 1.05^2\cdot 1.06\cdot 1.07\cdot 1.08 \notag \\
		&= 13504.92 \notag
	\end{align}
	Diese können jetzt in Renten aufgeteilt werden:
	\begin{align}
		13504.92 &= r\cdot \frac{1.04^5-1}{1.04^5\cdot 0.04} \notag \\
		r &= 3033.57 \notag
	\end{align}
	
\end{document}