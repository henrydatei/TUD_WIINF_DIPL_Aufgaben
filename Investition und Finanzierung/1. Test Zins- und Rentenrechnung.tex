\documentclass{article}

\usepackage{amsmath,amssymb}
\usepackage{tikz}
\usepackage{pgfplots}
\usepackage{xcolor}
\usepackage[left=2.1cm,right=3.1cm,bottom=3cm,footskip=0.75cm,headsep=0.5cm]{geometry}
\usepackage{enumerate}
\usepackage{enumitem}
\usepackage{marvosym}
\usepackage{tabularx}
\usepackage{multirow}
\usepackage[colorlinks = true, linkcolor = blue, urlcolor  = blue, citecolor = blue, anchorcolor = blue]{hyperref}
\usepackage{parskip}

\usepackage{listings}
\definecolor{lightlightgray}{rgb}{0.95,0.95,0.95}
\definecolor{lila}{rgb}{0.8,0,0.8}
\definecolor{mygray}{rgb}{0.5,0.5,0.5}
\definecolor{mygreen}{rgb}{0,0.8,0.26}
\lstdefinestyle{java} {language=java}
\lstset{language=java,
	basicstyle=\ttfamily,
	keywordstyle=\color{lila},
	commentstyle=\color{lightgray},
	stringstyle=\color{mygreen}\ttfamily,
	backgroundcolor=\color{white},
	showstringspaces=false,
	numbers=left,
	numbersep=10pt,
	numberstyle=\color{mygray}\ttfamily,
	identifierstyle=\color{blue},
	xleftmargin=.1\textwidth, 
	%xrightmargin=.1\textwidth,
	escapechar=§,
}

\usepackage[utf8]{inputenc}

\renewcommand*{\arraystretch}{1.4}

\newcolumntype{L}[1]{>{\raggedright\arraybackslash}p{#1}}
\newcolumntype{R}[1]{>{\raggedleft\arraybackslash}p{#1}}
\newcolumntype{C}[1]{>{\centering\let\newline\\\arraybackslash\hspace{0pt}}m{#1}}

\newcommand{\E}{\mathbb{E}}
\DeclareMathOperator{\rk}{rk}
\DeclareMathOperator{\Var}{Var}
\DeclareMathOperator{\Cov}{Cov}
\DeclareMathOperator{\SD}{SD}
\DeclareMathOperator{\Cor}{Cor}
\DeclareMathOperator{\RBF}{RBF}

\title{\textbf{Investition und Finanzierung, Test Zins- und Rentenrechnung}}
\author{\textsc{Henry Haustein}}
\date{}

\begin{document}
	\maketitle
	
	\section*{Zinseszinseffekt}
	Mit Zinseszinseffekt ist das Konto nach 8 Jahren auf
	\begin{align}
		K &= K_0 \cdot (1+i)^n \notag \\
		&= 25000\text{ \EUR} \cdot (1+0.02)^8 \notag \\
		&= 29291.48\text{ \EUR} \notag
	\end{align}
	angewachsen. Ohne Zinseszinseffekt bekommen wir jedes Jahr $2\%\cdot 25000\text{ \EUR} = 500\text{ \EUR}$ Zinsen, also nach 8 Jahren 4000 \EUR. Damit wäre das Konto bei 29000 \EUR. Der Zinseszinseffekt ist damit $29291.48\text{ \EUR} - 29000\text{ \EUR} = 291.48\text{ \EUR}$.

	\section*{Wert einer Zahlungsreihe}
	Da wir den Wert der Zahlungsreihe am Ende des Jahres 4 berechnen sollen, müssen wir die ersten 4 Jahre aufzinsen und die letzten beiden Jahre abzinsen, also
	\begin{align}
		KW &= 5000\text{ \EUR} \cdot 1.07^3 + 8000\text{ \EUR} \cdot 1.07^2 + 1000\text{ \EUR} \cdot 1.07 + 1000\text{ \EUR} + \frac{7000\text{ \EUR}}{1.07} + \frac{6000\text{ \EUR}}{1.07^2} \notag \\
		&= 29137.10\text{ \EUR} \notag
	\end{align}
	
	\section*{Effektiver Zinssatz}
	Für den durchschnittlichen Zinssatz muss man das geometrische Mittel der Zinsen nehmen:
	\begin{align}
		p_{eff} &= \sqrt[6]{1.04\cdot 1.01\cdot 1.06\cdot 1.07\cdot 1.01\cdot 1.10} - 1 \notag \\
		&= 0.0478 \notag \\
		&= 4.78\% \notag
	\end{align}
	
	\section*{Laufzeit einer Zahlungsreihe}
	Wir brauchen hier den vorschüssigen Rentenbarwertfaktor, da gilt:
	\begin{align}
		R_{0,v} &= R\cdot \RBF_{vor} \notag
	\end{align}
	Wir wissen, dass $\RBF_{vor} = \RBF_{nach} \cdot q$ gilt und weiterhin
	\begin{align}
		\RBF_{nach} &= \frac{q^n-1}{q^{n+1}-q^n} \notag
	\end{align}
	Setzen wir das alles zusammen, müssen wir folgende Gleichung lösen:
	\begin{align}
		R_{0,v} &= R\cdot \frac{q^n-1}{q^{n+1}-q^n} \cdot q \notag \\
		14000\text{ \EUR} &= 3000\text{ \EUR} \cdot \frac{1.03^n-1}{1.03^{n+1}-1.03^n} \cdot 1.03 \notag \\
		n &= 4.9424 \notag
	\end{align}

	\section*{Anlage}
	Der Betrag aus dem Jahr 2 wird 9 Jahre verzinst, der Betrag aus dem Jahr 3 wird 8 Jahre verzinst und der Betrag aus dem Jahr 8 wird 3 Jahre verzinst. Damit ergibt sich folgende Gleichung:
	\begin{align}
		30000\text{ \EUR} &= x\cdot 1.02^9 + x\cdot 1.02^8 + x\cdot 1.02^3 \notag \\
		x &= 8751.56\notag
	\end{align}
	
\end{document}