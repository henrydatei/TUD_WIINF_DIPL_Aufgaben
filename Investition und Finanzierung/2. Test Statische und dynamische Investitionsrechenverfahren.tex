\documentclass{article}

\usepackage{amsmath,amssymb}
\usepackage{tikz}
\usepackage{pgfplots}
\usepackage{xcolor}
\usepackage[left=2.1cm,right=3.1cm,bottom=3cm,footskip=0.75cm,headsep=0.5cm]{geometry}
\usepackage{enumerate}
\usepackage{enumitem}
\usepackage{marvosym}
\usepackage{tabularx}
\usepackage{multirow}
\usepackage[colorlinks = true, linkcolor = blue, urlcolor  = blue, citecolor = blue, anchorcolor = blue]{hyperref}
\usepackage{parskip}

\usepackage{listings}
\definecolor{lightlightgray}{rgb}{0.95,0.95,0.95}
\definecolor{lila}{rgb}{0.8,0,0.8}
\definecolor{mygray}{rgb}{0.5,0.5,0.5}
\definecolor{mygreen}{rgb}{0,0.8,0.26}
\lstdefinestyle{java} {language=java}
\lstset{language=java,
	basicstyle=\ttfamily,
	keywordstyle=\color{lila},
	commentstyle=\color{lightgray},
	stringstyle=\color{mygreen}\ttfamily,
	backgroundcolor=\color{white},
	showstringspaces=false,
	numbers=left,
	numbersep=10pt,
	numberstyle=\color{mygray}\ttfamily,
	identifierstyle=\color{blue},
	xleftmargin=.1\textwidth, 
	%xrightmargin=.1\textwidth,
	escapechar=§,
}

\usepackage[utf8]{inputenc}

\renewcommand*{\arraystretch}{1.4}

\newcolumntype{L}[1]{>{\raggedright\arraybackslash}p{#1}}
\newcolumntype{R}[1]{>{\raggedleft\arraybackslash}p{#1}}
\newcolumntype{C}[1]{>{\centering\let\newline\\\arraybackslash\hspace{0pt}}m{#1}}

\newcommand{\E}{\mathbb{E}}
\DeclareMathOperator{\rk}{rk}
\DeclareMathOperator{\Var}{Var}
\DeclareMathOperator{\Cov}{Cov}
\DeclareMathOperator{\SD}{SD}
\DeclareMathOperator{\Cor}{Cor}
\DeclareMathOperator{\RBF}{RBF}

\title{\textbf{Investition und Finanzierung, Test Statische und dynamische Investitionsrechenverfahren}}
\author{\textsc{Henry Haustein}}
\date{}

\begin{document}
	\maketitle
	
	\section*{Kritische Menge}
	Welche Kosten fallen auf der neuen Maschine an?
	\begin{itemize}
		\item Abschreibung: $\frac{I_0-L_n}{n} = \frac{24000\text{ \EUR} - 0\text{ \EUR}}{12} = 2000\text{ \EUR}$ pro Jahr
		\item kalkulatorische Kosten: $\frac{I_0+L_n}{2} \cdot i = \frac{24000\text{ \EUR} + 0\text{ \EUR}}{2}\cdot 0.11 = 1320\text{ \EUR}$ pro Jahr
		\item Materialkosten: 5 \EUR\, pro Teil
		\item Lohnkosten: 0.6 \EUR\, pro Teil
	\end{itemize}
	Damit sich die neue Maschine lohnt, müssen die Stückkosten kleiner als 20 \EUR\, werden:
	\begin{align}
		5\text{ \EUR} + 0.6\text{ \EUR} + \frac{2000\text{ \EUR}}{x} + \frac{1320\text{ \EUR}}{x} &< 20\text{ \EUR} \notag \\
		x &> 230.56 \notag 
	\end{align}
	
	\section*{Amortisationszeit}
	Folgende Kosten erzeugt die Maschine:
	\begin{align}
		\text{Abschreibung} &= \frac{450000\text{ \EUR} - 20000\text{ \EUR}}{5} = 86000\text{ \EUR} \notag \\
		\text{kalk. Kosten} &= \frac{450000\text{ \EUR} + 20000\text{ \EUR}}{2} \cdot 0.09 = 21150 \text{ \EUR} \notag
	\end{align}
	\begin{center}
		\begin{tabular}{l|r|r|r|r|r}
			\textbf{Periode} & 1 & 2 & 3 & 4 & 5 \\
			\hline
			\textbf{Abschreibung} & 86000 & 86000 & 86000 & 86000 & 86000 \\
			\hline
			\textbf{kalk. Kosten} & 21150 & 21150 & 21150 & 21150 & 21150 \\
			\hline
			\textbf{Reparatur} & 5000 & 5000 & 5000 & 5000 & 5000 \\
			\hline
			\textbf{Strom + Bedienung} & 160000 & 144000 & 128000 & 112000 & 96000 \\
			\hline\hline
			\textbf{Kosten} & 272150 & 256150 & 240150 & 224150 & 208150
		\end{tabular}
	\end{center}
	Der Erlös ist in jeder Periode gleich, damit ergibt sich folgender Gewinn:
	\begin{center}
		\begin{tabular}{l|r|r|r|r|r}
			\textbf{Periode} & 1 & 2 & 3 & 4 & 5 \\
			\hline
			\textbf{Gewinn} & 127850 & 143850 & 159850 & 175850 & 191850 \\
			\hline
			\textbf{Gewinn kumuliert} & 127850 & 271700 & 431550 & 607400 & 799250
		\end{tabular}
	\end{center}
	Es dauert also 3-4 Perioden, bis sich die Maschine amortisiert hat. Die lineare Interpolation zwischen den Perioden 3 und 4 ist:
	\begin{align}
		y(x) &= y_1 + \frac{y_2-y_1}{x_2-x_1} \cdot (x-x_1) \notag \\
		&= 431550 + \frac{607400 - 431550}{4-3} \cdot (x-3) \notag
	\end{align}
	Wir suchen die Stelle, wo diese Funktion den Wert 450000 erreicht:
	\begin{align}
		450000 &= 431550 + \frac{607400 - 431550}{4-3} \cdot (x-3) \notag \\
		x &= 3.1049 \notag
	\end{align}
	Leider ist das nicht das Ergebnis, was rauskommen soll, die richtige Lösung soll 2.02562538133 sein, aber ich komme da nicht drauf.
	
	\section*{Differenzinvestition}
	Die Differenzinvestition ist
	\begin{center}
		\begin{tabular}{l|r|r|r|r|r|r|r}
			Periode & 0 & 1 & 2 & 3 & 4 & 5 & 6 \\
			\hline
			$P_1$ & -2200 & 900 & 2200 & 100 & 700 & 2000 & 3000 \\
			\hline
			$P_2$ & -3400 & 1100 & 1800 & 1100 & 1100 & 2700 & 3200 \\
			\hline\hline
			$P_1-P_2$ & 1200 & -200 & 400 & -1000 & -400 & -700 & -200
		\end{tabular}
	\end{center}
	Der Kapitalwert $C_0$ dieser Investition beträgt -276.57 \EUR. Der Annuitätenfaktor ist
	\begin{align}
		a_n &= \frac{q^n\cdot (q-1)}{q^n-1} \notag \\
		&= \frac{1.09^6\cdot 0.09}{1.09^6-1} \notag \\
		&= 0.2229 \notag
	\end{align}
	Und damit ergibt sich eine Annuität von $A=C_0\cdot a_n = -61.65\text{ \EUR}$.

	\section*{Tangentennäherungsverfahren}
	Ich weiß nicht, was mit \textit{Endwertfunktion} gemeint ist, aber wahrscheinlich komme ich deswegen nicht auf das richtige Ergebnis. Herauskommen soll 29.85837391945.
	
	Die Funktion, deren Nullstelle zu suchen ist, ist
	\begin{align}
		f(i) &= -2400 + \frac{-600}{1+i} + \frac{1000}{(1+i)^2} + \frac{2700}{(1+i)^3} + \frac{1900}{(1+i)^4} \notag
	\end{align}
	$f(0.07)$ hat den Wert 1566.19, die Ableitung ist
	\begin{align}
		f'(i) &= \frac{100(6i^3 - 2i^2 - 103i - 171)}{(1+i)^5} \notag
	\end{align}
	und $f'(0.07) = -12706.68$. Damit ist der neue Wert für $i$
	\begin{align}
		i^\ast &= 0.07 - \frac{1566.19}{-12706.68} \notag \\
		&= 1.1933 \notag
	\end{align}
	also 19.33 \%.
	
	\section*{Kapitalwert unter Berücksichtigung von Steuern}
	Der Zinssatz nach Steuern ist
	\begin{align}
		i^S &= i\cdot (1-s^{ert}) \notag \\
		&= 0.05 \cdot (1-0.34) \notag \\
		&= 0.033 \notag
	\end{align}
	Die Abschreibung ist $\frac{230\text{ \EUR} - 30\text{ \EUR}}{4} = 50\text{ \EUR}$. Die Formel für den Kapitalwert mit Steuern ist
	\begin{align}
		C_0^S &= -I_0 + \sum_{t=1}^{n} \frac{P_t-s^{ert}(P_t-\text{Abschreibung})}{(1+i^S)^t} + \frac{L_n-s^{ert}(L_n-RBW_n)}{(1+i^S)^n} \notag \\
		&= -230 + \frac{50-0.34(50-50)}{1+0.033} + \frac{60-0.34(60-50)}{(1+0.033)^2} + \frac{130-0.34(130-50)}{(1+0.033)^3} + \frac{10-0.34(10-50)}{(1+0.033)^4} + \frac{50-0.34(50-30)}{(1+0.033)^4} \notag \\
		&= 23.37 \text{ \EUR} \notag
	\end{align}
	
\end{document}