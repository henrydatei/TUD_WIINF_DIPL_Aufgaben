\documentclass{article}

\usepackage{amsmath,amssymb}
\usepackage{tikz}
\usepackage{pgfplots}
\usepackage{xcolor}
\usepackage[left=2.1cm,right=3.1cm,bottom=3cm,footskip=0.75cm,headsep=0.5cm]{geometry}
\usepackage{enumerate}
\usepackage{enumitem}
\usepackage{marvosym}
\usepackage{tabularx}
\usepackage{multirow}
\usepackage[colorlinks = true, linkcolor = blue, urlcolor  = blue, citecolor = blue, anchorcolor = blue]{hyperref}
\usepackage{parskip}

\usepackage{listings}
\definecolor{lightlightgray}{rgb}{0.95,0.95,0.95}
\definecolor{lila}{rgb}{0.8,0,0.8}
\definecolor{mygray}{rgb}{0.5,0.5,0.5}
\definecolor{mygreen}{rgb}{0,0.8,0.26}
\lstdefinestyle{java} {language=java}
\lstset{language=java,
	basicstyle=\ttfamily,
	keywordstyle=\color{lila},
	commentstyle=\color{lightgray},
	stringstyle=\color{mygreen}\ttfamily,
	backgroundcolor=\color{white},
	showstringspaces=false,
	numbers=left,
	numbersep=10pt,
	numberstyle=\color{mygray}\ttfamily,
	identifierstyle=\color{blue},
	xleftmargin=.1\textwidth, 
	%xrightmargin=.1\textwidth,
	escapechar=§,
}

\usepackage[utf8]{inputenc}

\renewcommand*{\arraystretch}{1.4}

\newcolumntype{L}[1]{>{\raggedright\arraybackslash}p{#1}}
\newcolumntype{R}[1]{>{\raggedleft\arraybackslash}p{#1}}
\newcolumntype{C}[1]{>{\centering\let\newline\\\arraybackslash\hspace{0pt}}m{#1}}

\newcommand{\E}{\mathbb{E}}
\DeclareMathOperator{\rk}{rk}
\DeclareMathOperator{\Var}{Var}
\DeclareMathOperator{\Cov}{Cov}
\DeclareMathOperator{\SD}{SD}
\DeclareMathOperator{\Cor}{Cor}
\DeclareMathOperator{\RBF}{RBF}

\title{\textbf{Investition und Finanzierung, Vertiefung Tutorium 4}}
\author{\textsc{Henry Haustein}}
\date{}

\begin{document}
	\maketitle
	
	\section*{Zinsänderung}
	Wenn der Zins sinkt, so werden Erträge aus der Zukunft weniger abgezinst und damit sind sie heute mehr wert $\Rightarrow$ der Barwert steigt $\Rightarrow$ das Projekt wird vorteilhafter

	\section*{Lineare Interpolation}
	Die Formel für die lineare Interpolation ist
	\begin{align}
		f(x) = y_1 + (x - x_1)\cdot\frac{y_2-y_1}{x_2-x_1} \notag
	\end{align}
	Setzen wir auf die $x$-Achse den Kapitalwert und auf die $y$-Achse den Zinssatz, so erhalten wir folgendes Diagramm (lineare Interpolation schon mit eingezeichnet):
	\begin{center}
		\begin{tikzpicture}
			\begin{axis}[
				xmin=-6, xmax=48, xlabel={$C_0$},
				ymin=0, ymax=10, ylabel={$i$ in \%},
				samples=400,
				axis x line=middle,
				axis y line=middle,
				domain=-6:48,
				]
				\addplot[blue, mark=x,only marks] coordinates {
					(-5.65,7)
					(47.1,5)
				};
				\addplot[blue, mark=none] {-0.0379147*x + 6.78578};
				
			\end{axis}
		\end{tikzpicture}
	\end{center}
	Wir interessieren uns genau für den Punkt, wo $C_0=0$, also
	\begin{align}
		i = f(0) &= y_1 + (x - x_1)\cdot\frac{y_2-y_1}{x_2-x_1} \notag \\
		&= 7\% + (0-(-5.65))\cdot\frac{5\% - 7\%}{47.1 - (-5.65)} \notag \\
		&= 6.79\% \notag
	\end{align}

	\section*{Newton-Verfahren}
	Wir brauchen dazu den Kapitalwert in Abhängigkeit von $q$:
	\begin{align}
		C_0(q) &= -1000 + \frac{700}{q} + \frac{500}{q^2} \notag \\
		C_0'(q) &= -\frac{700}{q^2} - \frac{1000}{q^3} \notag
	\end{align}
	Damit gilt
	\begin{align}
		q_1 &= q_0 - \frac{C_0(q_0)}{C'(q_0)} \notag \\
		&= 1.12 - \frac{23.597}{-1269.82} \notag \\
		&= 1.1386 \notag
	\end{align}
	$\Rightarrow i_1 = 13.86\%$
	
	\section*{Kapitalwert}
	Für die verschiedenen Nutzungsdauern ergeben sich folgende Kapitalwerte:
	\begin{itemize}
		\item 0 Jahre: $C_0 = -300 + \frac{300}{1.15^0} = 0$ 
		\item 1 Jahr: $C_0 = -300 + \frac{120}{1.15} + \frac{240}{1.15} = 13.043$
		\item 2 Jahre: $C_0 = -300 + \frac{120}{1.15} + \frac{115}{1.15^2} + \frac{180}{1.15^2} = 27.41$
		\item 3 Jahre: $C_0 = -300 + \frac{120}{1.15} + \frac{115}{1.15^2} + \frac{95}{1.15^3} + \frac{120}{1.15^3} = 32.67$
		\item 4 Jahre: $C_0 = -300 + \frac{120}{1.15} + \frac{115}{1.15^2} + \frac{95}{1.15^3} + \frac{75}{1.15^4} + \frac{70}{1.15^4} = 36.67$
		\item 5 Jahre: $C_0 = -300 + \frac{120}{1.15} + \frac{115}{1.15^2} + \frac{95}{1.15^3} + \frac{75}{1.15^4} + \frac{40}{1.15^5} + \frac{0}{1.15^5} = 16.53$
	\end{itemize}
	$\Rightarrow$ optimale Haltedauer 4 Jahre, Kapitalwert 36.67
	
	\section*{Annuität}
	Die Annuitätenformel ist
	\begin{align}
		A = C_0\cdot\frac{q^n\cdot i}{q^n-1} \notag
	\end{align}
	Damit ergeben sich für die verschiedenen Haltedauern die folgenden Annuitäten:
	\begin{itemize}
		\item 1 Jahr: $A = 13.043 \cdot \frac{1.15^1\cdot 0.15}{1.15^1 - 1} = 14.999$
		\item 2 Jahre: $A = 27.41 \cdot \frac{1.15^2\cdot 0.15}{1.15^2 - 1} = 16.86$
		\item 3 Jahre: $A = 32.67 \cdot \frac{1.15^3\cdot 0.15}{1.15^3 - 1} = 14.309$
		\item 4 Jahre: $A = 36.67 \cdot \frac{1.15^4\cdot 0.15}{1.15^4 - 1} = 12.84$
		\item 5 Jahre: $A = 16.53 \cdot \frac{1.15^5\cdot 0.15}{1.15^5 - 1} = 4.93$
	\end{itemize}
	
	\section*{Ertragssteuern}
	Die Abschreibung beträgt $\frac{15000 - 3000}{4} = 3000$, der Zinssatz $i^S = 0.05\cdot (1-0.38) = 0.031$. Damit ergibt sich:
	\begin{align}
		C_0^S &= -15000 + \frac{4000 - 0.38\cdot (4000 - 3000)}{1.031^1} + \frac{2500 - 0.38\cdot (2500 - 3000)}{1.031^2} + \frac{5500 - 0.38\cdot (5500 - 3000)}{1.031^3} \notag \\
		&+ \frac{4000 - 0.38\cdot (4000 - 3000)}{1.031^4} + \frac{3000 - 0.38\cdot (3000 - 3000)}{1.031^5} \notag \\
		&= 1052.60 \notag
	\end{align}
	
\end{document}