\documentclass{article}

\usepackage{amsmath,amssymb}
\usepackage{tikz}
\usepackage{pgfplots}
\usepackage{xcolor}
\usepackage[left=2.1cm,right=3.1cm,bottom=3cm,footskip=0.75cm,headsep=0.5cm]{geometry}
\usepackage{enumerate}
\usepackage{enumitem}
\usepackage{marvosym}
\usepackage{tabularx}
\usepackage{multirow}
\usepackage[colorlinks = true, linkcolor = blue, urlcolor  = blue, citecolor = blue, anchorcolor = blue]{hyperref}
\usepackage{parskip}

\usepackage{listings}
\definecolor{lightlightgray}{rgb}{0.95,0.95,0.95}
\definecolor{lila}{rgb}{0.8,0,0.8}
\definecolor{mygray}{rgb}{0.5,0.5,0.5}
\definecolor{mygreen}{rgb}{0,0.8,0.26}
\lstdefinestyle{java} {language=java}
\lstset{language=java,
	basicstyle=\ttfamily,
	keywordstyle=\color{lila},
	commentstyle=\color{lightgray},
	stringstyle=\color{mygreen}\ttfamily,
	backgroundcolor=\color{white},
	showstringspaces=false,
	numbers=left,
	numbersep=10pt,
	numberstyle=\color{mygray}\ttfamily,
	identifierstyle=\color{blue},
	xleftmargin=.1\textwidth, 
	%xrightmargin=.1\textwidth,
	escapechar=§,
}

\usepackage[utf8]{inputenc}

\renewcommand*{\arraystretch}{1.4}

\newcolumntype{L}[1]{>{\raggedright\arraybackslash}p{#1}}
\newcolumntype{R}[1]{>{\raggedleft\arraybackslash}p{#1}}
\newcolumntype{C}[1]{>{\centering\let\newline\\\arraybackslash\hspace{0pt}}m{#1}}

\newcommand{\E}{\mathbb{E}}
\DeclareMathOperator{\rk}{rk}
\DeclareMathOperator{\Var}{Var}
\DeclareMathOperator{\Cov}{Cov}
\DeclareMathOperator{\SD}{SD}
\DeclareMathOperator{\Cor}{Cor}
\DeclareMathOperator{\RBF}{RBF}

\title{\textbf{Investition und Finanzierung, Vertiefung Tutorium 1}}
\author{\textsc{Henry Haustein}}
\date{}

\begin{document}
	\maketitle
	
	\section*{Kapitalanlange - Einfache Verzinsung}
	Wir müssen folgende Gleichung lösen:
	\begin{align}
		2\cdot K_0 &= K_0\cdot (1+i\cdot n) \notag \\
		2\cdot K_0 &= K_0\cdot (1 + 0.1\cdot n) \notag \\
		2 &= 1+0.1\cdot n \notag \\
		1 &= 0.1 \cdot n \notag \\
		n &= 10 \notag
	\end{align}

	\section*{Kapitalanlange - Zinseszinsrechnung}
	Wir müssen folgende Gleichung lösen:
	\begin{align}
		2\cdot K_0 &= K_0\cdot (1+i)^n \notag \\
		2\cdot K_0 &= K_0\cdot (1 + 0.05)^n \notag \\
		2 &= (1+0.05)^n \notag \\
		\log(2) &= n\cdot \log(1+0.05) \notag \\
		n &= \frac{\log(2)}{\log(1+0.05)} \notag \\
		n &\approx 14.21 \notag 
	\end{align}

	\section*{Rentenzahlung - Vorschüsse Ratenzahlung}
	Der Betrag ist
	\begin{align}
		K &= \underbrace{200\cdot \left(1+\frac{12}{12}\cdot 5\%\right)}_{\text{Januar}} + \underbrace{200\cdot \left(1+\frac{11}{12}\cdot 5\%\right)}_{\text{Februar}} + \dots + \underbrace{200\cdot \left(1+\frac{1}{12}\cdot 5\%\right)}_{\text{Dezember}} \notag \\
		&= \sum_{m=1}^{12} 200\cdot\left(1+\frac{12-m+1}{12}\cdot 5\%\right) \notag \\
		&= 2465 \notag
	\end{align}

	\section*{Rentenzahlung - Nachschüssige Ratenzahlung}
	Berechnen wir zuerst den Betrag bei nachschüssiger Rentenzahlung
	\begin{align}
		K &= \underbrace{200\cdot \left(1+\frac{11}{12}\cdot 5\%\right)}_{\text{Januar}} + \underbrace{200\cdot \left(1+\frac{10}{12}\cdot 5\%\right)}_{\text{Februar}} + \dots + \underbrace{200\cdot \left(1+\frac{0}{12}\cdot 5\%\right)}_{\text{Dezember}} \notag \\
		&= \sum_{m=1}^{12} 200\cdot\left(1+\frac{12-m}{12}\cdot 5\%\right) \notag \\
		&= 2455 \notag
	\end{align}
	Damit ist der Unterschied 10.
	
	\section*{Gemischte Verzinsung - Gemischte Verzinsung bei jährlicher Zinszahlung}
	Wir werden 3 Abschnitte betrachten:
	\begin{itemize}
		\item 10.6.2016 - 30.12.2016: 200 Tage
		\item 1.1.2017 - 30.12.2019: 3 Jahre
		\item 1.1.2020 - 20.12.2020: 350 Tage
	\end{itemize}
	Damit befinden sich auf dem Konto:
	\begin{align}
		K &= 10000\cdot \left(1+\frac{200}{360}\cdot 5\%\right) \cdot (1+0.05)^3 \cdot \left(1+\frac{350}{360}\cdot 5\%\right) \notag \\
		&= 12476.18 \notag
	\end{align}
	
	\section*{Barwert und Endwert - Endwert}
	Die Endwertberechnung nach dem 6. Jahr ist:
	\begin{align}
		EW_6 &= 100\cdot 1.1^5 + 400 \cdot 1.1^3 + 200\cdot 1.1^2 \notag \\
		&= 935.45 \notag
	\end{align}
	Für den Endwert nach dem 3. Jahr dürfen wir die Einzahlung im 4. Jahr nicht mehr betrachten:
	\begin{align}
		EW_3 &= 100\cdot1.1^2 + 400 \notag \\
		&= 521 \notag
	\end{align}

	\section*{Barwert und Endwert - Barwert}
	Der Barwert zu Beginn des Jahres 1 ist
	\begin{align}
		BW_1 &= \frac{200}{1.05} + \frac{600}{1.05^2} + \frac{50}{1.05^3} + \frac{450}{1.05^4} + \frac{500}{1.05^5} \notag \\
		&= 1539.86 \notag
	\end{align}
	Für den Barwert zu Beginn des dritten Jahres müssen wir die ersten Zahlungen aufzinsen und die letzten Zahlungen abzinsen, also
	\begin{align}
		BW_3 &= 200\cdot 1.05 + 600 + \frac{50}{1.05} + \frac{450}{1.05^2} + \frac{500}{1.05^3} \notag \\
		&= 1697.70 \notag
	\end{align}

	\section*{Umschuldung - Unternehmen A}
	Schauen wir uns an, wie sich die Verbindlichkeiten über die 6 Jahre entwickeln. Die zu zahlende Rate wird mit $R$ bezeichnet.
	\begin{center}
		\begin{tabular}{l|r|r|r|r}
			\textbf{Jahr} & \textbf{Anfang des Jahres} & \textbf{Zinsen} & \textbf{Ratenzahlung} & \textbf{Ende des Jahres} \\
			\hline
			1 & -10000 & -500 & & -10500 \\
			2 & -10500 & -525 & & -11025 \\
			3 & -11025 & 551.25 & $R$ & $K_3$ \\
			4 & $K_3$ & $Z_4$ & $R$ & $K_4$ \\
			5 & $K_4$ & $Z_5$ & & $K_5$ \\
			6 & $K_5$ & $Z_6$ & $R$ & 0
		\end{tabular}
	\end{center}
	Eine Möglichkeit wäre die Tabelle in Excel umzusetzen und mittels Excel-Solver lösen zu lassen, aber wir versuchen die Vorgänge in den 6 Jahren in eine Gleichung zu schreiben:
	\begin{itemize}
		\item Wir starten mit 10000 Schulden, die 3 mal verzinst werden bevor die erste Ratenzahlung kommt.
		\item Dieser Betrag wird noch mal verzinst und es wird noch eine Rate bezahlt.
		\item Jetzt folgen noch 2 Verzinsungen bevor die letzte Rate gezahlt wird.
		\item Am Ende ist das Konto bei 0.
	\end{itemize}
	\begin{align}
		0 &= (((-10000 \cdot 1.05^3) + R) \cdot 1.05 + R) \cdot 1.05^2 + R \notag \\
		R &\approx 4110.57 \notag
	\end{align}

	\section*{Umschuldung - Unternehmen B}
	Analog zur vorherigen Aufgabe
	\begin{center}
		\begin{tabular}{l|r|r|r|r}
			\textbf{Jahr} & \textbf{Anfang des Jahres} & \textbf{Zinsen} & \textbf{Ratenzahlung} & \textbf{Ende des Jahres} \\
			\hline
			1 & -50000 & -4000 & & -54000 \\
			2 & -54000 & -4320 & $R$ & $K_2$ \\
			3 & $K_2$ & $Z_3$ & & $K_3$ \\
			4 & $K_3$ & $Z_4$ & & $K_4$ \\
			5 & $K_4$ & $Z_5$ & $R$ & $K_5$ \\
			6 & $K_5$ & $Z_6$ & $R$ & 0
		\end{tabular}
	\end{center}
	Wieder versuchen wir das in eine Gleichung zu schreiben:
	\begin{itemize}
		\item Wir starten mit 50000 Schulden, die 2 mal verzinst werden bevor die erste Ratenzahlung kommt.
		\item Dieser Betrag wird noch 3 mal verzinst und es wird noch eine Rate bezahlt.
		\item Jetzt folgt noch 1 Verzinsung bevor die letzte Rate gezahlt wird.
		\item Am Ende ist das Konto bei 0.
	\end{itemize}
	\begin{align}
		0 &= (((-50000 \cdot 1.08^2) + R) \cdot 1.08^3 + R) \cdot 1.08 + R \notag \\
		R &\approx 23061.76 \notag
	\end{align}
	
\end{document}